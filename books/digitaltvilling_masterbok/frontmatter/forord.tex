\chapter*{Forord}
\addcontentsline{toc}{chapter}{Forord}

Digitale tvillinger har på kort tid etablert seg som en nøkkelkomponent i utviklingen av fremtidens produkter, tjenester og samfunnsinfrastruktur. Norske virksomheter står overfor et stort behov for kandidater som både forstår den teoretiske bakgrunnen og kan anvende teknologien i praksis. Denne boken er skrevet for masterstudenter som ønsker å bygge en solid kompetanseplattform som kombinerer modellering, dataanalyse, systemforståelse og organisatorisk innsikt.

Boken er strukturert i tre deler. Første del gir fundamentet gjennom begrepsforståelse, modellering og datahåndtering. Andre del fokuserer på metodikk, simulering og læring, mens tredje del knytter teorien til implementering, styring og norske case. Vi legger vekt på å presentere både teknisk dybde og refleksjon rundt etikk, bærekraft og endringsledelse.

Utviklingen av boken skjer iterativt. Vi inviterer lesere, studenter og fagmiljø til å bidra med erfaringer, eksempler og faglige innspill. Gjennom prosjektstrukturen som er beskrevet i \texttt{instruksjoner.md} og oppgaveoversikten i \texttt{task\_queue.md}, kan vi sammen bygge en ressurs som støtter masterstudenter i å lykkes med digitale tvillinger i praksis.

\bigskip
\begin{flushright}
Redaksjonen
\end{flushright}
