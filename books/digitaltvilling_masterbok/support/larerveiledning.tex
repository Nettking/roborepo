% Lærerveiledning for «Digitale tvillinger i praksis for masterstudenter»
% Denne filen gir undervisere et utgangspunkt for planlegging av kursopplegg.

\chapter{Lærerveiledning}

\section{Formål og målgruppe}
Denne lærerveiledningen er laget for fagansvarlige og undervisere som skal bruke \emph{Digitale tvillinger i praksis for masterstudenter} i et emne eller modul. Veiledningen beskriver hvordan kapitlene kan fordeles over en seks ukers intensivperiode, foreslår læringsaktiviteter, vurderingsformer og tilpasninger for ulike studentgrupper. Den bygger på at studentene har grunnleggende forkunnskaper i programmering, systemmodellering eller dataanalyse, men åpner for differensiering mellom studenter fra teknologi, økonomi og informatikk.

\section{Overordnede læringsmål}
Etter gjennomført undervisningsopplegg skal studentene kunne:
\begin{itemize}
    \item Forklare sentrale begreper, metoder og verktøy knyttet til digitale tvillinger.
    \item Analysere hvordan dataflyt, modellering og styring henger sammen i et helhetlig digitalt tvilling-prosjekt.
    \item Anvende metodikk for å planlegge, validere og presentere en digital tvilling i en valgt sektor.
    \item Vurdere etiske, organisatoriske og tekniske konsekvenser av å innføre digitale tvillinger i praksis.
\end{itemize}

\section{Struktur og gjennomføring}
Veiledningen tar utgangspunkt i en seks ukers modul, men aktivitetene kan skaleres til et helt semester ved å utvide hvert tema med flere caser, laboratorier eller gjesteforelesninger. Tabellen under foreslår en rytme som kombinerer forelesning, arbeidsøkter og vurdering.

\begin{tabular}{p{1.3cm}p{4cm}p{5.2cm}p{4cm}}
\toprule
\textbf{Uke} & \textbf{Kapitler} & \textbf{Fokus og aktiviteter} & \textbf{Vurderingspunkter} \\
\midrule
1 & Kapittel 1--2 & Introduksjon til begreper, historiske linjer og modellering. Workshop der studentene kartlegger egne caser. & Refleksjonsnotat (1 side) om faglig bakgrunn og relevante modeller. \\
2 & Kapittel 3 & Dataflyt, integrasjonsmønstre og arkitektur. Gruppearbeid med systemkartlegging. & Leveranse: Dataflytdiagram og liste over datakilder med risikoanalyse. \\
3 & Kapittel 4 & Simulering og analyse. Dataverksted med demonstrasjon av verktøy (AnyLogic, Simulink, Modelica). & Kort laboratorierapport som beskriver valgt simuleringsopplegg. \\
4 & Kapittel 5 & Maskinlæring og optimalisering. Seminar om algoritmevalg og modellovervåking. & Pitch av ML-basert forbedring til caseprosjektet. \\
5 & Kapittel 6--7 & Kvalitetssikring, validering, styring og livssyklus. Rolle- og ansvarsøvelse. & RACI-matrise for caseprosjektet og plan for kontinuerlig validering. \\
6 & Kapittel 8--9 & Bruksområder, sektorvise case og fremtidsblikk. Prosjektpresentasjoner. & Sluttpresentasjon (15 minutter) og skriftlig prosjektoppsummering (4--6 sider). \\
\bottomrule
\end{tabular}

\section{Anbefalte læringsaktiviteter}
\subsection{Før oppstart}
\begin{itemize}
    \item Del ut en forhåndsundersøkelse der studentene beskriver forkunnskaper, sektorinteresse og ønsket case.
    \item Sett sammen tverrfaglige team (3--4 studenter) basert på utfyllende kompetanse.
    \item Del ressurser som forklarer grunnleggende begreper i IoT, dataforvaltning og modellbasert systemutvikling.
\end{itemize}

\subsection{Ukentlig struktur}
Hver uke kan organiseres rundt tre faste elementer:
\begin{enumerate}
    \item \textbf{Forelesning/innledning (2 timer):} Oppsummer hovedteori fra kapitlene, vis norske bransjeeksempler og introduksjon til ukens oppgave.
    \item \textbf{Arbeidsseminar (2 timer):} Studentene jobber i grupper med å anvende teorien på eget case. Underviser og eventuelle mentorer gir veiledning.
    \item \textbf{Refleksjon eller faglig journal (30 minutter):} Individuell innsending med læringspunkter, spørsmål og planer for neste uke.
\end{enumerate}

\section{Casevalg og modenhetsvurdering}
Planen for caser bør bygge på de samme kriteriene som presenteres i Kapittel~8, slik at studenter, fagfeller og eksterne partnere får et felles språk for prioritering. Introduser matrisen under allerede i uke~1 når gruppene kartlegger mulige caser, og bruk den aktivt når dere beslutter hvilke prosjekter som skal videreføres i undervisningen.

\subsection*{Vurderingsmatrise for casevalg}
Hvert case skåres fra 1 (lav modenhet) til 5 (høy modenhet). Noter begrunnelse for hver skår og lenk til dokumentasjon slik at vurderingen kan etterprøves.

\begin{longtable}{p{0.23\textwidth}p{0.26\textwidth}p{0.45\textwidth}}
\toprule
\textbf{Kriterium} & \textbf{Hva vurderes?} & \textbf{Eksempel på skåringsskala (1--5)} \\
\midrule
\endfirsthead
\toprule
\textbf{Kriterium} & \textbf{Hva vurderes?} & \textbf{Eksempel på skåringsskala (1--5)} \\
\midrule
\endhead
Datagrunnlag og tilgang & Tilgjengelighet, kvalitet og rettigheter til data som trengs for tvillingen. & 1~= fragmenterte kilder uten tilgangsavklaringer. 3~= etablerte datastrømmer, men mangler kvalitetssikring. 5~= sanntidsdata med avtalte API-er, eierskap og datakvalitetsrutiner. \\
\addlinespace
Organisatorisk forankring & Hvor godt caset er støttet av ledelse, fagmiljø og brukere. & 1~= initiativ fra enkeltperson uten mandat. 3~= prosjektgruppe med midlertidig finansiering. 5~= tverrfaglig styringsmodell med dedikerte roller og gevinstansvar. \\
\addlinespace
Teknologisk modenhet & Modenhet på modeller, integrasjonsplattform og driftsoppsett. & 1~= konseptskisse uten implementert plattform. 3~= pilot i begrenset produksjon. 5~= robust løsning med DevOps-/MLOps-sløyfer og overvåkning i drift. \\
\addlinespace
Gevinst- og læringspotensial & Forventet effekt for virksomheten og læringsverdi for studenter. & 1~= uklare mål og lite overføringsverdi. 3~= tydelige KPI-er, men begrenset dokumentasjon. 5~= målbare gevinster, åpne data/artefakter og tydelig kobling til læringsmål. \\
\addlinespace
Risiko og etterlevelse & Regulatoriske, sikkerhets- og etiske forutsetninger. & 1~= ukjente krav eller høy personvernrisiko. 3~= identifiserte tiltak, men ufullstendig dokumentasjon. 5~= komplette risikovurderinger, etterlevelsesplan og jevnlig revisjon. \\
\bottomrule
\end{longtable}

\paragraph{Arbeidsprosess}
\begin{enumerate}
    \item Sett sammen et tverrfaglig vurderingsteam (for eksempel faglærer, dataarkitekt og representant fra partnerbedrift) og avklar beslutningen som skal tas.
    \item Samle prosjektmandat, tekniske beskrivelser, datalister og gevinstplaner før vurderingen. Bruk ressursoversikten i \autoref{appendix:ressurser} dersom caset mangler støttedata eller verktøy.
    \item La studentgruppene skåre caset sitt individuelt før dere gjør en felles vurdering i plenum. Diskuter avvik og dokumenter en enighet i fagfellelogg eller læringsplattform.
    \item Identifiser tiltak for kriterier som får skår 1--2, og kommuniser tydelig om caset skal prioriteres, videreutvikles eller settes på vent. Del resultatet med studentene slik at forventninger og forberedelser blir transparente.
\end{enumerate}

\section{Vurderingsdesign}
Et balansert vurderingsopplegg kan inkludere:
\begin{itemize}
    \item \textbf{Underveisvurdering (40\%):} Består av ukentlige del-leveranser (refleksjonsnotat, dataflytdiagram, laboratorierapport, pitch, valideringsplan).
    \item \textbf{Prosjektarbeid (40\%):} Gruppevis caseprosjekt med skriftlig rapport og munnlig presentasjon. Oppmuntre til samarbeid med en ekstern industripartner for økt relevans.
    \item \textbf{Individuell muntlig vurdering (20\%):} Kort samtale hvor hver student forklarer sin rolle, hvordan teori ble anvendt og hvilke forbedringer som er mulige.
\end{itemize}
Vektleggingen kan tilpasses lokale krav. For intensive moduler kan underveisvurdering slås sammen til en samlet prosjektleveranse.

\subsection{Rubrikker for vurdering}
Rubrikkene under gir et felles språk for vurdering og gjør det enklere å gi målrettede tilbakemeldinger. Skalaen er inndelt i fire nivåer:
\emph{Utviklingsbehov}, \emph{Grunnleggende}, \emph{God} og \emph{Fremragende}. En enkel metode er å knytte nivåene til poeng (1--4) og multiplisere med vektingen i vurderingsplanen. Rubrikkene kan også brukes formativt ved at studentene gjør egenvurdering før innsending.

\subsubsection{Underveisleveranser}
\begin{table}[h]
    \centering
    \caption{Vurderingsrubrikk for underveisleveranser}
    \label{tab:rubrikk-underveis}
    \begin{tabular}{p{2.8cm}p{3.0cm}p{3.0cm}p{3.0cm}p{3.0cm}}
        \toprule
        \textbf{Kriterium} & \textbf{Utviklingsbehov} & \textbf{Grunnleggende} & \textbf{God} & \textbf{Fremragende} \\
        \midrule
        Problemforståelse & Oppgaven tolkes uklart; mål og avgrensning mangler. & Beskriver målet, men kopler det svakt til kapittelstoffet. & Klargjør mål, antakelser og kobler til relevante modeller eller teorier. & Formulerer tydelige mål, begrunner valg med teori og foreslår alternative tilnærminger. \\
        Metode og verktøy & Metodevalg er uklare eller feil for caset. & Velger metode, men begrunner den kun delvis og dokumenterer ikke arbeidsflyt. & Anvender hensiktsmessig metode med kort begrunnelse og enkel dokumentasjon. & Kombinerer flere metoder, viser arbeidsflyt og reflekterer over kvalitet og begrensninger. \\
        Samarbeid og leveranse & Teamet leverer fragmentert produkt uten felles struktur. & Leveransen er sammenhengende, men roller og ansvar er uklare. & Roller, versjonskontroll og kommunikasjonskanaler er beskrevet. & Viser strukturert samarbeid, læringspunkter og plan for neste iterasjon. \\
        \bottomrule
    \end{tabular}
\end{table}

\subsubsection{Prosjektpresentasjon}
\begin{table}[h]
    \centering
    \caption{Vurderingsrubrikk for prosjektpresentasjon}
    \label{tab:rubrikk-presentasjon}
    \begin{tabular}{p{2.8cm}p{3.0cm}p{3.0cm}p{3.0cm}p{3.0cm}}
        \toprule
        \textbf{Kriterium} & \textbf{Utviklingsbehov} & \textbf{Grunnleggende} & \textbf{God} & \textbf{Fremragende} \\
        \midrule
        Struktur og historiefortelling & Presentasjonen mangler rød tråd og tidsstyring. & Innledning og avslutning finnes, men budskapet er uklart. & Tidsbruken er balansert og hovedbudskap formidles tydelig. & Dramaturgien bygger opp et overbevisende narrativ med klare overganger. \\
        Faglig innhold & Ufullstendige eller feilaktige beskrivelser av data, modell og gevinst. & Dekker hovedkomponenter, men med begrenset dybde eller evidens. & Underbygger løsning med relevante data, modeller og gevinstindikatorer. & Viser kritisk refleksjon, risikoanalyse og kobling til forskningsfront eller bransjepraksis. \\
        Visualisering og formidling & Lysbilder er uleselige eller mangler kilder. & Visualiseringer støtter budskapet, men kan forbedres for tilgjengelighet. & Grafikk følger designprinsipp, kilder oppgis og budskapet styrkes. & Bruker visualiseringer og demonstrasjoner aktivt for å engasjere publikum og forklare komplekse sammenhenger. \\
        Respons på spørsmål & Svarer uklart og viser liten innsikt i begrensninger. & Kan svare på faktaspørsmål, men må ofte støtte seg på notater. & Gir presise svar, diskuterer usikkerhet og mulige forbedringer. & Inviterer til dialog, knytter svar til strategiske implikasjoner og foreslår videre arbeid. \\
        \bottomrule
    \end{tabular}
\end{table}

\subsubsection{Skriftlig rapport}
\begin{table}[h]
    \centering
    \caption{Vurderingsrubrikk for skriftlig rapport}
    \label{tab:rubrikk-rapport}
    \begin{tabular}{p{2.8cm}p{3.0cm}p{3.0cm}p{3.0cm}p{3.0cm}}
        \toprule
        \textbf{Kriterium} & \textbf{Utviklingsbehov} & \textbf{Grunnleggende} & \textbf{God} & \textbf{Fremragende} \\
        \midrule
        Struktur og språk & Rapporten mangler tydelig struktur, og språket hemmer forståelsen. & Følger foreslått disposisjon, men har flere uklarheter eller språkfeil. & Klart språk med logisk oppbygning, figurer og tabeller forklart i teksten. & Meget godt språk, tydelige overganger, konsistent terminologi og språkvask. \\
        Analyse og argumentasjon & Analysen er overflatisk eller støtter seg på uverifiserte antakelser. & Trekker inn nøkkeldata, men begrenser drøftingen av funn. & Diskuterer funn med støtte i data, teori og usikkerhet. & Gjennomfører helhetlig analyse med scenarioer, sensitivitetsdrøfting og handlingsanbefalinger. \\
        Kildebruk og etterprøvbarhet & Manglende referanser eller uklar kildebruk. & Bruker noen kilder, men med varierende format eller relevans. & Konsistent bruk av relevante kilder og vedlegg som viser metode og data. & Omfattende og kritisk kildebruk, vedlagt kode/data med lisens og beskrivelse av reproduksjonsløp. \\
        Gevinst og implementering & Gevinst og implementering behandles overflatisk. & Beskriver gevinstmuligheter, men uten plan for oppfølging. & Presenterer gevinstindikatorer, tiltak og ansvar. & Gir helhetlig implementeringsplan med risikohåndtering, gevinststyring og modenhetsløft. \\
        \bottomrule
    \end{tabular}
\end{table}

\subsubsection{Individuell muntlig prøve}
\begin{table}[h]
    \centering
    \caption{Vurderingsrubrikk for individuell muntlig prøve}
    \label{tab:rubrikk-muntlig}
    \begin{tabular}{p{2.8cm}p{3.0cm}p{3.0cm}p{3.0cm}p{3.0cm}}
        \toprule
        \textbf{Kriterium} & \textbf{Utviklingsbehov} & \textbf{Grunnleggende} & \textbf{God} & \textbf{Fremragende} \\
        \midrule
        Egen rolle og bidrag & Klarer ikke beskrive eget arbeid eller læringsutbytte. & Beskriver hovedoppgaver, men uten refleksjon over valg og konsekvenser. & Forklarer rolle, leveranser og hvordan beslutninger ble tatt. & Viser helhetlig forståelse av teamets arbeid, begrunner valg og knytter til faglig utvikling. \\
        Faglig resonnering & Sliter med å svare på oppfølgingsspørsmål eller bruke fagbegreper korrekt. & Bruker sentrale begreper, men må ha hjelp til å knytte dem til caset. & Resonerer sikkert med begreper, standarder og teorier fra boken. & Demonstrerer avansert resonnering, refererer til forskning og foreslår videreutvikling. \\
        Kritisk vurdering & Identifiserer få begrensninger eller risikoer. & Nevner noen svakheter, men uten tiltak. & Reflekterer over begrensninger og foreslår realistiske forbedringer. & Gjør helhetlig risikovurdering, prioriterer tiltak og knytter dem til organisasjonens strategi. \\
        Kommunikasjon & Fremføringen er usammenhengende eller ustrukturert. & Gir forståelige svar, men med enkelte avsporinger. & Kommuniserer klart og strukturert med passende terminologi. & Leverer presise, relevante svar og knytter dem til lytterens perspektiv. \\
        \bottomrule
    \end{tabular}
\end{table}

\paragraph{Praktiske tips}
Informer studentene om rubrikkene allerede i uke~1 og legg dem inn i læringsplattformen. Underveisleveranser kan vurderes til bestått/ikke bestått ved å kreve minst nivået \emph{Grunnleggende} på alle kriterier. For summative vurderinger anbefales sensorpar som scorer uavhengig før de enes om sluttkarakter.

\section{Tilpasning og differensiering}
\subsection{Studenter med ulik bakgrunn}
\begin{itemize}
    \item \textbf{Teknologistudenter:} Utfordres med dypere modellering, simulering og kodeimplementasjon.
    \item \textbf{Økonomi- og ledelsesstudenter:} Fokus på forretningsmodeller, gevinstrealisering og governance.
    \item \textbf{Informatikkstudenter:} Tilleggsoppgaver knyttet til dataplattformer, API-er og skaleringsstrategier.
\end{itemize}
Tilby frivillige støttesesjoner i matematisk modellering og programmering ved behov.

\subsection{Digitale og hybride formater}
\begin{itemize}
    \item Gjennomfør forelesninger som synkrone webinarer med opptak. Bruk digitale tavler for å illustrere modeller.
    \item Sett opp digitale labmiljøer (for eksempel JupyterHub eller skybaserte simuleringstjenester) slik at studentene kan jobbe uten lokal installasjon.
    \item Bruk samarbeidsverktøy (Miro, Teams, Notion) for å dele modeller, sjekklister og prosjektstatus.
\end{itemize}

\section{Samarbeid med eksterne partnere}
Et strukturert samarbeid med en ekstern virksomhet gir studentene realistiske rammer og tydelig forventningsavklaring. Be læringsgruppene avtale partner så tidlig som mulig (senest uke~2) slik at datatilgang, møter og eventuelle konfidensialitetskrav kan håndteres. Involver fakultetets juridiske rådgivere dersom det kreves særskilte avtaler om data, personvern eller immaterielle rettigheter.

\paragraph{Bruk av prosjektkontrakt}
Kontrakten bør gjennomgås i et oppstartsmøte med partneren. Studentene fyller ut utkastet, underviser gir tilbakemelding og begge parter signerer digitalt eller på papir. Avtal at kontrakten revideres ved større endringer, og legg den i prosjektets felles mappe eller læringsplattform.

\subsection*{Mal for prosjektkontrakt}
\noindent\textbf{Prosjektkontrakt for digital tvilling-prosjekt}\par
\noindent\textit{Denne malen tilpasses av studentgruppen og partneren. Fjern veiledende tekst i klammer og fyll inn konkrete detaljer.}

\begin{enumerate}
    \item \textbf{Parter}\newline
    Studentgruppe: \rule{0.6\linewidth}{0.4pt}\\
    Ekstern partner (navn/avdeling): \rule{0.6\linewidth}{0.4pt}\\
    Kontaktpersoner (student/partner/underviser): \rule{0.6\linewidth}{0.4pt}

    \item \textbf{Formål og læringsmål}\newline
    Kort beskrivelse av hvorfor prosjektet gjennomføres, og hvilke faglige mål studentene skal oppnå. Inkluder eventuelle innovasjons- eller bærekraftsambisjoner partneren har.

    \item \textbf{Omfang og avgrensning}\newline
    Beskriv prosess, system eller tjeneste som tvillingen skal dekke. List eksplisitt hva som er utenfor scope for å unngå scope creep.

    \item \textbf{Planlagte leveranser}\newline
    Angi leveranser med dato (for eksempel behovskartlegging, dataflytanalyse, prototype, rapport, presentasjon). Oppgi format og eventuelle godkjenningspunkter.

    \item \textbf{Roller og ansvar}\newline
    Definer ansvar for studentgruppen (analyse, modellering, dokumentasjon), partneren (datatilgang, faglig mentor, testarena) og underviser (kvalitetssikring, vurdering). Referer gjerne til RACI-matrisen fra uke~5.

    \item \textbf{Datahåndtering og konfidensialitet}\newline
    Beskriv datakilder, sikkerhetsnivå, anonymisering og lagringsløsninger. Noter om det kreves NDA, og hvordan personvern (GDPR) ivaretas.

    \item \textbf{Ressurser og støtte}\newline
    Avklar hvilke verktøy, lokaler eller systemtilganger partneren stiller til disposisjon, og hvilke ressurser studentene må skaffe selv.

    \item \textbf{Møteplan og kommunikasjon}\newline
    Oppgi møtefrekvens (f.eks. ukentlig stand-up), kanaler (Teams, e-post, Slack) og responstider. Legg inn rutine for eskalering ved hindringer.

    \item \textbf{Kvalitetssikring og tilbakemelding}\newline
    Beskriv hvordan leveranser vurderes fortløpende, hvem som godkjenner iterasjoner og hvordan tilbakemeldinger dokumenteres.

    \item \textbf{Varighet og revisjon}\newline
    Registrer prosjektperiode (fra--til), milepæler og hvordan kontrakten revideres ved endrede forutsetninger. Inkluder klausul om mulighet for tidlig avslutning.

    \item \textbf{Signaturer}\newline
    \begin{tabular}{p{0.45\linewidth}p{0.45\linewidth}}
        \rule{0.45\linewidth}{0.4pt} & \rule{0.45\linewidth}{0.4pt}\\
        Studentrepresentant & Partnerrepresentant\\[1.0em]
        \rule{0.45\linewidth}{0.4pt} & \rule{0.45\linewidth}{0.4pt}\\
        Dato & Dato\\
    \end{tabular}
\end{enumerate}

\paragraph{Oppfølging}
Be studentene loggføre status mot kontrakten i den ukentlige refleksjonsjournalen. Ved avvik setter underviser opp et ekstra møte med partneren for å avklare forventninger og revidere planen.

\section{Ressurser og verktøy}
\subsection{Faglitteratur og rapporter}
\begin{itemize}
    \item ISO 23247-serien for rammeverk knyttet til produksjonsindustri.
    \item DNVs anbefalte praksis for digitale tvillinger i maritim sektor.
    \item SINTEF-rapporter om digitale tvillinger i energiforsyning og smarte bygg.
\end{itemize}

\subsection{Verktøyanbefalinger}
\begin{itemize}
    \item Simulering: AnyLogic, Simulink, OpenModelica.
    \item Data- og skyplattformer: Azure Digital Twins, AWS IoT TwinMaker, Siemens MindSphere.
    \item Samhandling og visualisering: Grafana, Power BI, Unity Reflect.
\end{itemize}

\section{Evaluering og kontinuerlig forbedring}
Etter avsluttet modul bør underviserne samle tilbakemeldinger fra studentene og eventuelle eksterne partnere. Diskuter hva som fungerte, hvilke caser som skapte størst engasjement, og hvordan dataflyt eller verktøyoppsett kan forbedres. Dokumenter funn i et eget refleksjonsnotat og oppdater denne lærerveiledningen med nye anbefalinger.

\section{Forslag til videre utvikling}
\begin{itemize}
    \item Utvid veiledningen med detaljerte caser fra offentlig sektor (f.eks. smart by-planlegging) og prosessindustri.
    \item Pilotér vurderingsrubrikkene i kommende modul og juster kriteriene basert på tilbakemeldinger fra sensorer og studenter.
    \item Test og forbedre prosjektkontraktmalen sammen med eksterne partnere etter første modulgjennomføring.
    \item Koordiner med universitetsbibliotek for å sikre tilgang til relevante databaser og tidsskrifter.
\end{itemize}

\section{Kort oppsummering for lærere}
\begin{itemize}
    \item Planlegg tidlig med tverrfaglige grupper og tydelige forventninger.
    \item Kombiner teori med praktisk casearbeid hver uke.
    \item Sikre balansert vurdering som belønner både prosess og sluttprodukt.
    \item Revider opplegget fortløpende basert på erfaringer og innspill.
\end{itemize}

