% Lærerveiledning for «Digitale tvillinger i praksis for masterstudenter»
% Denne filen gir undervisere et utgangspunkt for planlegging av kursopplegg.

\chapter{Lærerveiledning}

\section{Formål og målgruppe}
Denne lærerveiledningen er laget for fagansvarlige og undervisere som skal bruke \emph{Digitale tvillinger i praksis for masterstudenter} i et emne eller modul. Veiledningen beskriver hvordan kapitlene kan fordeles over en seks ukers intensivperiode, foreslår læringsaktiviteter, vurderingsformer og tilpasninger for ulike studentgrupper. Den bygger på at studentene har grunnleggende forkunnskaper i programmering, systemmodellering eller dataanalyse, men åpner for differensiering mellom studenter fra teknologi, økonomi og informatikk.

\section{Overordnede læringsmål}
Etter gjennomført undervisningsopplegg skal studentene kunne:
\begin{itemize}
    \item Forklare sentrale begreper, metoder og verktøy knyttet til digitale tvillinger.
    \item Analysere hvordan dataflyt, modellering og styring henger sammen i et helhetlig digitalt tvilling-prosjekt.
    \item Anvende metodikk for å planlegge, validere og presentere en digital tvilling i en valgt sektor.
    \item Vurdere etiske, organisatoriske og tekniske konsekvenser av å innføre digitale tvillinger i praksis.
\end{itemize}

\section{Struktur og gjennomføring}
Veiledningen tar utgangspunkt i en seks ukers modul, men aktivitetene kan skaleres til et helt semester ved å utvide hvert tema med flere caser, laboratorier eller gjesteforelesninger. Tabellen under foreslår en rytme som kombinerer forelesning, arbeidsøkter og vurdering.

\begin{tabular}{p{1.3cm}p{4cm}p{5.2cm}p{4cm}}
\toprule
\textbf{Uke} & \textbf{Kapitler} & \textbf{Fokus og aktiviteter} & \textbf{Vurderingspunkter} \\
\midrule
1 & Kapittel 1--2 & Introduksjon til begreper, historiske linjer og modellering. Workshop der studentene kartlegger egne caser. & Refleksjonsnotat (1 side) om faglig bakgrunn og relevante modeller. \\
2 & Kapittel 3 & Dataflyt, integrasjonsmønstre og arkitektur. Gruppearbeid med systemkartlegging. & Leveranse: Dataflytdiagram og liste over datakilder med risikoanalyse. \\
3 & Kapittel 4 & Simulering og analyse. Dataverksted med demonstrasjon av verktøy (AnyLogic, Simulink, Modelica). & Kort laboratorierapport som beskriver valgt simuleringsopplegg. \\
4 & Kapittel 5 & Maskinlæring og optimalisering. Seminar om algoritmevalg og modellovervåking. & Pitch av ML-basert forbedring til caseprosjektet. \\
5 & Kapittel 6--7 & Kvalitetssikring, validering, styring og livssyklus. Rolle- og ansvarsøvelse. & RACI-matrise for caseprosjektet og plan for kontinuerlig validering. \\
6 & Kapittel 8--9 & Bruksområder, sektorvise case og fremtidsblikk. Prosjektpresentasjoner. & Sluttpresentasjon (15 minutter) og skriftlig prosjektoppsummering (4--6 sider). \\
\bottomrule
\end{tabular}

\section{Anbefalte læringsaktiviteter}
\subsection{Før oppstart}
\begin{itemize}
    \item Del ut en forhåndsundersøkelse der studentene beskriver forkunnskaper, sektorinteresse og ønsket case.
    \item Sett sammen tverrfaglige team (3--4 studenter) basert på utfyllende kompetanse.
    \item Del ressurser som forklarer grunnleggende begreper i IoT, dataforvaltning og modellbasert systemutvikling.
\end{itemize}

\subsection{Ukentlig struktur}
Hver uke kan organiseres rundt tre faste elementer:
\begin{enumerate}
    \item \textbf{Forelesning/innledning (2 timer):} Oppsummer hovedteori fra kapitlene, vis norske bransjeeksempler og introduksjon til ukens oppgave.
    \item \textbf{Arbeidsseminar (2 timer):} Studentene jobber i grupper med å anvende teorien på eget case. Underviser og eventuelle mentorer gir veiledning.
    \item \textbf{Refleksjon eller faglig journal (30 minutter):} Individuell innsending med læringspunkter, spørsmål og planer for neste uke.
\end{enumerate}

\section{Vurderingsdesign}
Et balansert vurderingsopplegg kan inkludere:
\begin{itemize}
    \item \textbf{Underveisvurdering (40\%):} Består av ukentlige del-leveranser (refleksjonsnotat, dataflytdiagram, laboratorierapport, pitch, valideringsplan).
    \item \textbf{Prosjektarbeid (40\%):} Gruppevis caseprosjekt med skriftlig rapport og munnlig presentasjon. Oppmuntre til samarbeid med en ekstern industripartner for økt relevans.
    \item \textbf{Individuell muntlig vurdering (20\%):} Kort samtale hvor hver student forklarer sin rolle, hvordan teori ble anvendt og hvilke forbedringer som er mulige.
\end{itemize}
Vektleggingen kan tilpasses lokale krav. For intensive moduler kan underveisvurdering slås sammen til en samlet prosjektleveranse.

\section{Tilpasning og differensiering}
\subsection{Studenter med ulik bakgrunn}
\begin{itemize}
    \item \textbf{Teknologistudenter:} Utfordres med dypere modellering, simulering og kodeimplementasjon.
    \item \textbf{Økonomi- og ledelsesstudenter:} Fokus på forretningsmodeller, gevinstrealisering og governance.
    \item \textbf{Informatikkstudenter:} Tilleggsoppgaver knyttet til dataplattformer, API-er og skaleringsstrategier.
\end{itemize}
Tilby frivillige støttesesjoner i matematisk modellering og programmering ved behov.

\subsection{Digitale og hybride formater}
\begin{itemize}
    \item Gjennomfør forelesninger som synkrone webinarer med opptak. Bruk digitale tavler for å illustrere modeller.
    \item Sett opp digitale labmiljøer (for eksempel JupyterHub eller skybaserte simuleringstjenester) slik at studentene kan jobbe uten lokal installasjon.
    \item Bruk samarbeidsverktøy (Miro, Teams, Notion) for å dele modeller, sjekklister og prosjektstatus.
\end{itemize}

\section{Ressurser og verktøy}
\subsection{Faglitteratur og rapporter}
\begin{itemize}
    \item ISO 23247-serien for rammeverk knyttet til produksjonsindustri.
    \item DNVs anbefalte praksis for digitale tvillinger i maritim sektor.
    \item SINTEF-rapporter om digitale tvillinger i energiforsyning og smarte bygg.
\end{itemize}

\subsection{Verktøyanbefalinger}
\begin{itemize}
    \item Simulering: AnyLogic, Simulink, OpenModelica.
    \item Data- og skyplattformer: Azure Digital Twins, AWS IoT TwinMaker, Siemens MindSphere.
    \item Samhandling og visualisering: Grafana, Power BI, Unity Reflect.
\end{itemize}

\section{Evaluering og kontinuerlig forbedring}
Etter avsluttet modul bør underviserne samle tilbakemeldinger fra studentene og eventuelle eksterne partnere. Diskuter hva som fungerte, hvilke caser som skapte størst engasjement, og hvordan dataflyt eller verktøyoppsett kan forbedres. Dokumenter funn i et eget refleksjonsnotat og oppdater denne lærerveiledningen med nye anbefalinger.

\section{Forslag til videre utvikling}
\begin{itemize}
    \item Utvid veiledningen med detaljerte caser fra offentlig sektor (f.eks. smart by-planlegging) og prosessindustri.
    \item Utarbeid vurderingsrubrikker for muntlig presentasjon, rapport og underveisnotater.
    \item Inkluder maler for prosjektkontrakt mellom studentgrupper og eksterne partnere.
    \item Koordiner med universitetsbibliotek for å sikre tilgang til relevante databaser og tidsskrifter.
\end{itemize}

\section{Kort oppsummering for lærere}
\begin{itemize}
    \item Planlegg tidlig med tverrfaglige grupper og tydelige forventninger.
    \item Kombiner teori med praktisk casearbeid hver uke.
    \item Sikre balansert vurdering som belønner både prosess og sluttprodukt.
    \item Revider opplegget fortløpende basert på erfaringer og innspill.
\end{itemize}

