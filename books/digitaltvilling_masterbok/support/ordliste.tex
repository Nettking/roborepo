\chapter{Begrepsliste og ordliste}
\label{appendix:ordliste}

Denne ordlisten samler sentrale faguttrykk som brukes gjennom boken. For hvert begrep angis norsk term, vanlig engelskspråklig
betegnelse og en kort forklaring slik at leseren raskt kan finne tilbake til definisjoner i kapitlene.

\section{Hvordan bruke ordlisten}
\begin{itemize}
    \item Slå opp ukjente begrep før du går videre i kapitlene for å sikre felles begrepsforståelse.
    \item Kryssjekk engelske kilder ved å bruke kolonnen for engelske uttrykk.
    \item Oppdater listen når nye uttrykk tas i bruk i kapitlene eller støttefilene.
\end{itemize}

\begin{longtable}{p{0.27\textwidth}p{0.28\textwidth}p{0.35\textwidth}}
\toprule
\textbf{Begrep (norsk)} & \textbf{Engelsk term} & \textbf{Forklaring} \\
\midrule
\endfirsthead
\toprule
\textbf{Begrep (norsk)} & \textbf{Engelsk term} & \textbf{Forklaring} \\
\midrule
\endhead
Digital tvilling & Digital twin & Virtuell representasjon av et fysisk system som oppdateres med data for analyse, beslutningsstøtte og automatisering. \\
Fysikkbasert modell & Physics-based model & Modell som beskriver systemdynamikk gjennom fysikklovene og differensialligninger, ofte brukt for presise simuleringer. \\
Datadrevet modell & Data-driven model & Modell som lærer sammenhenger direkte fra data ved hjelp av statistikk eller maskinlæring uten eksplisitte fysikklover. \\
Hybrid modell & Hybrid model & Kombinasjon av fysikkbaserte og datadrevne komponenter for å utnytte styrker fra begge tilnærminger. \\
Sensorfusjon & Sensor fusion & Metode for å kombinere målinger fra flere sensorer for å redusere usikkerhet og forbedre tilstandsestimat. \\
Tilstandsestimering & State estimation & Prosess for å beregne interne tilstander i et system basert på modeller og tilgjengelige observasjoner. \\
Dataassimilering & Data assimilation & Integrasjon av sanntidsdata i en modell for å korrigere prognoser og redusere avvik mellom tvilling og virkelighet. \\
Model predictive control (MPC) & Model predictive control & Optimaliserende styringsstrategi som bruker modell og prediksjoner for å beregne kontrollsignaler innenfor gitte begrensninger. \\
Kalibrering & Calibration & Justering av modellparametre mot observasjoner for å sikre at digital tvilling og fysisk system samsvarer. \\
Livssyklusforvaltning & Lifecycle management & Helhetlig styring av systemets faser fra idé til avvikling, inkludert kontinuerlig oppdatering av den digitale tvillingen. \\
Endringsledelse & Change management & Metodikk for å håndtere organisatoriske og menneskelige aspekter ved innføring av digitale tvillinger. \\
Tvillingplattform & Twin platform & Programvare- eller skyplattform som tilbyr modellforvaltning, dataintegrasjon og visualisering for digitale tvillinger. \\
Edge-prosessering & Edge computing & Databehandling som skjer nær sensorkilder for å redusere latenstid og avhengighet av skytilkobling. \\
Sanntidsdatastrøm & Real-time data stream & Kontinuerlig strøm av data som gjøres tilgjengelig umiddelbart for overvåking, analyse eller styring. \\
Datasjø & Data lake & Sentralisert lagringsarkitektur for rådata i ulike formater som kan benyttes av tvillingapplikasjoner. \\
Sensor gateway & Sensor gateway & Enhet eller programvare som aggregerer lokale sensordata og eksponerer dem til tvillingplattformen via standardprotokoller. \\
OPC UA & OPC UA & Åpen kommunikasjonsstandard for industriell dataintegrasjon som støtter semantikk og sikkerhet. \\
SCADA-system & Supervisory Control and Data Acquisition & System for overvåking og kontroll av industrielle prosesser som ofte integreres med digitale tvillinger. \\
MLOps & MLOps & Praksis for å drifte og vedlikeholde maskinlæringsmodeller, inkludert versjonering, deployering og overvåking. \\
Syntetiske data & Synthetic data & Kunstig genererte datasett som brukes til å trene eller teste modeller når reelle data er begrenset eller sensitive. \\
\bottomrule
\end{longtable}

\section{Notater}
Denne ordlisten ble etablert for å støtte kapitler som introduserer nye begreper. Utvid tabellen når kapitlene tar i bruk flere tekniske eller organisatoriske uttrykk.
