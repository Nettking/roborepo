\chapter{Begrepsliste og ordliste}
\label{appendix:ordliste}

Denne ordlisten samler sentrale faguttrykk som brukes gjennom boken. For hvert begrep angis norsk term, vanlig engelskspråklig betegnelse og en kort forklaring slik at leseren raskt kan finne tilbake til definisjoner i kapitlene.

\section{Hvordan bruke ordlisten}
\begin{itemize}
    \item Slå opp ukjente begrep før du går videre i kapitlene for å sikre felles begrepsforståelse.
    \item Kryssjekk engelske kilder ved å bruke kolonnen for engelske uttrykk.
    \item Noter forkortelsene når du lager figurer, tabeller og kodeeksempler for å holde terminologien konsistent.
    \item Bruk referansesøylen i tabellene til å fordype deg i standarder, lærebøker og fagartikler.
\end{itemize}

\subsection{Organisasjon og styring}
\begin{longtable}{p{0.23\textwidth}p{0.16\textwidth}p{0.23\textwidth}p{0.34\textwidth}}
\toprule
\textbf{Begrep (norsk)} & \textbf{Forkortelse} & \textbf{Engelsk term} & \textbf{Forklaring} \\
\midrule
\endfirsthead
\toprule
\textbf{Begrep (norsk)} & \textbf{Forkortelse} & \textbf{Engelsk term} & \textbf{Forklaring} \\
\midrule
\endhead
Digital tvilling & DT & Digital twin & Virtuell representasjon av et fysisk system som oppdateres løpende med data for analyse, beslutningsstøtte og automatisering. \citep{iso23247-2021} \\
Livssyklusforvaltning & -- & Lifecycle management & Helhetlig styring av systemets faser fra idé til avvikling, inkludert kontinuerlig oppdatering av tvillingen og dokumentert eierskap. \citep{dnv2021rp} \\
Endringsledelse & -- & Change management & Metodikk for å håndtere organisatoriske og menneskelige aspekter når digitale tvillinger tas i bruk i drifts- og prosjektløp. \citep{dnv2021a204} \\
Tvillingplattform & -- & Twin platform & Programvare- eller skyplattform som tilbyr modellforvaltning, dataintegrasjon, visualisering og livssyklusoppfølging for tvillinger. \citep{dnv2021rp} \\
Styringsmodell & -- & Governance model & Struktur som definerer beslutningsmyndighet, roller og prosesser for hvordan tvillinginitiativ etableres, prioriteres og følges opp. \citep{dnv2021rp} \\
Porteføljestyring & -- & Portfolio management & Koordinert prioritering og gevinstoppfølging av flere tvillingprosjekter slik at ressurser, gevinster og avhengigheter håndteres samlet. \citep{dnv2021rp} \\
Center of Excellence & CoE & Center of Excellence (CoE) & Kjernegruppe som etablerer standarder, gir veiledning og sikrer gjenbruk og kvalitet på tvers av tvillingprosjekter. \citep{dnv2021rp} \\
Change Advisory Board & CAB & Change Advisory Board (CAB) & Tverrfaglig forum som vurderer, tester og godkjenner endringsforslag før produksjonsmiljøet til tvillingen oppdateres. \citep{dnv2021a204} \\
Dataforvaltningsstyre & -- & Data governance board & Organ som setter retningslinjer for datakvalitet, tilgang og deling, og som sikrer etterlevelse av juridiske krav og interne policyer. \citep{idsa2023ram} \\
Datadelingavtale & -- & Data sharing agreement & Avtale som beskriver eierskap, sikkerhetstiltak og bruksvilkår når data deles mellom partnere i et tvillingøkosystem. \citep{idsa2023ram} \\
Modenhetsnivå & -- & Maturity level & Trinn i en modenhetsmodell som beskriver hvor langt organisasjonen har kommet i å integrere digitale tvillinger i arbeidsprosesser. \citep{dnv2021rp} \\
Kontinuerlig forbedring & -- & Continuous improvement & Systematisk arbeid med å analysere erfaringer og implementere tiltak som øker verdi, sikkerhet og bærekraft i tvillingløsninger over tid. \citep{dnv2021rp} \\
Forprosjektcanvas & -- & Project initiation canvas & Visuelt rammeverk som strukturerer problemforståelse, interessenter, gevinsthypoteser og ressursbehov før et tvillingprosjekt formaliseres. \citep{dnv2021rp} \\
Datastrømskartlegging & -- & Data flow mapping & Kartlegging av hvordan data beveger seg mellom kilder, plattformer og tvillingkomponenter, inkludert lisens- og etterlevelseskrav. \citep{dnv2021rp} \\
Gevinsthypotese & -- & Benefit hypothesis & Begrunnet antakelse om hvilken verdi et tiltak skal gi, brukt til å prioritere og evaluere tvillinginitiativ. \citep{dnv2021rp} \\
Risiko- og gevinstjournal & -- & Risk and benefit log & Register som sporer risikofaktorer, tiltak, forventede gevinster og status for oppfølging på tvers av tvillingprosjekter. \citep{dnv2021rp} \\
Etterprøvbarhetslogg & -- & Traceability log & Oversikt over beslutninger, versjoner og dokumentasjon som gjør det mulig å etterprøve sammenheng mellom data, modeller og leveranser. \citep{dnv2021rp} \\
Fagfelleklareringssjekkliste & -- & Peer review clearance checklist & Kontrolliste som bekrefter at dokumentasjon, arbeidsark og kvalitetskrav er oppfylt før leveranser sendes til fagfellevurdering. \citep{dnv2021rp} \\
Tillitsankertjeneste & -- & Trust anchor service & Sertifiseringskomponent i et dataspace som verifiserer identitet, sikkerhetsnivå og samsvar før parter får tilgang til tvillingdata. \citep{idsa2023ram} \\
Dataspace-operatør & -- & Dataspace operator & Rolle som drifter infrastrukturen, policyene og onboarding-prosessen i et dataspace slik at tvillingaktører kan dele data sikkert. \citep{digitalnorway2024dataspace} \\
Styringshåndbok & -- & Governance playbook & Dokumentasjon som beskriver roller, møtearenaer og beslutningsflyt for tvillingporteføljen, brukt til å forankre endringer i hele organisasjonen. \citep{digitalnorway2024dataspace} \\
\bottomrule
\end{longtable}

\subsection{Modellering og analyse}
\begin{longtable}{p{0.23\textwidth}p{0.16\textwidth}p{0.23\textwidth}p{0.34\textwidth}}
\toprule
\textbf{Begrep (norsk)} & \textbf{Forkortelse} & \textbf{Engelsk term} & \textbf{Forklaring} \\
\midrule
\endfirsthead
\toprule
\textbf{Begrep (norsk)} & \textbf{Forkortelse} & \textbf{Engelsk term} & \textbf{Forklaring} \\
\midrule
\endhead
Fysikkbasert modell & -- & Physics-based model & Modell som beskriver systemdynamikk gjennom fysikklovene og differensialligninger, brukt for presise simuleringer og regulatoriske analyser. \citep{law2015simulation} \\
Datadrevet modell & -- & Data-driven model & Modell som lærer sammenhenger fra historiske eller sanntidsdata ved hjelp av statistikk og maskinlæring uten eksplisitte fysikklover. \citep{bishop2006pattern} \\
Hybrid modell & -- & Hybrid model & Kombinasjon av fysikkbaserte og datadrevne komponenter for å utnytte styrker fra begge tilnærminger og håndtere usikkerhet. \citep{tao2018digital} \\
Sensorfusjon & -- & Sensor fusion & Metode for å kombinere målinger fra flere sensorer for å redusere usikkerhet og forbedre estimater av systemtilstanden. \citep{gustafsson2010statistical} \\
Tilstandsestimering & -- & State estimation & Prosess for å beregne interne tilstander i et system basert på modeller og tilgjengelige observasjoner, ofte med filtreringsmetoder. \citep{gustafsson2010statistical} \\
Dataassimilering & -- & Data assimilation & Integrasjon av sanntidsdata i en modell for å korrigere prognoser og redusere avvik mellom tvilling og virkelighet. \citep{evensen2009data} \\
Model predictive control & MPC & Model predictive control & Optimaliserende styringsstrategi som bruker modell og prediksjoner for å beregne kontrollsignaler innenfor gitte begrensninger. \citep{rawlings2017model} \\
Kalibrering & -- & Calibration & Justering av modellparametre mot observasjoner for å sikre at tvilling og fysisk system samsvarer innen toleranser. \citep{dnv2021rp} \\
Modell- og simuleringsplan & -- & Model and simulation plan & Dokument som beskriver mål, metodikk, scenarier og sjekkpunkter for modell- og simuleringsarbeid i tvillingprosjekter. \citep{dnv2021rp} \\
Virtuell kommisjonering & -- & Virtual commissioning & Test av automasjonslogikk, sikkerhetsfunksjoner og prosessforløp i et simulert miljø før fysisk oppstart av anlegg. \citep{boschert2018digital} \\
Scenario-co-simulering & -- & Scenario co-simulation & Kobling av flere simuleringsmodeller for å evaluere tverrfaglige hendelser og beslutninger i samme tvilling. \citep{boschert2018digital} \\
Flerfidelitetskalibrering & -- & Multi-fidelity calibration & Systematisk bruk av modeller med ulik detaljeringsgrad for å kombinere raske analyser med høy presisjon der det trengs. \citep{kennedy2000predicting} \\
\bottomrule
\end{longtable}

\subsection{Data og infrastruktur}
\begin{longtable}{p{0.23\textwidth}p{0.16\textwidth}p{0.23\textwidth}p{0.34\textwidth}}
\toprule
\textbf{Begrep (norsk)} & \textbf{Forkortelse} & \textbf{Engelsk term} & \textbf{Forklaring} \\
\midrule
\endfirsthead
\toprule
\textbf{Begrep (norsk)} & \textbf{Forkortelse} & \textbf{Engelsk term} & \textbf{Forklaring} \\
\midrule
\endhead
Edge-prosessering & -- & Edge computing & Databehandling som skjer nær sensorkilder for å redusere latenstid og avhengighet av skytilkobling i operative tvillinger. \citep{dnv2021a204} \\
Sanntidsdatastrøm & -- & Real-time data stream & Kontinuerlig strøm av data som gjøres tilgjengelig umiddelbart for overvåking, analyse eller styring av tvillingtjenester. \citep{dnv2021a204} \\
Datasjø & -- & Data lake & Sentralisert lagringsarkitektur for rådata i ulike formater som senere struktureres for tvillingapplikasjoner. \citep{meldst22datasomressurs} \\
Sensor gateway & -- & Sensor gateway & Enhet eller programvare som aggregerer lokale sensordata og eksponerer dem mot plattformen via standardiserte protokoller. \citep{dnv2021a204} \\
OPC UA & OPC UA & OPC UA & Åpen kommunikasjonsstandard for industriell dataintegrasjon med semantikk og innebygde sikkerhetsmekanismer. \citep{dnv2021a204} \\
SCADA-system & SCADA & Supervisory Control and Data Acquisition & System for overvåking og kontroll av industrielle prosesser som ofte integreres med tvillingplattformen. \citep{dnv2021a204} \\
Dataspace-arkitektur & -- & Dataspace architecture & Referansearkitektur som beskriver tillitsrammer, tilkobling og semantikk for deling av data mellom organisasjoner. \citep{idsa2023ram} \\
Asset-administrasjonsskal & AAS & Asset Administration Shell & Digital representasjon av industriobjektets egenskaper, tjenester og grensesnitt i samsvar med Industrie~4.0-standarden. \citep{plattformi40aas2023} \\
Digital Twin Definition Language & DTDL & Digital Twins Definition Language & Modellbeskrivelse som definerer komponenter, telemetri og relasjoner i Azure Digital Twins og lignende plattformer. \citep{microsoft2023dtdl} \\
MLOps & MLOps & MLOps & Praksis for å drifte og vedlikeholde maskinlæringsmodeller med versjonering, overvåking og automatisert leveranse. \citep{kreuzberger2023mlops} \\
Syntetiske data & -- & Synthetic data & Kunstig genererte datasett som brukes til å trene eller teste modeller når reelle data er begrenset eller sensitive. \citep{mittal2023synthetic} \\
Dataspace-konnektor & -- & Dataspace connector & Programvarekomponent som håndhever policyer, logging og kryptert datautveksling mellom deltakere i et dataspace. \citep{digitalnorway2024dataspace} \\
Policy Enforcement Point & PEP & Policy Enforcement Point (PEP) & Kontrollpunkt som stopper eller tillater datastrømmer basert på sikkerhets- og delingsregler før data når tvillingplattformen. \citep{idsa2023ram} \\
Observabilitetsplattform & -- & Observability platform & Verktøykjede for innsamling av logger, spor og metrikk som gir innsikt i ytelse og hendelser i tvillingens data- og modellpipeline. \citep{kreuzberger2023mlops} \\
\bottomrule
\end{longtable}

\subsection{Sikkerhet og beredskap}
\begin{longtable}{p{0.23\textwidth}p{0.16\textwidth}p{0.23\textwidth}p{0.34\textwidth}}
\toprule
\textbf{Begrep (norsk)} & \textbf{Forkortelse} & \textbf{Engelsk term} & \textbf{Forklaring} \\
\midrule
\endfirsthead
\toprule
\textbf{Begrep (norsk)} & \textbf{Forkortelse} & \textbf{Engelsk term} & \textbf{Forklaring} \\
\midrule
\endhead
NIS2-etterlevelsesplan & -- & NIS2 compliance plan & Tiltaksliste som beskriver ansvar, tidsfrister og tekniske kontroller for å oppfylle NIS2-krav i tvillingprosjekter. \citep{eu2022nis2} \\
Sikkerhetssoner & -- & Security zones & Segmentering av nettverk og systemer i adskilte soner for å begrense spredning av hendelser og sikre kritisk infrastruktur. \citep{iec62443-2-1} \\
Beredskapsøvelse & -- & Preparedness exercise & Planlagt trening som tester responsevne, roller og kommunikasjon ved cyberhendelser mot tvillingplattformen. \citep{dsb2023ovelser} \\
\bottomrule
\end{longtable}

\section{Notater}
Tabellene er gruppert etter tema slik at begreper knyttet til styring, modellering og infrastruktur kan oppdateres uavhengig av hverandre. Utvid seksjonene når kapitlene introduserer nye tekniske eller organisatoriske uttrykk, og følg samme kolonneoppsett for forkortelser og referanser.
