% Mal for prosjektkontrakt mellom studentteam og samarbeidspartner
% Brukes sammen med rubrikkene i support/larerveiledning.tex

\section*{Prosjektkontrakt for digital tvilling-case}
Denne malen brukes for å forankre samarbeidet mellom studentteam, faglærer og ekstern partner.
Utfylt kontrakt skal beskrive mål, datasett, ansvar og vurderingsformer, og knytte leveranser til
rubrikkene for underveisarbeid, presentasjon og rapport. Malen kan publiseres i læringsplattformen
som utfyllbart skjema.

\subsection*{1. Parter og kontaktinformasjon}
\begin{tabular}{p{0.35\textwidth}p{0.6\textwidth}}
\toprule
\textbf{Rolle} & \textbf{Navn, organisasjon og kontaktpunkt} \\
\midrule
Studentteam & \\
Faglærer/mentor & \\
Samarbeidspartner & \\
Teknisk støtte (IT/data) & \\
Andre ressurser (f.eks. grafikk, simulering) & \\
\bottomrule
\end{tabular}

\subsection*{2. Casebeskrivelse og læringsmål}
\begin{itemize}
    \item Kort beskrivelse av caset og hvorfor det er relevant.
    \item Forventede læringsmål for teamet (knytt til kapitlene som dekkes, f.eks. Kapittel~4 om
    simulering og Kapittel~8 om sektorcase).
    \item Suksesskriterier og nøkkelindikatorer (tekniske, organisatoriske og bærekraftsrelaterte).
\end{itemize}

\subsection*{3. Leveranser, tidsplan og kobling til rubrikker}
\begin{tabular}{p{0.12\textwidth}p{0.32\textwidth}p{0.22\textwidth}p{0.28\textwidth}}
\toprule
\textbf{Uke/dato} & \textbf{Leveranse eller aktivitet} & \textbf{Relevant kapittel/case} & \textbf{Vurderingsgrunnlag (rubrikk)} \\
\midrule
 & Forprosjektcanvas og interessentkart & Kap.~1--2 & Underveisrubrikk (Tab.~\texttt{rubrikk-underveis}) \\
 & Datainnsamlingsark og tilgangsplan & Kap.~3 & Underveisrubrikk + Rapport \\
 & Simuleringsscenario/AR-\&-VR-storyboard & Kap.~4 og planlagt caseutvidelse & Underveisrubrikk + Presentasjon \\
 & Algoritmeskisse eller læringssløyfe & Kap.~5 & Underveisrubrikk \\
 & Valideringsplan og risikovurdering & Kap.~6--7 & Rapport \\
 & Sluttleveranse (rapport + presentasjon) & Kap.~8--9 & Presentasjon + Rapport \\
\bottomrule
\end{tabular}

\subsection*{4. Ressurser, data og avklaringer}
\begin{tabular}{p{0.32\textwidth}p{0.62\textwidth}}
\toprule
\textbf{Tema} & \textbf{Avtalt tilgang eller tiltak} \\
\midrule
Datasett og API-er & \\
Lisens- og personvernkrav & \\
Tilgang til simulatorer/AR- eller VR-lab & \\
Tilgang til edge-plattformer eller skyressurser & \\
Behov for ekstra maler eller arbeidsark (angi navn) & \\
\bottomrule
\end{tabular}

\subsection*{5. Roller, ansvar og eskalering}
\begin{tabular}{p{0.28\textwidth}p{0.3\textwidth}p{0.34\textwidth}}
\toprule
\textbf{Aktivitet} & \textbf{Ansvarlig(e)} & \textbf{Eskalering/kontakt ved avvik} \\
\midrule
Faglig veiledning og ukentlige statusmøter & & \\
Data- og teknisk støtte & & \\
Sikkerhet, etikk og juridiske avklaringer & & \\
Koordinering av illustrasjoner og visualisering & & \\
Pilot- eller brukertesting & & \\
\bottomrule
\end{tabular}

\subsection*{6. Kvalitetssikring og kobling til kommende caseutvidelser}
\begin{itemize}
    \item Marker hvilke kommende caseutvidelser som påvirker kontrakten (for eksempel
    \emph{immersivt beslutningsrom}, \emph{havvind-case}, \emph{landbrukscase}).
    \item Beskriv hvordan leveranser skal oppdateres når nye rubrikker eller arbeidsark innføres.
    \item Avtal hvordan teamet dokumenterer versjoner slik at fagfellepakker kan oppdateres uten tap av sporbarhet.
\end{itemize}

\subsection*{7. Godkjenning og signaturer}
\begin{tabular}{p{0.32\textwidth}p{0.62\textwidth}}
\toprule
\textbf{Part} & \textbf{Signatur og dato} \\
\midrule
Studentteam (representant) & \\
Faglærer & \\
Samarbeidspartner & \\
\bottomrule
\end{tabular}

\paragraph{Oppfølging}
Sett opp revisjonstidspunkter (for eksempel midtveis og ved avslutning) og koble dem til vurderingsplanen.
Dokumenter eventuelle endringer i kontrakten gjennom korte tillegg som refererer til denne malen.
