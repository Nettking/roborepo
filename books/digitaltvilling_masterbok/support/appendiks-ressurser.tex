\chapter{Datasett og verktøyressurser}
\label{appendix:ressurser}

Dette appendikset samler forslag til datasett og verktøy som kan brukes i prosjekt-
eller masteroppgaver knyttet til digitale tvillinger. Ressursene er valgt med tanke på norske forhold, åpen tilgang der det er mulig, og relevans for modellering, overvåking og analyse.

\section{Åpne datasett}
\begin{longtable}{p{0.28\textwidth}p{0.32\textwidth}p{0.30\textwidth}}
\toprule
\textbf{Datasett} & \textbf{Beskrivelse} & \textbf{Hvordan bruke i digital tvilling?} \\
\midrule
\endfirsthead
\toprule
\textbf{Datasett} & \textbf{Beskrivelse} & \textbf{Hvordan bruke i digital tvilling?} \\
\midrule
\endhead
Felles datakatalog (Digdir) & Samleportal for over 2000 offentlige norske datasett på tvers av sektorer. Metadata inkluderer format, API-endepunkter og lisens. & Brukes til å finne referansedata for scenariomodellering, for eksempel demografi, infrastruktur eller miljøvariabler. \\
\addlinespace
Statnett – Åpne nettdata & Tidsserier for kraftforbruk og produksjon, netttopologi og flaskehalser. Tilgjengelig via \href{https://www.statnett.no/vare-tjenester/elanett/}{Statnett elhub} og \href{https://transparency.entsoe.eu/}{ENTSO-E Transparency}. & Understøtter energitvillingers prognosemodeller, belastningsanalyse og sanntidsvisualisering av strømnettet. \\
\addlinespace
Meteorologisk institutt – Frost API & Historiske og sanntids værmålinger fra norske stasjoner, tilgjengelig via REST-API etter registrering. & Knyt værpåvirkning til modeller for bygg, transport og fornybar energi, og test robusthet mot klimavariasjoner. \\
\addlinespace
Statens vegvesen – Trafikkdata & Åpne trafikk- og sensordata (veglys, trafikkmengde, hendelser) via \href{https://developer.vegdata.no/}{Vegdata API}. & Modellér logistikk- og mobilitetsløsninger, optimaliser ruter og vedlikehold basert på sanntidssignaler. \\
\addlinespace
Norsk Petroleumsdirektorat – Oljedata & Brønndata, produksjonsprofiler og reservoarmodeller. Åpne datasett finnes via \href{https://factpages.npd.no/en/}{FactPages}. & Brukes til å bygge tvillinger av subsurface-systemer, planlegge boreprogram eller analysere produksjonsoptimalisering. \\
\bottomrule
\end{longtable}

\section{Verktøy og plattformer}
\begin{longtable}{p{0.28\textwidth}p{0.32\textwidth}p{0.30\textwidth}}
\toprule
\textbf{Verktøy} & \textbf{Type} & \textbf{Relevans for studenter} \\
\midrule
\endfirsthead
\toprule
\textbf{Verktøy} & \textbf{Type} & \textbf{Relevans for studenter} \\
\midrule
\endhead
Azure Digital Twins & Skyplattform for modellering av komplekse miljøer med graforienterte tvillingmodeller og live dataingest. Studentabonnement gir gratis kvoter. & Hurtig prototyping av bygnings- eller campus-tvilling, med integrasjon mot IoT Hub og Power BI for visualisering. \\
\addlinespace
OpenModelica & Åpen kildekode-MBSE-verktøy basert på Modelica-standarden. Støtter fysiske modeller og co-simulering. & Utvikle detaljerte prosess- og energimodeller som kan kobles til sanntidsdata gjennom FMI/FMUs. \\
\addlinespace
Apache Kafka + ksqlDB & Distribuert hendelsesstrøm-plattform for sanntidsdata. & Bygg data pipelines mellom sensorer og digitale tvillinger, implementer streaming-analytics og integrer med verktøy som Spark eller Flink. \\
\addlinespace
Siemens Industrial Edge Trial & Industrinær plattform med støtte for OPC UA, containerdeployering og analysetjenester. Gratis testlisens via partnerprogram. & Evaluér edge-arkitekturer, test maskinlæring nær produksjonsutstyr og sammenlign med skybaserte tvillinger. \\
\addlinespace
CesiumJS & Bibliotek for 3D-geovisualisering i nettleser. Støtter tidsdimensjon og streaming av sensordata. & Lag interaktive dashboards for digitale tvillinger av byrom, transport og offshore installasjoner. \\
\bottomrule
\end{longtable}

\section{Forslag til arbeidsmåte}
\begin{enumerate}
    \item Definér et konkret problem (for eksempel energistyring i en campusbygning) og velg hvilke datasett som best dekker behovet for historiske, sanntids- og kontekstdata.
    \item Etabler en pipeline med streamingverktøy (Kafka eller tilsvarende) og en modell- eller visualiseringsplattform (Azure Digital Twins, OpenModelica, CesiumJS).
    \item Dokumenter datakvalitet, lisenskrav og eventuelle personvernhensyn i prosjektets styringsdokumenter.
    \item Lag forslag til videre arbeid i prosjektet, for eksempel supplering med syntetiske data eller integrasjon mot læringsmoduler i Kapittel~5.
\end{enumerate}

\section{Siteringspraksis og bibliografi}
For å sikre en enhetlig referansestil er et felles Bib\TeX-bibliotek tilgjengelig i filen \texttt{support/referanser.bib}. Kapittelforfattere bør:
\begin{itemize}
    \item bruke \verb+\citet{}+ når kilden skal inngå i selve setningen (for eksempel ``\citet{grieves2017digital} introduserer konseptet").
    \item bruke \verb+\citep{}+ for parentesreferanser (for eksempel ``norske myndigheter vektlegger datadeling \citep{regjeringen2022datastrategi}'').
    \item legge til nye kilder i Bib\TeX-filen med konsistente nøkler og utfylt \texttt{doi} eller \texttt{url} når det er tilgjengelig.
    \item bestille fagfellevurdering av referanselisten når nye kapitler publiseres for å sikre kvalitet på kildene.
\end{itemize}

Bibliografien settes automatisk inn i bokens bakmaterie via \LaTeX-kommandoen \verb+\bibliography{support/referanser}+.

\section{Notater}
Denne ressursoversikten ble opprettet for å støtte oppgaveløsning og casestudier. Oppdater listen ved nye sektorkrav eller når verktøy får endret lisens. Seksjonen om siteringspraksis ble lagt til for å informere kapittelforfattere om bruk av det nye Bib\TeX-arkivet.
