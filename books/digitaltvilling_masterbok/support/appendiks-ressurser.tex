\chapter{Datasett og verktøyressurser}
\label{appendix:ressurser}

Dette appendikset samler forslag til datasett, verktøy og styringsmaler som kan brukes i prosjekt- eller masteroppgaver knyttet til digitale tvillinger. Ressursene er valgt med tanke på norske forhold, åpen tilgang der det er mulig, og relevans for modellering, overvåking og analyse. Ressursbanken er nå strukturert etter modenhetsnivå slik at studenter, veiledere og partnere raskt kan velge riktige kombinasjoner til casets ståsted.

\section{Modenhetsnivåer og scenarier}
\begin{description}
    \item[Utforske (idé og konsept)] Brukes når teamet kartlegger behov, tilgjengelige datasett og mulige arkitekturer. Fokus ligger på raske analyser og samskaping i verksted eller undervisning.
    \item[Pilotere (kontrollert utprøving)] Passer når en prototype skal kobles til utvalgte datakilder, og når prosesser for tilgang, sikkerhet og vurdering skal testes med begrenset omfang.
    \item[Skalere (operasjonell drift og gevinst)] Aktuelt når løsningen skal levere verdi i daglig drift, synkroniseres med styringsmodeller og rapportere effekter på tvers av organisasjoner.
\end{description}

\begin{table}[h]
    \centering
    \caption{Eksempelscenarier for bruk av ressursbanken}
    \label{tab:ressurs-scenarier}
    \begin{tabular}{p{0.26\textwidth}p{0.64\textwidth}}
        \toprule
        \textbf{Modenhetsnivå} & \textbf{Typiske leveranser og casekobling} \\
        \midrule
        Utforske & Case «Klimatilpasset kommune»: kombiner Felles datakatalog, Frost API og CesiumJS for å identifisere hvilke data som trengs til scenariomodeller i kapittel~3 og kapittel~8. \\
        Pilotere & Case «Smart terminaldrift» og case «Virtuell legevakt»: koble Entur-, Bane~NOR-, Avinor- og FHI-data til Kafka, Azure Digital Twins og Siemens Industrial Edge for å teste nye arbeidsprosesser i transport og helse. \\
        Skalere & Case «Regional energikoordinering» og case «Autonom havn»: bruk Statnett-, Nord~Pool-, BarentsWatch- og RACI-S-ressurser for å sikre gevinst og styring i tråd med kapittel~6 og kapittel~7. \\
        \bottomrule
    \end{tabular}
\end{table}

\section{Casebeskrivelser}
\subsection{Case «Klimatilpasset kommune» (Utforske)}
Dette scenarioet tar utgangspunkt i en kommune som skal vurdere klimatilpasning for bygg og kritisk infrastruktur. Studenten bruker åpne datasett fra Digdir, Meteorologisk institutt og Statens vegvesen til å lage raske analyser, visualisere sårbarhet i CesiumJS og koble behovene til læringsmålene i kapittel~3.

\subsection{Case «Smart terminaldrift» (Pilotere)}
Her testes nye arbeidsprosesser for kollektivknutepunkt og lufthavn i samspill med logistikkaktører. Datakilder fra Entur, Bane~NOR, Avinor og BarentsWatch kobles via Kafka og Azure Digital Twins for å lage en felles situasjonsforståelse, mens Siemens Industrial Edge validerer at algoritmer kan kjøres nær operasjonen før skalering.

\subsection{Case «Regional energikoordinering» (Skalere)}
Caset retter seg mot energiselskap, fylkeskommune og industripartnere som skal samkjøre produksjon, fleksibilitet og beredskap. Statnett, Nord~Pool, NVE og Enova-prosjektbanken gir datagrunnlaget, mens gevinstplan og RACI-S-mal sikrer at tiltak følges opp etter styringsmodellene i kapittel~6 og kapittel~7.

\subsection{Case «Virtuell legevakt» (Pilotere)}
Scenarioet omfatter et samarbeid mellom kommunehelsetjenesten og sykehus som skal teste en digital legevakt. Helsedirektoratets nøkkeltall avdekker kapasitet og behov, mens data fra Sykdomspulsen og Felleskatalogen brukes til sanntidsbeslutninger i pilotlaboratoriet. Norsk Helsenetts norm gir rammene for informasjonssikkerhet før løsningen skaleres.

\subsection{Case «Autonom havn» (Skalere)}
Her samarbeider havn, logistikkaktører og myndigheter om å automatisere trafikkavvikling og beredskap. Kartverket og Mareano-data gir grunnlag for modellering, HUB~Ocean og BarentsWatch gir situasjonsdata i pilotfasen, mens Sjøfartsdirektoratets register og styringsmalene i dette appendikset sikrer etterlevelse ved overgang til ordinær drift.

\section{Åpne datasett}
\begin{longtable}{p{0.22\textwidth}p{0.30\textwidth}p{0.30\textwidth}p{0.14\textwidth}}
\toprule
\textbf{Datasett} & \textbf{Beskrivelse} & \textbf{Bruks-case} & \textbf{Modenhet} \\
\midrule
\endfirsthead
\toprule
\textbf{Datasett} & \textbf{Beskrivelse} & \textbf{Bruks-case} & \textbf{Modenhet} \\
\midrule
\endhead
Felles datakatalog (Digdir) & Samleportal for over 2000 offentlige norske datasett på tvers av sektorer. Metadata inkluderer format, API-endepunkter og lisens. & Case «Klimatilpasset kommune»: kartlegg referansedata for scenarioverksted og prioriter innsikt til kapittel~3 og kapittel~7. & Utforske \\
\addlinespace
Statnett – Åpne nettdata & Tidsserier for kraftforbruk og produksjon, netttopologi og flaskehalser. Tilgjengelig via \href{https://www.statnett.no/vare-tjenester/elanett/}{Statnett elhub} og \href{https://transparency.entsoe.eu/}{ENTSO-E Transparency}. & Case «Regional energikoordinering»: etabler sanntidsdashbord for kapasitet, fleksibilitet og beredskap i regionale energitvillinger. & Skalere \\
\addlinespace
Meteorologisk institutt – Frost API & Historiske og sanntids værmålinger fra norske stasjoner, tilgjengelig via REST-API etter registrering. & Case «Klimatilpasset kommune»: koble værprofiler til modellantakelser for bygg og mobilitet før pilotering. & Utforske \\
\addlinespace
Statens vegvesen – Trafikkdata & Åpne trafikk- og sensordata (veglys, trafikkmengde, hendelser) via \href{https://developer.vegdata.no/}{Vegdata API}. & Case «Klimatilpasset kommune»: vurdér beredskap for kritiske transportårer og prioriter datainnsamling i pilotfasen. & Utforske \\
\addlinespace
Norsk Petroleumsdirektorat – Oljedata & Brønndata, produksjonsprofiler og reservoarmodeller. Åpne datasett finnes via \href{https://factpages.npd.no/en/}{FactPages}. & Case «Regional energikoordinering»: sammenlign produksjonsdata og boringstiltak mot energiforsyning og beredskap i skalering. & Skalere \\
\bottomrule
\end{longtable}

\section{Verktøy og plattformer}
\begin{longtable}{p{0.22\textwidth}p{0.26\textwidth}p{0.30\textwidth}p{0.14\textwidth}}
\toprule
\textbf{Verktøy} & \textbf{Type} & \textbf{Bruks-case} & \textbf{Modenhet} \\
\midrule
\endfirsthead
\toprule
\textbf{Verktøy} & \textbf{Type} & \textbf{Bruks-case} & \textbf{Modenhet} \\
\midrule
\endhead
Azure Digital Twins & Skyplattform for modellering av komplekse miljøer med graforienterte tvillingmodeller og live dataingest. Studentabonnement gir gratis kvoter. & Case «Smart terminaldrift»: bygg en felles datastruktur for tog, buss og flyoperasjoner og visualiser KPI-er fra kapittel~4. & Pilotere \\
\addlinespace
OpenModelica & Åpen kildekode-MBSE-verktøy basert på Modelica-standarden. Støtter fysiske modeller og co-simulering. & Case «Klimatilpasset kommune»: etabler raske prototyper av energibruk i bygg før du bestemmer hvilke sensorer som må kobles på. & Utforske \\
\addlinespace
Apache Kafka + ksqlDB & Distribuert hendelsesstrøm-plattform for sanntidsdata. & Case «Smart terminaldrift»: test datalinjer fra Entur, Avinor og BarentsWatch og definér hendelsesregler for pilotering. & Pilotere \\
\addlinespace
Siemens Industrial Edge Trial & Industrinær plattform med støtte for OPC UA, containerdeployering og analysetjenester. Gratis testlisens via partnerprogram. & Case «Smart terminaldrift»: valider at maskinlæring kan kjøres på kanten før løsningen skalering og integreres med kapittel~5. & Pilotere \\
\addlinespace
CesiumJS & Bibliotek for 3D-geovisualisering i nettleser. Støtter tidsdimensjon og streaming av sensordata. & Case «Klimatilpasset kommune»: visualiser klima- og trafikkdata i et delt dashboard for å prioritere tiltak i tidligfase. & Utforske \\
\bottomrule
\end{longtable}

\section{Helsesektorspesifikke ressurser}
\begin{longtable}{p{0.24\textwidth}p{0.28\textwidth}p{0.32\textwidth}p{0.12\textwidth}}
\toprule
\textbf{Ressurs} & \textbf{Beskrivelse} & \textbf{Bruks-case} & \textbf{Modenhet} \\
\midrule
\endfirsthead
\toprule
\textbf{Ressurs} & \textbf{Beskrivelse} & \textbf{Bruks-case} & \textbf{Modenhet} \\
\midrule
\endhead
Helsedirektoratet – Åpne helsedata & Portal med statistikk om kommunale helse- og omsorgstjenester, kvalitetsindikatorer og folkehelsetall. Tilgjengelig som nedlastbare tabeller og API. & Case «Virtuell legevakt»: kartlegg volum, responstid og forebyggingsbehov før du designer datastrømmer til pilotmiljøet. & Utforske \\
\addlinespace
Folkehelseinstituttet – Sykdomspulsen & Åpne, anonymiserte tidsserier for luftveisinfeksjoner, sykehusinnleggelser og andre beredskapsindikatorer. Leveres via \href{https://www.fhi.no/hn/overvaking/sykdomspulsen/}{FHI API}. & Case «Virtuell legevakt»: overvåk belastning og triggere for beredskap når du tester beslutningsstøtte med klinikere. & Pilotere \\
\addlinespace
Felleskatalogen – Legemiddel-API & FHIR-kompatibelt API med virkestoff, dosering og interaksjonsdata for legemidler tilgjengelig i Norge. Krever gratis registrering. & Case «Virtuell legevakt»: gi klinikere kontekstuell legemiddelinformasjon og sjekk interaksjoner før skalering. & Pilotere \\
\addlinespace
Norsk Helsenett – Norm for informasjonssikkerhet & Retningslinjer, maler og risikovurderingsverktøy for behandling av helseopplysninger i norske virksomheter. Tilgjengelig via \href{https://normen.no}{normen.no}. & Case «Virtuell legevakt»: dokumenter styring, tilgang og loggføring slik at løsningen kan godkjennes for ordinær drift. & Skalere \\
\bottomrule
\end{longtable}

\section{Transport- og logistikkressurser}
\label{sec:transportressurser}
\begin{longtable}{p{0.24\textwidth}p{0.28\textwidth}p{0.32\textwidth}p{0.12\textwidth}}
\toprule
\textbf{Ressurs} & \textbf{Beskrivelse} & \textbf{Bruks-case} & \textbf{Modenhet} \\
\midrule
\endfirsthead
\toprule
\textbf{Ressurs} & \textbf{Beskrivelse} & \textbf{Bruks-case} & \textbf{Modenhet} \\
\midrule
\endhead
Entur – Mobilitetsdata (GTFS/NeTEx) & Nasjonal plattform for kollektivtrafikkdata med ruteplaner, stoppesteder og sanntidsavvik tilgjengelig via åpne API-er. & Case «Smart terminaldrift»: dimensjonér kapasitet og test køhåndtering før løsningen tas inn i pilot. & Pilotere \\
\addlinespace
Bane NOR – Banedata og arbeidsplan-API & Datasett over infrastruktur, kapasitet, vedlikeholdsplaner og operative meldinger for jernbanenettet. & Case «Smart terminaldrift»: synkroniser togplaner med gate-bruk og bemanningsskift i pilotlaboratoriet. & Pilotere \\
\addlinespace
Avinor – Operasjonelle flyplassdata & Åpne data om ankomster, avganger, stand-tildeling og passasjerstatistikk for norske lufthavner. & Case «Smart terminaldrift»: verifiser hvordan flyplassoperasjoner påvirker multimodale koblinger og KPI-er. & Pilotere \\
\addlinespace
Kystverket/BarentsWatch – AIS og havneoversikt & Sanntidsposisjoner, farleder, havneinformasjon og sikkerhetsmeldinger for norskekysten tilgjengelig via API og nedlasting. & Case «Autonom havn»: overvåk trafikk og beredskap når beslutningsstøtten skal videre til operativ drift. & Skalere \\
\addlinespace
Statens vegvesen – Nasjonal vegdatabank (NVDB) & Objektdata om vegnett, fartsgrenser, tunneler og tilstandsanalyser, eksponert som REST-API og nedlastbare datasett. & Case «Klimatilpasset kommune»: identifiser kritiske knutepunkt som trenger sensorer og vedlikehold i tidligfasen. & Utforske \\
\bottomrule
\end{longtable}

\section{Maritime og offshore-ressurser}
\label{sec:maritimressurser}
\begin{longtable}{p{0.24\textwidth}p{0.28\textwidth}p{0.32\textwidth}p{0.12\textwidth}}
\toprule
\textbf{Ressurs} & \textbf{Beskrivelse} & \textbf{Bruks-case} & \textbf{Modenhet} \\
\midrule
\endfirsthead
\toprule
\textbf{Ressurs} & \textbf{Beskrivelse} & \textbf{Bruks-case} & \textbf{Modenhet} \\
\midrule
\endhead
BarentsWatch – Havbase og AIS-tjenester & \href{https://www.barentswatch.no/}{BarentsWatch} tilbyr sanntidsdata for fartøysposisjoner, fiskeriaktivitet, miljøparametere og beredskapsressurser. & Case «Autonom havn»: bygg situasjonsforståelse for trafikkavvik og nødhendelser før drift overtas av digitale tjenester. & Skalere \\
\addlinespace
Sjøfartsdirektoratet – NOR/NIS skipsregister & Åpne data over alle registrerte fartøy, tekniske spesifikasjoner og sertifikatstatus via \href{https://www.sdir.no/digitalt/apne-data/}{Sjøfartsdirektoratet}. & Case «Autonom havn»: oppdater flåteoversikt, risikoklasser og sertifikater som grunnlag for styring og rapportering. & Skalere \\
\addlinespace
Kartverket – Dybdedata og navigasjonsgrunnlag & \href{https://www.geonorge.no/}{Geonorge} gjør tilgjengelig sjøkart, dybdemodeller og farledsinformasjon i standard GIS-formater. & Case «Autonom havn»: simuler ruter, anløp og sikkerhetssoner før automatiserte operasjoner slippes løs. & Pilotere \\
\addlinespace
Havforskningsinstituttet – Mareano og økosystemdata & \href{https://www.hi.no/hi/forskning/mareano}{Mareano-programmet} publiserer kartlagte bunnforhold, biologi og miljøparametere for norsk sokkel. & Case «Autonom havn»: vurder miljøkonsekvenser og lokasjonsvalg i idé- og planfase sammen med kapittel~8. & Utforske \\
\addlinespace
HUB Ocean – Ocean Data Platform & \href{https://portal.hubocean.no/}{Ocean Data Platform} samler delte datasett fra industrien og forskningsmiljøer med API-tilgang etter registrering. & Case «Autonom havn»: koble industridata og forskningsfunn for å teste algoritmer og samarbeidsmodeller i pilot. & Pilotere \\
\bottomrule
\end{longtable}

\section{Energi- og kraftressurser}
\begin{longtable}{p{0.24\textwidth}p{0.28\textwidth}p{0.32\textwidth}p{0.12\textwidth}}
\toprule
\textbf{Ressurs} & \textbf{Beskrivelse} & \textbf{Bruks-case} & \textbf{Modenhet} \\
\midrule
\endfirsthead
\toprule
\textbf{Ressurs} & \textbf{Beskrivelse} & \textbf{Bruks-case} & \textbf{Modenhet} \\
\midrule
\endhead
NVE – Energidataportalen & Åpen portal og API for timeserier om produksjon, magasinfylling, nettdrift og strømforbruk via \href{https://nedlasting.nve.no/}{nedlasting.nve.no}. & Case «Regional energikoordinering»: bygg modeller for last og produksjon som grunnlag for scenarioverksted. & Utforske \\
\addlinespace
Nord Pool – Markedsdata API & Gratis tilgang til spotpriser, områdepriser og regulerkraft via \href{https://www.nordpoolgroup.com/en/Market-data1/}{Nord Pool Market Data}. & Case «Regional energikoordinering»: test optimaliseringsstrategier og avtaleverk før løsningen rulles bredt ut. & Skalere \\
\addlinespace
Enova – Prosjektbank & Prosjektregister med beskrivelser av gjennomførte energieffektiviserings- og innovasjonsprosjekter tilgjengelig via \href{https://www.enova.no/bedrift/prosjekter/prosjektbank/}{enova.no}. & Case «Regional energikoordinering»: velg tiltakspakker og indikatorer for pilotprosjekter i kommunenettverk. & Pilotere \\
\addlinespace
Statistisk sentralbyrå – Energiregnskap & Offisielle tall for energiforbruk, produksjon og utslipp per sektor publisert på \href{https://www.ssb.no/energi-og-industri/energi/statistikk/energiregnskap}{ssb.no}. & Case «Regional energikoordinering»: mål gevinst, rapporter klimaeffekter og koble styring til kapittel~7. & Skalere \\
\bottomrule
\end{longtable}

\section{Personvern og bruk av sensitive data}
\begin{enumerate}
    \item Kartlegg behandlingsgrunnlag i henhold til GDPR artikkel~6 og 9 før datainnsamling. For forskningsprosjekter bør det innhentes godkjenning via \emph{Helsedataservice} og relevant etisk komité.
    \item Bruk pseudonymisering eller syntetiske datasett (for eksempel \emph{Synthea} eller data generert med verktøy fra Norsk Helsenett) i tidlige prototyper, og etabler klare rutiner for når ekte pasientdata kan benyttes.
    \item Dokumenter dataflyt, tilgangskontroller og logging i prosjektets styringsdokumenter, og knytt tiltakene til kapittel~6 om validering og tillit.
    \item Sørg for at modell- og beslutningsstøtte evalueres for bias og klinisk gyldighet før utrulling. Logg versjoner slik at fagfeller kan etterprøve resultater.
\end{enumerate}

\section{Forslag til arbeidsmåte}
\begin{enumerate}
    \item \textbf{Utforske:} Definér problemet (for eksempel klimatilpasning i kommunen eller energistyring i regionen) og bruk ressursene merket «Utforske» til å kartlegge datatilgang, indikatorer og interessenter.
    \item \textbf{Pilotere:} Sett opp datastrømmer og prototyper med verktøy og datasett merket «Pilotere», gjennomfør laboratorieøkter og dokumenter læring i tråd med kapittel~4 og kapittel~5.
    \item \textbf{Skalere:} Implementer styringsmaler, sikkerhetskrav og rapportering med ressursene merket «Skalere», og knytt gevinstoppfølging til kapittel~6 og kapittel~7.
    \item \textbf{Iterere:} Evaluer resultater og beslutningsgrunnlag mot casene i dette appendikset, og planlegg neste forbedringsrunde eller nye scenarier i samarbeid med partnere.
\end{enumerate}

\section{Styringsmaler og beslutningsverktøy}
Beslutningsverktøyene under kan brukes sammen med Kapittel~7 og Kapittel~8 for å strukturere ansvar, gevinster og datasamarbeid i casene.

\subsection{RACI-S-mal for dataspace-program}
\begin{table}[h]
    \centering
    \caption{Mal for RACI-S-matrise}
    \label{tab:appendix-raci}
    \begin{tabular}{p{4.2cm}ccccc}
        \toprule
        Aktivitet & Rolle 1 & Rolle 2 & Rolle 3 & Rolle 4 & Rolle 5 \\
        \midrule
        Definere mål og scope & & & & & \\
        Etablere dataprodukter og metadata & & & & & \\
        Implementere sikkerhet og tilgang & & & & & \\
        Overvåke datakvalitet og hendelser & & & & & \\
        Rapportere gevinster og tiltak & & & & & \\
        Revidere kontrakter og policy & & & & & \\
        \bottomrule
    \end{tabular}
\end{table}

Bruk kolonnetitler som beskriver faktiske roller (for eksempel «Operatør», «Plattformteam», «TSO», «Myndighet», «Leverandør») og fyll inn R, A, C, I eller S for hver aktivitet.

\subsection{Gevinstplan-mal}
\begin{table}[h]
    \centering
    \caption{Mal for gevinstplan med KPI-er}
    \label{tab:appendix-gevinstplan}
    \begin{tabular}{p{2.8cm}p{3.8cm}p{3.5cm}p{3.2cm}}
        \toprule
        Livssyklusfase & KPI og mål & Datakilde & Oppfølging \\
        \midrule
        Initiativ &  &  &  \\
        Design og implementering &  &  &  \\
        Drift og operasjon &  &  &  \\
        Kontinuerlig forbedring &  &  &  \\
        Avvikling og kunnskapsdeling &  &  &  \\
        \bottomrule
    \end{tabular}
\end{table}

Suppler tabellen med ansvarlige personer og koble den til RACI-S-matrisen slik at indikatorene følges opp på riktig beslutningsarena.

\section{Siteringspraksis og bibliografi}
For å sikre en enhetlig referansestil er et felles Bib\TeX-bibliotek tilgjengelig i filen \texttt{support/referanser.bib}. Kapittelforfattere bør:
\begin{itemize}
    \item bruke \verb+\citet{}+ når kilden skal inngå i selve setningen (for eksempel ``\citet{grieves2017digital} introduserer konseptet").
    \item bruke \verb+\citep{}+ for parentesreferanser (for eksempel ``norske myndigheter vektlegger datadeling \citep{regjeringen2022datastrategi}'').
    \item legge til nye kilder i Bib\TeX-filen med konsistente nøkler og utfylt \texttt{doi} eller \texttt{url} når det er tilgjengelig.
    \item bestille fagfellevurdering av referanselisten når nye kapitler publiseres for å sikre kvalitet på kildene.
\end{itemize}

Bibliografien settes automatisk inn i bokens bakmaterie via \LaTeX-kommandoen \verb+\bibliography{support/referanser}+.

\section{Notater}
Denne ressursoversikten ble opprettet for å støtte oppgaveløsning og casestudier. Oppdater listen ved nye sektorkrav eller når verktøy får endret lisens. Seksjonen om siteringspraksis ble lagt til for å informere kapittelforfattere om bruk av det nye Bib\TeX-arkivet.

