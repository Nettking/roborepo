\chapter{Datasett og verktøyressurser}
\label{appendix:ressurser}

Dette appendikset samler forslag til datasett og verktøy som kan brukes i prosjekt-
eller masteroppgaver knyttet til digitale tvillinger. Ressursene er valgt med tanke på norske forhold, åpen tilgang der det er mulig, og relevans for modellering, overvåking og analyse.

\section{Åpne datasett}
\begin{longtable}{p{0.28\textwidth}p{0.32\textwidth}p{0.30\textwidth}}
\toprule
\textbf{Datasett} & \textbf{Beskrivelse} & \textbf{Hvordan bruke i digital tvilling?} \\
\midrule
\endfirsthead
\toprule
\textbf{Datasett} & \textbf{Beskrivelse} & \textbf{Hvordan bruke i digital tvilling?} \\
\midrule
\endhead
Felles datakatalog (Digdir) & Samleportal for over 2000 offentlige norske datasett på tvers av sektorer. Metadata inkluderer format, API-endepunkter og lisens. & Brukes til å finne referansedata for scenariomodellering, for eksempel demografi, infrastruktur eller miljøvariabler. \\
\addlinespace
Statnett – Åpne nettdata & Tidsserier for kraftforbruk og produksjon, netttopologi og flaskehalser. Tilgjengelig via \href{https://www.statnett.no/vare-tjenester/elanett/}{Statnett elhub} og \href{https://transparency.entsoe.eu/}{ENTSO-E Transparency}. & Understøtter energitvillingers prognosemodeller, belastningsanalyse og sanntidsvisualisering av strømnettet. \\
\addlinespace
Meteorologisk institutt – Frost API & Historiske og sanntids værmålinger fra norske stasjoner, tilgjengelig via REST-API etter registrering. & Knyt værpåvirkning til modeller for bygg, transport og fornybar energi, og test robusthet mot klimavariasjoner. \\
\addlinespace
Statens vegvesen – Trafikkdata & Åpne trafikk- og sensordata (veglys, trafikkmengde, hendelser) via \href{https://developer.vegdata.no/}{Vegdata API}. & Modellér logistikk- og mobilitetsløsninger, optimaliser ruter og vedlikehold basert på sanntidssignaler. \\
\addlinespace
Norsk Petroleumsdirektorat – Oljedata & Brønndata, produksjonsprofiler og reservoarmodeller. Åpne datasett finnes via \href{https://factpages.npd.no/en/}{FactPages}. & Brukes til å bygge tvillinger av subsurface-systemer, planlegge boreprogram eller analysere produksjonsoptimalisering. \\
\bottomrule
\end{longtable}

\section{Verktøy og plattformer}
\begin{longtable}{p{0.28\textwidth}p{0.32\textwidth}p{0.30\textwidth}}
\toprule
\textbf{Verktøy} & \textbf{Type} & \textbf{Relevans for studenter} \\
\midrule
\endfirsthead
\toprule
\textbf{Verktøy} & \textbf{Type} & \textbf{Relevans for studenter} \\
\midrule
\endhead
Azure Digital Twins & Skyplattform for modellering av komplekse miljøer med graforienterte tvillingmodeller og live dataingest. Studentabonnement gir gratis kvoter. & Hurtig prototyping av bygnings- eller campus-tvilling, med integrasjon mot IoT Hub og Power BI for visualisering. \\
\addlinespace
OpenModelica & Åpen kildekode-MBSE-verktøy basert på Modelica-standarden. Støtter fysiske modeller og co-simulering. & Utvikle detaljerte prosess- og energimodeller som kan kobles til sanntidsdata gjennom FMI/FMUs. \\
\addlinespace
Apache Kafka + ksqlDB & Distribuert hendelsesstrøm-plattform for sanntidsdata. & Bygg data pipelines mellom sensorer og digitale tvillinger, implementer streaming-analytics og integrer med verktøy som Spark eller Flink. \\
\addlinespace
Siemens Industrial Edge Trial & Industrinær plattform med støtte for OPC UA, containerdeployering og analysetjenester. Gratis testlisens via partnerprogram. & Evaluér edge-arkitekturer, test maskinlæring nær produksjonsutstyr og sammenlign med skybaserte tvillinger. \\
\addlinespace
CesiumJS & Bibliotek for 3D-geovisualisering i nettleser. Støtter tidsdimensjon og streaming av sensordata. & Lag interaktive dashboards for digitale tvillinger av byrom, transport og offshore installasjoner. \\
\bottomrule
\end{longtable}

\section{Helsesektorspesifikke ressurser}
\begin{longtable}{p{0.28\textwidth}p{0.32\textwidth}p{0.30\textwidth}}
\toprule
\textbf{Ressurs} & \textbf{Beskrivelse} & \textbf{Hvordan støtte digitale tvillinger i helse?} \\
\midrule
\endfirsthead
\toprule
\textbf{Ressurs} & \textbf{Beskrivelse} & \textbf{Hvordan støtte digitale tvillinger i helse?} \\
\midrule
\endhead
Helsedirektoratet – Åpne helsedata & Portal med statistikk om kommunale helse- og omsorgstjenester, kvalitetsindikatorer og folkehelsetall. Tilgjengelig som nedlastbare tabeller og API. & Bruk aggregerte nøkkeltall som grunnlag for scenarioanalyse av kapasitet, ressursbehov og effekt av forebyggende tiltak i helsetvillinger. \\
\addlinespace
Folkehelseinstituttet – Sykdomspulsen & Åpne, anonymiserte tidsserier for luftveisinfeksjoner, sykehusinnleggelser og andre beredskapsindikatorer. Leveres via \href{https://www.fhi.no/hn/overvaking/sykdomspulsen/}{FHI API}. & Understøtt sanntidsoppdaterte overvåkingsmodeller, tidlig varsling og simulering av smittespredning koblet til lokale kapasitetsberegninger. \\
\addlinespace
Felleskatalogen – Legemiddel-API & FHIR-kompatibelt API med virkestoff, dosering og interaksjonsdata for legemidler tilgjengelig i Norge. Krever gratis registrering. & Knytt legemiddelopplysninger til beslutningsstøtte i digitale tvillinger for kliniske arbeidsflyter og pasientsimuleringer. \\
\addlinespace
Norsk Helsenett – Norm for informasjonssikkerhet & Retningslinjer, maler og risikovurderingsverktøy for behandling av helseopplysninger i norske virksomheter. Tilgjengelig via \href{https://normen.no}{normen.no}. & Sikre etterlevelse i datainnsamling, modelltrening og deling av innsikt ved å bygge styringspakker og kontrollpunkter inn i helsetvillinger. \\
\bottomrule
\end{longtable}

\section{Transport- og logistikkressurser}
\label{sec:transportressurser}
\begin{longtable}{p{0.28\textwidth}p{0.32\textwidth}p{0.30\textwidth}}
\toprule
\textbf{Ressurs} & \textbf{Beskrivelse} & \textbf{Hvordan støtte digitale tvillinger i transport?} \\
\midrule
\endfirsthead
\toprule
\textbf{Ressurs} & \textbf{Beskrivelse} & \textbf{Hvordan støtte digitale tvillinger i transport?} \\
\midrule
\endhead
Entur – Mobilitetsdata (GTFS/NeTEx) & Nasjonal plattform for kollektivtrafikkdata med ruteplaner, stoppesteder og sanntidsavvik tilgjengelig via åpne API-er. & Modellér passasjerflyt, koordinér multimodale reiser og test scenarioer for kapasitetsstyring i by- og regiontvillinger. \\
\addlinespace
Bane NOR – Banedata og arbeidsplan-API & Datasett over infrastruktur, kapasitet, vedlikeholdsplaner og operative meldinger for jernbanenettet. & Synkroniser vedlikeholdsplanlegging med simulerte togbevegelser og vurder effekten av avvik på punktlighet og ressursbruk. \\
\addlinespace
Avinor – Operasjonelle flyplassdata & Åpne data om ankomster, avganger, stand-tildeling og passasjerstatistikk for norske lufthavner. & Bygg digitale tvillinger av terminallogistikk, analysér bakketjenester og optimaliser ressursallokering gjennom dagen. \\
\addlinespace
Kystverket/BarentsWatch – AIS og havneoversikt & Sanntidsposisjoner, farleder, havneinformasjon og sikkerhetsmeldinger for norskekysten tilgjengelig via API og nedlasting. & Simuler seilingsmønstre, planlegg kaiutnyttelse og vurder beredskapstiltak i maritime tvillinger. \\
\addlinespace
Statens vegvesen – Nasjonal vegdatabank (NVDB) & Objektdata om vegnett, fartsgrenser, tunneler og tilstandsanalyser, eksponert som REST-API og nedlastbare datasett. & Kombiner statiske vegdata med trafikkstrømmer for å analysere vedlikeholdsetterslep, kapasitetsutvidelser og sikkerhetstiltak. \\
\bottomrule
\end{longtable}

\section{Maritime og offshore-ressurser}
\label{sec:maritimressurser}
\begin{longtable}{p{0.28\textwidth}p{0.32\textwidth}p{0.30\textwidth}}
\toprule
\textbf{Ressurs} & \textbf{Beskrivelse} & \textbf{Hvordan støtte digitale tvillinger i maritime miljøer?} \\
\midrule
\endfirsthead
\toprule
\textbf{Ressurs} & \textbf{Beskrivelse} & \textbf{Hvordan støtte digitale tvillinger i maritime miljøer?} \\
\midrule
\endhead
BarentsWatch – Havbase og AIS-tjenester & \href{https://www.barentswatch.no/}{BarentsWatch} tilbyr sanntidsdata for fartøysposisjoner, fiskeriaktivitet, miljøparametere og beredskapsressurser. & Understøtter situasjonsforståelse, logistikkplanlegging og risikoanalyse for offshore-operasjoner og kysttrafikk. \\
\addlinespace
Sjøfartsdirektoratet – NOR/NIS skipsregister & Åpne data over alle registrerte fartøy, tekniske spesifikasjoner og sertifikatstatus via \href{https://www.sdir.no/digitalt/apne-data/}{Sjøfartsdirektoratet}. & Bruk dataene til å modellere flåtesammensetning, vedlikeholdsplaner og regulatoriske krav i maritime tvillinger. \\
\addlinespace
Kartverket – Dybdedata og navigasjonsgrunnlag & \href{https://www.geonorge.no/}{Geonorge} gjør tilgjengelig sjøkart, dybdemodeller og farledsinformasjon i standard GIS-formater. & Gir nøyaktig geometri for simuleringsmodeller, ruteoptimalisering og vurdering av operasjonsgrenser i havner og fjorder. \\
\addlinespace
Havforskningsinstituttet – Mareano og økosystemdata & \href{https://www.hi.no/hi/forskning/mareano}{Mareano-programmet} publiserer kartlagte bunnforhold, biologi og miljøparametere for norsk sokkel. & Integrer miljø- og økosystemvariabler i digitale tvillinger som analyserer bærekraft, påvirkning og arealplanlegging. \\
\addlinespace
HUB Ocean – Ocean Data Platform & \href{https://portal.hubocean.no/}{Ocean Data Platform} samler delte datasett fra industrien og forskningsmiljøer med API-tilgang etter registrering. & Koble operative data med forskningsresultater for å teste nye algoritmer, samarbeidsmodeller og tjenesteinnovasjon. \\
\bottomrule
\end{longtable}

\section{Energi- og kraftressurser}
\begin{longtable}{p{0.28\textwidth}p{0.32\textwidth}p{0.30\textwidth}}
\toprule
\textbf{Ressurs} & \textbf{Beskrivelse} & \textbf{Hvordan støtte digitale tvillinger i energi?} \\
\midrule
\endfirsthead
\toprule
\textbf{Ressurs} & \textbf{Beskrivelse} & \textbf{Hvordan støtte digitale tvillinger i energi?} \\
\midrule
\endhead
NVE – Energidataportalen & Åpen portal og API for timeserier om produksjon, magasinfylling, nettdrift og strømforbruk via \href{https://nedlasting.nve.no/}{nedlasting.nve.no}. & Kalibrer last- og produksjonsmodeller, og analyser flaskehalser eller fleksibilitet i regionale energitvillinger. \\
\addlinespace
Nord Pool – Markedsdata API & Gratis tilgang til spotpriser, områdepriser og regulerkraft via \href{https://www.nordpoolgroup.com/en/Market-data1/}{Nord Pool Market Data}. & Legg inn pris- og ubalanseprofiler for å teste optimaliseringsstrategier og nye forretningsmodeller i energimarkeds-tvillinger. \\
\addlinespace
Enova – Prosjektbank & Prosjektregister med beskrivelser av gjennomførte energieffektiviserings- og innovasjonsprosjekter tilgjengelig via \href{https://www.enova.no/bedrift/prosjekter/prosjektbank/}{enova.no}. & Bruk casene som referanse for tiltakspakker og parameterisering av scenarier i bygg- og industrirelaterte tvillinger. \\
\addlinespace
Statistisk sentralbyrå – Energiregnskap & Offisielle tall for energiforbruk, produksjon og utslipp per sektor publisert på \href{https://www.ssb.no/energi-og-industri/energi/statistikk/energiregnskap}{ssb.no}. & Etabler baseline for gevinstmåling, scenarioanalyse og rapportering i energi- og klimarelaterte digitale tvillinger. \\
\bottomrule
\end{longtable}

\section{Personvern og bruk av sensitive data}
\begin{enumerate}
    \item Kartlegg behandlingsgrunnlag i henhold til GDPR artikkel~6 og 9 før datainnsamling. For forskningsprosjekter bør det innhentes godkjenning via \emph{Helsedataservice} og relevant etisk komité.
    \item Bruk pseudonymisering eller syntetiske datasett (for eksempel \emph{Synthea} eller data generert med verktøy fra Norsk Helsenett) i tidlige prototyper, og etabler klare rutiner for når ekte pasientdata kan benyttes.
    \item Dokumenter dataflyt, tilgangskontroller og logging i prosjektets styringsdokumenter, og knytt tiltakene til kapittel~6 om validering og tillit.
    \item Sørg for at modell- og beslutningsstøtte evalueres for bias og klinisk gyldighet før utrulling. Logg versjoner slik at fagfeller kan etterprøve resultater.
\end{enumerate}

\section{Forslag til arbeidsmåte}
\begin{enumerate}
    \item Definér et konkret problem (for eksempel energistyring i en campusbygning) og velg hvilke datasett som best dekker behovet for historiske, sanntids- og kontekstdata.
    \item Etabler en pipeline med streamingverktøy (Kafka eller tilsvarende) og en modell- eller visualiseringsplattform (Azure Digital Twins, OpenModelica, CesiumJS).
    \item Dokumenter datakvalitet, lisenskrav og eventuelle personvernhensyn i prosjektets styringsdokumenter.
    \item Lag forslag til videre arbeid i prosjektet, for eksempel supplering med syntetiske data eller integrasjon mot læringsmoduler i Kapittel~5.
\end{enumerate}

\section{Styringsmaler og beslutningsverktøy}
Beslutningsverktøyene under kan brukes sammen med Kapittel~7 og Kapittel~8 for å strukturere ansvar, gevinster og datasamarbeid i casene.

\subsection{RACI-S-mal for dataspace-program}
\begin{table}[h]
    \centering
    \caption{Mal for RACI-S-matrise}
    \label{tab:appendix-raci}
    \begin{tabular}{p{4.2cm}ccccc}
        \toprule
        Aktivitet & Rolle 1 & Rolle 2 & Rolle 3 & Rolle 4 & Rolle 5 \\
        \midrule
        Definere mål og scope & & & & & \\
        Etablere dataprodukter og metadata & & & & & \\
        Implementere sikkerhet og tilgang & & & & & \\
        Overvåke datakvalitet og hendelser & & & & & \\
        Rapportere gevinster og tiltak & & & & & \\
        Revidere kontrakter og policy & & & & & \\
        \bottomrule
    \end{tabular}
\end{table}

Bruk kolonnetitler som beskriver faktiske roller (for eksempel «Operatør», «Plattformteam», «TSO», «Myndighet», «Leverandør») og fyll inn R, A, C, I eller S for hver aktivitet.

\subsection{Gevinstplan-mal}
\begin{table}[h]
    \centering
    \caption{Mal for gevinstplan med KPI-er}
    \label{tab:appendix-gevinstplan}
    \begin{tabular}{p{2.8cm}p{3.8cm}p{3.5cm}p{3.2cm}}
        \toprule
        Livssyklusfase & KPI og mål & Datakilde & Oppfølging \\
        \midrule
        Initiativ &  &  &  \\
        Design og implementering &  &  &  \\
        Drift og operasjon &  &  &  \\
        Kontinuerlig forbedring &  &  &  \\
        Avvikling og kunnskapsdeling &  &  &  \\
        \bottomrule
    \end{tabular}
\end{table}

Suppler tabellen med ansvarlige personer og koble den til RACI-S-matrisen slik at indikatorene følges opp på riktig beslutningsarena.

\section{Siteringspraksis og bibliografi}
For å sikre en enhetlig referansestil er et felles Bib\TeX-bibliotek tilgjengelig i filen \texttt{support/referanser.bib}. Kapittelforfattere bør:
\begin{itemize}
    \item bruke \verb+\citet{}+ når kilden skal inngå i selve setningen (for eksempel ``\citet{grieves2017digital} introduserer konseptet").
    \item bruke \verb+\citep{}+ for parentesreferanser (for eksempel ``norske myndigheter vektlegger datadeling \citep{regjeringen2022datastrategi}'').
    \item legge til nye kilder i Bib\TeX-filen med konsistente nøkler og utfylt \texttt{doi} eller \texttt{url} når det er tilgjengelig.
    \item bestille fagfellevurdering av referanselisten når nye kapitler publiseres for å sikre kvalitet på kildene.
\end{itemize}

Bibliografien settes automatisk inn i bokens bakmaterie via \LaTeX-kommandoen \verb+\bibliography{support/referanser}+.

\section{Notater}
Denne ressursoversikten ble opprettet for å støtte oppgaveløsning og casestudier. Oppdater listen ved nye sektorkrav eller når verktøy får endret lisens. Seksjonen om siteringspraksis ble lagt til for å informere kapittelforfattere om bruk av det nye Bib\TeX-arkivet.

