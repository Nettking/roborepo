\chapter{Simulering og analyse}

\section{Læringsmål}
\begin{itemize}
    \item Forklare ulike simuleringsmetoder og når de bør brukes.
    \item Designe arbeidsflyter for analyse av digitale tvillinger.
    \item Evaluere ytelse og kvalitet på simuleringsresultater.
\end{itemize}

\section{Typer simulering}
\begin{itemize}
    \item Deterministisk kontra stokastisk simulering.
    \item Diskret hendelsessimulering, agentbaserte modeller, kontinuerlige modeller.
    \item Samspill mellom simuleringsverktøy og sanntidsdata.
\end{itemize}

\section{Analysemetoder}
\begin{itemize}
    \item Sensitivitetsanalyse og scenarioanalyse.
    \item Optimalisering, parameterstudier og eksperimentdesign.
    \item Visualisering og beslutningsstøtte.
\end{itemize}

\section{Verktøy og arbeidsflyt}
\begin{itemize}
    \item Eksempler på kommersielle og åpne verktøy (AnyLogic, Simulink, Modelica, OpenFoam).
    \item Automatisering av simuleringer med CI/CD-prinsipper.
    \item Integrasjon med data science-plattformer.
\end{itemize}

\section{Refleksjonsspørsmål og øvinger}
\begin{enumerate}
    \item Beskriv når du ville velge agentbasert simulering fremfor kontinuerlige modeller.
    \item Lag en enkel plan for å automatisere en simuleringsstudie med Git og CI/CD.
    \item Diskuter hvordan visualisering kan forbedre beslutningsprosesser.
\end{enumerate}
