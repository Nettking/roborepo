\chapter{Simulering og analyse}

\section{Læringsmål}
\begin{itemize}
    \item Forklare ulike simuleringsmetoder og når de bør brukes.
    \item Designe arbeidsflyter for analyse av digitale tvillinger.
    \item Evaluere ytelse og kvalitet på simuleringsresultater.
\end{itemize}

\section{Typer simulering}
Digitale tvillinger kombinerer flere simuleringsprinsipper for å speile virkeligheten med tilstrekkelig presisjon og responstid. I praksis trengs en bevisst vurdering av hvordan fysiske prosesser, menneskelig samhandling og tilfeldige hendelser skal representeres i modellen.

\subsection{Deterministiske og stokastiske modeller}
Deterministiske modeller forutsetter at like input gir like output. De brukes når fysikkens lover er godt forstått og måleseriene er stabile, for eksempel til varmeoverføring i en byggmodell. Stokastiske modeller inkluderer usikkerhet eller støy i inputvariablene, noe som er nødvendig når sensorverdier varierer eller prosessen påvirkes av menneskelig atferd. En moderne digital tvilling kombinerer ofte begge tilnærmingene ved å bruke deterministiske kjerneberegninger sammen med stokastiske parameterfordelinger for å kvantifisere risiko og toleranser.

\subsection{Diskret hendelsessimulering}
Diskret hendelsessimulering (DES) beskriver systemer som endres ved identifiserbare hendelser, slik som køsystemer, vedlikeholdsplaner eller logistikkflyt. I en digital tvilling for produksjonslinjer kan DES beregne gjennomløpstid, identifisere flaskehalser og evaluere effekten av alternative operasjonssekvenser.

\subsection{Agentbaserte modeller}
Agentbaserte modeller (ABM) representerer individuelle aktører med egne regler og mål. De egner seg for digitale tvillinger der interaksjoner mellom mange aktører påvirker systemets dynamikk, for eksempel i energinett med mange distribuert produserende enheter eller i bylogistikk med autonome kjøretøy. ABM gir innsikt i kollektive mønstre, emergent atferd og hvordan incentiver påvirker helheten.

\subsection{Kontinuerlige og hybride modeller}
Kontinuerlige modeller benytter differensialligninger eller tilnærminger som Modelica for å beskrive prosesser over tid. Når kontinuerlige prosesser må kobles til diskrete hendelser eller beslutninger, brukes hybride modeller. Et eksempel er en digital tvilling av et vannkraftverk som kombinerer kontinuerlige strømningsegenskaper med diskrete beslutninger om når turbiner startes og stoppes.

\subsection{Kopling mot sanntidsdata}
Uavhengig av modelltype må simuleringsløsningen kunne kobles mot sanntidsdata. Det innebærer funksjoner for dataassimilering, filtrering av støy og oppdatering av parametre i forkant av hver simuleringsperiode. En robust digital tvilling har mekanismer som gjør det mulig å planlegge scenarioer offline og samtidig oppdatere simuleringene når nye sensordata ankommer.

\begin{figure}[htbp]
    \centering
    % Alt-tekst: kap04-simuleringsmatrise-v1.alt.md
    \fbox{\parbox{0.9\textwidth}{\centering\textit{Plassholder for simuleringsmatrise som viser kombinasjoner av deterministiske/stokastiske og diskrete/kontinuerlige modeller med norske eksempler.}}}
    \caption{Oversikt over sentrale simuleringsmetoder og hvordan de kombineres i praksis.}
    \label{fig:kap04-simuleringsmatrise}
\end{figure}

\section{Analysemetoder}
Validerte simuleringsmodeller gir grunnlag for analyse som støtter både taktiske og strategiske beslutninger. Analysearbeidet bør være strukturert slik at resultater kan spores tilbake til antakelser og datagrunnlag.

\begin{figure}[htbp]
    \centering
    % Alt-tekst: kap04-analyseflyt-v1.alt.md
    \fbox{\parbox{0.9\textwidth}{\centering\textit{Plassholder for analyseflyt fra datainntak til visualisering med kontrollpunkter og feedback-sløyfer.}}}
    \caption{Foreslått arbeidsflyt for analyse av digitale tvillinger med kvalitetssikring i hvert steg.}
    \label{fig:kap04-analyseflyt}
\end{figure}

\subsection{Sensitivitets- og scenarioanalyse}
Sensitivitetsanalyse identifiserer hvilke parametre som har størst innvirkning på målvariablene, og gjør det mulig å prioritere datainnsamling eller kalibrering. Scenarioanalyse handler om å variere inngangsverdier systematisk for å belyse framtidige muligheter, som ulike etterspørselsnivåer eller driftsstrategier. I en digital tvilling kan scenarioresultater lagres som konfigurasjoner som senere kan visualiseres og sammenlignes i dashbord.

\subsection{Optimalisering og eksperimentdesign}
Optimaliseringsmetoder brukes for å finne beste løsning gitt begrensninger, enten gjennom matematisk programmering, heuristikker eller evolusjonsalgoritmer. Når simuleringen er tidskrevende, kombineres den gjerne med metoder for design av eksperiment (DoE) eller metamodelering for å redusere antall kjøringer. Dette muliggjør presise anbefalinger om for eksempel vedlikeholdsintervaller, produksjonsoppsett eller energistyring.

\subsection{Visualisering og beslutningsstøtte}
Resultatene må presenteres på en måte som skaper tillit og handling. Kombinasjoner av 3D-visualisering, tidsseriegrafer og indikatorer for kvalitet eller risiko hjelper beslutningstakere med å forstå konsekvensene av ulike tiltak. Moderne verktøy gjør det mulig å overføre simuleringsresultater til VR/AR-miljøer eller integrere dem direkte i operative dashbord.

\subsection{Immersiv AR/VR-beslutningsstøtte}
Immersive arbeidsflater gjør det mulig for tverrfaglige team å oppleve de samme modellene samtidig, enten de står i et kontrollrom eller på en fjern lokasjon. Operatører kan «pinne» sensorer og KPI-er direkte på 3D-modellen, mens analytikere kjører scenariovariasjoner og deler resultatene i samme visuelle kontekst. Forskning på vedlikeholdsprosesser viser at slike løsninger gir raskere feilidentifisering og bedre felles situasjonsforståelse \citep{palmarini2018augmented}. Når AR/VR-panelet er koblet til tvillingens simuleringsmotor, kan brukere se hvilke antakelser som ligger bak en anbefaling og teste alternative tiltak før de aktiveres.

I norske industrimiljøer brukes immersive kontrollrom til å støtte både planlagte revisjoner og hendelseshåndtering. Operatørtrening ved onshore støttesentre kombinerer sanntidsdata fra offshore-felt med historiske hendelser og digitale prosedyrer slik at teamet kan øve på kritiske operasjoner uten å stoppe produksjonen. Internasjonale studier av AR/VR i digitale tvillinger viser at effekten blir størst når visualiseringsløsningen integreres tett med datakvalitetssporing, rollebaserte perspektiver og en felles beslutningslogg \citep{zhu2021augmented}. Dette gir både etterprøvbarhet og et læringsgrunnlag for senere forbedringer.

Erfaring fra norske energiselskaper tilsier at immersive flater bør designes etter tre prinsipper:
\begin{itemize}
    \item \textbf{Samtidig synlighet av modell og risikomål:} Visuelle lag må kunne slåes av og på slik at driftsoperatører raskt forstår konsekvenser av tiltak.
    \item \textbf{Brukerroller og perspektivfiltrering:} Kontrollrommet bør støtte både operatører, analytikere og vedlikeholdsplanleggere med egne dashbord, men samme datagrunnlag.
    \item \textbf{Sporbar beslutningslogg:} Tiltak bør loggføres direkte i den immersive flaten og eksporteres til arbeidsordre- og læringssystemer.
\end{itemize}

For fjernvarmecaset i dette kapitlet etableres et immersivt kontrollrom der sanntidsdata, historikk og planlagte tiltak samles i en arbeidsflate. Figur~\ref{fig:kap04-immersiv-beslutning} skisserer hvordan sensorer, scenariokjøringer og beslutningslogg henger sammen i en norsk kontekst. Støttenotatet \textit{kap04-immersiv-case.md} beskriver detaljene i laboratorieøvelsen, inkludert oppgavetekst og evaluering. Notatet inneholder også forslag til hvordan studentene kan dokumentere brukerreise, sikkerhetssjekker og universell utforming i den immersive løsningen.

\begin{figure}[htbp]
    \centering
    % Alt-tekst: kap04-immersiv-beslutning-v1.alt.md
    \fbox{\parbox{0.9\textwidth}{\centering\textit{Plassholder for immersivt kontrollrom som viser AR/VR-flater med sensorer, simulering og beslutningslogg.}}}
    \caption{Storyboard for immersivt beslutningsrom knyttet til fjernvarmecaset.}
    \label{fig:kap04-immersiv-beslutning}
\end{figure}

\subsection{Simuleringscase: Samordnet beredskapsøvelse}
Som supplement til fjernvarmecaset beskrives en beredskapsøvelse for et norsk prosessanlegg der både operatører, driftsingeniører og eksterne beredskapsteam møtes i et immersivt kontrollrom. Tvillingen kombinerer kontinuerlige prosessmodeller, agentbaserte evakueringsscenarioer og diskret hendelsessimulering av vedlikeholdssituasjoner. Formålet er å teste koordinering og beslutningskvalitet når flere samtidige avvik oppstår.

Øvelsen følger tre faser:
\begin{enumerate}
    \item \textbf{Før øvelsen:} Simuleringsplanleggere bygger scenarier med ulike feiltyper og definerer hva som skal logges (alarmer, tiltak, kommunikasjon).
    \item \textbf{Under øvelsen:} Deltakerne bruker AR-briller for å se virtuelle hjelpetekster i kontrollrommet, mens VR-deltakere visualiserer rømningsveier og trykkfordeling i anlegget.
    \item \textbf{Etter øvelsen:} Beslutningsloggen eksporteres til beredskapssystemet, og tiltak evalueres opp mot responstid, avvik fra prosedyrer og læringspunkter.
\end{enumerate}

I støttefilen \textit{kap04-immersiv-case.md} er det lagt inn sjekklister for hvordan en slik øvelse kan gjennomføres i masterkurs, inkludert roller, datakilder og nødvendige simulatorer. Figur~\ref{fig:kap04-beredskap-case} reserverer plass til en illustrasjon som viser koblingen mellom tvilling, AR/VR-grensesnitt og beredskapsprosesser.

\begin{figure}[htbp]
    \centering
    % Alt-tekst: kap04-beredskap-case-v1.alt.md
    \fbox{\parbox{0.9\textwidth}{\centering\textit{Plassholder for beredskapscase som viser samspillet mellom tvilling, AR/VR og responsteam.}}}
    \caption{Oversikt over samordnet beredskapsøvelse støttet av immersiv digital tvilling.}
    \label{fig:kap04-beredskap-case}
\end{figure}

\section{Verktøy og arbeidsflyt}
Et velfungerende verktøyslandskap består av modellering, simulering, datahåndtering og automasjon.

\begin{itemize}
    \item \textbf{Modellerings- og simuleringsverktøy:} Kommersielle alternativer som AnyLogic, Simulink og Ansys Twin Builder tilbyr grafiske modellbyggere og co-simuleringsmuligheter. Åpne verktøy som OpenModelica, OpenFOAM og JuliaSim gir fleksibilitet og kan integreres med egendefinerte bibliotek.
    \item \textbf{Data- og integrasjonsplattform:} ETL-verktøy, meldingskøer (for eksempel Kafka) og skybaserte datasjøer sørger for kontinuerlig oppdatering av tvillingen med historikk og sanntidsstrømmer.
    \item \textbf{Automasjon og kvalitet:} Infrastruktur som Git, container-baserte byggeoppsett, CI/CD-rør og automatiserte tester sikrer reproduserbare simuleringer. Orkestrering med Jupyter Notebooks eller workflow-motorer som Airflow kan styre eksperimenter og rapportering.
    \item \textbf{Ytelsesmiljø:} For store beregninger brukes HPC-klynger eller skyplattformer med GPU/FPGA-akselerasjon. Skaleringsstrategien bør beskrives eksplisitt slik at kostnader kan estimeres og kapasitet kan reserveres.
\end{itemize}

\begin{figure}[htbp]
    \centering
    % Alt-tekst: kap04-verktoystakk-v1.alt.md
    \fbox{\parbox{0.9\textwidth}{\centering\textit{Plassholder for verktøystakk med lag for modellering, integrasjon, automasjon og ytelse, inkludert norske eksempler.}}}
    \caption{Forslag til lagdelt verktøystøtte for simulering og analyse.}
    \label{fig:kap04-verktoystakk}
\end{figure}

Arbeidsflyten starter med versjonskontrollert modellkode, etterfulgt av automatisert validering av parametrisering, simulering og analyse. Resultater pakkes i rapporter eller API-er og lagres i et beslutningsarkiv slik at tidligere scenarioer kan gjenbrukes ved nye vurderinger.

\section{Praksiseksempel: Digital tvilling for fjernvarmenett}
Et norsk energiselskap ønsker å styre et fjernvarmenett mer effektivt og redusere toppbelastninger vinterstid. Teamet utvikler først en kontinuerlig modell basert på rørnettets geometri, termiske egenskaper og pumpelogikk. Ved å kombinere deterministiske varmeoverføringsligninger med stokastiske profiler for kundeuttak fanges variasjonene i forbruksmønsteret.

Deretter bygges en diskret hendelsessimulering av driftsoperasjoner som vedlikeholdsavvik, nødstopper og lastskifte mellom kjeler. Agentbaserte komponenter modellerer kundesegmenter, slik at kampanjer for energisparing kan evalueres. Data fra IoT-målere strømmer kontinuerlig inn via en skyplattform, og parametre kalibreres hver time.

Analysefasen kombinerer sensitivitet mot utetemperatur og energipriser med optimalisering av pumpesetpunkter. Resultatene visualiseres i et dashboard der driftsoperatører kan teste scenarioer før de aktiveres. Etter ett års bruk dokumenterer selskapet 12\% reduksjon i energitopper og et bedre beslutningsgrunnlag for investeringer i nye varmesentraler.

\begin{figure}[htbp]
    \centering
    % Alt-tekst: kap04-fjernvarmecase-v1.alt.md
    \fbox{\parbox{0.9\textwidth}{\centering\textit{Plassholder for fjernvarmecase som kobler dataflyt, modelltyper og styringsdashboard.}}}
    \caption{Illustrasjon av hvordan fjernvarmecaset kombinerer ulike simuleringsmetoder og beslutningsstøtte.}
    \label{fig:kap04-fjernvarmecase}
\end{figure}

\section{Laboratorieøving: Immersivt beslutningsrom}
Laboratorieøvelsen bygger på scenariet i \textit{kap04-immersiv-case.md} og tar utgangspunkt i at studentene får tilgang til et immersivt kontrollrom der fjernvarmeoperasjoner kan testes. Økten varer 90 minutter og gjennomføres i grupper på fire.

\begin{enumerate}
    \item \textbf{Forberedelse:} Studentene analyserer historiske logger og definerer tre hypoteser for lastbalansering. De etablerer et minimumssett av sensorer som skal «festes» til AR-objekter.
    \item \textbf{Gjennomføring:} I laben utfører gruppen scenariokjøringer i AnyLogic/Modelica mens de følger KPI-er og alarmer i VR-panelet. De dokumenterer tiltak, fallback-prosedyrer og hvordan informasjon deles mellom roller.
    \item \textbf{Etterarbeid:} Hver gruppe leverer en to-siders refleksjon som beskriver læring, forbedringer og risikovurdering. Evalueringsrubrikken fra støttenotatet brukes av faglærer og medstudenter til å gi karakteren «godkjent/ikke godkjent».
\end{enumerate}

Rubrikken vektlegger modellforankring, beslutningslogg, universell utforming og risikohåndtering, og kobler eksplisitt til kravene i Kapittel~5 og Kapittel~6. Tabell~\ref{tab:kap04-immersiv-rubrikk} oppsummerer kriteriene som brukes i vurderingen.

\begin{table}[htbp]
    \centering
    \begin{tabular}{p{0.26\textwidth}p{0.34\textwidth}p{0.30\textwidth}}
        \toprule
        \textbf{Kriterium} & \textbf{Beskrivelse} & \textbf{Avansert nivå}\
        \midrule
        Modellforankring & Alle tiltak begrunnes med data fra tvillingen, og forutsetninger dokumenteres i beslutningsloggen. & Hypoteser testes mot minst to scenarier og knyttes til konkrete usikkerheter. \\
        Samhandling og kommunikasjon & Roller, ansvar og overleveringer synliggjøres i AR/VR-flaten og i loggen. & Teamet demonstrerer parallelle arbeidsstrømmer og eksplisitt håndtering av konflikter. \\
        Universell utforming & Visualiseringene følger etablerte retningslinjer for kontrast, språk og navigasjon. & Løsningen inkluderer alternative presentasjoner (tekst/lyd) og tilrettelegging for fjernbrukere. \\
        Risikohåndtering & Tiltak evalueres mot sannsynlighet/konsekvens og beredskapsplaner. & Gruppen identifiserer sekundæreffekter og anbefaler forbedringer til prosedyrer og sensornett. \\
        Læringsrefleksjon & Teamet beskriver læringsutbytte, forbedringsforslag og kobling til teori. & Refleksjonen sammenligner AR/VR-løsningen med tradisjonelle kontrollrom og foreslår videre eksperimenter. \\
        \bottomrule
    \end{tabular}
    \caption{Vurderingskriterier for laboratorieøvelsen i immersivt beslutningsrom.}
    \label{tab:kap04-immersiv-rubrikk}
\end{table}

Rubrikken er harmonisert med vurderingsopplegget i lærerveiledningen og kan brukes både til egenvurdering og faglig sensur. Etter gjennomført øving anbefales det å sammenligne resultatene med måltallene i Kapittel~7 for governance og Kapittel~8 for sektorspesifikke KPI-er.

\section{Refleksjonsspørsmål og øvinger}
\begin{enumerate}
    \item Beskriv når du ville velge agentbasert simulering fremfor kontinuerlige modeller.
    \item Lag en enkel plan for å automatisere en simuleringsstudie med Git og CI/CD.
    \item Diskuter hvordan visualisering kan forbedre beslutningsprosesser.
    \item Bruk praksiseksempelet som inspirasjon og skisser hvordan en digital tvilling kan forbedre en annen norsk energitjeneste.
    \item Foreslå en vurderingsrubrikk for en AR/VR-basert øving i eget fagmiljø og begrunn hvilke kriterier som bør vektlegges.
\end{enumerate}
