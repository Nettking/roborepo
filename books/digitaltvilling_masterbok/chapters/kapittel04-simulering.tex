\chapter{Simulering og analyse}

\section{Læringsmål}
\begin{itemize}
    \item Forklare ulike simuleringsmetoder og når de bør brukes.
    \item Designe arbeidsflyter for analyse av digitale tvillinger.
    \item Evaluere ytelse og kvalitet på simuleringsresultater.
\end{itemize}

\section{Typer simulering}
Digitale tvillinger kombinerer flere simuleringsprinsipper for å speile virkeligheten med tilstrekkelig presisjon og responstid. I praksis trengs en bevisst vurdering av hvordan fysiske prosesser, menneskelig samhandling og tilfeldige hendelser skal representeres i modellen.

\subsection{Deterministiske og stokastiske modeller}
Deterministiske modeller forutsetter at like input gir like output. De brukes når fysikkens lover er godt forstått og måleseriene er stabile, for eksempel til varmeoverføring i en byggmodell. Stokastiske modeller inkluderer usikkerhet eller støy i inputvariablene, noe som er nødvendig når sensorverdier varierer eller prosessen påvirkes av menneskelig atferd. En moderne digital tvilling kombinerer ofte begge tilnærmingene ved å bruke deterministiske kjerneberegninger sammen med stokastiske parameterfordelinger for å kvantifisere risiko og toleranser.

\subsection{Diskret hendelsessimulering}
Diskret hendelsessimulering (DES) beskriver systemer som endres ved identifiserbare hendelser, slik som køsystemer, vedlikeholdsplaner eller logistikkflyt. I en digital tvilling for produksjonslinjer kan DES beregne gjennomløpstid, identifisere flaskehalser og evaluere effekten av alternative operasjonssekvenser.

\subsection{Agentbaserte modeller}
Agentbaserte modeller (ABM) representerer individuelle aktører med egne regler og mål. De egner seg for digitale tvillinger der interaksjoner mellom mange aktører påvirker systemets dynamikk, for eksempel i energinett med mange distribuert produserende enheter eller i bylogistikk med autonome kjøretøy. ABM gir innsikt i kollektive mønstre, emergent atferd og hvordan incentiver påvirker helheten.

\subsection{Kontinuerlige og hybride modeller}
Kontinuerlige modeller benytter differensialligninger eller tilnærminger som Modelica for å beskrive prosesser over tid. Når kontinuerlige prosesser må kobles til diskrete hendelser eller beslutninger, brukes hybride modeller. Et eksempel er en digital tvilling av et vannkraftverk som kombinerer kontinuerlige strømningsegenskaper med diskrete beslutninger om når turbiner startes og stoppes.

\subsection{Kopling mot sanntidsdata}
Uavhengig av modelltype må simuleringsløsningen kunne kobles mot sanntidsdata. Det innebærer funksjoner for dataassimilering, filtrering av støy og oppdatering av parametre i forkant av hver simuleringsperiode. En robust digital tvilling har mekanismer som gjør det mulig å planlegge scenarioer offline og samtidig oppdatere simuleringene når nye sensordata ankommer.

\begin{figure}[htbp]
    \centering
    % Alt-tekst: kap04-simuleringsmatrise-v1.alt.md
    \fbox{\parbox{0.9\textwidth}{\centering\textit{Plassholder for simuleringsmatrise som viser kombinasjoner av deterministiske/stokastiske og diskrete/kontinuerlige modeller med norske eksempler.}}}
    \caption{Oversikt over sentrale simuleringsmetoder og hvordan de kombineres i praksis.}
    \label{fig:kap04-simuleringsmatrise}
\end{figure}

\section{Analysemetoder}
Validerte simuleringsmodeller gir grunnlag for analyse som støtter både taktiske og strategiske beslutninger. Analysearbeidet bør være strukturert slik at resultater kan spores tilbake til antakelser og datagrunnlag.

\begin{figure}[htbp]
    \centering
    % Alt-tekst: kap04-analyseflyt-v1.alt.md
    \fbox{\parbox{0.9\textwidth}{\centering\textit{Plassholder for analyseflyt fra datainntak til visualisering med kontrollpunkter og feedback-sløyfer.}}}
    \caption{Foreslått arbeidsflyt for analyse av digitale tvillinger med kvalitetssikring i hvert steg.}
    \label{fig:kap04-analyseflyt}
\end{figure}

\subsection{Sensitivitets- og scenarioanalyse}
Sensitivitetsanalyse identifiserer hvilke parametre som har størst innvirkning på målvariablene, og gjør det mulig å prioritere datainnsamling eller kalibrering. Scenarioanalyse handler om å variere inngangsverdier systematisk for å belyse framtidige muligheter, som ulike etterspørselsnivåer eller driftsstrategier. I en digital tvilling kan scenarioresultater lagres som konfigurasjoner som senere kan visualiseres og sammenlignes i dashbord.

\subsection{Optimalisering og eksperimentdesign}
Optimaliseringsmetoder brukes for å finne beste løsning gitt begrensninger, enten gjennom matematisk programmering, heuristikker eller evolusjonsalgoritmer. Når simuleringen er tidskrevende, kombineres den gjerne med metoder for design av eksperiment (DoE) eller metamodelering for å redusere antall kjøringer. Dette muliggjør presise anbefalinger om for eksempel vedlikeholdsintervaller, produksjonsoppsett eller energistyring.

\subsection{Visualisering og beslutningsstøtte}
Resultatene må presenteres på en måte som skaper tillit og handling. Kombinasjoner av 3D-visualisering, tidsseriegrafer og indikatorer for kvalitet eller risiko hjelper beslutningstakere med å forstå konsekvensene av ulike tiltak. Moderne verktøy gjør det mulig å overføre simuleringsresultater til VR/AR-miljøer eller integrere dem direkte i operative dashbord.

\subsection{Immersiv AR/VR-beslutningsstøtte}
Immersive arbeidsflater gjør det mulig for tverrfaglige team å oppleve de samme modellene samtidig, enten de står i et kontrollrom eller på en fjern lokasjon. Operatører kan «pinne» sensorer og KPI-er direkte på 3D-modellen, mens analytikere kjører scenariovariasjoner og deler resultatene i samme visuelle kontekst. Forskning på vedlikeholdsprosesser viser at slike løsninger gir raskere feilidentifisering og bedre felles situasjonsforståelse \citep{palmarini2018augmented}. Når AR/VR-panelet er koblet til tvillingens simuleringsmotor, kan brukere se hvilke antakelser som ligger bak en anbefaling og teste alternative tiltak før de aktiveres.

I norske industrimiljøer brukes immersive kontrollrom til å støtte både planlagte revisjoner og hendelseshåndtering. Operatørtrening ved onshore støttesentre kombinerer sanntidsdata fra offshore-felt med historiske hendelser og digitale prosedyrer slik at teamet kan øve på kritiske operasjoner uten å stoppe produksjonen. Internasjonale studier av AR/VR i digitale tvillinger viser at effekten blir størst når visualiseringsløsningen integreres tett med datakvalitetssporing, rollebaserte perspektiver og en felles beslutningslogg \citep{zhu2021augmented}. Dette gir både etterprøvbarhet og et læringsgrunnlag for senere forbedringer.

Partnere som Kongsberg Digital og Equinor har de siste årene demonstrert kontrollrom der AR/VR-overlegg ligger direkte oppå live datastrømmer fra produksjonsanlegg, og der beslutningslogger eksporteres til arbeidsordre-systemer \citep{kongsberg2023kognitwin}. Erfaringene viser at immersive grensesnitt må være tett koblet til modellforvaltning og hendelseshåndtering for å skape tillit i sikkerhetskritiske miljøer.

Erfaring fra norske energiselskaper tilsier at immersive flater bør designes etter tre prinsipper:
\begin{itemize}
    \item \textbf{Samtidig synlighet av modell og risikomål:} Visuelle lag må kunne slåes av og på slik at driftsoperatører raskt forstår konsekvenser av tiltak.
    \item \textbf{Brukerroller og perspektivfiltrering:} Kontrollrommet bør støtte både operatører, analytikere og vedlikeholdsplanleggere med egne dashbord, men samme datagrunnlag.
    \item \textbf{Sporbar beslutningslogg:} Tiltak bør loggføres direkte i den immersive flaten og eksporteres til arbeidsordre- og læringssystemer.
\end{itemize}

Et typisk implementeringsløp starter med å kuratere en referansemodell i modellforvaltningsverktøyet før scenarier lastes til den immersive flaten. Deretter kobles hendelsesstrømmer (for eksempel alarmer og driftslogger) gjennom en meldingskø som synkroniserer 3D-panelet med dashboard og beslutningslogg. Til slutt integreres AR/VR-klienten med hendelseshåndteringssystemet slik at tiltak kan publiseres som arbeidsordre og analyseres i etterkant \citep{cognite2023akerbp,statnett2024kontrolltarn}. Denne kjeden gjør at operatører kan skifte mellom strategisk planlegging og sanntidsdrift uten å forlate det samme økosystemet.

\begin{table}[htbp]
    \centering
    \begin{tabular}{p{0.25\textwidth}p{0.35\textwidth}p{0.28\textwidth}}
        \toprule
        \textbf{Scenario} & \textbf{Simuleringsfokus} & \textbf{Immersiv gevinst}\\
        \midrule
        Offshore vedlikeholdsplan & Kombinere stokastisk feilmodell og ressursoptimalisering for fartøy og personell. & Felles situasjonsbilde for operatører på land og mannskap offshore, med deling av tiltak i sanntid \citep{cognite2023akerbp}.\\
        Kraftnett-driftskontroll & Hybride modeller for lastflyt og hendelsestrigget switching av brytere. & Visualisering av belastning og beredskapsplaner i kontrolltårn som understøtter risikoavveiing \citep{statnett2024kontrolltarn}.\\
        Lufthavnlogistikk & Diskret hendelsessimulering av flybevegelser, bagasje og bakketjenester. & Operativ koordinering av ressurser med AR-overlegg på kart og gateplan for felles prioritering \citep{avinor2022digital}.\\
        \bottomrule
    \end{tabular}
    \caption{Eksempler på hvordan norske virksomheter kobler simulering og immersive grensesnitt.}
    \label{tab:kap04-immersive-scenarier}
\end{table}

Tabell~\ref{tab:kap04-immersiv-arkitektur} oppsummerer en operasjonsarkitektur som møter prinsippene ovenfor. Tabellen brukes både som bestilling til grafikkteamet og som sjekkliste når laboratoriet rigges.

\begin{table}[htbp]
    \centering
    \begin{tabular}{p{0.22\textwidth}p{0.38\textwidth}p{0.28\textwidth}}
        \toprule
        \textbf{Lag} & \textbf{Hovedinnhold} & \textbf{Norske referansepunkt}\\
        \midrule
        Datafangst og edge & SCADA-strømmer, IoT-sensorer, historikkbuffer med sanntidsreplay og kvalitetssjekk. & Fjernvarmesentraler i Trondheim og Stavanger, energioperasjoner i Equinor OMNIA.\\
        Modell- og analysemotor & Kontinuerlige og diskrete modeller, scenariobank og API-er for å trigge simuleringer fra AR/VR-panelet. & Kognitwin Live Operations, AnyLogic-modeller fra NTNU-pilot.\\
        Samhandling og beslutning & Rollebaserte dashboards, AR/VR-visninger med beslutningslogg, eksport til arbeidsordre og læringsplattform. & Kongsberg Digitals kontrollromsdemoer, læringsopplegget i kapittel 5.\\
        Sikkerhet og etterlevelse & Tilgangsstyring, hendelseskøer, logging i tråd med IEC 62443-2-1 og norske øvingskrav. & Sikkerhetsregime fra kapittel 6 og DSBs øvelsesveileder.\\
        \bottomrule
    \end{tabular}
    \caption{Kjernekomponenter i et immersivt beslutningsrom for fjernvarmedrift.}
    \label{tab:kap04-immersiv-arkitektur}
\end{table}

For fjernvarmecaset i dette kapitlet etableres et immersivt kontrollrom der sanntidsdata, historikk og planlagte tiltak samles i en arbeidsflate. Figur~\ref{fig:kap04-immersiv-beslutning} skisserer hvordan sensorer, scenariokjøringer og beslutningslogg henger sammen i en norsk kontekst. Støttenotatet \textit{kap04-immersiv-case.md} beskriver detaljene i laboratorieøvelsen, inkludert oppgavetekst og evaluering. Notatet inneholder også forslag til hvordan studentene kan dokumentere brukerreise, sikkerhetssjekker og universell utforming i den immersive løsningen.

\begin{figure}[htbp]
    \centering
    % Alt-tekst: kap04-immersiv-beslutning-v1.alt.md
    \fbox{\parbox{0.9\textwidth}{\centering\textit{Plassholder for immersivt kontrollrom som viser AR/VR-flater med sensorer, simulering og beslutningslogg.}}}
    \caption{Storyboard for immersivt beslutningsrom knyttet til fjernvarmecaset.}
    \label{fig:kap04-immersiv-beslutning}
\end{figure}

\subsection{Simuleringscase: Samordnet beredskapsøvelse}
Som supplement til fjernvarmecaset beskrives en beredskapsøvelse for et norsk prosessanlegg der både operatører, driftsingeniører og eksterne beredskapsteam møtes i et immersivt kontrollrom. Tvillingen kombinerer kontinuerlige prosessmodeller, agentbaserte evakueringsscenarioer og diskret hendelsessimulering av vedlikeholdssituasjoner. Formålet er å teste koordinering og beslutningskvalitet når flere samtidige avvik oppstår.

Øvelsen følger tre faser i tråd med nasjonale øvingskrav \citep{dsb2023ovelser}:
\begin{enumerate}
    \item \textbf{Før øvelsen:} Simuleringsplanleggere bygger scenarier med ulike feiltyper og definerer hva som skal logges (alarmer, tiltak, kommunikasjon).
    \item \textbf{Under øvelsen:} Deltakerne bruker AR-briller for å se virtuelle hjelpetekster i kontrollrommet, mens VR-deltakere visualiserer rømningsveier og trykkfordeling i anlegget.
    \item \textbf{Etter øvelsen:} Beslutningsloggen eksporteres til beredskapssystemet, og tiltak evalueres opp mot responstid, avvik fra prosedyrer og læringspunkter.
\end{enumerate}

I støttefilen \textit{kap04-immersiv-case.md} er det lagt inn sjekklister for hvordan en slik øvelse kan gjennomføres i masterkurs, inkludert roller, datakilder og nødvendige simulatorer. Figur~\ref{fig:kap04-beredskap-case} reserverer plass til en illustrasjon som viser koblingen mellom tvilling, AR/VR-grensesnitt og beredskapsprosesser.

\begin{figure}[htbp]
    \centering
    % Alt-tekst: kap04-beredskap-case-v1.alt.md
    \fbox{\parbox{0.9\textwidth}{\centering\textit{Plassholder for beredskapscase som viser samspillet mellom tvilling, AR/VR og responsteam.}}}
    \caption{Oversikt over samordnet beredskapsøvelse støttet av immersiv digital tvilling.}
    \label{fig:kap04-beredskap-case}
\end{figure}

\section{Verktøy og arbeidsflyt}
Et velfungerende verktøyslandskap består av modellering, simulering, datahåndtering og automasjon.

\begin{itemize}
    \item \textbf{Modellerings- og simuleringsverktøy:} Kommersielle alternativer som AnyLogic, Simulink og Ansys Twin Builder tilbyr grafiske modellbyggere og co-simuleringsmuligheter. Åpne verktøy som OpenModelica, OpenFOAM og JuliaSim gir fleksibilitet og kan integreres med egendefinerte bibliotek.
    \item \textbf{Data- og integrasjonsplattform:} ETL-verktøy, meldingskøer (for eksempel Kafka) og skybaserte datasjøer sørger for kontinuerlig oppdatering av tvillingen med historikk og sanntidsstrømmer.
    \item \textbf{Automasjon og kvalitet:} Infrastruktur som Git, container-baserte byggeoppsett, CI/CD-rør og automatiserte tester sikrer reproduserbare simuleringer. Orkestrering med Jupyter Notebooks eller workflow-motorer som Airflow kan styre eksperimenter og rapportering.
    \item \textbf{Ytelsesmiljø:} For store beregninger brukes HPC-klynger eller skyplattformer med GPU/FPGA-akselerasjon. Skaleringsstrategien bør beskrives eksplisitt slik at kostnader kan estimeres og kapasitet kan reserveres.
\end{itemize}

\begin{figure}[htbp]
    \centering
    % Alt-tekst: kap04-verktoystakk-v1.alt.md
    \fbox{\parbox{0.9\textwidth}{\centering\textit{Plassholder for verktøystakk med lag for modellering, integrasjon, automasjon og ytelse, inkludert norske eksempler.}}}
    \caption{Forslag til lagdelt verktøystøtte for simulering og analyse.}
    \label{fig:kap04-verktoystakk}
\end{figure}

Arbeidsflyten starter med versjonskontrollert modellkode, etterfulgt av automatisert validering av parametrisering, simulering og analyse. Resultater pakkes i rapporter eller API-er og lagres i et beslutningsarkiv slik at tidligere scenarioer kan gjenbrukes ved nye vurderinger.

\section{Sirkulær simulering og bærekraft}
Simuleringsmodeller for digitale tvillinger brukes i økende grad til å støtte sirkulærøkonomiske beslutninger. Norske virksomheter som Statsbygg og industripartnere innen batteri- og prosessindustri kobler tvillingene til materialregistre for å dokumentere ombrukspotensial og klimafotavtrykk \citep{statsbygg2022ombruk,statsbygg2023loopfront}. I masterkurset bør studentene lære å modellere både tekniske strømmer og ressursgjenvinning slik at tiltak kan begrunnes med tallfestede effekter.

\subsection{Materialstrømmer og ombruksscenarier}
En sirkulær tvilling kombinerer produksjons- og driftssimuleringer med livssyklusdata for komponentene. I praksis etableres et bibliotek som angir vekt, karbonintensitet og forventet levetid for hvert objekt. Når komponentene demonteres eller oppgraderes, registreres de i et markedsgrensesnitt (for eksempel Loopfront) slik at nye prosjekter kan reservere dem \citep{statsbygg2023loopfront}. Simuleringsløpet består av tre steg:
\begin{enumerate}
    \item \textbf{Kartlegg materialstrømmer:} Bruk BIM-data og sensorer for å estimere volum, tilstand og demonteringskostnader for hver komponent.
    \item \textbf{Kjør ombruksscenarier:} Simuler alternative planer for gjenbruk lokalt, regionalt eller hos partnere og vurder effekten på kost og CO$_2$-utslipp.
    \item \textbf{Oppdater tiltakslogg:} Dokumenter hvilke deler som faktisk gjenbrukes og synkroniser med indikatorene i kapittel 7 slik at gevinstoppfølgingen blir konsistent.
\end{enumerate}

Simuleringsresultater deles med innkjøps- og prosjektledere gjennom dashboards som viser materialbalanser, utslippsbaner og økonomiske gevinster. Når studentene jobber med caser innen bygg og industri, bør de sammenligne ombruksscenarier mot referansebanen som beskriver lineær utskifting. Dette gjør det mulig å diskutere både tekniske avhengigheter og regulatoriske krav fra blant annet EU sitt sirkulærøkonomiprogram \citep{miljodir2022sirkular,regjeringen2021sirkulaer}.

\subsection{Bærekraftsindikatorer for tvillinglaboratoriet}
For å sikre at simuleringene faktisk leverer bærekraftige resultater, må laboratoriet definere indikatorer som er kompatible med styringsmodellen i kapittel 7. Tabell~\ref{tab:kap04-sirkular-indikatorer} foreslår et utvalg indikatorer som kan beregnes direkte fra simuleringsdata. Indikatorene kan knyttes til nasjonale veikart og bransjeforpliktelser, noe som gjør det enklere å benchmarke mot partnere i norsk industri \citep{norskindustri2023sirkular}.

\begin{table}[htbp]
    \centering
    \begin{tabular}{p{0.23\textwidth}p{0.34\textwidth}p{0.31\textwidth}}
        \toprule
        \textbf{Indikator} & \textbf{Simuleringsgrunnlag} & \textbf{Bruksområde i kurset}\\
        \midrule
        Andel ombrukte komponenter & Antall komponenter som går inn i nye scenarier vs. total avhendet mengde. & Planlegging av demonteringssekvenser og logistikk for bygg- og industriprosjekter.\\
        CO$_2$-besparelse per tiltak & Livssyklusdata for materialer kombinert med transport- og energimodeller. & Prioritering av tiltak i tiltakslogg og rapportering til bærekraftspanel i kapittel 7.\\
        Materialgjenvinningsgrad & Masse som forblir i lukket kretsløp dividert på total masse. & Evaluering av sirkulære mål for kommunale bygg og energisystemer.\\
        Ressursintensitet per tjeneste & Simulerte timer drift eller levert energi satt opp mot ressursbruk per scenario. & Sammenligning av scenarier i fjernvarme- og industrilabben for å synliggjøre gevinst.\\
        \bottomrule
    \end{tabular}
    \caption{Forslag til indikatorer for sirkulær simulering i digitale tvillinglaboratorier.}
    \label{tab:kap04-sirkular-indikatorer}
\end{table}

Under laboratorieøkter bør studentgruppene bruke indikatorene til å evaluere egne modeller. Dette innebærer å dokumentere antakelser, datakilder og følsomhet overfor usikkerhet i materialdata. Resultatene kan presenteres i en felles «sirkulær tavle» hvor hvert team oppdaterer status mot målverdier, og kobler funnene til tiltakene fra kapittel 3 og styringsmodellen i kapittel 7.

\section{Hydrogenknutepunkt og forsyningssimulering}
Hydrogen får en stadig tydeligere rolle i norsk industri og maritim transport, og krever samspilt infrastruktur mellom produksjon, lagring og bunkring i havner. \citet{enova2024hydrogenknutepunkt} beskriver hvordan hydrogenknutepunkt skal dimensjoneres for å støtte flere fartøyskategorier samtidig, mens \citet{gassco2023hydrogen} peker på behovet for samordnede rør-, båt- og lastebilstrømmer når hydrogen flyttes mellom industriklynger. En digital tvilling for et slikt knutepunkt må kombinere kontinuerlig prosessmodellering med diskrete hendelser knyttet til fartøyanløp, vedlikehold og sikkerhetskontroller. Scenarioene må også speile usikkerhet i etterspørsel og energipriser slik at investeringer og operative tiltak kan prioriteres over tid.\citep{dnv2023hydrogenforecast}

\subsection{Scenario og datagrunnlag}
Et hydrogenknutepunkt består typisk av elektrolysekapasitet, lagertanker, distribusjonsrør og bunkringsstasjoner i havn. Studentene bør først kartlegge material- og energistrømmer med støtte fra datakatalogen i Kapittel~3. Det innebærer å registrere sensorer for trykk, temperatur, fyllingsgrad og ventiltider, samt logistikkdata for planlagte anløp og lastprofiler. Historiske data fra produksjonsstatistikk og bookinglister brukes som input til simuleringen, mens framtidsscenarier kan trekkes fra porteføljeplaner og markedsutsikter hos de involverte energiselskapene. Prosess- og sikkerhetskrav hentes fra standardkartet i Kapittel~6 slik at hendelser som trykkavlastning, lekkasjetest eller sikkerhetsinspeksjon får tydelige regler i modellen.

\subsection{Indikatorpakke for navet}
For å støtte styring må simuleringen levere indikatorer som kobler tekniske måltall til kapasitet, klimaeffekt og sikkerhet. Tabell~\ref{tab:kap04-hydrogen-indikatorer} viser et forslag til indikatorpakke som kombinerer data fra produksjon, logistikk og marked. Kolonnene speiler hvordan indikatorene må knyttes til tiltakslogg og rapporteringskrav i Kapittel~6.

\begin{table}[htbp]
    \centering
    \caption{Forslag til indikatorer for hydrogenknutepunkt}
    \label{tab:kap04-hydrogen-indikatorer}
    \begin{tabular}{p{0.25\textwidth}p{0.33\textwidth}p{0.32\textwidth}}
        \toprule
        \textbf{Indikator} & \textbf{Datagrunnlag og simuleringsstøtte} & \textbf{Tiltak ved avvik}\\
        \midrule
        Leveringsgrad per uke & Kombinerer elektrolyseproduksjon, lagernivå og bunkringsplan fra DES-modellen. & Planlegg ekstra skift eller alternative leveranser, oppdater leverandørkontrakter i dataspace.\\
        \addlinespace
        Utnyttelse av lagertank & Kontinuerlig nivåmåling koblet med scenario for fartøyanløp og sikkerhetsgrenser. & Optimaliser fyllingssekvenser, iverksett trykkavlastning og informer havneoperatører.\\
        \addlinespace
        CO$_2$-besparelse per leveranse & Beregn direkte substitusjon mot marin diesel basert på energiinnhold og logistikkprofiler. & Prioriter kunder eller ruter med størst klimaeffekt og rapporter i bærekraftrapporten.\\
        \addlinespace
        Sikkerhetsmargin i bunkring & Kombiner tidsplan, trykkmålinger og nødsystemtester i en risikomodell. & Utsett bunkring, kall inn ekstra sikkerhetspersonell og oppdater hendelseslogg.\\
        \bottomrule
    \end{tabular}
\end{table}

Indikatorene gir grunnlag for å styre kapasitet og risiko i realtid. Når simuleringen er integrert i dataspace-arkitekturen fra Kapittel~3, kan indikatorene deles med både operatører, leverandører og myndigheter gjennom standardiserte API-er. Dette gjør det enklere å etablere felles kontrolltårn mellom havn, energiselskap og transportaktører.

\subsection{Øvingsopplegg for masterkurset}
Et praktisk undervisningsopplegg kan følge tre hovedtrinn:
\begin{enumerate}
    \item \textbf{Bygg scenariobibliotek:} Studentene etablerer minst tre etterspørselsprofiler (base, høy vekst og krise) og kobler dem til sannsynlighetsfordelinger for fartøyanløp og industriell etterspørsel.
    \item \textbf{Koble til styringssløyfer:} Resultatene mates inn i indikatorpanelet i Tabell~\ref{tab:kap04-hydrogen-indikatorer} og kobles til rapporteringspakken i Kapittel~6 slik at hendelser automatisk flagges.
    \item \textbf{Evaluere tiltak:} Gruppene tester tiltak som å forskyve produksjon, aktivere mobile lagertanker eller samarbeide med nabohavner. Effekten måles på både leveringssikkerhet, klima og sikkerhetsmarginer.
\end{enumerate}
Refleksjonsdelen bør knytte læringspunkter til governance-modellen i Kapittel~7: Hvem beslutter investeringer, hvordan deles risiko med partnerne og hvilke nye indikatorer må inn i gevinstplanene? Når simuleringen kobles til lærerveiledningen får studentene også erfaring med å formidle funn til eksterne interessenter.

\section{Simulering av vinterdrift i fjellbaner}
Norske fjellbaner som Bergensbanen og Ofotbanen opplever krevende vinterforhold med sterk vind, ising og raske temperaturskift. Driftsselskapene kombinerer mekanisk snørydding, varmeelementer i sporveksler og beredskapslagre for kritiske komponenter, men trenger beslutningsstøtte for å vite når tiltak skal settes inn. En digital tvilling som kombinerer værprognoser, energistyring og beredskapslogistikk kan redusere nedetid og energibruk samtidig som sikkerheten ivaretas \citep{banenor2023vinterdrift,met2023fjellvaer}. I masterkurset kan studentene bruke caset til å trene på sammensatte simuleringer som koordinerer både tekniske og operative valg.

\subsection{Datasett og modelloppsett}
Caset krever at kontinuerlige modeller for energi- og varmeoverføring kombineres med diskrete hendelser for snøfall, avvik og beredskapsutrykning. Et anbefalt minimumsoppsett er:
\begin{itemize}
    \item \textbf{Meteorologiske data:} Værvarsel fra Meteorologisk institutt og historiske målinger fra fjellstasjoner for temperatur, vind, nedbør og isingsindeks \citep{met2023fjellvaer}.
    \item \textbf{Beredskapslogistikk:} Ressursoversikt over snøryddingsmaskiner, mannskap og reservedeler med oppdaterte responstider.
    \item \textbf{Energi og infrastruktur:} Effektbehov til sporvekselvarme, transformatorer og driftsbygg, inkludert avtaler for fleksibelt strømforbruk \citep{jernbanedirektoratet2022energi}.
    \item \textbf{Trafikkplan:} Planlagte togavganger, prioriterte godstog og kritiske tidsvinduer for passasjertrafikk.
\end{itemize}
Modellen må kunne simulere hvordan varmeelementer prioriteres når kapasiteten er begrenset, og hvordan alternative togoppsett påvirker både energibruk og regularitet. Når datagrunnlaget struktureres i dataspace-arkitekturen fra Kapittel~3, kan studentgrupper samarbeide om forskjellige delmodeller og integrere dem gjennom felles API-er.

\subsection{Scenarioer og eksperimentdesign}
Et undervisningsopplegg kan organisere scenarioene i tre nivåer:
\begin{enumerate}
    \item \textbf{Forebyggende drift:} Standard vinterdriftsplan med moderate værforhold. Studentene optimaliserer varmeplaner, vedlikeholdsintervaller og bruk av autonome sensormålinger.
    \item \textbf{Varslet ekstremvær:} Hurtig opptrapping av vind og snø, der beslutningen står mellom tidlig stenging, ekstra mannskap eller omdirigering av tog. Her må modellen kombinere prediktiv vedlikeholdsanalyse med logistikk for snørydding.
    \item \textbf{Uforutsett hendelse:} Plutselig ising i kontaktledning eller sporveksler, som krever nødprosedyrer og koordinering med trafikkstyring. Simuleringen må synkroniseres med hendelsesjournalen i Kapittel~6.
\end{enumerate}
For hvert scenario bør studentene definere hypoteser om energiforbruk, regularitet og sikkerhetsmarginer. Eksperimentene kan automatiseres med scripts som justerer værparametere og ressursfordeling, slik at resultatene kan analyseres i et dashboard under gruppediskusjoner.

\subsection{Indikatorer for styring og læring}
Tabell~\ref{tab:kap04-vinterdrift-indikatorer} viser et forslag til indikatorer som kombinerer simuleringens resultatdata med styringsmodellene i Kapittel~6 og Kapittel~7. Indikatorene kan kobles til tiltaksloggen og rapporteringen i Kapittel~7 slik at både teknisk og organisatorisk læring dokumenteres.

\begin{table}[htbp]
    \centering
    \caption{Indikatorpakke for vinterdrift i fjellbaner}
    \label{tab:kap04-vinterdrift-indikatorer}
    \begin{tabular}{p{0.26\textwidth}p{0.34\textwidth}p{0.28\textwidth}}
        \toprule
        \textbf{Indikator} & \textbf{Simuleringsgrunnlag} & \textbf{Foreslåtte tiltak ved avvik}\\
        \midrule
        Regularitet per tidsvindu & Diskrete hendelser for togbevegelser kombinert med sannsynlighet for stengte spor. & Prioriter alternative togoppsett, varsle trafikkstyring og oppdatere passasjerinformasjon.\\
        \addlinespace
        Energiforbruk i sporvekselvarme & Kontinuerlig modell av varmeelementer og energipriser i driftstimen. & Omprioriter varmeplaner, aktivere fleksibilitetsavtaler og dele effektdata med energiselskap.\\
        \addlinespace
        Responstid for beredskapsmannskap & Logistikkmodell for maskiner og personell med rutevalg og væravhengige forsinkelser. & Flytt utstyr til alternative beredskapsbaser, reforhandle kontrakter og oppdatere tiltakslogg.\\
        \addlinespace
        Sikkerhetsmargin for kontaktledninger & Kombinerer isingsprognoser med temperatursensornett og inspeksjonsplan. & Iverksett nødavising, redusér hastighet eller steng strekninger etter beredskapsprotokoll.\\
        \bottomrule
    \end{tabular}
\end{table}

Resultatene bør rapporteres i en månedlig vinterdriftsreview der baneforvalter, trafikkstyring og energipartnere vurderer tiltaksliste og læringspunkter. Når indikatorene deles gjennom dataspace-arkitekturen, kan oppdaterte erfaringer brukes i de nasjonale vinterdriftsforaene som Jernbanedirektoratet fasiliterer \citep{jernbanedirektoratet2022energi}. Dette gir studentene innsikt i hvordan tekniske anbefalinger må forankres i sektorens styringslinjer og samarbeidsarenaer.

\section{Praksiseksempel: Digital tvilling for fjernvarmenett}
Et norsk energiselskap ønsker å styre et fjernvarmenett mer effektivt og redusere toppbelastninger vinterstid. Teamet utvikler først en kontinuerlig modell basert på rørnettets geometri, termiske egenskaper og pumpelogikk. Ved å kombinere deterministiske varmeoverføringsligninger med stokastiske profiler for kundeuttak fanges variasjonene i forbruksmønsteret.

Deretter bygges en diskret hendelsessimulering av driftsoperasjoner som vedlikeholdsavvik, nødstopper og lastskifte mellom kjeler. Agentbaserte komponenter modellerer kundesegmenter, slik at kampanjer for energisparing kan evalueres. Data fra IoT-målere strømmer kontinuerlig inn via en skyplattform, og parametre kalibreres hver time.

Analysefasen kombinerer sensitivitet mot utetemperatur og energipriser med optimalisering av pumpesetpunkter. Resultatene visualiseres i et dashboard der driftsoperatører kan teste scenarioer før de aktiveres. Løsningen bygger videre på anbefalinger fra International Energy Agency om å bruke digitale tvillinger til å balansere last i fjernvarmenett \citep{iea2021district}. Etter ett års bruk dokumenterer selskapet 12\% reduksjon i energitopper og et bedre beslutningsgrunnlag for investeringer i nye varmesentraler.

\begin{figure}[htbp]
    \centering
    % Alt-tekst: kap04-fjernvarmecase-v1.alt.md
    \fbox{\parbox{0.9\textwidth}{\centering\textit{Plassholder for fjernvarmecase som kobler dataflyt, modelltyper og styringsdashboard.}}}
    \caption{Illustrasjon av hvordan fjernvarmecaset kombinerer ulike simuleringsmetoder og beslutningsstøtte.}
    \label{fig:kap04-fjernvarmecase}
\end{figure}

\section{Praksiseksempel: Immersivt kontrolltårn for kraftsystemet}
Statnett utvikler digitale kontrolltårn der hybride simuleringer av lastflyt kombineres med hendelsesstyrt beredskapstrening. Operasjonsmiljøet bygger på en digital tvilling av sentralnettet, supplert med detaljmodeller av utvalgte stasjonsnoder og sanntidsdata fra sensorer og verneanlegg \citep{statnett2024kontrolltarn}. Scenarioene genereres i forkant av øvelser ved å kombinere historiske hendelser, værprognoser og planlagte vedlikeholdsaktiviteter. Når driftsplanleggere starter en økt, lastes scenarioet inn i et immersivt rom der nettoperatører, analysestøtte og beredskapsledelse kan dele samme visuelle flate.

Simuleringsmotoren beregner kontinuerlige lastflytprognoser og foreslår bryteroperasjoner. Dersom en hendelse utløses, aktualiseres også diskrete sekvenser for feilsøking og gjeninnkobling. Disse kombinasjonene gjør det mulig å teste alternative tiltak som lastdeling, opp- og nedregulering eller omkoblinger uten å påvirke den faktiske driften. Resultatene visualiseres i AR-paneler som er projisert på vegger og individuelle skjermer, slik at hvert teammedlem kan filtrere visningen etter sin rolle samtidig som felles risikobilde bevares \citep{kongsberg2023kognitwin}.

Etter hver øvelse eksporteres beslutningsloggen til virksomhetens hendelseshåndteringssystem, der læringspunkter og anbefalte tiltak spores til konkrete simuleringskjøringer. Statnett rapporterer at denne kombinasjonen av simulator, immersiv visualisering og strukturert beslutningslogg gir raskere identifikasjon av kritiske tiltak og bedre overlevering til vedlikeholdsplanlegging \citep{statnett2024kontrolltarn}. Erfaringene deles med regionale nettselskaper gjennom felles treningsprogram, og casebeskrivelsen brukes som grunnlag for masterstudenters prosjektoppgaver om resilient strømnett.

\section{Praksiseksempel: Flom- og overvannslaboratorium for kommuner}
Flom- og overvannsrisiko er et økende problem i norske byer, og myndighetene anbefaler at kommuner kombinerer hydrologiske analyser med digitale tvillinger for å teste tiltak før neste ekstremnedbør \citep{nve2022kommunal,dsb2022beredskap}. Oslo kommune har utviklet en overvannsstrategi der digitale scenarier kobles til grønn infrastruktur, pumpestyring og kriseplaner slik at beslutninger kan forankres på tvers av etater \citep{oslo2023overvann}. Et kommunalt overvannslaboratorium bygger videre på dette ved å kombinere kontinuerlige strømningmodeller med diskrete hendelser som veistenging, kritiske bygg og kapasitet i beredskapslagre.

Arbeidsløpet organiseres i tre sløyfer som kobler modellering, dataassimilering og beredskap:
\begin{enumerate}
    \item \textbf{Datagrunnlag og kalibrering:} Kartlegg sensor- og kartdata fra NVE, kommunale målenoder og værprognoser. Strømningmodellen kalibreres mot historiske hendelser (for eksempel Vesleofsen 2023) og parameterne versjonskontrolleres i modelljournalen.
    \item \textbf{Scenario- og tiltaksvurdering:} Diskrete hendelser som stengte kulverter, trafikkavvikling og kritiske samfunnsfunksjoner simuleres sammen med kontinuerlige vannføringsmodeller. Tiltak som regnbed, midlertidige barrierer og styrt utslipp testes og rangeres etter effekt og ressursbruk.
    \item \textbf{Operativ oppfølging og læring:} Resultater synkroniseres mot kommunens beredskapsplaner, og indikatorer som vannstand mot kritiske terskler og responstid for tiltak loggføres i samme kontrollpanel som brukes i Kapittel~6.
\end{enumerate}

Tabell~\ref{tab:overvanns-lab} viser hvordan de tre sløyfene oversettes til konkrete leveranser i et overvannslaboratorium. Strukturen kan brukes i undervisningsprosjekter der studentgrupper jobber med egne kommunedata eller åpne datasett.

\begin{table}[ht]
    \centering
    \caption{Leveransepakke for kommunalt flom- og overvannslaboratorium}
    \label{tab:overvanns-lab}
    \begin{tabular}{|p{3.1cm}|p{4.4cm}|p{4.4cm}|p{3.1cm}|}
        \hline
        \textbf{Fase} & \textbf{Formål} & \textbf{Modell- og dataoppsett} & \textbf{Hovedleveranse} \\
        \hline
        Datagrunnlag og kalibrering & Etablere felles datakilder og sikre sporbarhet mot historiske hendelser & Integrert datakatalog med sensorer, kartlag, demografidata og hendelseslogger \citep{nve2022kommunal} & Kalibreringsrapport med parameterhistorikk og kvalitetsindikatorer \\
        \hline
        Scenario- og tiltaksvurdering & Evaluere effekten av tiltak på tvers av fag (samferdsel, bygg, grønn struktur) & Hybride modeller som kombinerer kontinuerlig hydraulikk og diskrete tiltaksskript, koblet til simuleringsmotor for nedbørsscenario & Tiltaksportefølje med kost/nytte, risikoreduserende effekt og krav til gjennomføring \citep{dsb2022beredskap} \\
        \hline
        Operativ oppfølging og læring & Forankre beslutninger i beredskapsplaner og etablere indikatorer for neste hendelse & Dashboard med nivåmålinger, kapasitetsindikatorer, avvikshåndtering og lenker til varslingssystem \citep{oslo2023overvann} & Oppdatert beredskapsplan med læringslogg, varslingsrutiner og kobling til kapittel 6 sitt tillitspanel \\
        \hline
    \end{tabular}
\end{table}

\subsection{Integrasjon med dataspace og kontrolltårn}
Etter hver øvelse dokumenteres læringspunkter i et felles register som deles med nabokommuner og regionale vannområder. Flere kommuner tester nå skybaserte overvannslaboratorier der digitale tvillinger integreres med 3D-bymodeller for å visualisere konsekvenser for kritisk infrastruktur og skoleveier \citep{asplan2023overvannslab}. Erfaringene bør kobles til kapittel 3 sine beskrivelser av dataspace-samarbeid, slik at sensor- og modelldata kan deles sikkert mellom kommuner, entreprenører og statlige myndigheter. Når caset brukes i undervisning, kan studentgrupper simulere effekten av grønne tak, permeable dekker og nødavledninger, og måle gevinstene med samme indikatorer som brukes i kapittel 6 sine beredskapsøvelser.

For å gjøre koblingen eksplisitt bør laboratoriet driftes med samme styringsmekanismer som dataspace-tavlen i kapittel~3 og kontrolltårnet i kapittel~6. Tabell~\ref{tab:kap04-overvann-integrasjon} konkretiserer hvilke beslutninger som må speiles i de andre kapitlene for at beredskapsarbeidet skal bli helhetlig.

\begin{table}[ht]
    \centering
    \caption{Integrasjonspunkter mellom overvannslaboratoriet og øvrige kapitler.}
    \label{tab:kap04-overvann-integrasjon}
    \begin{tabular}{p{0.28\textwidth}p{0.40\textwidth}p{0.24\textwidth}}
        \toprule
        \textbf{Integrasjonsområde} & \textbf{Tiltak i overvannslaboratoriet} & \textbf{Kobling} \\
        \midrule
        Sanntidsvarsling & Synkronisere flomvarsler, sensoralarmer og beslutningslogg i samme hendelsesprosess som dataspace-operatøren bruker. & Kapittel~3 dataspace-drift, Kapittel~6 hendelseshåndtering \citep{digdir2024sanntidsdata,dsb2022beredskap} \\
        Indikatorpanel & Oppdatere dashboardet med vannstand, responstid og tiltakseffekt slik at indikatorene kan leses direkte inn i tillitspanelet. & Kapittel~6 tillitsindikatorer, Kapittel~7 gevinstlogg \citep{digdir2023styringai} \\
        Data- og modellkvalitet & Versjonskontrollere hydrologiske modeller og dele kvalitetsrapporter gjennom dataspace-katalogen. & Kapittel~3 datakvalitetsstyring \citep{nve2022kommunal} \\
        Læringssløyfer & Samle forbedringsforslag fra øvelser og kommunale pilotprosjekter i samme tiltaksregister som brukes i styringskapittelet. & Kapittel~7 governance, Kapittel~8 sektorreiser \citep{dsb2022beredskap} \\
        \bottomrule
    \end{tabular}
\end{table}

En månedlig driftsgjennomgang mellom kommunens beredskapsteam, dataspace-operatøren og kontrolltårnansvarlige bør bekrefte at tabellen er oppdatert. Møtet følger sjekklisten fra kapittel~3: avvikslogg oppdateres, kommende endringer tidssettes og eventuelle behov for syntetiske datasett vurderes. Resultatene tas inn i kvalitetsjournalen fra kapittel~6 og i governance-oversikten fra kapittel~7 slik at tiltakene forblir sporbare.

\section{Laboratorieøving: Immersivt beslutningsrom}
Laboratorieøvelsen bygger på scenariet i \textit{kap04-immersiv-case.md} og tar utgangspunkt i at studentene får tilgang til et immersivt kontrollrom der fjernvarmeoperasjoner kan testes. Økten varer 90 minutter og gjennomføres i grupper på fire. Datastrømmer og roller speiles i Tabell~\ref{tab:kap04-immersiv-arkitektur} slik at studentene forstår hvordan teknologi og organisasjon henger sammen.

\begin{enumerate}
    \item \textbf{Forberedelse:} Studentene analyserer historiske logger og definerer tre hypoteser for lastbalansering. De etablerer et minimumssett av sensorer som skal «festes» til AR-objekter.
    \item \textbf{Gjennomføring:} I laben utfører gruppen scenariokjøringer i AnyLogic/Modelica mens de følger KPI-er og alarmer i VR-panelet. De dokumenterer tiltak, fallback-prosedyrer og hvordan informasjon deles mellom roller.
    \item \textbf{Etterarbeid:} Hver gruppe leverer en to-siders refleksjon som beskriver læring, forbedringer og risikovurdering. Evalueringsrubrikken fra støttenotatet brukes av faglærer og medstudenter til å gi karakteren «godkjent/ikke godkjent».
\end{enumerate}

Rubrikken vektlegger modellforankring, beslutningslogg, universell utforming og risikohåndtering, og kobler eksplisitt til kravene i Kapittel~5 og Kapittel~6. Tabell~\ref{tab:kap04-immersiv-rubrikk} oppsummerer kriteriene som brukes i vurderingen.

\begin{table}[htbp]
    \centering
    \begin{tabular}{p{0.26\textwidth}p{0.34\textwidth}p{0.30\textwidth}}
        \toprule
        \textbf{Kriterium} & \textbf{Beskrivelse} & \textbf{Avansert nivå}\
        \midrule
        Modellforankring & Alle tiltak begrunnes med data fra tvillingen, og forutsetninger dokumenteres i beslutningsloggen. & Hypoteser testes mot minst to scenarier og knyttes til konkrete usikkerheter. \\
        Samhandling og kommunikasjon & Roller, ansvar og overleveringer synliggjøres i AR/VR-flaten og i loggen. & Teamet demonstrerer parallelle arbeidsstrømmer og eksplisitt håndtering av konflikter. \\
        Universell utforming & Visualiseringene følger etablerte retningslinjer for kontrast, språk og navigasjon. & Løsningen inkluderer alternative presentasjoner (tekst/lyd) og tilrettelegging for fjernbrukere. \\
        Risikohåndtering & Tiltak evalueres mot sannsynlighet/konsekvens og beredskapsplaner. & Gruppen identifiserer sekundæreffekter og anbefaler forbedringer til prosedyrer og sensornett. \\
        Læringsrefleksjon & Teamet beskriver læringsutbytte, forbedringsforslag og kobling til teori. & Refleksjonen sammenligner AR/VR-løsningen med tradisjonelle kontrollrom og foreslår videre eksperimenter. \\
        \bottomrule
    \end{tabular}
    \caption{Vurderingskriterier for laboratorieøvelsen i immersivt beslutningsrom.}
    \label{tab:kap04-immersiv-rubrikk}
\end{table}

Rubrikken er harmonisert med vurderingsopplegget i lærerveiledningen og kan brukes både til egenvurdering og faglig sensur. Etter gjennomført øving anbefales det å sammenligne resultatene med måltallene i Kapittel~7 for governance og Kapittel~8 for sektorspesifikke KPI-er.

\section{Refleksjonsspørsmål og øvinger}
\begin{enumerate}
    \item Beskriv når du ville velge agentbasert simulering fremfor kontinuerlige modeller.
    \item Lag en enkel plan for å automatisere en simuleringsstudie med Git og CI/CD.
    \item Diskuter hvordan visualisering kan forbedre beslutningsprosesser.
    \item Bruk praksiseksempelet som inspirasjon og skisser hvordan en digital tvilling kan forbedre en annen norsk energitjeneste.
    \item Foreslå en vurderingsrubrikk for en AR/VR-basert øving i eget fagmiljø og begrunn hvilke kriterier som bør vektlegges.
\end{enumerate}
