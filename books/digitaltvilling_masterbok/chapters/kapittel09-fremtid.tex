\chapter{Fremtidstrender og forskning}

\section{Læringsmål}
\begin{itemize}
    \item Identifisere forskningsfronten innen digitale tvillinger.
    \item Diskutere fremtidige teknologier og samfunnsmessige konsekvenser.
    \item Formulere forskningsspørsmål og prosjektidéer.
\end{itemize}

\section{Forskningslandskapet}
\begin{itemize}
    \item Sentrale internasjonale initiativer og forskningsprogrammer.
    \item Norske satsinger: SFI, Forskningsrådet, EU-prosjekter.
    \item Forskningsmetoder og samarbeid mellom akademia og industri.
\end{itemize}

\section{Teknologitrender}
\begin{itemize}
    \item Semantiske tvillinger og autonomi.
    \item Sammensmelting med metaverset, utvidet virkelighet og digitale plattformer.
    \item Bærekraft, grønn omstilling og regulatoriske krav.
\end{itemize}

\section{Fra idé til prosjekt}
\begin{itemize}
    \item Hvordan identifisere forskningshull og formulere hypoteser.
    \item Valg av metodikk: eksperimenter, case, mixed-methods.
    \item Publiseringsstrategi og nettverksbygging.
\end{itemize}

\section{Refleksjonsspørsmål og øvinger}
\begin{enumerate}
    \item Formuler tre forskningsspørsmål knyttet til digitale tvillinger i din sektor.
    \item Beskriv hvordan en fremtidig teknologi kan påvirke digitale tvillinger.
    \item Lag en disposisjon for et masterprosjekt basert på boken.
\end{enumerate}
