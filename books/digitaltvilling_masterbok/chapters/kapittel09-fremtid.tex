\chapter{Fremtidstrender og forskning}

\section{Læringsmål}
\begin{itemize}
    \item Identifisere forskningsfronten innen digitale tvillinger.
    \item Diskutere fremtidige teknologier og samfunnsmessige konsekvenser.
    \item Formulere forskningsspørsmål og prosjektidéer.
\end{itemize}

\section{Forskningslandskapet}
Digitale tvillinger har etablert seg som et tverrfaglig forskningsfelt der systemteknikk, informatikk og domeneekspertise møtes. De mest synlige trendene springer ut av store internasjonale programmer som kobler akademia, leverandørindustri og offentlige aktører, samtidig som nasjonale prioriteringer legger føringer for hvilke problemstillinger som løftes frem i Norge.

\subsection{Internasjonale drivere}
EU har gjort digitale tvillinger til et kjerneelement i \emph{Horizon Europe} gjennom klynger som fokuserer på industriell digitalisering, energi og helseteknologi. Programmer som \emph{Destination Earth} og Gaia-X arbeider med felles dataplattformer og modeller for kritisk infrastruktur, og setter standarder som også påvirker norske miljøer. På forskningssiden markerer IEEE, ISO og Plattform Industrie 4.0 seg med veikart for interoperabilitet, semantikk og sikkerhet. Det gir en rik portefølje av referansemodeller, åpne datasett og calls for proposals som masterstudenter kan bruke som utgangspunkt for problemstillinger.

\subsection{Norske satsinger}
Forskningsrådet finansierer flere store miljøer, blant annet SFI Manufacturing, SFI Autoship og FME Neutron som alle utforsker digitale tvillinger i produksjon, maritime operasjoner og energi. Disse prosjektene kombinerer laboratorier, pilotanlegg og partnerskap med selskaper som Equinor, Kongsberg Gruppen, Elkem og offentlige etater. I tillegg gir regionale initiativ som katapultsentrene og Grønn plattform-prosjekter tilgang til testfasiliteter for masterstudenter. Norske kommuner og helseregioner stiller etter hvert krav om digitale tvillinger i anskaffelser, noe som åpner dører for anvendt forskning på mobilitet, helse og byggforvaltning.

\subsection{Samarbeidsformer og metoder}
Feltet domineres av tverrfaglige metoder som kobler modellering, dataanalyse og brukerinnsikt. Levende laboratorier (living labs) og sandkasser gjør det mulig å eksperimentere med datatilgang, mens dataspaces og standardiserte API-er legger til rette for datadeling på tvers av virksomheter. Publikasjoner i tidsskrifter som \emph{Computers in Industry}, \emph{Advanced Engineering Informatics} og \emph{IEEE Access} viser en utvikling fra konseptstudier til empiriske evalueringer av driftseffekter og bærekraftsgevinster. En masterstudent bør derfor kombinere kvantitative indikatorer (for eksempel energi, kvalitet eller risiko) med kvalitative intervjuer som belyser organisering og beslutninger.

\section{Teknologitrender}
Teknologisk utvikling setter nye rammer for hva digitale tvillinger kan løse og hvordan de driftes. Det skjer en sammensmelting mellom simuleringsverdenen, sanntidsdata og intelligent beslutningsstøtte, noe som øker autonomien i både industrielle og samfunnsrettede systemer.

\subsection{Generativ AI og beslutningsassistenter}
Foundation-modeller og multimodale generative teknikker brukes nå til å foreslå designalternativer, generere syntetiske datasett og veilede operatører i komplekse beslutninger. Industripartnere eksperimenterer med copiloter som kombinerer naturlig språk, visualisering og historiske tvillingdata for å foreslå tiltak i sanntid.\citep{siemens2023copilot} Den europeiske AI-handlingsplanen fremhever behovet for ansvarlige, dokumenterte modeller, noe som innebærer at tvillinger må ha forklaringslag og revisjonsspor før generativ AI kan tas i bruk i regulerte bransjer.\citep{eu2023ai}

\subsection{Edge-native tvillinger}
For prosesser som krever lav responstid kombineres digitale tvillinger med edge-komponenter der databehandling skjer tett på utstyret. Standarder for multi-access edge computing gjør det mulig å orkestrere containere og AI-modeller i fabrikkhaller, på fartøy eller i landbruksmaskiner, samtidig som policyer for dataspace-tilgang ivaretas.\citep{etsi2023mec} Edge-native mønstre utfordrer tradisjonell skyarkitektur og krever at DevOps- og MLOps-prosesser inkluderer distribuerte oppdateringer, overvåking og fallback-rutiner.

\subsection{Semantiske tvillinger og autonomi}
De mest modne tvillingene bygger nå på semantiske modeller der kunnskapsgrafer og ontologier beskriver forholdet mellom komponenter, datastrømmer og forretningsregler. Kombinert med maskinlæring, probabilistiske modeller og formelle verifikasjonsmetoder gir dette grunnlag for beslutninger som kan automatiseres. Autonome tvillinger brukes til å orkestrere vedlikehold, balansere energinett og styre logistikksystemer, men krever transparens og sporbarhet slik at operatører kan forstå og overstyre kritiske valg.

\subsection{Immersive arbeidsflater og metaverset}
Utvidet virkelighet, digitale arbeidsrom og metaverse-plattformer gjør det mulig å visualisere komplekse systemer for operatører og borgere. Kombinasjonen av 3D-modeller, sanntidsstrømmer og samarbeidsverktøy lar tverrfaglige team samhandle om scenarioanalyse, opplæring og fjernoperasjon. Forskningen fokuserer på hvordan slike grensesnitt kan gi situasjonsforståelse uten å overbelaste brukeren, og hvordan sikkerhet og personvern ivaretas når data distribueres til mange enheter.

\subsection{Bærekraft, dataspace og regulering}
Digitaliseringsstrategier forventes å dokumentere klima- og ressursgevinster, og digitale tvillinger brukes som beslutningsgrunnlag for grønn omstilling. Forskningen undersøker hvordan tvillingene kan kobles til EU-taksonomien, rapportere Scope 1–3 utslipp og støtte sirkulærøkonomiske modeller. Dataspace-initiativer i transport og energi legger føringer for hvordan dataprodukter beskrives, lisensieres og revideres, og gjør det enklere å dele gevinster på tvers av aktører.\citep{ec2023mobilitydataspace,idsa2023ram} Dette stiller krav til datakvalitet, revisjonsspor og etterlevelse av regelverk som NIS2 og personvernforordningen, samtidig som det åpner for nye tjenester knyttet til sertifisering og revisjon.

\section{Fra idé til prosjekt}
Å utvikle en masteroppgave eller et forskningsprosjekt innen digitale tvillinger innebærer å knytte tekniske muligheter til et meningsfullt problem i en valgt sektor. Prosessen starter med å forstå eksisterende praksis, avdekke kunnskapshull og sette mål som kan måles.

\subsection{Identifisere forskningshull}
Begynn med en systematisk litteraturgjennomgang i databaser som Scopus og Oria, og kartlegg hvilke prosesser, teknologier eller brukergrupper som er lite omtalt. Sammenlign funnene med behov fra industripartnere eller offentlige virksomheter. Gap analyseres ofte i grensesnittet mellom teknologi og organisasjon: Hvordan påvirker nye datakilder beslutninger, hvilke barrierer finnes for deling, og hvilke usikkerheter må håndteres?

\subsection{Metodevalg og datainnsamling}
Valg av metode bør speile både teknologiens modenhet og tilgjengelige ressurser. Eksperimentelle design kan brukes når man har tilgang til laboratorier eller simuleringsmiljøer, mens casestudier gir verdi når feltdata må samles i samarbeid med virksomheter. Mixed-methods med både kvantitative målinger og kvalitative intervjuer blir stadig vanligere for å dokumentere effekter og endringsledelse. Masterstudenter bør tidlig avklare datatilgang, lisensbehov og etiske avklaringer (for eksempel REK eller NSD) slik at prosjektet blir gjennomførbart.

\subsection*{Indikatorer for forskningsfremdrift}
Fagfelleløp og masterprosjekter kan bruke indikatorene i Tabell~\ref{tab:forskningsindikatorer} for å holde fremdrift og datasamarbeid under kontroll. Indikatorene er hentet fra porteføljestyring i norsk forskningsfinansiering og kobler resultater til samarbeid i dataspace.

\begin{table}[h]
    \centering
    \caption{Forslag til KPI-er for forsknings- og innovasjonsprosjekter}
    \label{tab:forskningsindikatorer}
    \begin{tabular}{p{2.6cm}p{3.8cm}p{3.8cm}p{3.2cm}}
        \toprule
        Fase & KPI & Beskrivelse & Datakilde \\
        \midrule
        Problemforståelse & Andel kartlagte datasett og partnere & Deling av metadatakort i dataspace-katalog & Prosjektlogg, dataspace-dashboard \\
        Metodeutvikling & Andel eksperimenter med reproduserbare skript & Kode- og modellpakker publisert med versjonering & Git-repositorier, laboratorienotat \\
        Evaluering & Dokumenterte effekter (for eksempel energi, kvalitet, sikkerhet) & Sammenligning av før/etter-KPI-er og fagfelleuttalelser & Evalueringsrapporter, gevinstplan \\
        Formidling & Antall åpne dataprodukter eller artikler & Publisering i kanaler definert av finansieringsprogram & Publiseringsdatabase, dataspace-katalog \\
        Videreføring & Finansiert oppskalering eller pilot & Nye kontrakter, videre forskningsmidler & Prosjektregnskap, Forskningsrådets porteføljerapporter \\
        \bottomrule
    \end{tabular}
\end{table}

Indikatorene støtter kravene i porteføljeanalysen for digitalisering og muliggjør datadeling gjennom standardiserte policyer og kontrakter.\citep{rcn2024digitalisering}

\subsection{Publisering og nettverk}
Planlegg formidling parallelt med forskningen. Målrett konferanser som \emph{Model-Based Systems Engineering} (MBSE), \emph{Simulation-Based Digital Twins} eller norske arenaer som Smart Industri og Tekna-fagkvelder. Delta i faglige nettverk som DigSam, Data Space Norway og relevante forskerskoler. Aktiv deltakelse gir tilgang til datasett, veiledere og potensielle medforfattere, og øker sjansen for at prosjektet bidrar til pågående initiativer.

\subsection{Forslag til masterprosjekter}
\begin{enumerate}
    \item \textbf{Energinett i omstilling:} Modellere en digital tvilling for lokal energideling i en bydel, med fokus på å koble simulering, sanntidsdata og regulatoriske krav til fleksibilitet.
    \item \textbf{Autonom maritim logistikk:} Utforske hvordan semantiske tvillinger kan støtte beslutninger på autonome fartøy i norske havner gjennom samarbeid med maritime testarenaer.
    \item \textbf{Helse og velferdsteknologi:} Utvikle en prototype for digital tvilling av pasientforløp der personvern og datadeling håndteres via tillitstjenester og differensiell personvern.
    \item \textbf{Sirkulær industri:} Analysere hvordan digitale tvillinger kan følge materialstrømmer og karbonavtrykk for en produsent, og dokumentere effekter på rapportering etter EU-taksonomien.
    \item \textbf{Kommunal mobilitet:} Sammenligne scenarier for autonome busser ved å koble trafikkdata, simulering av reisevaner og brukerinvolvering for å evaluere samfunnsvirkninger.
\end{enumerate}

\section{Refleksjonsspørsmål og øvinger}
\begin{enumerate}
    \item Formuler tre forskningsspørsmål for en valgt sektor der du spesifiserer hvilke data, metoder og samarbeidspartnere som er nødvendige for å besvare dem.
    \item Beskriv hvordan en av teknologitrendene kan endre arbeidsprosesser, forretningsmodell og regulatoriske krav i en virksomhet du kjenner.
    \item Lag en disposisjon for et masterprosjekt som inkluderer problemstilling, metodevalg, plan for datainnsamling og strategi for publisering.
\end{enumerate}
