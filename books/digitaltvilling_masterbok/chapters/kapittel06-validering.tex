\chapter{Validering, verifikasjon og tillit}


\section{Læringsmål}
\begin{itemize}
    \item Etablere prosesser for validering og verifikasjon av digitale tvillinger.
    \item Analysere usikkerhet og risiko knyttet til modellresultater.
    \item Kommunisere tillit og transparens til interessenter.
\end{itemize}

\section{Valideringsrammeverk}
Validering og verifikasjon er to komplementære kontrollsløyfer som må inngå i samme kvalitetsledelsessystem. \textit{Verifikasjon} handler om å teste at den digitale tvillingen er riktig implementert i henhold til spesifikasjoner, for eksempel ved å sjekke at koden følger modellgrunnlaget og at grensesnittene mot sensorer er konsistente. \textit{Validering} handler om å demonstrere at tvillingen er egnet til formålet den skal støtte, ved å sammenligne modellens resultater med observasjoner, operatørkunnskap og forretningsmessige krav. I praksis bør disse aktivitetene planlegges i et V\&V-program med tydelige akseptkriterier, roller og beslutningspunkter.

Flere etablerte standarder kan brukes som rammeverk. ISO~9001 gir føringer for styring av kvalitetsprosesser og dokumentasjon, mens DNV-RP-A204 og ASME V\&V 40 beskriver hvordan digitale tvillinger og prediktive modeller kan kvalifiseres i sikkerhetskritiske domener. Offentlige aktører som Avinor og Statnett krever at leverandører følger slike prinsipper når nye tvillinger skal tas i bruk. Prosjekter med stor usikkerhet kan også hente metodikk fra NASA sin modellmodenhetsskala, der testplaner og datakvalitetskrav utvikles trinnvis i takt med at tvillingen utvides.

Validering er ikke en engangsaktivitet, men en kontinuerlig praksis gjennom tvillingens livsløp. Hver gang modellen får nye data, algoritmer eller styringsparametere må baseline-modeller, referansescenarier og feltmålinger brukes til å bekrefte at presisjonen er innenfor toleranse. Dette krever en styrt datastrøm inn i DevOps- og MLOps-prosessene, versjonskontroll av modellartefakter, og periodiske revisjoner der både tekniske og domeneeksperter signerer på at tvillingen fortsatt er egnet.

\section{Sikkerhetsstandarder og regulatorisk etterlevelse}
Digitale tvillinger som påvirker kritisk infrastruktur må etterleve etablerte sikkerhetsstandarder. \citet{iec62443-2-1} gir rammeverk for å dele systemet inn i soner og konduiter med definerte sikkerhetsnivåer, slik at både IT- og OT-komponenter får tilpasset beskyttelse. For organisatoriske prosesser supplerer \citet{iso27001-2022} med krav til risikostyring, dokumentasjon og kontinuerlig forbedring av informasjonssikkerhet. I tillegg stiller \citet{eu2022nis2} krav til hendelsesrapportering, styring og tilsyn, noe som innebærer tydelige roller og oppfølging i hele verdikjeden. Prosjekter i energisektoren må samtidig dokumentere samsvar med nasjonale beredskapsforskrifter og kontrollprogram, slik Energi Norge beskriver for nettselskapers operative kontrolltårn.\citep{energinorge2023beredskap}

Et praktisk grep er å knytte hvert modell- og datagrensesnitt til en kontroll i standardene. Tiltak som tofaktor-autentisering, segmentering og revisjonslogger må beskrives både teknisk og organisatorisk. Når nye funksjoner lanseres, bør et standardkart vise hvilke krav som berøres og hvem som eier tiltakene. Tabell~\ref{tab:standardkart} viser et utdrag av en slik oversikt for energibransjen. Støttenotatet \textit{kap06-sikkerhetscase.md} gir en caseoppgave der studentene kartlegger tiltak for et regionalt kraftselskap.

\section{Hendelsesrespons og kvalitetsjournal i dataspace}
Når en digital tvilling inngår i et dataspace for kritisk infrastruktur må hendelsesrespons og kvalitetsjournal styres som én sammenhengende prosess. De nasjonale føringene fra \citet{nsm2023grunnprinsipper} legger til grunn at hendelser håndteres etter prinsippene for forebygging, avdekking, håndtering og læring. Samtidig krever både NIS2-implementeringen og DSB sin nasjonale digitaliseringsstrategi for samfunnssikkerhet at virksomheter kan dokumentere beslutninger, responstid og eskalering i tverrsektorielle situasjoner.\citep{dsb2024nser} For helsesektoren må loggføringen harmoniseres med Helsedirektoratets krav til beredskapsjournal og varsling mot regionale helseforetak.\citep{helsedir2023beredskap}

En robust prosess bør forankres i styringssystemet fra Kapittel~7 og dataforvaltningsmodellen i Kapittel~3. Følgende fire aktiviteter binder hendelsesresponsen til kvalitetsjournalen:
\begin{enumerate}
    \item \textbf{Felles situasjonsbilde:} Dataspace-operatøren setter opp en hendelseskanal der overvåkingsdata, modellalarmer og feltobservasjoner samles i sanntid. Beslutningsgrunnlaget lagres i journalen slik at revisjoner kan se hvilke indikatorer som utløste tiltak.
    \item \textbf{Koordinert responstrinn:} Beredskapsleder, tvillingforvalter og leverandør følger en forhåndsdefinert responsplan med varsling til myndigheter innen tidsfrister fra \citet{eu2022nis2}. Tiltak logges som oppgaver med ansvar og tidsstempler.
    \item \textbf{Læring og forbedring:} Etter hendelsen gjennomfører teamet en strukturert debrief der journaldata kobles til tiltakslogg og fagfellelogg. Metodikken bygger på \citet{dsb2023ovelser} og sørger for at læringspunkter blir synlige i påfølgende øvelser.
    \item \textbf{Rapportering til dataspace-partnere:} Resultatet deles via dataspace-kontraktene slik at andre deltakere kan oppdatere sine kontrolltiltak, i tråd med kravene til åpenhet i \citet{nho2023datadeling} og datadelingsprinsippene fra \citet{dssc2023skills}.
\end{enumerate}

Tabell~\ref{tab:hendelsesroller} viser hvordan roller og responstid kan organiseres på tvers av kritiske sektorer. Den bygger på erfaringer fra energibransjen, kommunal beredskap og helsetjenesten, og gjør det enkelt å gjenbruke strukturen i caseoppgaver og øvelser.

\begin{table}[ht]
    \centering
    \caption{Roller og responstid i hendelsesrespons for dataspace-tilknyttede tvillinger}
    \label{tab:hendelsesroller}
    \begin{tabular}{|p{3.4cm}|p{4.4cm}|p{4.6cm}|p{3.0cm}|}
        \hline
        \textbf{Rolle} & \textbf{Hovedansvar} & \textbf{Journalføring og rapportering} & \textbf{Krav til responstid} \\
        \hline
        Beredskapsleder (kommune eller energiselskap) & Koordinere tverrsektoriell situasjonsforståelse, aktivere kriseledelse og beslutte tiltak & Oppdatere tiltakslogg, ROS-plan og kommunal beredskapsjournal \citep{dsb2022beredskap,dsb2023ovelser} & Varsle statsforvalter innen 2 timer og myndigheter innen 24 timer ved alvorlige hendelser \citep{dsb2024nser} \\
        \hline
        Tvillingforvalter / produktleder & Sikre modellstabilitet, fryse versjoner og koordinere tekniske korrigeringer med leverandører & Dokumentere parameterendringer og modellpålitelighet i kvalitetsjournal og MLOps-logg \citep{iso10007-2017,iso20000-1-2018} & Gjennomføre midlertidige sikringstiltak innen 60 minutter og validere normaltilstand før driftssetting \\
        \hline
        Dataspace-operatør / data steward & Overvåke delte datastrømmer, stenge kompromitterte API-er og informere partnere & Registrere datakontraktsbrudd, varsle gjennom dataspace-portalen og arkivere beslutninger \citep{dataspaces2023skills,nhn2024dataspace} & Aktivere hendelsesprosedyre innen 15 minutter etter alarm og sende statusoppdateringer hver time \\
        \hline
        Personvernombud / helsefaglig ansvarlig & Vurdere konsekvenser for personvern og pasientsikkerhet, koordinere med helsemyndigheter & Oppdatere DPIA, journalføre vurderinger i henhold til helsemessig beredskap \citep{helsedir2023beredskap,helseplattformen2023kontinuitet} & Varsle Datatilsynet innen 72 timer og berørte pasienter uten ugrunnet opphold \citep{eu2022nis2} \\
        \hline
    \end{tabular}
\end{table}

For masterstudentene kan tabellen brukes som sjekkliste når de utvikler hendelsesscenarier i laboratoriet. Kombineres oversikten med triageprosessen tidligere i kapitlet, får studentene trening i å dokumentere både tekniske og organisatoriske beslutninger på en måte som tåler revisjon fra eksterne fagfeller.

\begin{table}[ht]
    \centering
    \caption{Eksempel på standardkart for kontrolltårn i energisektoren}
    \label{tab:standardkart}
    \begin{tabular}{|p{3.2cm}|p{4.6cm}|p{4.6cm}|p{3.2cm}|}
        \hline
        \textbf{Funksjon} & \textbf{Kjernekrav} & \textbf{Tiltak og dokumentasjon} & \textbf{Ansvarlig rolle} \\
        \hline
        Dataoppsamling fra SCADA & IEC~62443-3-3 SR 3, NIS2 art. 21 & Segmenterte konduiter, logging av tjenestekontoer, endringsjournal & OT-sikkerhetsarkitekt \\
        \hline
        Modelloppdatering i tvillingplattform & ISO~27001 A.12, DNV-RP-A204 steg 3 & Godkjenningsprotokoll, versjonskontroll, testlogg for modellendringer & Produktleder digital tvilling \\
        \hline
        Hendelsesrespons & NIS2 art. 23, IEC~62443-2-1 kap. 5 & Alarmgrenser, responsskript, rapport til myndigheter innen 24 timer & Beredskapsleder \\
        \hline
        Visualisering i kontrolltårn & Energi Norge veileder kap. 4, ISO~27001 A.13 & Rollebasert tilgang, sanntidsdashboard med audit trail, sporbarhet til datakilder & Kontrolltårnleder \\
        \hline
    \end{tabular}
\end{table}

\subsection{Kontrolltårn-case for regionalt kraftsystem}
Flere norske nettselskaper etablerer kontrolltårn for å kombinere feltdata, tvillingmodeller og beslutningsstøtte i én operativ arena. \citet{statnett2024kontrolltarn} beskriver hvordan overvåkingssenteret koordinerer sanntidssignaler fra produksjon, nett og værdata for å planlegge vedlikehold og håndtere avvik. For å overføre praksisen til undervisningen anbefales følgende struktur:
\begin{enumerate}
    \item \textbf{Kartlegg datastrømmer:} Studentene tegner et dataflytsdiagram fra sensorer via SCADA og datasjø til tvillingplattformen, og identifiserer hvor tilgangskontroll og logging må styrkes.
    \item \textbf{Definer kontrollpaneler:} Gruppen designer KPI-er for forsyningssikkerhet, tap, beredskap og bærekraft. Hvert panel må kobles til en standardreferanse i tabellen over.
    \item \textbf{Planlegg koordinering:} Scenarioverksteder kjøres med operatører, fagansvarlige og leverandører for å teste hendelseshåndtering, samtidig som roller og varslingsrutiner dokumenteres etter NIS2.
\end{enumerate}
Kontrolltårn-caset forsterker koplingen mellom modellvalidering, cybersikkerhet og organisasjonsdesign. Studentene skal levere både teknisk dokumentasjon og en kort tiltakslogg for oppfølging i styringssystemet.

\section{Indikatorramme for dataspace-valideringslaboratorium}
Dataspace-laboratoriet binder valideringsprosesser fra kapittel\,3, 4 og 7, og trenger derfor en indikatorramme som kombinerer kvalitetskontroller, hendelseshåndtering og gevinstoppfølging. \citet{digdir2024datasamarbeid} anbefaler at dataspace-partnere etablerer felles måltall og responstider før sandkasseaktiviteter igangsettes, mens \citet{idsa2023operational} beskriver hvordan kontrollpunkter må logges for å kunne sertifisere dataspace-komponenter. For å sikre at laben oppfyller norske krav til bærekraftsrapportering bør indikatorene også speile rammeverket i \citet{efrag2023esrs} og styringsprosessene som \citet{dfo2023baerekraft} anbefaler for offentlig sektor.

En helhetlig indikatorramme kan bygges opp i fire arbeidssteg som studentteamet repeterer ved hver iterasjon:
\begin{enumerate}
    \item \textbf{Planlegg måleprogrammet:} Koble dataspace-kontrakter, modeller og kontrolltårn-panelet til et felles målekart med definisjoner, eiere og akseptgrenser. Bruk sjekklistene for modelljournal og gevinstoppfølging for å sikre at både tekniske og organisatoriske effekter spores.\citep{digdir2023modelljournal,digdir2022gevinst}
    \item \textbf{Synkroniser datakvalitet:} Etabler dataprodukter med tydelig beskrivelse av kvalitet, oppdateringsfrekvens og policy, og registrer dem i katalogen slik dataspace-partnere kan verifisere grunnlaget for testene.\citep{dssc2024dataproducts}
    \item \textbf{Test modeller og prosesser:} Kjør modellscenarier, ytre påvirkninger og hendelsesøvelser i sandkassen mens indikatorene logges automatisk til kvalitetsjournalen. Hver test får sin egen versjonslenke slik at revisjoner kan etterprøve resultatet.\citep{idsa2023operational}
    \item \textbf{Analysér effekt og gevinst:} Del målerapporter i dataspace-portalen, oppdater tiltakslogg og gevinstkart, og bestem hvilke tiltak som skal overføres til drift. Kombinerer teamet dette med målstyringen fra kapittel\,7, får de en helhetlig vurdering av både teknisk kvalitet og virksomhetsverdi.\citep{digdir2022gevinst}
\end{enumerate}

Tabell~\ref{tab:dataspaceindikatorer} viser et eksempel på indikatorer som brukes i laben for å demonstrere sammenhengen mellom modellpresisjon, datakvalitet, hendelsesberedskap og bærekraftsrapportering. Tersklene kan justeres for sektor- og casetilpasning, men strukturen gir studentene en mal for å dokumentere kvantitative resultater før pilotering.

\begin{table}[ht]
    \centering
    \caption{Kjerneindikatorer for dataspace-valideringslaboratorium}
    \label{tab:dataspaceindikatorer}
    \begin{tabular}{|p{3.2cm}|p{4.8cm}|p{4.2cm}|p{3.2cm}|}
        \hline
        \textbf{Måling} & \textbf{Datagrunnlag} & \textbf{Terskel og eskalering} & \textbf{Oppfølging} \\
        \hline
        Modellpresisjon mot feltdata & Sammenligning av simuleringer fra kapittel\,4 og sanntidsdata fra dataspace & Avvik \textless{} 5\% over tre iterasjoner; ellers trigges ekstra verifikasjon og sign-off fra fageier & Modelljournal og kontrolltårn-protokoll \\
        \hline
        Dataproduktkvalitet & Profilering av dataprodukter, metadata og policy fra dataspace-katalogen & Minimum 98\% fullstendige metadata; avvik \textgreater{} 2\% varsler data steward og oppdatering av kontrakt & Datakatalog og dataspace-varsel \\
        \hline
        Hendelsesrespons & Tidsstempler fra hendelsesøvelser og kvalitetsjournal & Varsling til partnere innen 15 minutter og full hendelsesrapport innen 24 timer; brudd eskaleres til beredskapsleder & Hendelseslogg og tiltaksplan \\
        \hline
        Bærekrafts- og gevinstindikatorer & KPI-er for energi, klima og tjenestegevinst fra gevinstplanen & Rapportering av ESRS E1-indikatorer hver sprint; manglende tall fører til midlertidig stopp i pilotering & Gevinstlogg og bærekraftsrapport \\
        \hline
    \end{tabular}
\end{table}

Ved å bruke indikatorsettet som mal får masterstudentene trening i å kombinere tekniske målinger med styringskrav. Resultatene kan gjenbrukes i fagfellelogg, tiltaksplan og planfil, og gir samtidig dokumentasjon for å flytte sandkasseaktiviteter inn i ordinær drift når kriteriene er oppfylt.

\section{Kvartalsrapport for valideringspanelet}
Kvalitetsteamet oppsummerer indikatorene i en kvartalsrapport slik at både dataspace-partnere og styringsgrupper ser utviklingen over tid. Rapporten kombinerer tall fra kontrolltårnet, modelljournalen og gevinstplanen, og brukes aktivt når Statsbygg, Bane NOR og Ruter vurderer om pilotene kan skalere videre.\citep{statsbygg2023digitalmodenhet,banenor2024leverandor,ruter2024mobilitetslab} Tallfestingen gjør det enklere for masterstudentene å argumentere for tiltak i tiltaksloggen fra Kapittel~7 og koble funnene til porteføljestyringen.

Arbeidet følger tre gjentakende steg som bygger på indikatorrammen over:
\begin{enumerate}
    \item \textbf{Konsolider data:} Samle eksport fra kontrolltårn, kvalitetsjournal og bærekraftsdashboard før tallene verifiseres mot dataspace-kontraktene.
    \item \textbf{Analysér avvik:} Sammenlign resultatene med akseptgrenser og avklar om avvik skyldes modelloppdateringer, sensorfeil eller organisatoriske endringer.
    \item \textbf{Planlegg tiltak:} Prioriter forbedringstiltak i tiltaksloggen, tildel ansvar og koble til gevinstindikatorer før neste sprint.
\end{enumerate}

Tabell~\ref{tab:kvartalsrapport} viser et eksempel på utdrag fra kvartalsrapporten for første kvartal. Verdiene er syntetiske, men strukturen følger rapporteringsformatet som brukes i pilotene.

\begin{table}[ht]
    \centering
    \caption{Utdrag fra kvartalsrapport for valideringspanelet}
    \label{tab:kvartalsrapport}
    \begin{tabular}{|p{3.2cm}|p{4.4cm}|p{3.2cm}|p{4.2cm}|}
        \hline
        \textbf{Indikator} & \textbf{Datagrunnlag} & \textbf{Resultat Q1 2024} & \textbf{Tiltak og oppfølging} \\
        \hline
        Modellpresisjon Statsbygg ombrukslab & Sammenstilling av simuleringsresultater og målinger fra materialombruk & 96,5\% mot akseptgrense 95\% & Fortsette månedlige verifikasjoner og dele dokumentasjon med eiendomsteamet før neste styringsmøte \citep{statsbygg2022ombruk,statsbygg2023digitalmodenhet} \\
        \hline
        Sensorstabilitet Bane NOR vinterdrift & Driftsovervåking fra jernbanesensorer og kontrolltårn-logg & 12 avvik over 1,5\% av målepunktene (grense 15) & Oppdatere kalibreringsplanen og synkronisere hendelser med kapittel 4 sitt vinterdriftsscenario \citep{banenor2023vinterdrift,banenor2024leverandor} \\
        \hline
        Hendelsesrespons kollektivdataspace & Varslingslogg fra dataspace-portalen og tiltaksjournal & Varslingstid 11 minutter (krav \leq{}15) og full rapport innen 6 timer & Trene responsteamet i sandkasseøvelse og koble læringspunkter til beredskapsjournalen \citep{ruter2023dataplattform,ruter2024mobilitetslab} \\
        \hline
        Bærekraftsgevinster energiportefølje & KPI-er fra gevinstplan og ESRS-rapportering & 4,2\% reduksjon i energibruk mot mål 4,0\% & Eskalere tiltak til porteføljestyret og dele analyse med bærekraftsteamet \citep{efrag2023esrs,dfo2023baerekraft} \\
        \hline
    \end{tabular}
\end{table}

Rapporten blir et felles beslutningsgrunnlag for videre pilotering. Når tallene deles i dataspace-portalen får partnerne innsikt i hvilke tiltak som fungerer, og planfilen kan oppdateres med neste steg for hvert case. I undervisningen bruker studentene formatet til å øve på å formulere forbedringshypoteser, argumentere for investeringer og dokumentere effekter i gevinstplanen.

\section{Avviksanalyse og tiltakssporing i kvalitetsjournalen}
Selv med solide indikatorpakker vil kvartalsrapporten avdekke avvik som må behandles systematisk. Internkontrollforskriften og DFØ sine anbefalinger krever at virksomheter dokumenterer både årsak, konsekvens og korrigerende tiltak før avvik kan lukkes.\citep{dfo2024internkontroll} For dataspace-piloter må analysen knytte modellresultater til kontraktsfestede kvalitetsnivåer slik at både leverandører og offentlige partnere kan vise etterlevelse av styringsmodellen i Kapittel~7.

Prosessen bygger videre på modelljournalen og dataspace-operasjonene som allerede er etablert i laboratoriet. Når måledata og simuleringer registreres med versjonslenker kan hvert avvik kobles til konkrete endringer i modellparametere, hendelser eller datakilder.\citep{digdir2023modelljournal,idsa2023operational} Det gir et spor som dataspace-operatøren kan dele med partnere i tråd med sanntidsdataveilederen og kravene til samstyring fra Digitaliseringsdirektoratet.\citep{digdir2024sanntidsdata,digdir2024samstyring}

En praktisk arbeidsflyt følger fire steg:
\begin{enumerate}
    \item \textbf{Kategoriser avviket:} Klassifiser hendelsen som modellavvik, datakvalitetsavvik, prosessavvik eller bærekraftsavvik og registrer den med referanse til indikatorpanelet.
    \item \textbf{Analyser årsak:} Bruk modelljournalen og kvalitetsjournalen til å spore parameterendringer, datakilder og ansvarlige roller. Samkjør funnene med hendelsesloggen for å se om avviket skyldes hendelser eller planlagte endringer.
    \item \textbf{Planlegg tiltak:} Definer korrigerende tiltak, ansvarlig rolle og frist, og koble tiltaket til gevinstplanen og tiltaksloggen fra Kapittel~7 slik at effekten følges opp.\citep{digdir2022gevinst}
    \item \textbf{Verifiser lukking:} Test modellen eller prosessen på nytt, dokumenter resultatet og godkjenn lukking i valideringspanelet før status endres til lukket i dataspace-portalen.
\end{enumerate}

Tabell~\ref{tab:avviksanalyse} gir et eksempel på hvordan masterstudentene kan tallfeste og følge opp avvik i laboratoriet. Tersklene knyttes til indikatorene i kvartalsrapporten slik at fremdriften kan spores i samme rapporteringspakke.

\begin{table}[ht]
    \centering
    \caption{Avviksmatrise for dataspace-validering}
    \label{tab:avviksanalyse}
    \begin{tabular}{|p{3.4cm}|p{4.4cm}|p{4.0cm}|p{3.6cm}|}
        \hline
        \textbf{Avvikstype} & \textbf{Trigger og målt verdi} & \textbf{Korrigerende tiltak} & \textbf{Sporing og lukking} \\
        \hline
        Modellpresisjon & Avvik \textgreater{} 7\% mellom simulering og feltdata i tre måleserier & Revider kalibrering, oppdater modellparametere og gjennomfør ekstra verifikasjon med fageier & Nye testresultater lastes opp i modelljournalen og signeres av valideringspanelet \\
        \hline
        Datakvalitet & Metadatafullstendighet faller til 94\% for et kritisk dataprodukt & Data steward kompletterer metadata, oppdaterer kontraktskrav og dokumenterer kvalitetssikringsplan & Dataspace-portalen viser lukket status når komplett metadata er verifisert av partnerne \\
        \hline
        Prosess etterlevelse & Varslingstid i hendelsesprosessen overstiger 20 minutter i to øvelser & Oppdatere responsskript, trene teamet og automatisere eskaleringsvarsel i portalen & Hendelsesloggen viser normalisert responstid i neste øvelse og lukkes av beredskapsleder \\
        \hline
        Bærekraft og gevinst & Energireduksjon i pilotporteføljen faller til 3,5\% mot mål 4,0\% & Tiltakslogg prioriterer energieffektivisering, ansvarlig følger opp investeringsplan og rapporterer gevinst & Kvartalsrapporten viser korrigert trend og gevinstansvarlig signerer lukking \citep{dfo2023baerekraft} \\
        \hline
    \end{tabular}
\end{table}

Avviksmatrisen gjør det tydelig hvilke tiltak som gir raskest effekt, og hvordan studenter kan dokumentere forbedringene i både kvalitetsjournal, tiltakslogg og gevinstrapport. Når lukking signeres av valideringspanelet får dataspace-partnerne et revidert beslutningsgrunnlag før de eventuelt skalerer tvillingen videre.

\section{Scenariobasert stresstesting for hydrogenknutepunkt}
Utbygging av hydrogenknutepunkt i Norge kombinerer produksjon, lagring, bunkring og distribusjon til maritim og landbasert transport.\citep{enova2024hydrogenknutepunkt} Digitale tvillinger brukes for å dimensjonere anlegget, koordinere logistikk og dokumentere sikkerhetstiltak, men komplekse verdikjeder gjør at både energiselskaper, nettoperatører og havner må samordne risikobilde og beredskap.\citep{gassco2023hydrogen,dnv2023hydrogenforecast} For å sikre at modellene tåler forstyrrelser må kapittel 4 sitt hydrogenknutepunkt-case suppleres med en strukturert stresstest i valideringsløpet.

En scenariobasert stresstest følger fire gjentakende aktiviteter:
\begin{enumerate}
    \item \textbf{Velg risikoscenarier:} Kombiner forsyningsavbrudd, produksjonsavvik og hendelser i havna for å dekke både tekniske og operasjonelle risikoer. Bruk DNV sitt veikart for å identifisere sannsynlige utviklingstrekk frem mot 2030.\citep{dnv2023hydrogenforecast}
    \item \textbf{Konfigurer modellparametere:} Hent lastprofiler, trykkgrenser og leveranseavtaler fra kapittel 4-modellen og koble dem til indikatorer i kontrolltårn-panelet. Hver simulering skal dokumenteres med parameterfil og referansesporing i kvalitetsjournalen.
    \item \textbf{Planlegg målepunkter:} Definer hvilke KPI-er som må oppdateres i sanntid for å vise effekten av stresset, for eksempel fyllingsgrad i lagertanker, antall sikkerhetsventil-aktiveringer og tidsbruk i logistikkjeden.
    \item \textbf{Evaluér respons og tiltak:} Sammenlign modellresultater med beredskapsplaner fra nettselskap, havn og produsent. Tildel oppfølgingstiltak og risikoaksept i tiltaksloggen slik at erfaringene synkroniseres med leverandør- og styringsprosessene.
\end{enumerate}

Tabell~\ref{tab:hydrogenstress} viser hvordan scenariene kan struktureres for undervisnings- og pilotformål. Oppsettet knytter modellparametere, indikatorer og ansvarlige funksjoner sammen, slik at stresstesten blir et felles planleggingsverktøy for aktørene i verdikjeden.

\begin{table}[ht]
    \centering
    \caption{Scenarioportefølje for stresstesting av hydrogenknutepunkt}
    \label{tab:hydrogenstress}
    \begin{tabular}{|p{3.2cm}|p{4.6cm}|p{4.6cm}|p{3.0cm}|}
        \hline
        \textbf{Scenario} & \textbf{Modellparametere} & \textbf{Nøkkelindikatorer} & \textbf{Primært ansvar} \\
        \hline
        Strømavbrudd i elektrolysør & Reduser tilgjengelig effekt med 60\%, aktiver backup fra nettet og tidsforskyv produksjon & Leveranserate til fyllestasjoner, energikostnad per kg, tidsbruk på sikker nedstenging & Kraftsystemoperatør + produksjonsleder \\
        \hline
        Leveransehopp fra maritim kunde & Doble bunkringsvolum i tre timer, prioriter terminalplass og transportlogistikk & Ventetid for fartøy, lagernivå i buffer, antall samtidige sikkerhetskontroller & Havnedirektør + logistikkkoordinator \\
        \hline
        Sensorfeil i lagersystem & Sett trykksensorer til \pm{}10\% avvik, simuler datatap i 15 minutter & Antall manuelle inspeksjoner, trigger for failsafe-ventiler, responstid i kontrolltårn & Driftstekniker + sikkerhetsingeniør \\
        \hline
        Regional transportstans & Steng hovedvei i fire timer, aktiver alternative distribusjonsruter og ekstra lagring & Beholdning ved kunde, utslippsnivå fra ekstra logistikk, antall eskaleringer til kriseledelse & Logistikksjef + beredskapsleder \\
        \hline
    \end{tabular}
\end{table}

For å gjøre stresstesten praktisk anvendbar bør teamet bruke samme rapporteringspakke som for kontrolltårn-caset: indikatorer skrives til styringspanelet, hendelser loggføres i kvalitetsjournalen og beslutninger refereres til i planfilen. Workshopene kan organiseres i tre faser:
\begin{itemize}
    \item \textbf{Forberedelse:} Fordel rollene fra Tabell~\ref{tab:hydrogenstress} til student- og industrideltakere, og oppdater tiltaksloggen med forventede observasjoner.
    \item \textbf{Gjennomføring:} Kjør simuleringene i kapittel 4-plattformen, og la deltakerne registrere tiltak, avvik og kommunikasjon i sanntid via kontrolltårn-panelet.
    \item \textbf{Etterarbeid:} Evaluér data fra stresstesten, oppdater risikoaksept og beredskapsplaner, og del en kort rapport i fagfellelogg og planfil. Erfaringene skal også mates inn i leverandørkvalifiseringen for å vise hvordan krav håndteres i praksis.
\end{itemize}

Koblingen mellom simulering, indikatorer og styringssystem gjør at hydrogenknutepunkt-caset blir en bro mellom kapittel 4 og 6. Stresstesten viser samtidig hvordan digital tvilling, dataspace og beredskap må spille sammen for å sikre trygg utrulling av hydrogeninfrastruktur i Norge.

\section{Kontinuerlig modellovervåking for kommunale tvillinger}
Kommunale digitale tvillinger for vann, avløp og byplanlegging er avhengige av kontinuerlig overvåking for å oppdage avvik tidlig og koordinere beredskapstiltak. \citet{dsb2022beredskap} fremhever at kommunene må dokumentere helhetlig ROS-analyse, varslingsrutiner og øvingsplaner, mens \citet{nve2022kommunal} anbefaler at overvåkingen knyttes til konkrete hendelsesnivåer for flom og skred. Når en digital tvilling integrerer sensorstrømmer, simuleringer og hendelseslogg, må disse kravene oversettes til automatiserte kontroller slik at driftsorganisasjonen får varsler i tide. Prosjekter som det digitale overvannslaboratoriet utviklet av Asplan Viak viser hvordan 3D-bydata og sanntidssensorer kan brukes til å modellere kapasitet, responstid og effekten av blågrønne tiltak.\citep{asplan2023overvannslab}

\subsection{Overvåkingssløyfe og ansvar}
En kommunal overvåkingssløyfe kan bygges opp med fire gjentagende aktiviteter som involverer både tekniske ressurser og beredskapsledelse:
\begin{enumerate}
    \item \textbf{Innsamling og filtrering:} Sensorer for vannstand, pumpestasjoner og værdata normaliseres i en delt datasjø. Datakatalogen fra Kapittel~3 må utvides med felt for kvalitet, oppdateringshyppighet og ansvarlig etat slik at avvik spores raskt.\citep{nve2022kommunal}
    \item \textbf{Modellkjøring og prediksjon:} Tvillingen genererer prognoser for kapasitetsutnyttelse og risiko for oversvømmelse. Resultatene lagres med modellversjon, parameteroppsett og referansescenario slik at testresultatene kan gjenbrukes i sertifiseringsjournalen.
    \item \textbf{Varsling og tiltak:} Når indikatorer passerer terskler, utløses automatiske varsler til vaktlag, teknisk etat og beredskapsledelse. Tiltakene følger trinnene i kommunens beredskapsplan og loggføres i kontrolltårn-panelet.\citep{dsb2022beredskap}
    \item \textbf{Læring og revisjon:} Etter hendelser gjennomføres mini-revisjoner der datakvalitet, modellpresisjon og tiltak evalueres. Erfaringene brukes til å oppdatere risikologgen og indikatorpanelet i Kapittel~7.
\end{enumerate}

\subsection{Indikatorpanel for vann- og overvannstjenester}
Tabell~\ref{tab:kommunaleindikatorer} viser hvordan en kommune kan sette opp indikatorer som kombinerer modellresultater, feltmålinger og organisatoriske tiltak. Struktur og roller speiler kontrolltårn-caset, men referansene og tersklene er tilpasset kommunale tjenester.

\begin{table}[ht]
    \centering
    \caption{Eksempel på indikatorpanel for kommunal modellovervåking}
    \label{tab:kommunaleindikatorer}
    \begin{tabular}{|p{3.2cm}|p{4.6cm}|p{4.4cm}|p{3.0cm}|}
        \hline
        \textbf{Indikator} & \textbf{Datakilde og modellstøtte} & \textbf{Terskel og tiltak} & \textbf{Ansvarlig} \\
        \hline
        Magasinstatus i overvannsnett & SCADA-nivåer, tvillingprognoser for kapasitet \citep{oslo2023overvann} & Gult nivå ved 80\% utnyttelse; aktiver tømming og fleksible basseng, logg tiltak i beredskapsjournal & Vann- og avløpsetaten \\
        \hline
        Kritiske kjellere og bygg & Kombinasjon av høyde-/flomdata og risikokart fra tvillingen \citep{nve2022kommunal} & Orange nivå når simulert vannstand \textgreater{} 20 cm; utplasser mobile barrierer og informer beboere & Teknisk etat og beredskap \\
        \hline
        Sensorhelse & Datakvalitetsdashbord fra overvannslaboratoriet \citep{asplan2023overvannslab} & Rød status ved mer enn 2 sensorer offline i samme sone; send utrykning og aktiver reservekilder & Driftsovervåking \\
        \hline
        Øvings- og læringsstatus & Hendelseslogg, ROS-plan og fagfellelogg & Oppdateres etter hver øvelse; avvik eskaleres til kommunedirektør og rapporteres i styringspanelet fra Kapittel~7 & Beredskapsleder \\
        \hline
    \end{tabular}
\end{table}

Indikatorene bør visualiseres i samme dashboard som brukes i kontrolltårn-caset, slik at drifts- og beredskapsroller får en samlet situasjonsforståelse. Når terskler nås, skal loggskjemaene kobles til personvern- og sikkerhetskontrollene i Kapittel~5 og Kapittel~7 for å sikre etterlevelse på tvers av sektorer.

\subsection{Månedsrapport for modellovervåking}
For at indikatorene skal føre til kontinuerlig forbedring trenger kommunen en fast rapporteringssløyfe der tekniske funn og beredskapstiltak vurderes samlet. Oslo kommunes overvannsprogram har etablert en felles månedsrapport som kobler modellresultater, feltobservasjoner og planstatus i de samme møtene som drift og byutvikling behandles.\citep{oslo2023overvann} Rapporten må også synliggjøre samhandling med nabokommuner og regionale aktører slik \citet{ks2022samhandling} anbefaler, og dokumentere læringspunkter i tråd med kravene i kommunal beredskapsplikt.\citep{dsb2022beredskap}

En strukturert rapporteringssløyfe kan organiseres slik:
\begin{enumerate}
    \item \textbf{Forbered data}: Fagansvarlige eksporterer indikatorene fra Tabell~\ref{tab:kommunaleindikatorer}, oppdaterer hendelsesloggen og samler modellnotater fra siste simuleringer.
    \item \textbf{Gjennomfør møte}: Drift, beredskap og planleggere samles månedlig for å gjennomgå avvik, foreslå tiltak og validere at beslutningene kan spores i kvalitetsjournalen.
    \item \textbf{Fang læring}: Tiltak og beslutninger registreres i tiltakslogg og fagfellelogg, og kobles til oppdateringer i kapittel 7 sitt styringspanel og kapittel 3 sin dataspace-policy.
\end{enumerate}

Tabell~\ref{tab:manedsrapport} viser en anbefalt mal som binder indikatorpanelet til styringssystemet.

\begin{table}[ht]
    \centering
    \caption{Forslag til månedsrapport for kommunal modellovervåking}
    \label{tab:manedsrapport}
    \begin{tabular}{|p{3.1cm}|p{4.8cm}|p{4.6cm}|p{3.0cm}|}
        \hline
        \textbf{Del} & \textbf{Innhold} & \textbf{Datasett og kilder} & \textbf{Ansvarlig} \\
        \hline
        Drift og beredskap & Oppsummering av hendelser, terskelbrudd og responstid, koblet til tiltak i beredskapsplanen & Indikatorpanel, hendelseslogg, ROS-plan \citep{dsb2022beredskap} & Beredskapsleder + driftsansvarlig \\
        \hline
        Modellkvalitet & Analyse av prognoseavvik, modellversjoner og planlagte endringer i simuleringsoppsett & Modelljournal, parameterendringer, kvalitetsjournal & Tvillingforvalter \\
        \hline
        Datasamarbeid & Status for dataspace-avtaler, delte datasett og varsling mot nabokommuner og sektorpartnere & Dataspace-policy, delingsavtaler, samhandlingslogg \citep{ks2022samhandling} & Data steward \\
        \hline
        Bærekraft og plan & Effekt på klima- og overvannsmål, prioriterte investeringsprosjekter og kobling til planprosesser & Klimaregnskap, prosjektportefølje, indikatorer fra Kapittel~7 & Byutviklingssjef \\
        \hline
    \end{tabular}
\end{table}

Rapportstrukturen gjør det enklere å følge opp læringspunkter etter øvelser og hendelser. Når kommunen gjenbruker malen i fagfellemøter og pilotering med partnere, skapes en sammenhengende historikk som styrker kontinuerlig forbedring og gjør det enklere å dele innsikt på tvers av kapitlene.

\subsection{Kobling til læringsaktiviteter}
For masterstudentene gir overvåkingssløyfen et rammeverk for praktiske oppgaver:
\begin{itemize}
    \item \textbf{Workshop i digitalt laboratorieløp:} Studentene bruker datasett fra kommunale partnere til å validere indikatorene og tester hvordan kvalitetsjournalen fanger opp avvik mellom modell og feltobservasjoner.
    \item \textbf{Beredskapsøvelse:} En simulert ekstremnedbørshendelse kjøres i tvillingen. Gruppene dokumenterer beslutninger, varslingsløp og tiltak i journalen, og kobler dem til kravene fra \citet{dsb2022beredskap}.
    \item \textbf{Rapportering og refleksjon:} Leveransene inkluderer en kort rapport som viser hvordan indikatorpanelet støtter kommunens mål for klimaadaptasjon og bærekraft, i tråd med \citet{oslo2023overvann}.
\end{itemize}

Koblingen mellom kontinuerlig overvåking, beredskapsplaner og læringsaktiviteter gjør at studentene opplever hvordan en kommunal tvilling må forvaltes i praksis. Samtidig styrkes samspillet mellom Kapittel~3 (datastyring), Kapittel~4 (simulering) og Kapittel~7 (governance) fordi indikatorene deles på tvers av arenaer.

\section{Valideringsløp for byggdrift-sandkasse}
Kommunale eiendomsavdelinger har etablert dataspace-sandkasser for å teste integrasjoner mellom energiledelse, inneklimakontroll og vedlikeholdslogg før løsninger slippes inn i produksjonsmiljøet.\citep{osloeiendom2023strategi,bergen2024smartbygg} Sandkassen i Kapittel~3 beskriver prosessene for å hente data fra skole- og helsebygg, men for å gå fra eksperimentering til ordinær drift trengs et eksplisitt valideringsløp. Dette løftet binder sensordata, modellprediksjoner og styringsindikatorer til kvalitetsjournalen i dette kapittelet og gevinststyringen i Kapittel~7, slik at kommunedirektør, eiendomssjef og leverandører har en felles sannhet om status.

\subsection{Testtrinn og indikatorer}
Valideringsløpet må dekke både tekniske integrasjoner og organisatoriske beslutninger. Erfaringer fra Oslo, Bergen og Statsbygg viser at kommunene lykkes når hvert trinn avsluttes med måledata og styringssignaler som kan spores i dataspace-katalogen og planverkene.\citep{osloeiendom2023strategi,oslo2024klimaeiendom,statsbygg2023digitalmodenhet}

\begin{table}[ht]
    \centering
    \caption{Valideringsløp for dataspace-sandkasse i kommunal byggdrift}
    \label{tab:byggdrift-validering}
    \begin{tabular}{|p{3.2cm}|p{4.6cm}|p{4.6cm}|p{3.0cm}|}
        \hline
        \textbf{Fase} & \textbf{Hovedmål} & \textbf{Tester og datagrunnlag} & \textbf{Ansvarlig} \\
        \hline
        Sandkasseoppstart & Prioritere bygg, datasett og gevinster i tråd med klima- og energiambisjoner & Sammenligne bygningsportefølje mot energimerker, importere sanntidsdata til sandkasse, etablere datapolicy med personvernnotat \citep{osloeiendom2023strategi,ks2024eiendomsdrift} & Eiendomssjef + data steward \\
        \hline
        Integrasjonstest & Sikre at BMS, IoT-sensorer og vedlikeholdssystem leverer konsistente strømmer & Automatisk validering av tidsstempel og måleenheter, syntetiske avvikstester, oppdatering av modelljournal \citep{bergen2024smartbygg} & Teknisk integrasjonsteam \\
        \hline
        Pilotdrift & Dokumentere effekten på energi, inneklima og brukeropplevelse i utvalgte bygg & Sammenligne modellprediksjoner mot energiregister, CO$_2$-målinger og brukerklager; rapportere avvik i kvalitetsjournal og tiltakslogg \citep{oslo2024klimaeiendom} & Driftssjef + HMS-koordinator \\
        \hline
        Overlevering til ordinær drift & Forankre gevinstoppfølging og leverandøransvar i styringssystemet & Revisjon av SLA og tjenestekontrakter, måling av sparte kWh og utslipp, kobling til porteføljestyring i Kapittel~7 \citep{statsbygg2023digitalmodenhet} & Programkontor + innkjøp \\
        \hline
    \end{tabular}
\end{table}

Tabellen bygger videre på arbeidsflyten i Kapittel~3 og gjør det tydelig hvilke datapakker som må godkjennes før nye bygg kobles på dataspace. Når fasene loggføres i kvalitetsjournalen får kommunen et revisjonsspor som kan brukes både overfor Riksrevisjonen og interne kontrollutvalg.

\subsection{Rapporteringssløyfe og læringsarena}
Valideringsløpet bør støttes av en fast rapporteringsstruktur som kobler drift, økonomi og bærekraft:
\begin{enumerate}
    \item \textbf{Forankre i porteføljestyret:} Presentér status på energimål, investeringsbehov og brukeropplevelse hver tertial, og knytt tiltak til gevinstplanen fra Kapittel~7.\citep{ks2024eiendomsdrift}
    \item \textbf{Integrer med kommunens klimaregnskap:} Eksporter måledata fra sandkassen til klimaregnskapet slik at effekten av tiltak rapporteres til byråd og kommunestyre.\citep{oslo2024klimaeiendom}
    \item \textbf{Lukk avvik i leverandørkjeden:} Bruk kvalitetsjournalen til å spore hvordan leverandører oppdaterer dokumentasjon, sensorkalibrering og vedlikeholdsplaner etter hvert kvartalsmøte.\citep{statsbygg2023digitalmodenhet}
\end{enumerate}

\subsection{Læringsaktiviteter}
For masterstudentene gir byggdrift-sandkassen et konkret case som kombinerer dataintegrasjon, modelltesting og styring:
\begin{itemize}
    \item \textbf{Data- og indikatorverksted:} Studentene kobler datasett fra Kapittel~3 til tabell~\ref{tab:byggdrift-validering}, beregner indikatorer for energi- og inneklimamål og dokumenterer avvik i kvalitetsjournalen.
    \item \textbf{Revisjonssimulering:} Gruppene spiller gjennom en kvartalsvis styringsdialog der de må argumentere for videre utrulling basert på målt effekt, brukerfeedback og leverandørstatus.
    \item \textbf{Rapporteringsleveranse:} Resultatene oppsummeres i et kort notat til kommunedirektør som kobler funnene til klima-, budsjett- og arbeidsmiljømål, og viser hvordan tiltakene følges opp i planfilen.
\end{itemize}

\section{Auditplan for dataspace-valideringspanelet}
Tverrsektorielle piloter i dataspace-laboratoriet trenger en felles auditplan som binder tekniske tester, styringskrav og gevinstoppfølging.\citep{iso19011-2018,dfo2024internkontroll} Planen skal gjøre det mulig å følge revisjonssporet fra datakilder til beslutninger i kontrolltårn-panelet, og vise hvordan tiltakene kobles til porteføljestyringen i Kapittel~7 og dataspace-policyen i Kapittel~3. Digitaliseringsdirektoratet anbefaler at offentlige virksomheter etablerer tverrsektorielle styringsfora der partnere deler indikatorer og kontrollfunn, slik at læringen fra sandkasseprosjekter kan tas inn i ordinær drift.\citep{digdir2024samstyring}

En auditplan for valideringspanelet bør dekke fire hovedfaser gjennom året:
\begin{enumerate}
    \item \textbf{Forbered revisjonsgrunnlag:} Oppdater indikatorpanelet, kvalitetsjournalen og tiltaksloggen slik at hver kontroll kan spores til ansvarlige roller og tidligere beslutninger.
    \item \textbf{Gjennomfør datasporing:} Kryssjekk datakontrakter, modelljournal og hendelseslogg for å bekrefte at datasettet har gyldig behandlingsgrunnlag og at modellversjoner er dokumentert i tråd med AI-forordningen.
    \item \textbf{Test scenarier og kontroller:} Gjennomfør representative modellscenarier fra kapittel 4 og 5, og vurder hvordan alarmgrenser, menneskelig tilsyn og beredskapsplaner aktiveres underveis.
    \item \textbf{Lukk funn og rapporter:} Prioriter tiltak i tiltaksloggen, synkroniser med gevinstpanelet i Kapittel~7 og rapporter status til dataspace-partnere og tilsynsmyndigheter.
\end{enumerate}

Tabell~\ref{tab:dataspace-audit} viser en struktur som kan brukes både i undervisning og i samarbeid med industripartnere. Hvert trinn angir hvilke dokumenter som må foreligge, hvilke tester som må gjennomføres og hvem som er ansvarlig for å lukke funn.

\begin{table}[ht]
    \centering
    \caption{Auditplan for dataspace-valideringspanelet}
    \label{tab:dataspace-audit}
    \begin{tabular}{|p{3.2cm}|p{4.8cm}|p{4.6cm}|p{3.0cm}|}
        \hline
        \textbf{Fase} & \textbf{Fokus} & \textbf{Dokumentasjon og tester} & \textbf{Ansvar} \\
        \hline
        Planleggingsmøte & Etablere mandat, mål og risikoomfang for revisjonen & Revidert årsplan, oppdatert indikatorpanel og oversikt over dataspace-partnere \citep{digdir2024samstyring} & Programleder valideringspanel + dataspace-operatør \\
        \hline
        Datasporing og kontroll & Verifisere datakontrakter, tilgangsstyring og modelljournal & Dataspace-kontrakter, DPIA-oppdatering, logguttrekk fra kontrolltårn-panelet \citep{dfo2024internkontroll} & Data steward + personvernombud \\
        \hline
        Scenario- og stresstest & Reprodusere nøkkelscenarier fra kapittel 4 og 5 og vurdere alarmgrenser & Testprotokoller, modelljournal, hendelseslogg og måledata fra pilotene \citep{iso19011-2018} & Tvillingforvalter + fagansvarlige for casene \\
        \hline
        Forbedring og rapportering & Lukke funn, oppdatere tiltakslogg og koble resultat til gevinstoppfølging & Tiltaksplan, gevinstpanel fra Kapittel~7, revisjonsrapport til partnere og tilsyn & Kvalitetsteam + programleder Kapittel~7 \\
        \hline
    \end{tabular}
\end{table}

I praksis gir tabellen en sjekkliste for hvordan auditteamet skal forberede workshops, dele dokumentasjon og følge opp funn. Når strukturen brukes i masterkurs, kan studentene tilordnes roller i panelet og få ansvar for å dokumentere et komplett revisjonsspor fra scenariotest til tiltakslogg. Leveranser fra auditplanen bør lastes opp i dataspace-portalen slik at partnerne har et felles beslutningsgrunnlag og kan planlegge neste iterasjon av sandkasse eller pilot.

For undervisning anbefales følgende opplegg:
\begin{itemize}
    \item \textbf{Audit-sprint i laboratoriet:} Studentgrupper får utdelt en casepakke fra kontrolltårn-panelet og bruker Tabell~\ref{tab:dataspace-audit} til å planlegge revisjon, gjennomføre scenariotest og rapportere funn.
    \item \textbf{Refleksjonsnotat:} Hver gruppe beskriver hvordan kontrollene ivaretar krav fra AI-forordningen og dataspace-avtalene, og foreslår forbedringer i tiltaksloggen.
    \item \textbf{Felles beslutningsmøte:} Resultatene presenteres for et simulerte porteføljestyre som vurderer hvilke tiltak som skal prioriteres før neste kvartal.
\end{itemize}

\section{Mobilitetsdataspace for kollektivtransport}
Ruter, Entur og Bane NOR bygger nå en felles mobilitetsdataspace der sanntidsdata fra buss, bane, tog og delte mikromobilitetstjenester synkroniseres med digitale tvillinger for planlegging og beredskap.\citep{ruter2024mobilitetslab,entur2023dataplattform,banenor2024digitalspor} Dataspacet kombinerer reiseplanleggingsdata, billettinformasjon, tilstandsmålinger fra infrastruktur og trafikkinformasjon fra Statens vegvesen for å gi et samlet bilde av kapasitetsutnyttelse og flaskehalser.\citep{vegvesen2023beredskap} For å sikre tillit må valideringsregimet dekke datakvalitet, modellpresisjon og koordinering mellom etatene, samtidig som personvernkrav og sikkerhetsnivåer følges slik Kapittel~3 beskriver for mobilitetsdataspace og Kapittel~7 adresserer for porteføljestyring.

\subsection{Valideringsløp for sanntidsplanlegging}
Et felles V\&V-løp bør koble modelloppdateringer i mobilitetstvillingen til operative beslutninger hos trafikkledelsen. Tabell~\ref{tab:mobilitet-validering} viser en anbefalt struktur for å teste både dataflyt og modellrespons når rutetilbudet endres. Opplegget bygger på praksis fra Mobilitetslab-prosjektet der Ruter og Entur har etablert en delt modelljournal og hendelseslogg for å kunne spore hvordan sanntidsoppdateringer påvirker kundereiser og kapasitet.\citep{ruter2024mobilitetslab,entur2023dataplattform}

\begin{table}[ht]
    \centering
    \caption{Valideringsplan for mobilitetstvilling ved endring i rutetilbud}
    \label{tab:mobilitet-validering}
    \begin{tabular}{|p{3.4cm}|p{4.6cm}|p{4.6cm}|p{3.0cm}|}
        \hline
        \textbf{Trinn} & \textbf{Formål} & \textbf{Datagrunnlag og tester} & \textbf{Ansvar} \\
        \hline
        Datainnsamling og harmonisering & Sikre at sanntidsdata fra buss, trikk, T-bane og tog er tidsstemplede og kvalitetssikret før modelloppdatering & GTFS-RT-feeder, IoT-sensorer fra kjøretøy, tilstandsmålinger fra infrastruktur, datakvalitetsskript for duplikater og forsinkede meldinger & Entur datahub + Ruter trafikkdata \\
        \hline
        Modelloppdatering og simulering & Oppdatere mobilitetstvillingen med nytt rutetilbud og kapasitetsparametere & Simuleringsscenario i AnyLogic eller Aimsun, modelljournal med parameterendringer, versjonskontroll i dataspace & Mobilitetslab-teamet \\
        \hline
        Operativ verifikasjon & Kontrollere at dashboards, API-er og trafikkinformasjon viser riktige forbindelser og kapasitet & Integrasjonstest mellom mobilitetsdataspace og kundeorienterte apper, KPI-panel for regularitet og fyllingsgrad & Ruter og Bane NOR trafikkledelse \\
        \hline
        Etterlevelse og rapportering & Dokumentere tiltak, personvernvurderinger og koordinering med beredskapsplaner & Tiltakslogg, DPIA-oppdatering, kobling til beredskapsplan fra Statens vegvesen og kvalitetsjournal i Kapittel~7 & Programleder mobilitetstvilling \\
        \hline
    \end{tabular}
\end{table}

Gjennomføringen krever at sanntidskilder overvåkes for avvik, og at hendelser som forsinkelser eller feil i informasjonssystemene utløser triage-prosessen beskrevet tidligere i kapitlet. Når modellresponsen avviker fra faktiske passasjertall må nye datasett merkes og testes i samsvar med guardrails for syntetiske data og bias fra Kapittel~5.

\subsection{Tverretatlig beredskapssimulering}
Mobilitetstvillingen gir også grunnlag for felles beredskapsøvelser mellom kollektivaktører, politi, nødetater og kommunal kriseledelse. Statens vegvesen anbefaler at byområder har en oppdatert beredskapsplan for stengte knutepunkt og store arrangementer, med tydelig ansvarsdeling og alternative transportmidler.\citep{vegvesen2023beredskap} Tabellen under skisserer hvordan en stresstest kan organiseres for hendelser som strømbrudd i T-banetunneler eller signalfeil på jernbanen.

\begin{table}[ht]
    \centering
    \caption{Scenarioportefølje for kollektivtransport og mobilitetsberedskap}
    \label{tab:beredskap-mobilitet}
    \begin{tabular}{|p{3.2cm}|p{4.6cm}|p{4.6cm}|p{3.0cm}|}
        \hline
        \textbf{Scenario} & \textbf{Modellparametere og data} & \textbf{Indikatorer og tiltak} & \textbf{Ledende aktør} \\
        \hline
        Signalfeil på jernbane & Reduser kapasitet på hovedlinjer med 50\%, aktiver buss-for-tog og vurder kapasitet i knutepunkt & Regularitet, reisetidsavvik, passasjerstrøm i stasjoner, sikkerhetstiltak og bemanningsplan & Bane NOR trafikkstyring + Ruter operativ sentral \\
        \hline
        Strømbrudd i T-banetunnel & Sett strømtilførsel til null, modeller evakueringsløp og alternative ruter & Evakueringstid, kapasitet i erstatningslinjer, samsvar med beredskapsplan, kommunikasjonstiltak & Sporveien beredskap + Oslo politidistrikt \\
        \hline
        Massearrangement i sentrum & Øk passasjeretterspørsel med 70\%, replanlegg frekvens og micromobility-soner & Fyllingsgrad, ventetid, CO$_2$-utslipp, behov for midlertidig infrastruktur og tilgjengelighetstiltak & Oslo kommune + mobilitetsdataspace-koordinator \\
        \hline
    \end{tabular}
\end{table}

Beredskapssimuleringen bør dokumenteres med indikatorpanel som speiler styringsmodellen i Kapittel~7 og med tiltakslogg som kan deles med dataspace-partnerne i Kapittel~3. Etter hver øvelse analyseres læringspunkter sammen med passasjerdata og hendelseslogger for å oppdatere både modellparametere og operative prosedyrer.

\subsection{Læringsaktiviteter}
For masterstudentene kan mobilitetstvillingen brukes i tverrfaglige oppgaver som kobler dataanalyse, simulering og styring:
\begin{itemize}
    \item \textbf{Dataharmonisering}: Studentene utvider guardrail-skriptene for syntetiske datasett med kollektivindikatorer slik at dataspace-delingen mellom Ruter, Entur og Bane NOR kan testes mot feilsløyfer og forsinkelser.
    \item \textbf{Scenarioverksted}: Gruppene spiller gjennom Tabell~\ref{tab:beredskap-mobilitet}-scenariene og bruker tiltaksloggen fra tidligere seksjoner til å vurdere ansvar og eskalering.
    \item \textbf{Rapportering}: Leveransen inkluderer en kort rapport til byrådsavdeling for miljø og samferdsel der beslutninger og indikatorresultater knyttes til klimamål og samfunnssikkerhetskrav.
\end{itemize}

\section{AI-forordningens krav til helsetvillingene}
AI-forordningen klassifiserer kliniske beslutningsstøttesystemer og digitale tvillinger for pasientforløp som høyrisiko-systemer.\citep{eu2024aiact} Dermed må både leverandører og helsevirksomheter etablere et kvalitetsledelsessystem som dokumenterer hvordan risiko håndteres før og under drift. For masterstudentene betyr det at hver caseleveranse skal kunne peke på hvor i modellen kravene adresseres, og hvilke loggføringer som viser etterlevelse over tid.

Kjernen i regelverket kan kondenseres til seks gjensidig avhengige forpliktelser:
\begin{enumerate}
    \item \textbf{Risikostyring i hele livssyklusen:} Høyrisiko-systemer skal ha et dokumentert risikostyringssystem med kontinuerlig oppdatering av identifiserte farer, testplaner og mitigeringstiltak.\citep{eu2024aiact} Arbeidsflyten bør speile modenhetstrinnene i DNV sitt digital assurance-rammeverk slik at tekniske og organisatoriske kontroller kobles sammen.\citep{dnv2023digitalassurance}
    \item \textbf{Datastyring og datakvalitet:} Opplærings-, validerings- og testdatasett må beskrives med behandlingsgrunnlag, representativitet og eventuelle skjevheter.\citep{eu2024aiact} Helsesektoren kan gjenbruke praksis fra Direktoratet for e-helse sin veileder for tredjepartsvurdering ved å kreve datasettjournal og dokumentert kvalitetssikring før modelloppdateringer.\citep{ehelse2024tilsyn}
    \item \textbf{Teknisk dokumentasjon:} Leverandøren skal til enhver tid kunne fremvise modelljournal, algoritmebeskrivelse og valideringsresultater.\citep{eu2024aiact} Dokumentasjonen bør inngå i kvalitetsjournalen som allerede brukes for sertifiseringsløpet i dette kapittelet.
    \item \textbf{Journalføring og sporbarhet:} AI-forordningen krever automatiske loggfiler over modellbeslutninger, inputdata og brukerinteraksjoner for å muliggjøre revisjon og etterprøving.\citep{eu2024aiact} Dette må integreres med hendelsesloggene i kontrolltårn-caset slik at både tekniske og kliniske avvik kan spores.
    \item \textbf{Transparens og menneskelig kontroll:} Systemet skal utformes slik at klinikere forstår begrensninger, kan overstyre anbefalinger og får varsler når tvillingen opererer utenfor validert område.\citep{eu2024aiact} Digdirs styringsveileder anbefaler å kombinere dette med et lederstyrt indikatorpanel som viser risikostatus og tiltak.\citep{digdir2023styringai}
    \item \textbf{Robusthet, cybersikkerhet og nøyaktighet:} Tvillingen skal tåle feil, angrep og datadrift uten å miste sikkerhetsegenskaper.\citep{eu2024aiact} Kravet kan operasjonaliseres ved å koble kontrollpunktene fra Tabell~\ref{tab:standardkart} med egne kliniske stresstester.
\end{enumerate}

Tabell~\ref{tab:ai-helsekrav} viser hvordan et universitetssykehus kan tildele ansvar og dokumentasjon for å møte forpliktelsene.

\begin{table}[ht]
    \centering
    \caption{Oversettelse av AI-forordningens høyrisiko-krav til helsetvilling i sykehus}
    \label{tab:ai-helsekrav}
    \begin{tabular}{|p{3.4cm}|p{4.6cm}|p{4.6cm}|p{3.0cm}|}
        \hline
        \textbf{Forpliktelse} & \textbf{Tiltak i tvillingprosjekt} & \textbf{Dokumentasjon og verktøy} & \textbf{Primær ansvarlig} \\
        \hline
        Risikostyring & Etabler livssyklusdrevne fareanalyser og release-gates før klinisk bruk & Risikologg koblet til kvalitetsjournal, revisjonsprotokoll \citep{dnv2023digitalassurance} & Klinisk kvalitetsleder \\
        \hline
        Datastyring & Kontroll av datasett mot samtykke og representativitet, bias-sjekker før opplæring & Datasettjournal, datakatalog, tredjepartsvurdering \citep{ehelse2024tilsyn} & Data steward \\
        \hline
        Teknisk dokumentasjon & Versjonsstyrt modelljournal, testprotokoll og algoritmespesifikasjon & DevOps-logger, modellarkiv, API-beskrivelser & Produktleder digital tvilling \\
        \hline
        Loggføring & Automatisert hendelseslogg for modellavgjørelser og brukerhandlinger & Audit trail integrert med kontrolltårn, varslingspanel & Sikkerhetsarkitekt \\
        \hline
        Transparens og kontroll & Brukergrensesnitt med begrunnede anbefalinger, overstyringsknapp og opplæring & Kursjournal, beslutningsstøtte-håndbok, indikatorpanel \citep{digdir2023styringai} & Medisinsk fagansvarlig \\
        \hline
        Robusthet og cybersikkerhet & Stresstest mot feilmodi, penetrasjonstest av modellgrensesnitt, fail-safe-plan & Testjournal, penetrasjonstest-rapport, beredskapsplan & Plattformteam leder \\
        \hline
    \end{tabular}
\end{table}

\subsection{Operativ etterlevelse i undervisningscaset}
For å gjøre kravene håndgripelige kan kurslaboratoriet følge en firestegs arbeidsflyt:
\begin{enumerate}
    \item \textbf{Forberedelser og DPIA:} Studentgrupper identifiserer personvern- og sikkerhetsrisikoer gjennom en forenklet DPIA før modellen bygges, og registrerer funn i risikologgen fra sertifiseringsløpet.\citep{datatilsynet2023dpia}
    \item \textbf{Konfigurer styringspanelet:} Indikatorpanelet fra forrige seksjon utvides med AI-forordningens nøkkelindikatorer, som modellpresisjon per pasientgruppe og antall overstyringer per uke.
    \item \textbf{Verifiser menneskelig kontroll:} Under simuleringene dokumenterer studentene hvordan kliniske beslutninger tas, hvilke varslingsgrenser som brukes og hvordan opplæringsmateriellet sikrer forståelse hos brukerne.
    \item \textbf{Revider og rapporter:} Etter hver sprint gjennomføres en mini-revisjon der loggfiler, indikatorer og tiltak vurderes mot forpliktelsene i Tabell~\ref{tab:ai-helsekrav}. Resultatet legges i kvalitetsjournalen og oppsummeres til styringsgruppen for pilotkurset.
\end{enumerate}

Denne tilnærmingen gir studentene innsikt i hvordan europeisk regulering og norske veiledere må kombineres for å bygge tillit til helserettede digitale tvillinger.

\section{AI-forordningen for mobilitetstvillinger}
Mobilitetstvillinger for kollektivtransport, jernbane og veibaserte tjenester påvirker fysisk sikkerhet, kritisk infrastruktur og persontransport. Dermed klassifiseres de som høyrisikosystemer under AI-forordningen, og må dokumentere hvordan sikkerhet, transparens og menneskelig kontroll ivaretas før nye funksjoner tas i bruk.\citep{eu2024aiact} Norske piloter som Ruters mobilitetslab og Enturs dataplattform viser at tvillinger allerede styrer rutetilpasninger, billettflyt og hendelseshåndtering i sanntid.\citep{ruter2024mobilitetslab,entur2023dataplattform} Samtidig forventer Bane NOR og Statens vegvesen at digitale tvillinger støtter beredskapsplaner, trafikkstyring og kriseinformasjon.\citep{banenor2024digitalspor,vegvesen2023beredskap} For masterstudentene betyr det at mobilitetscaset må koble regulatoriske krav til indikatorpanelet og hendelsesjournalen fra de foregående seksjonene.

\subsection{Sjekkliste for kollektivtransport og jernbane}
Arbeidsgruppen kan bruke følgende sjekkliste for å planlegge og dokumentere etterlevelse i mobilitetstvillingen:
\begin{enumerate}
    \item \textbf{Risikoklassifisering og mandat:} Kartlegg hvilke beslutninger tvillingen påvirker (frekvens, omdirigering, kapasitetsprioritering) og dokumenter hvorfor de faller inn under AI-forordningens sikkerhetskritiske kategori.\citep{eu2024aiact,banenor2024digitalspor}
    \item \textbf{Datastyring og sporbarhet:} Beskriv hvordan data fra sanntidssensorer, billettsystem og tredjepartsleverandører kvalitetssikres, og merk datasett med formål, eierskap og oppdateringshyppighet i Enturs datakatalog.\citep{entur2023dataplattform}
    \item \textbf{Hendelseshåndtering og logging:} Integrer tvillingen med beredskapsjournalen slik at avvik, overstyringer og kommunikasjon mot publikum logges i tråd med NIS2 og beredskapsplanene til Statens vegvesen.\citep{vegvesen2023beredskap}
    \item \textbf{Menneskelig kontroll og opplæring:} Definer roller for trafikkledere, beredskapsvakter og operasjonssentre som kan stanse, tilpasse eller forkaste modellforslag, og dokumenter hvilke prosedyrer som sikrer forståelse av modellbegrensninger.\citep{ruter2024mobilitetslab}
    \item \textbf{Evaluering og rapportering:} Planlegg faste revisjoner der indikatorer, loggfiler og brukerinnspill vurderes opp mot AI-forordningens krav og kommunenes mobilitets- og klimamål.\citep{eu2024aiact,vegvesen2023beredskap}
\end{enumerate}

Sjekklisten kan brukes som leveransekrav i prosjektoppgaver: hver gruppe leverer en kort rapport som viser status, ansvarlig rolle og videre tiltak for hvert punkt.

\subsection{Indikatorpanel for mobilitetstvillinger}
For å demonstrere etterlevelse i drift bør mobilitetstvillingen ha et indikatorpanel som kombinerer operasjonelle data med regulatoriske kontrollpunkter. Tabell~\ref{tab:ai-mobilitet} viser et forslag som kan kobles direkte til kontrolltårn-panelet fra energicaset.

\begin{table}[ht]
    \centering
    \caption{Indikatorpanel som viser AI-forordningens krav i mobilitetstvillingen}
    \label{tab:ai-mobilitet}
    \begin{tabular}{|p{3.2cm}|p{4.6cm}|p{4.6cm}|p{3.0cm}|}
        \hline
        \textbf{Indikator} & \textbf{Datagrunnlag og modellstøtte} & \textbf{AI-forordningens kontrollpunkt} & \textbf{Ansvarlig funksjon} \\
        \hline
        Trafikksikkerhetsscore & Hendelsesdata fra trafikkledelse, simulert risikovurdering av ruter og kapasitetsgrenser \citep{banenor2024digitalspor} & Art. 9 krav til risikostyring og dokumenterte mitigeringstiltak \citep{eu2024aiact} & Trafikksikkerhetssjef (Bane NOR / Sporveien) \\
        \hline
        Algoritmisk overstyring & Logg av manuelle beslutninger i mobilitetslabben, antall forslag som stanses eller endres \citep{ruter2024mobilitetslab} & Art. 14 krav til menneskelig tilsyn og overstyring \citep{eu2024aiact} & Operativ leder i kollektivselskap \\
        \hline
        Datakvalitet per datastrøm & Enturs dataplattform-metrikker for tilgjengelighet, forsinket oppdatering og datatap \citep{entur2023dataplattform} & Art. 10 krav til datastyring og representativitet \citep{eu2024aiact} & Data steward i mobilitetsdataspace \\
        \hline
        Hendelsesrespons og kommunikasjon & Beredskapsjournal, responstid og informasjonsmeldinger til publikum \citep{vegvesen2023beredskap} & Art. 17 krav om logging og rapportering til tilsyn \citep{eu2024aiact} & Beredskapsleder for mobilitet \\
        \hline
    \end{tabular}
\end{table}

Indikatorene bør presenteres i samme dashboard som brukes i beredskapsscenarioene tidligere i kapitlet. Når terskler brytes, skal tiltakslogg, kvalitetsjournal og planfil oppdateres med hvem som vurderte avviket, hvilke tiltak som ble iverksatt og hvordan læring spres til samarbeidspartnerne (for eksempel kommunens mobilitetsavdeling og dataspace-partnere).

\section{Tverrsektorielt tilsyn og rapportering}
Kommende EU-regulering for kunstig intelligens krever at høyrisiko-tvillingene dokumenterer tilsyn, rapportering og menneskelig kontroll på tvers av domener. \citet{datatilsynet2023sandkasse} viser hvordan regulatoriske sandkasser brukes til å teste tiltak i avgrensede løp, mens \citet{ehelse2024tilsyn} beskriver tredjepartsvurderinger for kliniske beslutningsstøttesystemer. For å holde styr på slike krav må tvillingteamet koble kvalitetsjournalen fra Kapittel~6 til indikatorpanelene i Kapittel~5 og styringsmodellene i Kapittel~7, slik at hvert tiltak har et tydelig eierskap og en plan for rapportering til relevante tilsynsmyndigheter.

\subsection{Sandkassefunn og tilsynsspor}
Sandkasseprosjektene peker på tre gjentakende læringspunkter: å tydeliggjøre risikoaksept, å dokumentere vurderinger av forklarbarhet og å involvere sluttbrukere i godkjenningssløyfer.\citep{datatilsynet2023sandkasse} Når erfaringene skal omsettes i driftssetting, bør de skrives inn i kvalitetsjournalen som eksplisitte tilsynsspor. I praksis innebærer det at hver revisjon knytter modellversjoner, dataopphav og operasjonelle tiltak sammen med en ansvarlig funksjon. Helsetilsynet etterspør den samme sammenhengen når de vurderer internkontroll i spesialisthelsetjenesten, og energitilsynet krever tilsvarende logg når kontrolltårn håndterer beredskapshendelser.\citep{energinorge2023beredskap,helsetilsynet2024internkontroll}

En anbefalt arbeidsflyt er å etablere et felles tilsynsregister som viser hvilke krav som gjelder for hver tjeneste, hvilke måleparametere som følger opp kravene og hvilke eskaleringsrutiner som skal brukes dersom avvik oppstår. Registeret kan deles mellom prosjekteier, compliance-team og fagansvarlige gjennom samme dataspace som brukes for indikatorene i Kapittel~3. Når nye funksjoner lanseres, oppdateres registeret som en del av releaseprosessen i kvalitetsjournalen.

\subsection{Rapporteringspakke for høyrisikotvillinger}
For å møte kravene i AI-forordningen og sektorregelverk må tvillingen levere en helhetlig rapporteringspakke med faste leveranser gjennom året. Pakken kan bygges opp av fire komponenter:
\begin{enumerate}
    \item \textbf{Kvartalsvis statusrapport}: Oppsummerer modellendringer, indikatorer for ytelse og menneskelig kontroll samt eventuelle avvik som er håndtert i perioden. Rapporten deler indikatorer med dashboards i Kapittel~5 for å sikre konsistent kommunikasjon til ledelsen.
    \item \textbf{Årlig revisjonsoppsummering}: Dokumenterer funn fra interne og eksterne revisjoner, refererer til oppdaterte risikovurderinger og kobler tiltak til styringssløyfene i Kapittel~7.
    \item \textbf{Hendelsesmeldinger}: Beskriver avvik innen 24 timer, slik NIS2 og helsesektorens tilsyn forventer, og forklarer hvilke fallback-modeller som er aktivert.\citep{eu2022nis2,ehelse2024tilsyn}
    \item \textbf{Brukerlogg og forklaringsarkiv}: Samler spørsmål og beslutninger fra operatører, og viser hvordan anbefalingene er forklart og eventuelt overstyrt.
\end{enumerate}

Tabell~\ref{tab:kap06-tilsynsplan} gir en oversikt over hvordan rapporteringspakken kan struktureres. Kolonnene speiler koblingen mellom regulatoriske krav, ansvarlige roller og verktøyene som er introdusert i tidligere kapitler.

\begin{table}[htbp]
    \centering
    \caption{Rapporteringsplan for høyrisiko digitale tvillinger}
    \label{tab:kap06-tilsynsplan}
    \begin{tabular}{p{3.2cm}p{4.6cm}p{3.8cm}p{3.4cm}}
        \toprule
        \textbf{Leveranse} & \textbf{Innhold} & \textbf{Ansvar og godkjenning} & \textbf{Verktøy og koblinger} \\
        \midrule
        Kvartalsvis statusrapport & Modellendringer, indikatorer for presisjon, forklarbarhet og menneskelig kontroll. & Produktleder tvilling signerer, compliance-team kvalitetssikrer. & Kvalitetsjournal, indikatorpanel i Kapittel~5, dataspace-kontrakter i Kapittel~3. \\
        \addlinespace
        Årlig revisjonsoppsummering & Sammendrag av interne/eksterne revisjoner, risikologg og forbedringstiltak. & Programstyre godkjenner etter høring i styringsmodell fra Kapittel~7. & Revisjonsmodul i kvalitetsjournal, tiltakslogg og sertifiseringsløp. \\
        \addlinespace
        Hendelsesmelding & Varsling av avvik, aktiverte fallback-modeller og varslede myndigheter. & Beredskapsleder og sikkerhetsansvarlig koordinerer, faglig ansvarlig bekrefter tiltak. & Hendelsesjournal, beredskapsplan og NIS2-rapportmal. \\
        \addlinespace
        Brukerlogg og forklaringsarkiv & Operatørspørsmål, forklaringer, manuelle overstyringer og læringspunkter. & Kontrolltårnleder følger opp, opplæringsteamet oppdaterer materiell. & Modelljournal, lærerveiledning og øvingsnotater fra Kapittel~5 og Kapittel~8. \\
        \bottomrule
    \end{tabular}
\end{table}

\subsection{Case: Felles rapporteringsløp for helse og energi}
Et praktisk undervisningsopplegg kan la studentgrupper kombinere pasientlogistikk-caset og kontrolltårn-caset. Hensikten er å vise hvordan samme rapporteringspakke kan dekke både helsesektorens krav til klinisk beslutningsstøtte og energisektorens krav til beredskap. Gruppen fordeler ansvar mellom et klinisk team og et energiteam, men bruker felles kvalitetsjournal og tilsynsregister. Under øvelsen skal de:
\begin{itemize}
    \item Kartlegge relevante tilsyn (Datatilsynet, Direktoratet for e-helse, NVE) og sette terskler for hendelsesmeldinger.
    \item Konfigurere indikatorpanelet slik at både pasientsikkerhet og forsyningssikkerhet måles med like metodikker.
    \item Levere en samlet revisjonsoppsummering som viser hvordan tiltak fra hvert domene henger sammen og hvor læring overføres.
\end{itemize}

Ved å bruke én rapporteringspakke på tvers av caser får studentene erfaring med hvordan organisasjoner kan utnytte samme styringsmodell på tvers av sektorer uten å miste domenespesifikke krav av syne. Erfaringene mates tilbake til planfilen i Kapittel~7 og til støttefilene for undervisning slik at kommende kull kan bygge videre.

\section{Samvirkeøvelser for dataspace-beredskap}
Tverrsektorielle øvelser blir et krav når digitale tvillinger skal støtte nasjonal beredskap. Totalberedskapskommisjonen understreker at felles situasjonsforståelse og deling av beslutningsdata må øves jevnlig for å tåle samtidige kriser.\citep{dsb2023totalberedskap} I praksis betyr det at tvillingteamet må planlegge samvirkeøvelser som kombinerer dataspace-kontrakter, kontrolltårn og kliniske beslutningsstøtter i ett scenario. Direktoratet for samfunnssikkerhet og beredskap anbefaler at nasjonale samvirkeøvelser tester både teknologi og styring, med tydelige evalueringskriterier for datautveksling, varslingsrutiner og læringssløyfer.\citep{dsb2024nser} Kommunesektoren løfter samtidig behovet for å koble lokale beredskapsplaner til slike øvelser slik at kommunale og statlige aktører kan dele måledata, tiltakslogger og beslutningsgrunnlag i sanntid.\citep{ks2022samhandling}

\subsection{Scenario og roller}
Et undervisningsopplegg kan bygge på et 24-timers scenario der et ekstremvær påvirker både strømforsyning og helsetjenester. Tabellen nedenfor viser et forslag til struktur, roller og verktøy som brukes i hvert trinn. Scenarioet kan integreres i kurslaboratoriet ved å gjenbruke kontrolltårn-panelet fra energicaset, pasientlogistikkpakken og kvalitetsjournalen.

\begin{table}[ht]
    \centering
    \caption{Forslag til samvirkeøvelse for dataspace-baserte høyrisikotvillinger}
    \label{tab:samvirkeovelse}
    \begin{tabular}{|p{2.8cm}|p{5.0cm}|p{4.8cm}|p{3.0cm}|}
        \hline
        \textbf{Fase} & \textbf{Aktivitet og dataflyt} & \textbf{Tvilling- og styringsverktøy} & \textbf{Ansvarlig(e)} \\
        \hline
        Varsling (0--2 t) & Energikontrolltårn sender varsler via dataspace til helsesektorens operasjonssenter; hendelser tagges med prioritet og lokasjon. & Hendelsesjournal, dataspace-kontrakter, varslingsdashbord & Vaktleder energi, beredskapskoordinator helse \\
        \hline
        Analyse (2--8 t) & Simulerer nettkapasitet og pasientstrøm; vurderer fallback-modeller og ressursbehov. & Tvillingmodeller fra Kapittel~5, kvalitetsjournal, indikatorpanel & Modellansvarlig, klinisk fagansvarlig \\
        \hline
        Tiltak (8--16 t) & Aktiverer mobile aggregater, omdirigerer pasienter, etablerer felles informasjonslinje; logger avvik og beslutninger. & Tiltakslogg, styringsmodell fra Kapittel~7, beslutningsstøtte & Programleder tvilling, kommunikasjonsteam \\
        \hline
        Evaluering (16--24 t) & Samler læringspunkter, oppdaterer datasett og modellparametre, utarbeider rapport til tilsyn. & Revisjonsmodul, indikatoranalyse, rapporteringspakke & Compliance-team, fagfellepanel \\
        \hline
    \end{tabular}
\end{table}

\subsection{Evalueringskriterier}
For å sikre at øvelsen dekker både tekniske og organisatoriske mål, anbefales tre evalueringskriterier:
\begin{enumerate}
    \item \textbf{Datatilgjengelighet og kvalitet:} Måle hvor raskt datasett åpnes i dataspace og om metadata oppfyller kravene fra Kapittel~3, inkludert logging av etiske og juridiske avklaringer.
    \item \textbf{Samordnet beslutningstaking:} Evaluere hvor godt tiltak i energi- og helsesektoren harmoniseres med styringssløyfene i Kapittel~7, samt hvordan menneskelig kontroll dokumenteres.
    \item \textbf{Læringssløyfe og rapportering:} Kontrollere at kvalitetsjournalen oppdateres med forbedringstiltak og at rapporteringspakken i dette kapittelet brukes til å informere tilsyn og samarbeidspartnere.
\end{enumerate}

Når øvelsen er gjennomført, bør resultatene sammenlignes med tidligere scenarier i fagfelleloggen og brukes til å oppdatere indikatorpanelene i Kapittel~5 og casene i Kapittel~8. En slik iterativ praksis gir studentene erfaring med hvordan tverrsektoriell beredskap kan støttes av digitale tvillinger og dataspace-løsninger.

\section{Bærekraftsrapportering og klimarisiko i valideringsløpet}
Valideringsarbeidet må også håndtere kravene til bærekraft og klimarisiko som følger av EUs taksonomi og den nye rapporteringsplikten i CSRD.\citep{eu2020taxonomy,eu2022csrd} For digitale tvillinger betyr det å dokumentere hvordan modellresultater påvirker miljø- og klimamål, og hvordan kontrollsløyfene bidrar til å redusere negativ påvirkning. De europeiske bærekraftsstandardene (ESRS) beskriver eksplisitte indikatorer for klima, ressurser og sirkularitet, og krever at virksomheten kan spore beregningsgrunnlaget frem til datakilder og beslutninger.\citep{efrag2023esrs} Dermed må kvalitetsjournalen og rapporteringspakken i dette kapittelet suppleres med bærekraftsdata, og indikatorpanelet fra Kapittel~5 bør utvides med nøkkeltall som måles i samme DevOps-prosess som presisjon og sikkerhet.

\subsection{Koble validering til bærekraftsrapportering}
For å tilfredsstille offentlige virksomheters rapporteringsforventninger anbefaler DFØ at bærekraftsdata integreres i de ordinære styringssløyfene og at ansvar fordeles i linjen.\citep{dfo2023baerekraft} I praksis kan tvillingteamet følge tre prinsipper når bærekraftsindikatorene bygges inn i valideringsløpet:
\begin{enumerate}
    \item \textbf{Kartlegg mål og taksonomikrav:} Knytt tvillingens formål til miljømålene i EU-taksonomien og dokumenter hvordan modellresultater påvirker kriteriene for vesentlighet og vesentlig skade.
    \item \textbf{Synkroniser datakilder:} Merk datasett med hvilke ESRS-indikatorer de støtter, og bruk kvalitetsjournalen til å vise hvordan beregninger kan revideres fra rådata til rapporterte tall.
    \item \textbf{Fordel ansvar:} Speil ansvarslinjene fra styringsmodellen i Kapittel~7 slik at bærekraftsansvarlige, fagledere og compliance-team godkjenner måltall i samme sign-off som sikkerhets- og kvalitetskontroller.
\end{enumerate}

Tabell~\ref{tab:baerekraft-validering} foreslår en indikatorpakke som binder bærekraftsrapporteringen til valideringssløyfen. Strukturen gjør det mulig å kombinere klimascenarier, ressursbruk og sosial effekt i samme rapporteringspakke som allerede brukes for høyrisiko-tvillingene.

\begin{table}[htbp]
    \centering
    \caption{Bærekraftsindikatorer i validerings- og rapporteringspakken}
    \label{tab:baerekraft-validering}
    \begin{tabular}{|p{3.2cm}|p{4.5cm}|p{4.5cm}|p{3.0cm}|}
        \hline
        \textbf{Indikatorområde} & \textbf{Datagrunnlag og modellkobling} & \textbf{Rapporteringskrav og tiltak} & \textbf{Ansvarlig rolle} \\
        \hline
        Klimagassintensitet & Simulert energiforbruk, sanntidssensorer for utslipp og scenarioanalyse mot referanseår & ESRS~E1-krav til utslippsbaner og taksonomiens klimamål; registrer avvik og tiltak i kvalitetsjournalen & Bærekraftsansvarlig i linjen \\
        \hline
        Ressurseffektivitet & Tvillingens beregning av materialstrømmer, vedlikeholdsdata og sirkulasjonsgrad & Rapporter på ESRS~E5 og taksonomiens kriterier for sirkulær økonomi; oppdater tiltakslogg og dataspace-kontrakter & Produktleder digital tvilling \\
        \hline
        Klimarisikoscenarier & Stress-testing av modeller mot fysiske og overgangsrelaterte scenarier fra NOU~2018:17 \citep{nou2018klimarisiko} & Dokumenter konsekvens og rest-risiko i risikologg og rapport til styret i tråd med CSRD & Risikostyringsfunksjon \\
        \hline
        Sosial påvirkning og sikkerhet & Arbeidsmiljødata, brukerundersøkelser og hendelseslogg for leverandørkjeden & Vurder ESRS~S2 og styringskrav til anstendig arbeid; koordinér tiltak med rapporteringspakken for tilsyn & HR- og leverandøransvarlig \\
        \hline
    \end{tabular}
\end{table}

\subsection{Arbeidsflyt for klimarisikotesting}
Klimarisiko bør testes parallelt med de tekniske valideringene, slik at modelloppdateringer ikke skaper utilsiktet økning i utslipp eller eksponering. En anbefalt arbeidsflyt er:
\begin{enumerate}
    \item \textbf{Scenarioforberedelse:} Velg fysiske og overgangsscenarioer fra NOU~2018:17 og relevant bransjestandard, og koble dem til simuleringer i Kapittel~4.
    \item \textbf{Tverrfaglig analyse:} Kombiner finans-, drift- og bærekraftsressurser i samme valideringsmøte slik at konsekvenser for økonomi, sikkerhet og miljø vurderes samlet.
    \item \textbf{Integrert rapportering:} Oppdater kvalitetsjournal, rapporteringspakke og indikatorpanelet for tillit med klimarisikofunn, og sørg for at dataene kan eksporteres til virksomhetens bærekraftsrapport.
\end{enumerate}

Når klimarisikotesten inngår i valideringslaboratoriet gir det studentene innsikt i hvordan bærekraftsmål, regulatorisk etterlevelse og tekniske kvalitetskrav må balanseres. Erfaringene kan gjenbrukes i Kapittel~7 når gevinst- og styringsmodellene skal vise hvordan bærekraft og økonomi følges opp over tid.

\section{Valideringspakke for grønne industriparker}
Det grønne industriløftet krever at norske industriklynger dokumenterer hvordan nye produksjonsløp reduserer utslipp og energiintensitet før offentlig støtte og konsesjoner kan gis.\citep{regjeringen2023grontindustriloft} Digitale tvillinger brukes allerede i Mo Industripark og Herøya Industripark for å koble produksjonsdata, energisystemer og klimaregnskap i samme styringsplattform.\citep{moindustripark2024klimaplan,heroya2024hydrogenhub} For at resultatene skal være troverdige må valideringsløpet dekke både tekniske modeller, måledata og rapportering til investorer og myndigheter.

\subsection{Trinnvis kvalitetssikring av energi- og klimadata}
Et helhetlig V\&V-løp for grønne industriparker kan organiseres i fire gjentakende trinn:
\begin{enumerate}
    \item \textbf{Baselinemåling}: Kartlegg historiske energistrømmer, utslippskilder og produksjonsvolum før tiltak settes i verk, og etabler referansescenarier i tvillingen.\citep{moindustripark2024klimaplan}
    \item \textbf{Energiledelse og målepunkt}: Implementer automatisert energiledelse med kalibrerte målepunkter for elektrisitet, damp og varme slik at modellene kan valideres mot kontinuerlige tidsserier.\citep{enova2024energiledelse}
    \item \textbf{Integrasjon av nye prosesser}: Synkroniser hydrogen-, karbonfangst- og batteripiloter med eksisterende produksjonslinjer ved å bruke felles modelljournal og kvalitetsjournal.\citep{heroya2024hydrogenhub}
    \item \textbf{Klimarapportering og læring}: Knytt leverandørkrav, bærekraftsindikatorer og læringspunkter til samme rapporteringspakke som brukes i Kapittel~7, slik at forbedringstiltak kan spores gjennom verdikjeden.\citep{eyde2023batteri}
\end{enumerate}

\subsection{Validerings- og rapporteringspakke}
Tabell~\ref{tab:industripark-validering} viser en anbefalt struktur for test- og rapporteringspakken som binder tekniske tester til klimamål og myndighetskrav.

\begin{table}[htbp]
    \centering
    \caption{Valideringspakke for grønne industriparker}
    \label{tab:industripark-validering}
    \begin{tabular}{|p{3.2cm}|p{4.5cm}|p{4.8cm}|p{3.0cm}|}
        \hline
        \textbf{Fase} & \textbf{Formål} & \textbf{Tester og dokumentasjon} & \textbf{Ansvarlig} \\
        \hline
        Forstudie og referanse & Fastsette mål, tiltak og finansiering i tråd med grønt industriløft og lokale klimaplaner & Scenarioanalyse i tvilling, ROS for energi- og forsyningssikkerhet, baseline-rapport til investorer \citep{regjeringen2023grontindustriloft,moindustripark2024klimaplan} & Programleder industripark \\
        \hline
        Pilotering av energisystem & Validere energiledelse, fleksibilitet og sanntidsmålinger før oppskalering & Feltkalibrering av sensorer, sammenligning av modellprediksjoner mot produksjonsdata, revisjon av energiledelsesprosedyrer \citep{enova2024energiledelse,moindustripark2024klimaplan} & Energi- og bærekraftsteam \\
        \hline
        Integrasjon av nye prosesser & Sikre at hydrogen, karbonfangst og batterilinjer leverer forventet effekt uten å skape sikkerhetsavvik & HAZOP-/LOPA-gjennomgang i tvilling, test av dataspace-kontrakter for delte prosessdata, oppdatering av kvalitetsjournal \citep{heroya2024hydrogenhub} & Teknologiansvarlig prosess \\
        \hline
        Løpende rapportering og revisjon & Kombinere klimaregnskap, leverandørkrav og læringspunkter i én rapporteringspakke & ESRS- og taksonomirapport, leverandørevaluering, tiltakslogg for forbedringer og revisjonsnotat \citep{enova2024energiledelse,eyde2023batteri} & Styresekretariat og compliance \\
        \hline
    \end{tabular}
\end{table}

\subsection{Læringsopplegg for masterstudenter}
Caseopplegget i laboratoriet kan bygge på samme struktur ved å la studentgrupper arbeide med industridata:
\begin{itemize}
    \item \textbf{Data- og modellverksted}: Studentene setter opp en energi- og utslippsmodell med referansescenario fra Mo Industripark og tester hvordan tiltak påvirker indikatorene i Tabell~\ref{tab:industripark-validering}.\citep{moindustripark2024klimaplan}
    \item \textbf{Rapportering i praksis}: Gruppene leverer en integrert rapport som kombinerer energiledelse, klimaregnskap og leverandørkrav etter malene i Kapittel~7.\citep{enova2024energiledelse,eyde2023batteri}
    \item \textbf{Tverrsektorielt seminar}: Resultatene diskuteres sammen med mobilitets- og helsecasene for å vise hvordan samme rapporteringspakke kan brukes i ulike sektorer og kobles til dataspace-kravene i Kapittel~3.\citep{heroya2024hydrogenhub,regjeringen2023grontindustriloft}
\end{itemize}

\section{Usikkerhetsanalyse og robusthet}
Usikkerhetsanalyse gjør det mulig å forstå hvor pålitelig tvillingens prediksjoner er. En viktig første aktivitet er å klassifisere usikkerhet som \textit{aleatorisk} (tilfeldige variasjoner i prosessen) eller \textit{epistemisk} (mangel på kunnskap). For eksempel vil en energitvilling for et fjernvarmenett ha aleatorisk usikkerhet knyttet til vær og forbruksmønstre, mens epistemisk usikkerhet kan skyldes ufullstendige rørdata eller sjeldne feilmodi. Klassifiseringen påvirker hvilke tiltak som settes inn: bedre sensorer og datakvalitetsrutiner for epistemisk usikkerhet, og probabilistiske metoder for aleatorisk variasjon. Risikostyringsprinsippene i \citet{iso31000-2018} kan brukes som ramme for å prioritere tiltak og utforme beslutningspunkter.

Konkrete metoder for kvantifisering inkluderer sensitivitetsanalyse for å identifisere hvilke inputvariabler som styrer resultatene, Monte Carlo-simuleringer for å beregne sannsynlighetsfordelinger, og Bayesiansk oppdatering når feltmålinger gradvis forbedrer modellparametrene. Scenario-testing mot historiske driftsforstyrrelser, som produksjonsavvik i prosessindustrien eller værrelaterte avbrudd i kraftnettet, bidrar til å teste modellens robusthet. Resultatene bør uttrykkes gjennom konfidensintervaller, prediksjonsbånd og risikoindikatorer som kan deles med beslutningstakere. For kontrolltårn-caset bør indikatorer mates direkte inn i driftens dashboards, med terskelverdier som trigges av avvik i modellprediksjon versus faktisk måling.

\subsection{Arbeidsflyt for usikkerhetsstyring}
En helhetlig arbeidsflyt sikrer at usikkerhet håndteres systematisk i DevOps-prosessene:
\begin{enumerate}
    \item \textbf{Hypotese og antakelser:} Definer modellens formål, antakelser og datakilder i en risikologg som kobles til standardkartet. Antakelsene bør vurderes mot DNV-RP-A204 sine modenhetstrinn.
    \item \textbf{Kvantitative analyser:} Utfør sensitivitet og Monte Carlo-simuleringer i hver sprint. Resultatene dokumenteres i en modelljournal som inngår i kvalitetssystemet.
    \item \textbf{Overvåkning og alarmgrenser:} Implementer automatiske avvikstester i produksjon slik at driftsorganisasjonen varsles når prediksjonsfeil overskrider definerte grenser. Kontrolltårn-teamet bruker KPI-ene til å eskalere hendelser via beredskapsplanene.
    \item \textbf{Læring og forbedring:} Evaluér avvik i retrospektivene, oppdater datakilder eller algoritmer, og legg forbedringstiltak inn i styringssystemet for informasjonssikkerhet.
\end{enumerate}

Robust design handler også om å gjøre tvillingen og den fysiske prosessen motstandsdyktig mot feil. Dette omfatter redundante sensorer, feiltolerante algoritmer og beredskapsprosedyrer når modellen oppdager avvik. I industrielle DevOps-team betyr det at valideringsresultater må integreres i sprint- og releaseritualer, slik at usikkerhet blir aktivt styrt og ikke bare rapportert. Til slutt bør innsikt fra kontrolltårn-caset brukes som treningsgrunnlag for både operatører og ledelse.

\section{Valideringslaboratorium for dataspace-integrerte tvillinger}
Norske virksomheter som deler data gjennom dataspace-økosystemer trenger et valideringslaboratorium som kombinerer modelltester, sikkerhetskontroller og styring av datakontrakter. \citet{sintef2021digital} viser at industripartnere lykkes best når laboratoriet gjenskaper hele kjeden fra sensorer til beslutningsstøtte før løsningen flyttes til produksjon. I dataspace-konteksten betyr det å kunne koble sammen deltagernes connectorer, verifisere delingsregler og teste hvordan tvillingen reagerer på både syntetiske og reelle avvik. Derfor bør laboratoriet ha et eget dataspace-domene med felles policyer, slik \citet{idsa2023operational} anbefaler, samtidig som tillitsmekanismene fra Gaia-X sitt rammeverk brukes til å dokumentere identitet, sertifisering og etterlevelse.\citep{gaiax2023architecture}

\subsection{Designprinsipper for laboratoriet}
Et helhetlig laboratorium bør bygges rundt fire designprinsipper:
\begin{enumerate}
    \item \textbf{Dataspace som testbenk:} Sett opp et isolert dataspace-miljø med egen katalog, policy-håndhevelse og logganalyse slik at kontrakter og konnektorer kan valideres uten å påvirke produksjonssystemer.\citep{idsa2023operational}
    \item \textbf{Reelle driftsdata med styrt anonymisering:} Bruk delmengder av faktiske datastrømmer fra energi, helse eller mobilitet, men koble dem til anonymiserings- og datakvalitetsrutiner slik \citet{digdir2024sanntidsdata} beskriver for offentlig sektor.
    \item \textbf{Scenario- og stresstesting:} Kombiner modellscenarier, Monte Carlo-kjøringer og hendelsessimuleringer for å teste hvordan tvillingen reagerer på avvik i både data og algoritmer. Resultatene kobles direkte til indikatorpanelet i Kapittel~5.
    \item \textbf{Dokumentert sporbarhet:} Integrer kvalitetsjournalen, hendelsesloggen og dataspace-kontraktene slik at hvert teststeg kan spores fra krav i \citet{gaiax2023architecture} til beslutninger i styringsmodellen i Kapittel~7.
\end{enumerate}

\subsection{Testpakker og leveranser}
Tabell~\ref{tab:dataspace-lab} viser en anbefalt struktur for testpakker i laboratoriet. Pakken gjør det mulig å koble dataspace-krav til modelltester og operativ beredskap før løsningen settes i drift.

\begin{table}[htbp]
    \centering
    \caption{Testpakker i valideringslaboratorium for dataspace-integrerte tvillinger}
    \label{tab:dataspace-lab}
    \begin{tabular}{|p{3.4cm}|p{4.6cm}|p{4.6cm}|p{2.8cm}|}
        \hline
        \textbf{Testpakke} & \textbf{Formål} & \textbf{Dataspace-koblinger} & \textbf{Resultater og gjenbruk} \\
        \hline
        Inntaks- og katalogtest & Verifisere at datasett, metadata og tilgangspolicyer følger IDS-krav før onboarding. & Katalogsynkronisering, policy-håndhevelse og connector-autentisering i henhold til \citet{idsa2023operational}. & Godkjente datasett med sporbar behandlingslogg, klar for bruk i kapittel 3 og 5. \\
        \hline
        Sanntidsstresstest & Teste hvor raskt tvillingen reagerer på sensoravvik, latency og datatap. & Simulerte streaming-kanaler og hendelsesvarsler etter anbefalingene i \citet{digdir2024sanntidsdata}. & Oppdatert overvåkingspanel, terskler og fallback-strategier til kontrolltårn-caset. \\
        \hline
        Tillits- og etterlevelsespakke & Bekrefte at identitet, sertifikater og rapporteringskrav i dataspace oppfylles. & Gaia-X trustrammeverk, verifiserte attestasjonsprofiler og delingsavtaler.\citep{gaiax2023architecture} & Revisorpakke med sjekklister til kvalitetsjournal og rapporteringsløp i dette kapittelet. \\
        \hline
        Samspill mellom tvillinger & Øve på tverrsektorielle scenarier (helse og energi) med delte indikatorer og tiltak. & Kombinerte datakontrakter og kontrolltårn-visninger, styrt av samarbeidsmodellen i \citet{sintef2021digital}. & Scenariojournal som gjenbrukes i caseopplegget i Kapittel~8 og lærerveiledningen. \\
        \hline
    \end{tabular}
\end{table}

\subsection{Gjennomføring i kurslaboratoriet}
For å bruke laboratoriet i undervisningen kan masterstudentene følge fire arbeidspakker som knytter dataspace og validering sammen:
\begin{enumerate}
    \item \textbf{Planlegg testløpet:} Studentene lager et valideringskart som kobler dataspace-kontrakter til indikatorer, og registrerer hypoteser i kvalitetsjournalen.
    \item \textbf{Konfigurer connectorer og policyer:} Hver gruppe setter opp en IDS-konnektor med minimum én delingspolicy og demonstrerer at dataflyten loggeres med korrekte tidsstempler.\citep{idsa2023operational}
    \item \textbf{Kjør testpakker:} Gruppene gjennomfører tabellens tre første pakker og dokumenterer avvik, ytelse og forbedringstiltak med referanse til kontrolltårn-panelet og usikkerhetsanalysen.
    \item \textbf{Evaluer og dele erfaringer:} Resultatene deles i en fagfellesesjon der læringspunkter mates inn i planfilen, og forbedringstiltak prioriteres i styringsmodellen fra Kapittel~7.
\end{enumerate}

Laboratoriet gjør det enklere å koble dataspace-styring og modellkvalitet til konkrete leveranser i DevOps-løpet. Når caseopplegget brukes i fagfelleplanen, kan industripartnerne bidra med realistiske scenarier samtidig som studentene får praktisk trening i dokumentert etterlevelse.

\section{Avvikstriage og læringssløyfer}
Selv med grundige testpakker vil dataspace-integrerte tvillinger møte hendelser som må håndteres i fellesskap. For å unngå at avvik blir isolerte support-saker, bør teamet etablere en triagerutine som binder sammen kontrolltårn, kvalitetsjournal og rapporteringspakke. \citet{helsetilsynet2024internkontroll} anbefaler at avvik vurderes på tvers av kliniske og tekniske funksjoner, mens \citet{energinorge2023beredskap} viser at energibransjen lykkes når overvåkingspanelene er koblet til klare eskaleringsveier. Når rutinen bygges inn i dataspace-laboratoriet kan den samme strukturen testes i undervisningscasene, og erfaringene føres direkte inn i tiltakslogg og styringsmodeller.

\subsection{Triagerutine for dataspace-hendelser}
En praktisk triagerutine kan beskrives i fire trinn som alle grupperer funn i kvalitetsjournalen og vurderer dem mot regulatoriske krav:
\begin{enumerate}
    \item \textbf{Fang hendelsen:} Overvåkingsvarsel eller kvalitetsavvik registreres automatisk i journalen med referanse til dataspace-kontrakten og den relevante testpakken fra laboratoriet.\citep{digdir2024sanntidsdata}
    \item \textbf{Klassifiser alvorlighet:} Produktleder, sikkerhetsansvarlig og domeneekspert vurderer konsekvens og sannsynlighet. Resultatet kobles til indikatorpanelene for kontrolltårn og helsecase slik at påvirkede tjenester identifiseres.\citep{energinorge2023beredskap}
    \item \textbf{Velg tiltak:} Teamet aktiverer forhåndsdefinerte tiltak i rapporteringspakken, for eksempel fallback-modeller eller manuell overstyring. Tiltakene loggføres med ansvarlig rolle og forventet effekt.
    \item \textbf{Eskalér og lær:} Når hendelsen er lukket, gjennomføres mini-revisjon sammen med fagansvarlige. Læringspunkter mates inn i tiltakslogg, undervisningsopplegg og kommende sprintplan.
\end{enumerate}

\subsection{Læringssløyfe mellom sektorer}
Tabell~\ref{tab:triage-laringssloyfe} viser hvordan triagefunnet kan følge en standardisert læringssløyfe som dekker både helsesektorens og energisektorens krav. Kolonnene speiler arbeidsflyten i rapporteringspakken, mens radene gjør det enkelt å koordinere tiltak når flere domener er involvert i samme dataspace.

\begin{table}[ht]
    \centering
    \caption{Læringssløyfe for triagerte avvik i dataspace-integrerte tvillinger}
    \label{tab:triage-laringssloyfe}
    \begin{tabular}{|p{3.3cm}|p{4.8cm}|p{4.6cm}|p{3.0cm}|}
        \hline
        \textbf{Fase} & \textbf{Aktivitet} & \textbf{Dokumentasjon og rapportering} & \textbf{Videre læring} \\
        \hline
        Umiddelbar respons & Aktivere beredskapsplan, sikre manuell kontroll og varsle berørte aktører i dataspace. & Hendelsesmelding med referanse til dataspace-kontrakt og påslåtte kontrolltiltak. & Oppdatere kontrolltårn-panelet og undervisningscaset med ny alarmgrense. \\
        \hline
        Felles analyse & Samle tekniske logger, modelljournal og brukerobservasjoner i et tverrfaglig møte. & Analyseoppsummering i kvalitetsjournalen med kobling til regulatoriske krav. & Synkronisere funn med indikatorbiblioteket i Kapittel~5 og governance-tabellene i Kapittel~7. \\
        \hline
        Beslutning og implementering & Prioritere forbedringstiltak og planlegge releaser på tvers av domener. & Tiltaksliste i rapporteringspakken med ansvar, tidsfrist og forventet effekt. & Utarbeide nye øvingsoppgaver i laboratoriet for å teste tiltaket før produksjon. \\
        \hline
        Tilbakemelding og deling & Dele læringspunkter med partnerne i dataspace og med neste studentkohort. & Oppdatert fagfellelogg og referanse til lærerveiledningens oppgavesett. & Justere planfilen og oppdatere oppgavetavlen med nye prioriteringer. \\
        \hline
    \end{tabular}
\end{table}

Sløyfen gjør det enklere å vise hvordan én hendelse fører til konkrete forbedringer i både tekniske systemer og organisasjonens rutiner. Når triage-resultatene kobles til indikatorpanel og rapporteringspakke, får studentene et helhetlig bilde av hvordan dataspace-samarbeid kan drives trygt over tid.

\section{Sertifiseringsløp og kvalitetsjournal}
Et tydelig sertifiseringsløp gjør det mulig å dokumentere at tvillingen holder et konsekvent kvalitetsnivå gjennom hele livssyklusen. \citet{dnv2023digitalassurance} anbefaler at prosjekteier etablerer et rammeverk som kobler risikovurderinger, testregimer og ansvarslinjer. Dette kan struktureres som en trinnvis modenhetsstige der hvert steg utløser nye revisjoner og beslutninger før funksjoner settes i drift. I tillegg fremhever \citet{iso2020tr24028} at tillitskriterier må dekke både tekniske egenskaper (robusthet, sikkerhet, personvern) og organisatoriske forhold (kompetanse, ansvar, prosesskontroll).

\begin{table}[ht]
    \centering
    \caption{Foreslått sertifiseringsløp for kontrolltårn-case}
    \label{tab:sertifiseringslop}
    \begin{tabular}{p{2.2cm}p{4.3cm}p{4.5cm}p{3.0cm}}
        \toprule
        \textbf{Trinn} & \textbf{Formål} & \textbf{Dokumentasjon} & \textbf{Godkjenningsforum} \\
        \midrule
        Pilot & Validere modellantakelser og datakvalitet i isolert testmiljø. & Modelljournal, usikkerhetsanalyse og risikologg fra Tabell~\ref{tab:standardkart}. & Produktleder + sikkerhetsarkitekt \\
        \addlinespace
        Driftssimulering & Teste kontrolltårn-flyt mot realistiske scenarier og fallback-prosedyrer. & Scenarioresultater, hendelseslogger og vurdering mot \citet{nist2023airmf}. & Kontrolltårnleder + beredskapsleder \\
        \addlinespace
        Sertifisering & Verifisere samsvar mot standarder og krav til ansvarlighet. & Uavhengig revisjonsrapport, etterlevelsesbevis for \citet{iso2020tr24028} og \citet{dnv2023digitalassurance}. & Ledelsesutvalg + ekstern revisor \\
        \addlinespace
        Kontinuerlig forbedring & Sikre at oppdateringer ikke svekker sertifiseringsstatus. & Kvartalsvis kvalitetsjournal, avviksliste og tiltakslogg koblet til Kapittel~7. & Programstyre + personvernombud \\
        \bottomrule
    \end{tabular}
\end{table}

\subsection{Kvalitetsjournal og revisjon}
Når sertifiseringsløpet er etablert, trenger teamet en kvalitetsjournal som følger hver release. Journalen bør ligge i samme versjonskontroll som modellkoden og inneholde sjekklister for data, algoritmer, sikkerhet og brukeropplevelse. \citet{nist2023airmf} beskriver dette som en risikoregister-praksis der indikatorer og beslutninger spores fra første prototype til ferdig tjeneste. En anbefalt struktur er:
\begin{enumerate}
    \item \textbf{Releasekort:} Oppsummer formål, ansvarlige og dato for hver endring. Knytt kortet til indikatorene i Tabell~\ref{tab:tillitsindikatorer}.
    \item \textbf{Kvalitetsbevis:} Lagre resultater fra tester, scenariotrening og brukerworkshops som vedlegg. Merkes med referanser til hvilke krav i \citet{iso2020tr24028} eller NIS2 som dekkes.
    \item \textbf{Revisjonsmerknader:} Dokumenter avvik, beslutninger og oppfølging i henhold til ansvarslinjene i kontrolltårn-caset. Merknadene brukes i fagfellelogg og retrospektiver.
\end{enumerate}
Journalen gjør det enklere å bevise samsvar under tilsyn og når studenter gjennomfører caseoppgaven, fordi alle beslutninger er etterprøvbare.

\subsection{Kobling til undervisning og kontrolltårn-case}
For å omsette sertifiseringsløpet til læringsaktiviteter bør laboratorieøvelsene be studentene levere et utdrag fra kvalitetsjournalen sammen med tekniske resultater. Dermed lærer de hvordan kontrolltårn-teamet må samle bevis for ansvarlig AI. I kurslaboratoriet kan journalen brukes som evalueringsgrunnlag: faglærer vurderer om dokumentasjonen dekker kravene i Tabell~\ref{tab:sertifiseringslop}, mens medstudenter gir tilbakemelding på tydeligheten i prosessbeskrivelsene. Når tiltakene integreres med kontrolltårn-panelet fra Kapittel~5, blir det også mulig å visualisere hvilke sertifiseringskriterier som er oppfylt i sanntid.

\section{Etikk, transparens og forklarbarhet}
Tillit til den digitale tvillingen forutsetter at både data og beslutninger er sporbare. Prosjektet bør etablere en styringsmodell for datasett, med kildebeskrivelser, behandlingsgrunnlag og kontroll av personvern i tråd med GDPR og den kommende EUs AI-forordning. Revisjonsspor i form av dataversjoner, modellkonfigurasjoner og brukerhandlinger gjør det mulig å rekonstruere hvorfor en anbefaling ble gitt og hvem som godkjente den.

Forklarbarhet er spesielt viktig når tvillingen benytter maskinlæring. Kombiner visuelle dashboards med tekstlige forklaringer som gjør det klart hvilke faktorer som påvirker en prediksjon, og hvilke begrensninger som gjelder. For operatører kan dette være interaktive «hva om»-analyser, mens ledelsen trenger oversikter over KPI-er, sikkerhetsmarginer og regulatorisk etterlevelse. For offentlige aktører som helseforetak eller samferdselsmyndigheter er det også nødvendig å kommunisere åpent med borgere om hvordan data brukes og hvilke rettigheter de har.

Etikk omfatter også vurdering av konsekvenser for arbeidstakere, leverandører og miljø. En digital tvilling kan flytte beslutningsmyndighet fra erfarne fagarbeidere til automatiserte systemer; derfor bør organisasjonen ha tiltak for kompetansebygging, brukermedvirkning og varsling av uønskede hendelser. En klar etikkpolicy med kontaktpunkter for spørsmål og avvik styrker opplevd integritet og reduserer motstand mot innføringen.

\section{Tillitsindikatorer og styringspanel}
For å gjøre etikk- og sikkerhetsarbeidet operasjonelt trenger prosjektet et eksplisitt sett med tillitsindikatorer som knyttes til styringssystemet. \citet{digdir2023styringai} anbefaler at virksomheter etablerer styringspaneler som kombinerer tekniske måltall, prosessindikatorer og dokumentasjon av ansvarslinjer. I praksis bør indikatorene følge de samme kontrollpunktene som brukes i kvalitets- og sikkerhetsstandardene, slik at ledelsen får én samlet oversikt over modenhet, avvik og tiltak. Når indikatorene kobles til datastrømmer og modellversjoner, kan de automatisk oppdateres og inngå i sprint- og releaseritualer.

Tabell~\ref{tab:tillitsindikatorer} viser et forslag til indikatorer som dekker hele DevOps-kjeden. Oppsettet kan gjenbrukes både i kontrolltårn-caset og i pilotprosjekter innen helse, energi og samferdsel.

\begin{table}[ht]
    \centering
    \caption{Forslag til tillitsindikatorer for digital tvilling i kritisk infrastruktur}
    \label{tab:tillitsindikatorer}
    \begin{tabular}{|p{3.2cm}|p{4.2cm}|p{4.2cm}|p{3.2cm}|}
        \hline
        \textbf{Indikator} & \textbf{Målemetode} & \textbf{Datakilde} & \textbf{Oppfølging} \\
        \hline
        Modellpresisjon & Prediksjonsfeil mot feltmålinger per driftsmodus & DevOps-logger, kontrolltårn-dashboards & Eskaleres til modellforvalter når feil > terskel \citep{dnv2023digitalassurance} \\
        \hline
        Datasettgjennomsiktighet & Andel datasett med dokumentert behandlingsgrunnlag og kontaktpunkt & Datakatalog med tilgang til behandlingsprotokoller & Rapporteres til personvernombud kvartalsvis \\
        \hline
        Hendelseshåndtering & Tid fra avvik til lukket tiltak i NIS2-prosess & Beredskapssystem og hendelseslogger & Drøftes i beredskapsmøte, referanse til \citet{eu2022nis2} \\
        \hline
        Brukerinvolvering & Antall forbedringsforslag fra operatører per sprint & Retrospektivlogger, læringsplattform & Følges opp av produktleder og fagforening \\
        \hline
    \end{tabular}
\end{table}

Implementeringen av panelet kan organiseres i tre steg:
\begin{enumerate}
    \item \textbf{Definer indikatorene:} Knytt hvert måltall til ansvarlig rolle, datakilde og beslutningsforum. Bruk standardkartet fra Tabell~\ref{tab:standardkart} for å kontrollere at indikatorene dekker både tekniske og organisatoriske krav.
    \item \textbf{Automatiser oppdateringen:} Integrer indikatorene i eksisterende data pipelines og modelljournaler slik at verdiene oppdateres samtidig med nye releaser. Vedlikehold scripts og dashboards i samme versjonskontroll som modellkoden.
    \item \textbf{Kommuniser resultatene:} Presenter indikatorene i kontrolltårn, ledermøter og pilotundervisning. Kombiner tallmateriale med kvalitativ status slik \citet{statnett2024kontrolltarn} anbefaler for operativt samarbeid.
\end{enumerate}

Når tillitsindikatorene er på plass, kan de kobles til fagfellelogg og tiltaksplan i Kapittel~7. Dette gir transparens i hvordan innspill følges opp og sikrer at tiltak prioriteres etter risiko og modenhet. Panelet støtter også kravene i AI-forordningen om risikostyring og dokumentasjon av menneskelig kontroll, og gjør det enklere å demonstrere etterlevelse overfor tilsynsmyndigheter.

\section{Driftsgodkjenning og modellvedlikehold}
Driftsgodkjenning er bindeleddet mellom utviklingsteamet og driftsorganisasjonen. For å unngå at nye modellversjoner introduserer ukjente risikoer må endringer loggføres, risikovurderes og dokumenteres i samme styringssystem som brukes for øvrige kritiske applikasjoner. \citet{iso10007-2017} anbefaler at alle modellartefakter inngår i en konfigurasjonsstyringsplan med unike identifikatorer, avhengigheter og historikk. Når tvillingen kjøres som en forvaltet tjeneste, bør service management-prosessene i \citet{iso20000-1-2018} brukes til å koordinere releaser, endringsråd og support slik at operatører vet hvilke funksjoner som er testet og godkjent.

I praksis kombineres konfigurasjonsstyring med sikkerhetskravene fra \citet{nsm2023grunnprinsipper} og kvalitetskriteriene i \citet{dnv2023digitalassurance}. Det betyr at hver endring skal ha en definert eier, knyttes til målbare effekter og følge en forhåndsdefinert sjekkliste for testing, dokumentasjon og sign-off. Tabell~\ref{tab:driftsgodkjenning} viser et eksempel på godkjenningspunkter som dekker både tekniske og organisatoriske kontroller. Tabellen kan brukes direkte i kvalitetsjournalen ved å registrere status og referanse til relevante rapporter eller loggføringer.

\begin{table}[ht]
    \centering
    \caption{Godkjenningspunkter for endringer i digital tvilling}
    \label{tab:driftsgodkjenning}
    \begin{tabular}{|p{3.0cm}|p{4.8cm}|p{4.8cm}|p{3.0cm}|}
        \hline
        \textbf{Steg} & \textbf{Godkjenningskriterier} & \textbf{Dokumentasjon} & \textbf{Beslutningsforum} \\
        \hline
        Endringsregistrering & Endringen er klassifisert (funksjonell, sikkerhet, data) og koblet til mål i veikartet & Konfigurasjonslogg, kobling til sprint-mål og risikologg & Produktleder + endringsråd \\
        \hline
        Test og modellkvalitet & Dekning for regresjonstester, modellavvik \(<\) definerte grenser, datasett godkjent for bruk & Testprotokoll, modelljournal, datakvalitetsrapport & Teknisk ansvarlig + fagansvarlig \\
        \hline
        Pilotering i drift & Kontrollert utrulling mot representativt miljø, overvåkning av KPI-er og hendelser & Pilotnotat, overvåkningsdashboard, beslutningslogg & Driftsteam + sikkerhetsleder \\
        \hline
        Endelig sign-off & Risikoaksept dokumentert, avvik lukket, kommunikasjon til brukere forberedt & Godkjenningsprotokoll, oppdatert kvalitetsjournal, kommunikasjonspakke & Kvalitetsråd + linjeleder \\
        \hline
        Etterkontroll & Effektmåling gjennom indikatorer, læringspunkter registrert og tiltak satt i arbeid & Retrospektivrapport, KPI-panelet, tiltaksregister & Produktleder + styringsgruppe \\
        \hline
    \end{tabular}
\end{table}

For å sikre at tabellen ikke bare blir et compliance-dokument, bør teamet etablere en fast arbeidsflyt:
\begin{enumerate}
    \item \textbf{Planlegging av endringen:} Endringsbeskrivelsen kobles til relevante standardkrav og sikkerhetsmål, og vurderes av endringsrådet sammen med kvalitetsjournalen.
    \item \textbf{Gjennomføring og overvåkning:} Under pilotering logges modellavvik, brukererfaring og sikkerhetshendelser i sanntid. Avvik som bryter terskelverdier fra kontrolltårn-caset må behandles før sign-off.
    \item \textbf{Læring og forbedring:} Etterkontrollen skal identifisere hvordan data fra produksjon kan brukes til å forbedre modell og prosess. Tiltak registreres som nye backlog-poster eller oppdateringer av styringssystemet.
\end{enumerate}



\section{Leverand\o{}rkvalifisering og datakontrakter}
Valideringsarbeidet må også forankres hos leverandører som leverer komponenter, data eller driftstjenester til tvillingen. Offentlige virksomheter forventes å dokumentere hvordan anskaffelser av kunstig intelligens ivaretar krav til ansvarlighet, datasikkerhet og styring.\citep{dfo2024anskaffelseki} Samtidig må dataspace-partnere enes om delingsavtaler, policyer og tekniske kontroller før de får tilgang til laboratoriet eller produksjonsmiljøet.\citep{digdir2024datasamarbeid,idsa2023operational} Kapittelets drifts- og rapporteringsstrukturer kan bare fungere dersom leverandørkontraktene speiler samme krav til kvalitetsjournal, indikatorpanel og eskaleringsrutiner.

\subsection{Prekvalifisering og kravstilling}
En koordinert kravprosess reduserer risikoen for at leverandørene leverer løsninger uten nødvendig dokumentasjon eller styringsmekanismer. Følgende trinn anbefales før kontrakt inngås:
\begin{enumerate}
    \item \textbf{Definer styringskrav}: Bruk kravrammeverket fra \citet{dfo2024anskaffelseki} og tillitskriteriene i \citet{iso2020tr24028} til å formulere funksjonelle og ikke-funksjonelle krav for hver modul.
    \item \textbf{Kartlegg datakontrakter}: Beskriv datasett, delingsformål og policyer i en dataspace-avtale før testing starter, slik \citet{digdir2024datasamarbeid} og \citet{idsa2023operational} anbefaler.
    \item \textbf{Avklar kvalitetsjournal}: Krev at leverandøren benytter modelljournalen fra \citet{digdir2023modelljournal} og leverer testbevis som kan importeres direkte i den felles journalstrukturen.
    \item \textbf{Planlegg revisjoner}: Integrer sertifiserings- og revisjonsløpene fra \citet{dnv2023digitalassurance} i kontrakten, med tydelige milepæler for pilot, produksjon og videreutvikling.
\end{enumerate}

\subsection{Samsvarspakke for leverand\o{}rer}
Tabell~\ref{tab:leverandor-samsvar} viser hvordan leverandørleveranser kan struktureres slik at de støtter både dataspace-styring og valideringsjournalen. Hver fase bygger videre på kontrollpunktene i Tabell~\ref{tab:driftsgodkjenning} og gjør det mulig å følge leverandøroppfølgingen i samme styringspanel som interne team.

\begin{table}[ht]
    \centering
    \caption{Samsvarspakke for leverand\o{}rer i dataspace-integrerte tvillinger}
    \label{tab:leverandor-samsvar}
    \begin{tabular}{|p{3.0cm}|p{4.9cm}|p{4.6cm}|p{3.1cm}|}
        \hline
        \textbf{Fase} & \textbf{Leveranse} & \textbf{Dokumentasjon og sjekkpunkter} & \textbf{Kobling i styringssystemet} \\
        \hline
        Forhåndskvalifisering & Risikovurdering av løsning, referansearkitektur og datalandskap & Kravtabell fra \citet{dfo2024anskaffelseki}, dataspace-policy og ROS-analyse for deling \citep{digdir2024datasamarbeid} & Tiltaksregister og risikologg i kvalitetsjournalen \\
        \hline
        Pilotleveranse & Testpakker, modelljournal og konfigurasjonsoversikt & Modelljournal iht. \citet{digdir2023modelljournal}, connector-logg og policytester \citep{idsa2023operational} & Valideringslaboratoriets testresultater og avvikslogg \\
        \hline
        Driftskontrakt & Oppdateringsplan, indikatorpanel og rapporteringsmal & Sertifiseringsbevis fra \citet{dnv2023digitalassurance}, ajourførte indikatorer og hendelsesrapportmaler & Rapporteringspakken i Tabell~\ref{tab:kap06-tilsynsplan} og styringspanelet for tillit \\
        \hline
    \end{tabular}
\end{table}

\subsection{Oppf\o{}lging i valideringsjournalen}
Når leverandørpakkene er definert, må de følges opp gjennom samme kvalitetsjournal som interne aktiviteter. Dette kan gjøres ved å:
\begin{enumerate}
    \item \textbf{Etablere felles releasekort}: Leverandørens leveranser registreres som egne releasekort med ansvarlig kontaktperson og kobling til indikatorene i Tabell~\ref{tab:tillitsindikatorer}.\citep{digdir2023modelljournal,digdir2023styringai}
    \item \textbf{Gjennomføre kvartalsvise samsvarsmøter}: Kombiner rapportene fra Tabell~\ref{tab:leverandor-samsvar} med revisjonsmerknadene i sertifiseringsløpet slik at avvik lukkes i fellesskap.\citep{dnv2023digitalassurance}
    \item \textbf{Integrere læringssløyfer}: Dokumenter forbedringsforslag fra leverandørene i tiltaksloggen og vurder hvordan de påvirker dataspace-policyer og beredskapsplaner i kapittelets øvrige caser.\citep{digdir2024datasamarbeid}
\end{enumerate}

Med en slik struktur kan kapittelets laboratorier og samvirkeøvelser gjenbruke leverandørdata og beslutninger uten at dokumentasjonen spres i sidekanaler. Studentene får trening i å se hvordan anskaffelser, dataspace-avtaler og kvalitetsjournal henger sammen i praksis, og virksomheten står bedre rustet til å forsvare valgene sine i tilsyn og fagfelleprosesser.

\section{Maritimt tilsyn for autonome fartøy}
Autonome ferger og havneoperasjoner i Norge drives gjennom tett samarbeid mellom teknologiutviklere, rederier og myndigheter. Sjøfartsdirektoratet krever at hvert testløp dokumenteres med sikkerhetsvurderinger, navigasjonsgrenser og kontaktpunkter mot trafikkovervåkingen før en tvilling får tillatelse til å operere i norske farvann.\citep{sdir2023autonomefartoy} DNV beskriver samtidig hvordan digitale tvillinger skal brukes til å vise samsvar med regelverket gjennom kontinuerlig overvåkning av risiko og ytelse.\citep{dnv2024autonomous} 

Tilsynsløpet kombinerer modelloppfølging, sjødyktighet og operativ beredskap. DevOps-teamet må levere innsikt til rederiets driftssentral (for eksempel Massterly) slik at kapteinene på land kan verifisere modelloppdateringer mot navigasjonsregler og værgrenser.\citep{massterly2023operations} Dette krever at modelljournal, hendelseslogg og sikkerhetsrapporter er strukturert likt, og at tvillingen eksponerer nøkkelindikatorer som bremseavstand, energireserver og failsafe-status.

Arbeidsflyten kan organiseres i fire steg:
\begin{enumerate}
    \item \textbf{Forbered godkjenning}: Kartlegg operasjonsscenario, risikobildet og hvilke nautiske begrensninger som gjelder i det aktuelle farvannet. Dokumentasjonen sendes til Sjøfartsdirektoratet sammen med referanser til modellbaserte tester.
    \item \textbf{Integrer overvåkning}: Koble tvillingens sensorer og avviksvakter til driftssentralen slik at alarmer automatisk sammenlignes med definerte sikkerhetsmarginer fra klassereglene.
    \item \textbf{Revider etter hendelser}: Når tvillingen opplever nestenulykker eller navigasjonsavvik, gjennomfør en felles granskning med rederi, teknologileverandør og myndigheter for å validere læringspunkter og oppdatere modellparametere.
    \item \textbf{Publiser læringslogg}: Del oppsummeringer med operatørfellesskap og undervisningsteam slik at erfaringer fra feltet gjenbrukes i kurslaboratorier og nye pilotprosjekter.
\end{enumerate}

Tabell~\ref{tab:maritim-tilsyn} viser et forslag til revisjonspunkter for en autonom ferge. Strukturen speiler kvalitetsjournalen i kontrolltårn-caset og gjør det enklere å kombinere maritime krav med indikatorpanelet for tillit.

\begin{table}[ht]
    \centering
    \caption{Revisjonspunkter for autonom maritim digital tvilling}
    \label{tab:maritim-tilsyn}
    \begin{tabular}{|p{3.1cm}|p{4.7cm}|p{4.7cm}|p{2.8cm}|}
        \hline
        \textbf{Revisjonspunkt} & \textbf{Regelverk og mål} & \textbf{Kontroll og dokumentasjon} & \textbf{Ansvarlig} \\
        \hline
        Navigasjonsmodell & Sjøfartsdirektoratets testveileder, COLREG & Scenariojournal med værgrenser, simulert møte med trafikkseparasjon, avvikslogg & Teknisk sjef tvilling \\
        \hline
        Situasjonsforståelse & DNV autonomi-rammeverk, ISO~13849 & Sensordekning, redundansrapport, automatiske helsesjekker mot driftssentral & Driftssentralleder \\
        \hline
        Beredskap og failsafe & NMA sikkerhetsstyring, NIS2 rapportering & Emergency stop-test, logg for manuell overstyring, varsling til Kystverket & Beredskapskoordinator \\
        \hline
        Data og cyberhygiene & IEC~62443 soner/konduiter, NSM grunnprinsipper & Revisjon av tilgangslistene, penetrasjonstest, oppdateringsprotokoll & Sikkerhetsansvarlig \\
        \hline
        Operatøropplæring & Sjøfartsdirektoratet krav til vakthold & Kursplan for fjernoperatører, simulatoropptak, signerte kompetansekort & Programleder autonom drift \\
        \hline
    \end{tabular}
\end{table}

Ved å integrere tabellen i kvalitetsjournalen kan rederiet dokumentere samsvar før hver milepæl i driftsgodkjenningen. Rapporteringen gjenbrukes i fagfelleprosesser og undervisning, og sikrer at nye iterasjoner av tvillingen vurderes mot maritime risikokrav samtidig som indikatorpanelet for tillit oppdateres.

\subsection{Internasjonal samsvarsbenchmark for autonome tvillinger}
Internasjonal regulering for autonome fartøy utvikler seg raskt, og norske aktører må bevise at tvillingene deres følger både globale og nasjonale krav. \citet{imo2023masscode} legger rammen for hvordan Maritime Autonomous Surface Ships (MASS) skal testes i internasjonalt farvann, mens \citet{emsa2024remote} beskriver forventninger til fjernoperasjonssentre i EU. Disse må kombineres med norske krav til testløp, sikkerhetsstyring og digital pålitelighet fra \citet{sdir2023autonomefartoy} og \citet{dnv2024autonomous}. For tvillingteamet betyr det at modelljournalen må kunne vise samsvar med flere regimer i samme revisjon.

Tabell~\ref{tab:maritim-benchmark} viser en samsvarsbenchmark som binder sammen sentrale regelverk og foreslår indikatorer som kan hentes fra kvalitetsjournalen. Strukturen gjør det mulig å følge opp hvilke bevis som er levert for hvert regime, og å koble kravene til kontrolltårn-panelet fra kapittelets tidligere seksjoner.

\begin{table}[ht]
    \centering
    \caption{Benchmark for samsvar i autonome maritime digitale tvillinger}
    \label{tab:maritim-benchmark}
    \begin{tabular}{|p{3.2cm}|p{4.6cm}|p{4.6cm}|p{3.0cm}|}
        \hline
        \textbf{Regime} & \textbf{Kjernekrav} & \textbf{Dokumentasjon i tvillingprogrammet} & \textbf{Indikator for samsvar} \\
        \hline
        IMO MASS-kode & Risikovurdering av operasjonsscenarier, kommunikasjonsrutiner og datadeling i internasjonalt farvann \citep{imo2023masscode} & Scenariojournal fra simuleringer, kommunikasjonsplan for bro-til-bro og rapport fra sikkerhetscase & Andel scenarier validert mot IMO-checkliste, tid brukt på informasjonsdeling før seilas \\
        \hline
        EMSA fjernoperasjon & Krav til organisering av fjernoperasjonssentre, situasjonsforståelse og cyberresiliens \citep{emsa2024remote} & Driftsmanual for kontrollrom, redundanstest av sensorkjeder og cybersikkerhetssjekk fra Kapittel~5 & Oppetid på fjernoperasjonssenter, dokumentert responstid på alarmer \citep{massterly2023operations} \\
        \hline
        Sjøfartsdirektoratet testløp & Nasjonale testkorridorer, sikkerhetsstyring og rapportering av avvik \citep{sdir2023autonomefartoy} & Tillatelsesbrev, avvikslogg fra operasjoner i norske farvann og plan for manuell overstyring & Antall godkjente testfaser uten gule avvik, lukketid for hendelser \citep{eu2022nis2} \\
        \hline
        DNV digital assurance & Klasseregler for autonome systemer, kontinuerlig overvåkning og oppdateringskontroll \citep{dnv2024autonomous,dnv2023digitalassurance} & Revisjonsrapporter, modelljournal med versjonssporing og patch-protokoller & Andel leveranser med godkjent DNV-review, tid fra funn til oppdatert modell \\
        \hline
        EU AI-forordning & Krav til risikoklassifisering, transparens og menneskelig tilsyn for høyrisikosystemer \citep{eu2023ai} & Risikovurdering fra kapittelets AI-seksjon, brukerlogg for menneskelig overstyring og dokumentert XAI-analyse & Fullførte kontrollpunkter i høyrisiko-sjekklisten, registrerte manuelle inngrep per seilas \\
        \hline
    \end{tabular}
\end{table}

Benchmarken gjør det tydelig hvilke deler av styringssystemet som må aktiveres før en ny maritim funksjon settes i drift. Når tabellen oppdateres jevnlig, kan den brukes i dialog med både Sjøfartsdirektoratet og internasjonale sertifisører for å vise helhetlig etterlevelse.

For masterstudentene gir benchmarken en konkret ramme for prosjekt- og laboratorieoppgaver:
\begin{itemize}
    \item \textbf{Compliance-sprint i laboratoriet:} Studentgrupper får utdelt et scenario der fjernoperasjonssenteret må håndtere nytt farvann. De fyller ut tabellen med hvilke bevis som finnes og hvilke gap som må lukkes før lansering.
    \item \textbf{Tverrfaglig workshop:} Maritime teknologer, jurister og dataingeniører analyserer hvordan indikatorene fra Tabell~\ref{tab:maritim-benchmark} kobles til hendelsesresponsen i dataspace-seksjonen og styringspanelet i Kapittel~7.
    \item \textbf{Refleksjonsnotat:} Hver student beskriver hvordan internasjonale krav påvirker modelloppdateringer og menneskelig tilsyn, med henvisning til \citet{imo2023masscode} og \citet{emsa2024remote}.
\end{itemize}

Ved å bruke benchmarken som felles språk på tvers av team får organisasjonen en sporbar oversikt over hvilke krav som er dekket, og kapittel 6 kan fungere som bro mellom teknisk validering, regulatorisk etterlevelse og undervisningspraksis.

\section{Kommunal validering for overvann og kritisk infrastruktur}
Kommuner som bygger digitale tvillinger for overvann, kritisk infrastruktur og beredskap må kombinere tekniske modeller med lovpålagte krav til planverk og øving. \citet{dsb2022beredskap} legger vekt på at kommunal beredskapsplikt krever en helhetlig ROS-prosess, mens \citet{nve2022kommunal} anbefaler at flom- og skredscenarioer testes gjennom samkjørte øvelser med kritiske tjenesteområder. I Oslo brukes den digitale tvillingen for å koordinere blågrønn infrastruktur og driftsberedskap, med indikatorer for kapasitet, risiko og tiltakssoner.\citep{oslo2023overvann} Prosjekter som overvannslaboratoriet utviklet av Asplan Viak viser hvordan kommuner kan koble sensordata, værprognoser og 3D-bymodeller for å teste tiltak før de implementeres.\citep{asplan2023overvannslab}

For å sikre at tvillingen støtter både drift og beredskap bør valideringsløpet organiseres i fire steg:
\begin{enumerate}
    \item \textbf{Etabler felles datagrunnlag:} Kartlegg hvilke sensorer, planregistre og hendelseslogger som inngår i tvillingen, og dokumenter datakvalitet og behandlingsgrunnlag i kvalitetsjournalen.
    \item \textbf{Koble modeller og tiltak:} Synkroniser hydrauliske modeller, driftsscenarier og bærekraftsmål slik at simuleringene speiler prioriterte tiltak i kommunens handlingsplaner for overvann.\citep{oslo2023overvann}
    \item \textbf{Planlegg øving og samhandling:} Bruk \citet{dsb2022beredskap} som sjekkliste for å definere roller, varslingslinjer og beslutningspunkter før tverrfaglige øvelser.
    \item \textbf{Evaluér og forbedre:} Oppsummer funn fra øvelser og drift i fagfelleloggen, og registrer forbedringstiltak i planverk og styringspanel for tillit slik \citet{nve2022kommunal} anbefaler.
\end{enumerate}

Tabell~\ref{tab:kommunal-validering} oppsummerer hvordan kommunen kan strukturere leveranser og ansvar for å koble tvillingen til kravene i beredskapsplanen.

\begin{table}[ht]
    \centering
    \caption{Valideringssløyfe for kommunalt overvanns- og beredskapscase}
    \label{tab:kommunal-validering}
    \begin{tabular}{|p{3.2cm}|p{4.6cm}|p{4.6cm}|p{3.0cm}|}
        \hline
        \textbf{Steg} & \textbf{Leveranse} & \textbf{Datagrunnlag og verktøy} & \textbf{Ansvarlig} \\
        \hline
        Oppstart og kartlegging & ROS-analyse som kombinerer flomsoner, kritiske bygg og tjenesteområder & Historiske hendelseslogger, planregister, sensorstrømmer fra overvannsløsninger & Beredskapskoordinator \\
        \hline
        Modellering og scenariobygging & Kalibrerte hydrauliske modeller og tiltakslister med prioritert rekkefølge & Overvannslaboratorium, værprognoser, 3D-bymodell \citep{asplan2023overvannslab} & Teknisk fageier for vann og avløp \\
        \hline
        Øving og driftstøtte & Tabletop-øvelser og operativ beslutningsstøtte i kontrolltårn & Hendelseslogger, samarbeidsplattform, standardkart fra Kapittel~6 & Tverrfaglig innsatsgruppe \\
        \hline
        Evaluering og oppfølging & Tiltakstabell med status, læringspunkter og kobling til handlingsplan & Kvalitetsjournal, fagfellelogg, indikatorpanel for tillit & Kommunedirektørens kriseledelse \\
        \hline
    \end{tabular}
\end{table}

Seksjonen bygger bro mellom dataspace-arbeidet i Kapittel~3 og simuleringene i Kapittel~4 ved å tydeliggjøre hvordan kommunal beredskap kan bruke samme valideringspanel og styringsindikatorer som energisektoren. Når tvillingen kombinerer overvannslaboratoriet med kvalitetsjournalen fra dette kapittelet får studentene et helhetlig case som dekker modellering, datadeling og beredskapsledelse.

\subsection{Kontinuerlig modellovervåking og gjenkalibrering}
Når valideringssløyfen er etablert må kommunen sikre løpende kontroll med modellpresisjon, datastrømmer og tiltaksgjennomføring. \citet{dsb2022beredskap} anbefaler jevnlig rapportering til beredskapsrådet, mens Oslo VAV har gjort modellovervåking til en del av driftsmøtene mellom avdeling for blågrønn infrastruktur og beredskapsstaben.\citep{oslo2023overvann} Overvannslaboratoriet viser i praksis hvordan sensordata og scenarier fra \citet{asplan2023overvannslab} kan kobles direkte til tiltakslogg og kontrolltårn-panelet i Kapittel~4. Det samme rammeverket kan gjenbrukes for andre kommunale kritiske tjenester som vannforsyning og energistyring.

Tabell~\ref{tab:kommunal-overvaking} strukturerer et overvåkingsprogram som binder sammen kontrolltårn-funksjoner fra Kapittel~5 og indikatorpanelet for tillit i dette kapittelet. Kolonnene kan gjenbrukes som sjekkliste i kvalitetsjournalen og i samarbeidsmøter mellom teknisk drift, beredskap og miljø.

\begin{table}[ht]
    \centering
    \caption{Overvåkingsprogram for kommunal digital tvilling}
    \label{tab:kommunal-overvaking}
    \begin{tabular}{|p{3.2cm}|p{4.6cm}|p{4.6cm}|p{3.0cm}|}
        \hline
        \textbf{Overvåkingskomponent} & \textbf{Varslingskriterier} & \textbf{Data- og prosesskilder} & \textbf{Tiltak og ansvar} \\
        \hline
        Sensorhelse og datakvalitet & Avvik \(>5\%\) mellom sanntid og referansemåling, tapte målepunkter over 30 minutter & IoT-plattform, kalibreringslogger, avviksrapport fra laboratoriet & Teknisk drift iverksetter feltkontroll, dataspace-ansvarlig oppdaterer datakatalog \\
        \hline
        Modellpresisjon og scenariodekning & Prediksjonsfeil over terskler definert i kontrolltårn-panelet, manglende simulering av prioriterte tiltak & Hydrauliske modeller, scenarioarkiv, læringslogger fra Kapittel~4 & Fagansvarlig oppdaterer modellparametere, beredskapsleder beslutter nye scenariotester \\
        \hline
        Tiltakslogg og planstatus & Tiltak uten eier eller frist, øvelser ikke gjennomført innen 12 måneder & Kommunal handlingsplan, beredskapsportal, fagfellelogg & Beredskapskoordinator eskalerer til beredskapsråd, planansvarlig synkroniserer med planverket \\
        \hline
        Bærekrafts- og klimamål & Avvik fra utslippsbudsjett eller blågrønn kapasitet, KPI-er som faller under gult nivå & Klimaregnskap, miljødashboard, indikatorer fra Kapittel~3 & Miljøsjef initierer forbedringsprosjekter, kontrolltårnleder rapporterer status til politisk nivå \\
        \hline
    \end{tabular}
\end{table}

For å holde oversikten operasjonell bør kommunen standardisere en månedsrapport som kombinerer tabellen med hendelsesstatistikk og læringspunkter. Rapporten kan behandles i samme møte som drift av vann- og avløpssystemet og oppfølging av byutviklingsprosjekter, slik \citet{oslo2023overvann} praktiserer. Neste steg er å koble overvåkingen til automatiske varslingsregler og tiltaksplaner i dataspace-løsningen fra Kapittel~3, slik at hendelser i felt umiddelbart trigges som oppgaver i styringssystemet.

\section{Regulatoriske rammer og sikkerhetsstandarder}
Fagfellepanelet etterspurte en tydelig kobling mellom kvalitetsprosessen og de regulatoriske rammene som gjelder for norske virksomheter. NIS2-direktivet og IEC~62443-serien definerer krav til cybersikkerhet, hendelseshåndtering og styring av industrielle kontrollsystemer, mens DNV-RP-A204 beskriver hvordan digitale tvillinger skal kvalifiseres i sikkerhetskritiske domener.\citep{eu2022nis2,iec62443-2-1,dnv2021a204} Disse rammeverkene krever at valideringsplanen dokumenterer både tekniske og organisatoriske tiltak, med klar oversikt over ansvar og godkjenninger.

For prosjekter som skal inngå i pilotundervisningen anbefales følgende kontrollpunkter:
\begin{itemize}
    \item Kartlegg om tvillingen omfattes av NIS2-krav til rapportering av hendelser og sikkerhetsavvik, og etabler rutiner for å varsle relevante myndigheter innen tidsfristene.
    \item Bruk IEC~62443 som sjekkliste for å fordele ansvar mellom leverandør, drift og OT-team, inkludert segmentering, tilgangskontroll og oppdateringsprosedyrer.
    \item Dokumenter hvordan kvalifikasjonsløpet følger DNV-RP-A204, fra definisjon av bruksområde til testing og verifikasjon av oppdateringer.
\end{itemize}
\section{Laboratorieøving: Standard- og risikoanalyse}
Laboratorieøvelsen for kapittel 6 lar studentene omsette standardkrav til konkrete tiltak. Utgangspunktet er casebeskrivelsen i \textit{kap06-sikkerhetscase.md}, der et regionalt nettselskap skal dokumentere etterlevelse før utrulling av en operatørportal.

\begin{enumerate}
    \item \textbf{Forstudie:} Studentene analyserer avviksloggen og identifiserer hvilke hendelser som faller inn under NIS2 artikkel 23.
    \item \textbf{Kartlegging:} Gruppen fyller ut et compliance-canvas med standardreferanse, teknisk og organisatorisk tiltak, ansvarlig rolle og dokumentasjon.
    \item \textbf{Rapportering:} Resultatene oppsummeres i en tre-siders rapport med soner/konduiter-diagram og indikatorer for oppfølging.
\end{enumerate}

Evalueringsrubrikken vektlegger standardforståelse, risikohåndtering, dokumentasjon og tverrfaglig koordinering. En totalscore på 10 eller mer gir «godkjent» og brukes sammen med ALTAI-sjekklisten fra Kapittel~5.

\section{Beredskapssimulering i helsetjenesten}
Helsesektoren møter særskilte krav til beredskap, pasientsikkerhet og informasjonsforvaltning. Helsedirektoratet \citet{helsedir2023beredskap} anbefaler at helseforetak dokumenterer scenarioøvelser som kobler kliniske prosesser til digital infrastruktur. Digital tvilling-teknologi kan støtte dette ved å gi et felles situasjonsbilde, teste endringer i kapasitet og dokumentere sporbarhet mot hendelsesjournaler. Direktoratet for samfunnssikkerhet og beredskap \citet{dsb2023ovelser} fremhever at øvelser bør planlegges som helhetlige læringsløp med tydelige mål og indikatorer. Når tvillingen inngår i øvelsen, må både modellpresisjon og dataflyt måles i sanntid slik at kvalitetsjournalen oppdateres automatisk.

Et helsecase kan struktureres i tre hovedfaser:
\begin{enumerate}
    \item \textbf{Planlegging:} Kliniske fagansvarlige, IT-beredskap og sikkerhetsledelse kartlegger kritiske pasientforløp, datakilder og modellantakelser. Risikoene vurderes mot scenarioene i \citet{dsb2023nrb} og knyttes til tiltak i standardkartet fra Tabell~\ref{tab:standardkart}.
    \item \textbf{Gjennomføring:} Simuleringer av kapasitetsendringer, forsyningssvikt eller cyberhendelser kjøres i tvillingen parallelt med tabletop-øvelser. Operatører registrerer beslutninger og modellavvik direkte i kvalitetsjournalen slik at \citet{dnv2023digitalassurance} sine krav til sporbarhet ivaretas.
    \item \textbf{Læring:} Etter øvelsen oppdateres indikatorene i Tabell~\ref{tab:tillitsindikatorer}, og erfaringene integreres i styringssystemet for informasjonssikkerhet. Tiltak og anbefalinger synkroniseres med fagfelleloggen og undervisningsopplegget.
\end{enumerate}

Tabell~\ref{tab:helseberedskap} viser hvordan en beredskapssimulering kan struktureres for å koble kliniske behov til tvillingens datagrunnlag og indikatorer.

\begin{table}[ht]
    \centering
    \caption{Scenarioelementer for helsesektorens beredskapssimulering}
    \label{tab:helseberedskap}
    \begin{tabular}{|p{3.0cm}|p{4.6cm}|p{4.6cm}|p{3.0cm}|}
        \hline
        \textbf{Øvingsfase} & \textbf{Hovedmål} & \textbf{Datagrunnlag og tvillingstøtte} & \textbf{Indikatorer} \\
        \hline
        Før-øvelse & Oppdatere beredskapsplaner, kapasitetsgrenser og varslingslinjer & Pasientlogistikk fra EPJ, sensordata fra medisinsk teknisk utstyr, tilgangslogger & Fullført risikologg, andel datasett med behandlingsgrunnlag \\
        \hline
        Under øvelse & Validere beslutningspunkter ved samtidige hendelser (kapasitetsøkning og cyberangrep) & Sanntidskopi av driftstavle, simulering av ressursbelastning, automatisk hendelseslogging & Responstid på alarm, prediksjonsfeil i kapasitetsmodell, antall eskalerte tiltak \\
        \hline
        Etter øvelse & Oppdatere tiltak og ansvar, dele læringspunkter med hele virksomheten & Kvalitetsjournal, fagfellelogg, ALTAI-sjekkliste & Lukketiltak innen frist, implementerte forbedringer i styringssystemet, deltagertilfredshet \\
        \hline
    \end{tabular}
\end{table}

For undervisning anbefales det å kombinere beredskapssimuleringen med laboratorieøvelsen over. Studentgrupper kan få tildelt ulike roller (beredskapsleder, klinikksjef, IKT-sikkerhetsansvarlig) og bruke tabellen som sjekkliste for hvordan tvillingens data skal brukes i hver fase. Resultatene synliggjør sammenhengen mellom modellkvalitet, regulatorisk etterlevelse og pasientsikkerhet, og bygger bro til helsesektorcaset i Kapittel~8.

\section{Helsesektor-case: pasientlogistikk og beredskap}
Sykehus og kommunale akuttmottak bruker i økende grad digitale tvillinger for å balansere sengekapasitet, personell og logistikk når innleggelser varierer gjennom døgnet. \citet{helsedir2020dho} anbefaler at tjenesteinnovasjon kombineres med kontinuerlig datadeling mellom kommune, fastlege og spesialisthelsetjeneste for å sikre trygg utskriving og oppfølging. Et helhetlig valideringsopplegg må derfor koble operativ planlegging med beredskapskravene i \citet{hod2020beredskap} og informasjonssikkerhetsrammeverket i \citet{norm2023}. Dette krever at pasientflytmodellen dokumenterer hvordan prediksjoner påvirker beredskap, personvern og kliniske beslutninger.

Et anbefalt arbeidsforløp for helsecaset består av tre sløyfer som støtter både drift og kvalitetssikring:
\begin{enumerate}
    \item \textbf{Data- og integrasjonskontroll:} Oppdatér tvillingen med sanntidsdata fra triage, laboratorieprøver og kommunale tjenester, og loggfør behandlingsgrunnlag og tilgangsrettigheter i tråd med Normens minimumskrav.
    \item \textbf{Beslutningsstøtte og varsling:} Valider prediksjoner for beleggsgrad og ventetider mot historikk, og definer terskelverdier som utløser beredskapsplanens trinnvise tiltak når bemanningen blir kritisk.
    \item \textbf{Læring og etikk:} Dokumenter avvik, pasientsikkerhetstiltak og brukermedvirkning i kvalitetsjournalen slik at faglige råd fra tilsyn og pasientutvalg blir sporbare.
\end{enumerate}

\section{Personvern, dataminimering og tilsynslogg}
Personvernregimet i helse- og energisektoren krever at valideringsaktivitetene dokumenterer både behandlingsgrunnlag og nødvendighet. Normen understreker at datatilgang skal begrenses til det som er strengt nødvendig for formålet, og at alle oppslag skal kunne spores gjennom revisjonslogger.\citep{norm2023} Når digitale tvillinger kombinerer pasientdata, produksjonsmålinger og syntetiske scenarier må modellteamet derfor vise hvordan hvert datasett støtter et konkret beslutningspunkt i kvalitetsjournalen.

\subsection{Personvernkonsekvensvurdering som del av V\&V}
Digitale tvillinger som behandler helse- eller personopplysninger med høy risiko skal gjennomføre en personvernkonsekvensvurdering (DPIA) før nye funksjoner tas i bruk. \citet{datatilsynet2023dpia} beskriver fire hovedelementer som bør inngå i kapittel 6 sitt valideringsløp:
\begin{enumerate}
    \item \textbf{Beskriv behandling og formål:} Kartlegg hvilke datasett som inngår i tvillingens opplæring og operativ drift, og gjør rede for hvorfor hvert felt er nødvendig for å oppnå ønsket beslutningsstøtte.
    \item \textbf{Vurder nødvendighet og forholdsmessighet:} Dokumenter hvordan dataminimering er implementert i datainnsamling, modelltrening og dashboards. For helsesektoren betyr det å vise at simuleringsvariabler ikke inkluderer identifiserende felter når aggregerte indikatorer er tilstrekkelige.
    \item \textbf{Analyse av risiko og tiltak:} Knyt identifiserte risikoer til sikkerhetstiltakene i standardkartet (Tabell~\ref{tab:standardkart}), inkludert tilgangsstyring, pseudonymisering og logging.
    \item \textbf{Plan for oppfølging:} Definer milepæler for nye vurderinger når tvillingen får tilgang til nye datakilder eller når algoritmene oppdateres. Resultatene bør forankres i fagfelleloggen og i styringssystemet for informasjonssikkerhet.
\end{enumerate}

\subsection{Innebygd personvern og etterlevelsesbevis}
Innebygd personvern innebærer å bake kontrolltiltak inn i arkitekturen fremfor å legge dem på toppen etterpå. \citet{datatilsynet2022innebygd} anbefaler at løsninger bruker standardinnstillinger som beskytter sluttbrukere, detaljerte tilgangsmatriser og tekniske sikringstiltak som begrenser eksport av rådata. For kapittel 6 bør dette oversettes til konkrete arkitekturkrav:
\begin{itemize}
    \item \textbf{Standardiserte rolleprofiler:} Tilgang til treningsdata, modelljournal og operasjonelle dashboards gis kun via roller som er forhåndsgodkjent av personvernombud og sikkerhetsleder.
    \item \textbf{Automatisert sporbarhet:} Revisjonslogg for modellendringer, datauttrekk og hendelseshåndtering lagres i et separat tilsynsregister som kan eksporteres til Datatilsynet eller Helsetilsynet ved behov. Loggen bør kobles til tiltaksplanen i Kapittel~7 for å sikre helhetlig styring.
    \item \textbf{Tidsavgrenset datalagring:} Sett eksplisitte tidsgrenser for hvor lenge trenings- og valideringsdata lagres, og verifiser sletting i retrospektive møter. Kontrollpunktene kan inngå som indikator i Tabell~\ref{tab:tillitsindikatorer}.
\end{itemize}

For energisektoren anbefales tilsvarende praksis i kontrolltårn-caset: adgang til hendelseslogger begrenses til operatører på vakt, og eksport av loggdata skjer gjennom godkjente sikkerhetskanaler. Ved å samle DPIA-resultater, tilgangskontroller og loggutdrag i én tilsynslogg kan virksomheten dokumentere etterlevelse av både personvernforordningen og NIS2-direktivets rapporteringskrav. Dette styrker tillitspanelene i kapittelet og gir studentene en konkret mal for hvordan juridiske krav integreres i modellstyring.

Tabell~\ref{tab:helsevalidering} viser hvordan elementene over kan struktureres i en valideringspakke som dekker både tekniske tester og pasientsikkerhet.

\begin{table}[ht]
    \centering
    \caption{Valideringspakke for pasientlogistikk i helsesektorens digitale tvilling}
    \label{tab:helsevalidering}
    \begin{tabular}{|p{3.6cm}|p{4.2cm}|p{4.2cm}|p{3.0cm}|}
        \hline
        \textbf{Kontrollpunkt} & \textbf{Målemetode} & \textbf{Dokumentasjon} & \textbf{Ansvarlig} \\
        \hline
        Datakvalitet og sporbarhet & Kryssjekk mot elektronisk pasientjournal og kommunale meldinger & Tilgangslogg, behandlingsgrunnlag, samtykkeskjema & Klinisk informasjonsforvalter \\
        \hline
        Prediksjonsnøyaktighet & Sammenlikn prognoser for liggetid og ventetid mot historiske målepunkter & Modelljournal, ukentlig rapport til beredskapsråd & Analytiker for pasientlogistikk \\
        \hline
        Beredskapsrespons & Øvelse av beredskapstrinn (grønn–rød) og eskalering til kommunal koordineringsenhet & Hendelseslogger, evalueringsnotat etter øvelse & Akuttleder \\
        \hline
        Personvern og etikk & Revisjon av tilgangsstyring, vurdering av automatiserte anbefalinger mot kliniske retningslinjer & ROS-analyse, beslutningsprotokoll fra pasientsikkerhetsutvalg & Personvernombud \\
        \hline
    \end{tabular}
\end{table}

\section{Etikkstyring og AI-forordningen for helsesektorens tvillinger}
AI-forordningen klassifiserer kliniske beslutningsstøttesystemer som høyrisiko og krever at de dokumenterer en helhetlig styring av risiko, data og menneskelig tilsyn før de settes i produksjon.\citep{eu2024aiact} For helsesektorens digitale tvillinger betyr det at kontrolltårn-panelet, kvalitetsjournalen og beredskapsøvelsene må vise hvordan modellresultater kan forklares, justeres og stanses av autoriserte fagpersoner. Arbeidet bør forankres i den norske veilederen for styring av kunstig intelligens\citep{digdir2023styringai} og i ALTAI-sjekklisten for tillitsverdig AI, slik at etiske prinsipper oversettes til konkrete kontrollpunkter.\citep{ec2020trustworthyai}

\subsection{Styring av høyrisikosystemer}
Et systematisk styringsløp gjør det mulig å knytte AI-forordningens krav til eksisterende DevOps- og kvalitetsrutiner:
\begin{enumerate}
    \item \textbf{Klassifisering og risikoregister:} Kartlegg tvillingens funksjoner mot forordningens høyrisiko-kategorier og oppdater risikoregisteret med konsekvenser for pasientsikkerhet, diskriminering og tilgjengelighet.\citep{eu2024aiact}
    \item \textbf{Datastyring og tekniske kontroller:} Dokumenter datasettkvalitet, syntetisering og anonymisering i tråd med AI-forordningens artikkel~10, og koble kontrollene til standardkartet i Tabell~\ref{tab:standardkart}.\citep{nist2023airmf}
    \item \textbf{Menneskelig tilsyn og override:} Etabler prosedyrer for manuell overstyring, eskalering og pausemodus, og vis hvordan beslutningslogg og indikatorpanel gjør tilsyn mulig i sanntid.\citep{digdir2023styringai}
    \item \textbf{Transparens og forklarbarhet:} Utvikle forklaringsmoduler for sentrale prediksjoner og knytt dem til ALTAI-spørsmålene om transparens, ansvarlighet og robusthet.\citep{ec2020trustworthyai}
\end{enumerate}

\subsection{Tiltakslogg og ansvar}
Tabell~\ref{tab:aiforordning-tiltak} viser hvordan kravene kan oversettes til tiltak og dokumentasjon i kvalitetsjournalen. Strukturen bygger på samme logikk som tabellene for kontrolltårn og personvern, slik at styringssløyfene kan samordnes.

\begin{table}[ht]
    \centering
    \caption{Tiltak for å operasjonalisere AI-forordningens krav i helsetvilling}
    \label{tab:aiforordning-tiltak}
    \begin{tabular}{|p{3.3cm}|p{4.5cm}|p{4.5cm}|p{3.0cm}|}
        \hline
        \textbf{Kravområde} & \textbf{Praktisk gjennomføring} & \textbf{Dokumentasjon} & \textbf{Ansvarlig rolle} \\
        \hline
        Risikostyring og klassifisering & Oppdatere risikomatrise og akseptkriterier per klinisk tjeneste, inkludert terskler for å stanse automatiserte anbefalinger & Risikologg, beslutningsprotokoll fra pasientsikkerhetsutvalg, referanse til AI-forordningens art. 9 & Kvalitetsleder \citep{eu2024aiact} \\
        \hline
        Dataforvaltning og datakvalitet & Etablere datakatalog med behandlingsgrunnlag, syntetiseringsregler og sporbarhet til kilde-systemer & Datasettjournal, teknisk bevis for dataminimering, revisjonslogg for tilgang & Dataforvalter \citep{eu2024aiact} \\
        \hline
        Menneskelig tilsyn og kompetanse & Beskrive operasjonelle prosedyrer for manuell overstyring, varslingsrutiner og sertifisering av operatører & Beslutningslogg, opplæringsregister, scenarioresultater fra beredskapsøvelser & Klinisk driftsleder \citep{digdir2023styringai} \\
        \hline
        Transparens og rapportering & Publisere årlig redegjørelse for modellendringer, tilsynslogger og forbedringstiltak til ledelse og tilsynsmyndigheter & Årsrapport, ALTAI-sjekkliste, oppdatert tiltakslogg koblet til Tabell~\ref{tab:tillitsindikatorer} & Programleder digital tvilling \citep{ec2020trustworthyai} \\
        \hline
    \end{tabular}
\end{table}

Når tiltaksloggen er integrert i kvalitetsjournalen kan indikatorpanelet i Tabell~\ref{tab:tillitsindikatorer} utvides med AI-spesifikke mål, for eksempel andel beslutninger som får manuell gjennomgang og tid fra avvik til varslet tilsyn. Dette gjør det enklere å demonstrere etterlevelse overfor Datatilsynet og helsemyndigheter, samtidig som DevOps-teamet får tydeligere prioriteringer i sprintplanene.

Når tvillingen tas i bruk i undervisningslaboratoriet, bør studentgrupper teste scenarioer for akutt pågang, influensaepidemier og samtidige kommunale beredskapssituasjoner. Resultatene kobles til kontrolltårn-panelet fra energisektoren ved å oversette indikatorene til helsesektorens måleparametere, for eksempel andel planlagte utskrivinger som må flyttes og tid fra triage til første kliniske vurdering. Slik får studentene erfaring med å overføre strukturerte valideringsmetoder til et nytt domene og ser hvordan ansvarlige roller samarbeider for å sikre trygg tjenesteleveranse.

\section{Validering av generative modeller og syntetiske datasett}
Generative AI-assistenter og foundation-modeller tas i bruk for å generere nye scenarier, kode og beslutningsunderlag i digitale tvillinger.\citep{siemens2023copilot,rcn2024digitalisering} Når slike modeller koples til sanntidsdata må valideringsløpet utvides med kontroller som fanger opp hallusinasjoner, datastrømmer utenfor modellens treningsgrunnlag og risiko for at syntetiske datasett gjenskaper personopplysninger. \citet{nist2023airmf} fremhever at generative modeller krever eksplisitte guardrails og risikoscenarier, mens \citet{digdir2023styringai} anbefaler at offentlige virksomheter dokumenterer hvordan menneskelig kontroll og transparens sikres i hver iterasjon. Dette kapittelet integrerer derfor en egen kontrollpakke for generative komponenter slik at laboratoriet og undervisningsoppleggene kan kobles til kapittel~5 sitt arbeid med generativ og edge-basert AI.

Et generativt valideringsløp kan organiseres i fire aktiviteter som gjentas gjennom utvikling og drift:
\begin{enumerate}
    \item \textbf{Kartlegg bruksområder og risikobilde:} Dokumenter hvilke beslutninger, kodeendringer eller simuleringer som påvirkes av den generative modellen, og vurder konsekvenser for sikkerhet, beredskap og styring før modellen tas i bruk.\citep{digdir2023styringai}
    \item \textbf{Definer guardrails og evaluering:} Etabler tester for faktasjekk, referanseprompter og måling av avvik fra domenedata. Kombiner automatiserte metrikker med faglig gjennomgang slik at modellen kun leverer svar innenfor godkjent ansvarsområde.
    \item \textbf{Sikre datastrømmer og journal:} Loggfør hvilke treningsdata, syntetiske datasett og brukerprompter som benyttes, og knytt dem til kvalitetsjournalen slik at endringer kan spores tilbake ved hendelser.\citep{digdir2024datasamarbeid}
    \item \textbf{Evaluer påvirkning på operasjonell drift:} Sammenlign modellens forslag med beslutninger tatt i kontrolltårn, laboratorier og beredskapsøvelser. Registrer avvik i tiltaksloggen og oppdater indikatorpanelet for tillit med generative feilkilder.
\end{enumerate}

Tabell~\ref{tab:generativ-validering} viser hvordan kontrollpunktene kan struktureres for å integrere generative komponenter i kvalitetsjournalen.

\begin{table}[ht]
    \centering
    \caption{Kontrollpakke for generative tvillingmodeller}
    \label{tab:generativ-validering}
    \begin{tabular}{|p{3.2cm}|p{4.6cm}|p{4.6cm}|p{3.0cm}|}
        \hline
        \textbf{Kontrollområde} & \textbf{Valideringssteg} & \textbf{Dokumentasjon} & \textbf{Ansvarlig} \\
        \hline
        Modellformål og risiko & Klassifiser generative funksjoner, avklar risikonivå og beslutningsmyndighet før implementering & Risikoregister, beslutningsprotokoll, referanse til AI-forordningen \citep{eu2024aiact} & Programleder digital tvilling \\
        \hline
        Prompt- og outputkontroll & Test referanseprompter, etabler sperrer for sensitive forespørsler og mål presisjon mot domeneekspertenes fasit & Testlogg, eksempelbank, referansepanel med faglig vurdering & Kapittel 6 laboratorieteam \\
        \hline
        Datastrøm og sporbarhet & Dokumenter opprinnelse til trenings- og syntetiske datasett, loggfør brukerprompter og modellversjoner & Datasettjournal, modelljournal, tilgangslogg \citep{digdir2024datasamarbeid} & Data steward \\
        \hline
        Operasjonell integrasjon & Sammenlign modellforslag med beslutninger i kontrolltårn og beredskapsøvelser, registrer effekter på indikatorpanelet & Tiltakslogg, indikatorpanel for tillit, læringsnotat fra øvelser & Driftssjef/beredskapsleder \\
        \hline
    \end{tabular}
\end{table}

\subsection{Kvalitetskontroll av syntetiske datasett}
Syntetiske datasett brukes for å beskytte personopplysninger og dele treningsdata mellom partnere, men må kvalitetssikres for å unngå lekkasje av reelle individdata eller feil statistikk. \citet{datatilsynet2022anonymisering} anbefaler at organisasjoner kombinerer kvantitative tester for gjenidentifisering med manuell vurdering av om datasettet kan misbrukes. I dataspace-samarbeid bør kommuner og virksomheter også beskrive hvordan syntetiske datasett distribueres, hvem som har tilgang og hvilke scenarier de er gyldige for.\citep{digdir2024datasamarbeid}

\begin{itemize}
    \item \textbf{Representativitet:} Sammenlign nøkkelstatistikk (gjennomsnitt, varians, ekstreme verdier) med originale datasett for å sikre at syntetiske data bevarer mønstre som modellen er avhengig av.
    \item \textbf{Re-identifiseringstest:} Bruk risikoindikatorer fra \citet{datatilsynet2022anonymisering} for å måle sannsynlighet for å gjenskape enkeltindivider. Dersom risikoen overskrider terskler må datasettet re-genereres eller begrenses i distribusjon.
    \item \textbf{Kontinuerlig overvåking:} Registrer hvilke generasjoner som er brukt i laboratoriet, og oppdater kvalitetsjournalen når nye versjoner tas i bruk slik at devops-teamet kan spore avvik til konkrete datasett.
\end{itemize}

Resultatene kobles til kontrolltårn-panelet fra energicase og helsesektorcaset slik at indikatorene kan filtrere mellom observasjoner fra produksjonsdata og syntetiske testdata. Dette gjør det mulig å avsløre hvis genererte datasett skjuler risikoer som må håndteres i ordinær drift.

\subsection{Kobling til undervisnings- og laboratorieopplegg}
For masterstudentene gir generativ validering et aktuelt case som viser hvordan fremvoksende AI-verktøy kan brukes ansvarlig. Øvingspakken bør derfor utvides med følgende aktiviteter:
\begin{itemize}
    \item \textbf{Prompt-laboratorium:} Studentgrupper tester guardrails mot faktiske scenarier fra Kapittel~4 og dokumenterer hvordan feilutdata registreres i kvalitetsjournalen.
    \item \textbf{Syntetisk dataverksted:} Klassen sammenligner treningssett fra Kapittel~3 sine dataspace-case med syntetiske alternativer, og vurderer risiko basert på \citet{datatilsynet2022anonymisering}.
    \item \textbf{Refleksjonsnotat:} Hver gruppe beskriver hvordan kontrollene i Tabell~\ref{tab:generativ-validering} påvirker indikatorpanelet for tillit og leverandøroppfølgingen fra forrige seksjoner.
\end{itemize}

Når laboratoriet dokumenterer generative kontroller på samme måte som øvrige valideringssløyfer får virksomheten et helhetlig rammeverk som kan deles med fagfeller og tilsyn. Erfaringene brukes videre når nye leverandører skal kvalifiseres eller når dataspace-partnere ber om dokumentasjon på at syntetiske datasett er trygge å bruke.

\section{Regulatorisk sandkasse og tredjepartsrevisjon}
Datatilsynets regulatoriske sandkasse for ansvarlig kunstig intelligens gir offentlige virksomheter mulighet til å teste nye digitale løsninger sammen med tilsynet før de tas i full drift.\citep{datatilsynet2023sandkasse} For digitale tvillinger i helsesektoren betyr det at prosjektet kan eksperimentere med nye datakilder og algoritmer innenfor klare rammer for personvern, informasjonssikkerhet og etikk. Sandkasseløpet krever at tvillingteamet dokumenterer hvordan løsningen bygger inn nødvendige sikkerhetsmekanismer og menneskelig kontroll, og at erfaringene omsettes til forbedrede rutiner i kvalitetsjournalen.

En strukturert sandkasseprosess består av fire hovedaktiviteter:
\begin{enumerate}
    \item \textbf{Forberedelse og mandat:} Beskriv formålet, datagrunnlaget og forventede gevinster i en prosjektbeskrivelse som koordineres mellom klinisk ledelse, personvernombud og Datatilsynet. Identifiser relevante risikoområder og kartlegg hvilke deler av AI-forordningen som berøres.
    \item \textbf{Design og testscenarier:} Etabler scenarier for modelltesting, datasett og vurderingskriterier sammen med sandkassens fagteam. Dokumenter hvordan syntetiske data, anonymisering og tilgangsstyring skal brukes for å redusere risiko i eksperimenteringen.\citep{norm2023}
    \item \textbf{Evaluering og forbedring:} Analyser funn fra sandkassens workshops og tester, og oversett anbefalingene til konkrete tiltak i kvalitetsjournalen, kontrolltårn-panelet og beredskapsplanene.
    \item \textbf{Ekstern deling og implementering:} Publiser hovedfunn, risikovurderinger og anbefalte tiltak slik at både pasientorganisasjoner og tilsynsmyndigheter får innsikt i hvordan tvillingen håndterer risiko og personvern. Oppdater læringssløyfene i kapittelets øvrige seksjoner slik at erfaringene blir en del av kontinuerlig forbedring.
\end{enumerate}

For å sikre etterlevelse gjennom hele livsløpet bør sandkasseresultatene følges av en tredjepartsrevisjon. Direktoratet for e-helse anbefaler at slike revisjoner kombinerer kliniske fagmiljø, tekniske eksperter og juridiske rådgivere som vurderer både datakvalitet, modellstyring og pasientsikkerhet.\citep{ehelse2024tilsyn} Tabell~\ref{tab:tredjepartsrevisjon} viser et eksempel på hvordan revisjonen kan struktureres for en helsetvilling som skal rulles ut i et regionalt helseforetak.

\begin{table}[ht]
    \centering
    \caption{Tredjepartsrevisjon etter sandkasse for helsesektorens digitale tvilling}
    \label{tab:tredjepartsrevisjon}
    \begin{tabular}{|p{3.2cm}|p{4.6cm}|p{4.6cm}|p{3.0cm}|}
        \hline
        \textbf{Fokusområde} & \textbf{Revisjonstiltak} & \textbf{Dokumentasjon} & \textbf{Ekstern aktør} \\
        \hline
        Datagrunnlag og personvern & Sammenligne sandkassens databruk med behandlingsgrunnlag og tilgangsprotokoller, kontrollere anonymiseringsstrategi & Databehandlingsprotokoller, tilgangslogg, sandkasserapport & Personvernombud + Datatilsynet \\
        \hline
        Modell og algoritmer & Gjennomgå modelljournal, feilanalyser og valideringsresultater, reprodusere kritiske scenarier fra sandkassen & Modelljournal, testlogg, kodeutdrag & Ekstern modellrevisor \\
        \hline
        Klinisk prosess & Evaluere hvordan tvillingen påvirker pasientforløp, beslutningsstøtte og beredskap, sikre at menneskelig kontroll er tydelig dokumentert & Prosesskart, kontrolltårn-indikatorer, læringslogg fra øvelser & Klinisk fagråd \\
        \hline
        Styring og forbedring & Kontrollere at tiltak fra sandkassen er integrert i kvalitetsjournalen, at avvik behandles i styringssystemet og at kommunikasjon med pasienter er planlagt & Kvalitetsjournal, tiltaksregister, kommunikasjonsplan & Direktoratet for e-helse + styresekretariat \\
        \hline
    \end{tabular}
\end{table}

Resultatene fra sandkassen og tredjepartsrevisjonen bør brukes til å oppdatere indikatorene i kontrolltårn-panelet, sikre at nye risikoer loggføres i kvalitetsjournalen og planlegge videre pilotering i undervisningslaboratoriet. Dette styrker koblingen mellom helsesektorens tvillinger, de kommunale beredskapsløpene og kravene fra AI-forordningen som omtales tidligere i kapittelet.

\section{Refleksjonsspørsmål og øvinger}
\begin{enumerate}
    \item Definer en valideringsstrategi for en digital tvilling i helsevesenet, og identifiser hvilke standarder og datasett som må håndteres for å dokumentere pasientsikkerhet.
    \item Beskriv hvordan du ville gjennomføre en usikkerhetsanalyse for en energimodell, inkludert valg av statistiske metoder og hvordan resultatene skal presenteres for driftsorganisasjonen.
    \item Diskuter tiltak for å ivareta etikk og personvern i prosjektet, og foreslå hvordan funnene kan formidles til både tekniske team og eksterne tilsynsmyndigheter.
\end{enumerate}
