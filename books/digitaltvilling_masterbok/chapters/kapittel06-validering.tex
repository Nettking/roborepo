\chapter{Validering, verifikasjon og tillit}

\section{Læringsmål}
\begin{itemize}
    \item Etablere prosesser for validering og verifikasjon av digitale tvillinger.
    \item Analysere usikkerhet og risiko knyttet til modellresultater.
    \item Kommunisere tillit og transparens til interessenter.
\end{itemize}

\section{Valideringsrammeverk}
\begin{itemize}
    \item Forskjell mellom verifikasjon og validering.
    \item Standarder og beste praksis (ISO 9001, ASME V\&V, NASA).
    \item Kontinuerlig validering gjennom livssyklusen.
\end{itemize}

\section{Usikkerhetsanalyse og robusthet}
\begin{itemize}
    \item Typer usikkerhet: epistemisk og aleatorisk.
    \item Sensitivitetsanalyser, Monte Carlo og scenario-testing.
    \item Robust design og feiltoleranse.
\end{itemize}

\section{Etikk, transparens og forklarbarhet}
\begin{itemize}
    \item Krav til sporbarhet og revisjon.
    \item Forklarbar AI og ansvarlig bruk av data.
    \item Kommunikasjon med ledelse, operatører og myndigheter.
\end{itemize}

\section{Refleksjonsspørsmål og øvinger}
\begin{enumerate}
    \item Definer en valideringsstrategi for en digital tvilling i helsevesenet.
    \item Beskriv hvordan du ville gjennomføre en usikkerhetsanalyse for en energimodell.
    \item Diskuter tiltak for å ivareta etikk og personvern i prosjektet.
\end{enumerate}
