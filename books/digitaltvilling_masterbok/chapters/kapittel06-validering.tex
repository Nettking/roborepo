\chapter{Validering, verifikasjon og tillit}


\section{Læringsmål}
\begin{itemize}
    \item Etablere prosesser for validering og verifikasjon av digitale tvillinger.
    \item Analysere usikkerhet og risiko knyttet til modellresultater.
    \item Kommunisere tillit og transparens til interessenter.
\end{itemize}

\section{Valideringsrammeverk}
Validering og verifikasjon er to komplementære kontrollsløyfer som må inngå i samme kvalitetsledelsessystem. \textit{Verifikasjon} handler om å teste at den digitale tvillingen er riktig implementert i henhold til spesifikasjoner, for eksempel ved å sjekke at koden følger modellgrunnlaget og at grensesnittene mot sensorer er konsistente. \textit{Validering} handler om å demonstrere at tvillingen er egnet til formålet den skal støtte, ved å sammenligne modellens resultater med observasjoner, operatørkunnskap og forretningsmessige krav. I praksis bør disse aktivitetene planlegges i et V\&V-program med tydelige akseptkriterier, roller og beslutningspunkter.

Flere etablerte standarder kan brukes som rammeverk. ISO~9001 gir føringer for styring av kvalitetsprosesser og dokumentasjon, mens DNV-RP-A204 og ASME V\&V 40 beskriver hvordan digitale tvillinger og prediktive modeller kan kvalifiseres i sikkerhetskritiske domener. Offentlige aktører som Avinor og Statnett krever at leverandører følger slike prinsipper når nye tvillinger skal tas i bruk. Prosjekter med stor usikkerhet kan også hente metodikk fra NASA sin modellmodenhetsskala, der testplaner og datakvalitetskrav utvikles trinnvis i takt med at tvillingen utvides.

Validering er ikke en engangsaktivitet, men en kontinuerlig praksis gjennom tvillingens livsløp. Hver gang modellen får nye data, algoritmer eller styringsparametere må baseline-modeller, referansescenarier og feltmålinger brukes til å bekrefte at presisjonen er innenfor toleranse. Dette krever en styrt datastrøm inn i DevOps- og MLOps-prosessene, versjonskontroll av modellartefakter, og periodiske revisjoner der både tekniske og domeneeksperter signerer på at tvillingen fortsatt er egnet.

\section{Sikkerhetsstandarder og regulatorisk etterlevelse}
Digitale tvillinger som påvirker kritisk infrastruktur må etterleve etablerte sikkerhetsstandarder. \citet{iec62443-2-1} gir rammeverk for å dele systemet inn i soner og konduiter med definerte sikkerhetsnivåer, slik at både IT- og OT-komponenter får tilpasset beskyttelse. For organisatoriske prosesser supplerer \citet{iso27001-2022} med krav til risikostyring, dokumentasjon og kontinuerlig forbedring av informasjonssikkerhet. I tillegg stiller \citet{eu2022nis2} krav til hendelsesrapportering, styring og tilsyn, noe som innebærer tydelige roller og oppfølging i hele verdikjeden. Prosjekter i energisektoren må samtidig dokumentere samsvar med nasjonale beredskapsforskrifter og kontrollprogram, slik Energi Norge beskriver for nettselskapers operative kontrolltårn.\citep{energinorge2023beredskap}

Et praktisk grep er å knytte hvert modell- og datagrensesnitt til en kontroll i standardene. Tiltak som tofaktor-autentisering, segmentering og revisjonslogger må beskrives både teknisk og organisatorisk. Når nye funksjoner lanseres, bør et standardkart vise hvilke krav som berøres og hvem som eier tiltakene. Tabell~\ref{tab:standardkart} viser et utdrag av en slik oversikt for energibransjen. Støttenotatet \textit{kap06-sikkerhetscase.md} gir en caseoppgave der studentene kartlegger tiltak for et regionalt kraftselskap.

\begin{table}[ht]
    \centering
    \caption{Eksempel på standardkart for kontrolltårn i energisektoren}
    \label{tab:standardkart}
    \begin{tabular}{|p{3.2cm}|p{4.6cm}|p{4.6cm}|p{3.2cm}|}
        \hline
        \textbf{Funksjon} & \textbf{Kjernekrav} & \textbf{Tiltak og dokumentasjon} & \textbf{Ansvarlig rolle} \\
        \hline
        Dataoppsamling fra SCADA & IEC~62443-3-3 SR 3, NIS2 art. 21 & Segmenterte konduiter, logging av tjenestekontoer, endringsjournal & OT-sikkerhetsarkitekt \\
        \hline
        Modelloppdatering i tvillingplattform & ISO~27001 A.12, DNV-RP-A204 steg 3 & Godkjenningsprotokoll, versjonskontroll, testlogg for modellendringer & Produktleder digital tvilling \\
        \hline
        Hendelsesrespons & NIS2 art. 23, IEC~62443-2-1 kap. 5 & Alarmgrenser, responsskript, rapport til myndigheter innen 24 timer & Beredskapsleder \\
        \hline
        Visualisering i kontrolltårn & Energi Norge veileder kap. 4, ISO~27001 A.13 & Rollebasert tilgang, sanntidsdashboard med audit trail, sporbarhet til datakilder & Kontrolltårnleder \\
        \hline
    \end{tabular}
\end{table}

\subsection{Kontrolltårn-case for regionalt kraftsystem}
Flere norske nettselskaper etablerer kontrolltårn for å kombinere feltdata, tvillingmodeller og beslutningsstøtte i én operativ arena. \citet{statnett2024kontrolltarn} beskriver hvordan overvåkingssenteret koordinerer sanntidssignaler fra produksjon, nett og værdata for å planlegge vedlikehold og håndtere avvik. For å overføre praksisen til undervisningen anbefales følgende struktur:
\begin{enumerate}
    \item \textbf{Kartlegg datastrømmer:} Studentene tegner et dataflytsdiagram fra sensorer via SCADA og datasjø til tvillingplattformen, og identifiserer hvor tilgangskontroll og logging må styrkes.
    \item \textbf{Definer kontrollpaneler:} Gruppen designer KPI-er for forsyningssikkerhet, tap, beredskap og bærekraft. Hvert panel må kobles til en standardreferanse i tabellen over.
    \item \textbf{Planlegg koordinering:} Scenarioverksteder kjøres med operatører, fagansvarlige og leverandører for å teste hendelseshåndtering, samtidig som roller og varslingsrutiner dokumenteres etter NIS2.
\end{enumerate}
Kontrolltårn-caset forsterker koplingen mellom modellvalidering, cybersikkerhet og organisasjonsdesign. Studentene skal levere både teknisk dokumentasjon og en kort tiltakslogg for oppfølging i styringssystemet.

\section{Usikkerhetsanalyse og robusthet}
Usikkerhetsanalyse gjør det mulig å forstå hvor pålitelig tvillingens prediksjoner er. En viktig første aktivitet er å klassifisere usikkerhet som \textit{aleatorisk} (tilfeldige variasjoner i prosessen) eller \textit{epistemisk} (mangel på kunnskap). For eksempel vil en energitvilling for et fjernvarmenett ha aleatorisk usikkerhet knyttet til vær og forbruksmønstre, mens epistemisk usikkerhet kan skyldes ufullstendige rørdata eller sjeldne feilmodi. Klassifiseringen påvirker hvilke tiltak som settes inn: bedre sensorer og datakvalitetsrutiner for epistemisk usikkerhet, og probabilistiske metoder for aleatorisk variasjon. Risikostyringsprinsippene i \citet{iso31000-2018} kan brukes som ramme for å prioritere tiltak og utforme beslutningspunkter.

Konkrete metoder for kvantifisering inkluderer sensitivitetsanalyse for å identifisere hvilke inputvariabler som styrer resultatene, Monte Carlo-simuleringer for å beregne sannsynlighetsfordelinger, og Bayesiansk oppdatering når feltmålinger gradvis forbedrer modellparametrene. Scenario-testing mot historiske driftsforstyrrelser, som produksjonsavvik i prosessindustrien eller værrelaterte avbrudd i kraftnettet, bidrar til å teste modellens robusthet. Resultatene bør uttrykkes gjennom konfidensintervaller, prediksjonsbånd og risikoindikatorer som kan deles med beslutningstakere. For kontrolltårn-caset bør indikatorer mates direkte inn i driftens dashboards, med terskelverdier som trigges av avvik i modellprediksjon versus faktisk måling.

\subsection{Arbeidsflyt for usikkerhetsstyring}
En helhetlig arbeidsflyt sikrer at usikkerhet håndteres systematisk i DevOps-prosessene:
\begin{enumerate}
    \item \textbf{Hypotese og antakelser:} Definer modellens formål, antakelser og datakilder i en risikologg som kobles til standardkartet. Antakelsene bør vurderes mot DNV-RP-A204 sine modenhetstrinn.
    \item \textbf{Kvantitative analyser:} Utfør sensitivitet og Monte Carlo-simuleringer i hver sprint. Resultatene dokumenteres i en modelljournal som inngår i kvalitetssystemet.
    \item \textbf{Overvåkning og alarmgrenser:} Implementer automatiske avvikstester i produksjon slik at driftsorganisasjonen varsles når prediksjonsfeil overskrider definerte grenser. Kontrolltårn-teamet bruker KPI-ene til å eskalere hendelser via beredskapsplanene.
    \item \textbf{Læring og forbedring:} Evaluér avvik i retrospektivene, oppdater datakilder eller algoritmer, og legg forbedringstiltak inn i styringssystemet for informasjonssikkerhet.
\end{enumerate}

Robust design handler også om å gjøre tvillingen og den fysiske prosessen motstandsdyktig mot feil. Dette omfatter redundante sensorer, feiltolerante algoritmer og beredskapsprosedyrer når modellen oppdager avvik. I industrielle DevOps-team betyr det at valideringsresultater må integreres i sprint- og releaseritualer, slik at usikkerhet blir aktivt styrt og ikke bare rapportert. Til slutt bør innsikt fra kontrolltårn-caset brukes som treningsgrunnlag for både operatører og ledelse.

\section{Sertifiseringsløp og kvalitetsjournal}
Et tydelig sertifiseringsløp gjør det mulig å dokumentere at tvillingen holder et konsekvent kvalitetsnivå gjennom hele livssyklusen. \citet{dnv2023digitalassurance} anbefaler at prosjekteier etablerer et rammeverk som kobler risikovurderinger, testregimer og ansvarslinjer. Dette kan struktureres som en trinnvis modenhetsstige der hvert steg utløser nye revisjoner og beslutninger før funksjoner settes i drift. I tillegg fremhever \citet{iso2020tr24028} at tillitskriterier må dekke både tekniske egenskaper (robusthet, sikkerhet, personvern) og organisatoriske forhold (kompetanse, ansvar, prosesskontroll).

\begin{table}[ht]
    \centering
    \caption{Foreslått sertifiseringsløp for kontrolltårn-case}
    \label{tab:sertifiseringslop}
    \begin{tabular}{p{2.2cm}p{4.3cm}p{4.5cm}p{3.0cm}}
        \toprule
        \textbf{Trinn} & \textbf{Formål} & \textbf{Dokumentasjon} & \textbf{Godkjenningsforum} \\
        \midrule
        Pilot & Validere modellantakelser og datakvalitet i isolert testmiljø. & Modelljournal, usikkerhetsanalyse og risikologg fra Tabell~\ref{tab:standardkart}. & Produktleder + sikkerhetsarkitekt \\
        \addlinespace
        Driftssimulering & Teste kontrolltårn-flyt mot realistiske scenarier og fallback-prosedyrer. & Scenarioresultater, hendelseslogger og vurdering mot \citet{nist2023airmf}. & Kontrolltårnleder + beredskapsleder \\
        \addlinespace
        Sertifisering & Verifisere samsvar mot standarder og krav til ansvarlighet. & Uavhengig revisjonsrapport, etterlevelsesbevis for \citet{iso2020tr24028} og \citet{dnv2023digitalassurance}. & Ledelsesutvalg + ekstern revisor \\
        \addlinespace
        Kontinuerlig forbedring & Sikre at oppdateringer ikke svekker sertifiseringsstatus. & Kvartalsvis kvalitetsjournal, avviksliste og tiltakslogg koblet til Kapittel~7. & Programstyre + personvernombud \\
        \bottomrule
    \end{tabular}
\end{table}

\subsection{Kvalitetsjournal og revisjon}
Når sertifiseringsløpet er etablert, trenger teamet en kvalitetsjournal som følger hver release. Journalen bør ligge i samme versjonskontroll som modellkoden og inneholde sjekklister for data, algoritmer, sikkerhet og brukeropplevelse. \citet{nist2023airmf} beskriver dette som en risikoregister-praksis der indikatorer og beslutninger spores fra første prototype til ferdig tjeneste. En anbefalt struktur er:
\begin{enumerate}
    \item \textbf{Releasekort:} Oppsummer formål, ansvarlige og dato for hver endring. Knytt kortet til indikatorene i Tabell~\ref{tab:tillitsindikatorer}.
    \item \textbf{Kvalitetsbevis:} Lagre resultater fra tester, scenariotrening og brukerworkshops som vedlegg. Merkes med referanser til hvilke krav i \citet{iso2020tr24028} eller NIS2 som dekkes.
    \item \textbf{Revisjonsmerknader:} Dokumenter avvik, beslutninger og oppfølging i henhold til ansvarslinjene i kontrolltårn-caset. Merknadene brukes i fagfellelogg og retrospektiver.
\end{enumerate}
Journalen gjør det enklere å bevise samsvar under tilsyn og når studenter gjennomfører caseoppgaven, fordi alle beslutninger er etterprøvbare.

\subsection{Kobling til undervisning og kontrolltårn-case}
For å omsette sertifiseringsløpet til læringsaktiviteter bør laboratorieøvelsene be studentene levere et utdrag fra kvalitetsjournalen sammen med tekniske resultater. Dermed lærer de hvordan kontrolltårn-teamet må samle bevis for ansvarlig AI. I kurslaboratoriet kan journalen brukes som evalueringsgrunnlag: faglærer vurderer om dokumentasjonen dekker kravene i Tabell~\ref{tab:sertifiseringslop}, mens medstudenter gir tilbakemelding på tydeligheten i prosessbeskrivelsene. Når tiltakene integreres med kontrolltårn-panelet fra Kapittel~5, blir det også mulig å visualisere hvilke sertifiseringskriterier som er oppfylt i sanntid.

\section{Etikk, transparens og forklarbarhet}
Tillit til den digitale tvillingen forutsetter at både data og beslutninger er sporbare. Prosjektet bør etablere en styringsmodell for datasett, med kildebeskrivelser, behandlingsgrunnlag og kontroll av personvern i tråd med GDPR og den kommende EUs AI-forordning. Revisjonsspor i form av dataversjoner, modellkonfigurasjoner og brukerhandlinger gjør det mulig å rekonstruere hvorfor en anbefaling ble gitt og hvem som godkjente den.

Forklarbarhet er spesielt viktig når tvillingen benytter maskinlæring. Kombiner visuelle dashboards med tekstlige forklaringer som gjør det klart hvilke faktorer som påvirker en prediksjon, og hvilke begrensninger som gjelder. For operatører kan dette være interaktive «hva om»-analyser, mens ledelsen trenger oversikter over KPI-er, sikkerhetsmarginer og regulatorisk etterlevelse. For offentlige aktører som helseforetak eller samferdselsmyndigheter er det også nødvendig å kommunisere åpent med borgere om hvordan data brukes og hvilke rettigheter de har.

Etikk omfatter også vurdering av konsekvenser for arbeidstakere, leverandører og miljø. En digital tvilling kan flytte beslutningsmyndighet fra erfarne fagarbeidere til automatiserte systemer; derfor bør organisasjonen ha tiltak for kompetansebygging, brukermedvirkning og varsling av uønskede hendelser. En klar etikkpolicy med kontaktpunkter for spørsmål og avvik styrker opplevd integritet og reduserer motstand mot innføringen.

\section{Tillitsindikatorer og styringspanel}
For å gjøre etikk- og sikkerhetsarbeidet operasjonelt trenger prosjektet et eksplisitt sett med tillitsindikatorer som knyttes til styringssystemet. \citet{digdir2023styringai} anbefaler at virksomheter etablerer styringspaneler som kombinerer tekniske måltall, prosessindikatorer og dokumentasjon av ansvarslinjer. I praksis bør indikatorene følge de samme kontrollpunktene som brukes i kvalitets- og sikkerhetsstandardene, slik at ledelsen får én samlet oversikt over modenhet, avvik og tiltak. Når indikatorene kobles til datastrømmer og modellversjoner, kan de automatisk oppdateres og inngå i sprint- og releaseritualer.

Tabell~\ref{tab:tillitsindikatorer} viser et forslag til indikatorer som dekker hele DevOps-kjeden. Oppsettet kan gjenbrukes både i kontrolltårn-caset og i pilotprosjekter innen helse, energi og samferdsel.

\begin{table}[ht]
    \centering
    \caption{Forslag til tillitsindikatorer for digital tvilling i kritisk infrastruktur}
    \label{tab:tillitsindikatorer}
    \begin{tabular}{|p{3.2cm}|p{4.2cm}|p{4.2cm}|p{3.2cm}|}
        \hline
        \textbf{Indikator} & \textbf{Målemetode} & \textbf{Datakilde} & \textbf{Oppfølging} \\
        \hline
        Modellpresisjon & Prediksjonsfeil mot feltmålinger per driftsmodus & DevOps-logger, kontrolltårn-dashboards & Eskaleres til modellforvalter når feil > terskel \citep{dnv2023digitalassurance} \\
        \hline
        Datasettgjennomsiktighet & Andel datasett med dokumentert behandlingsgrunnlag og kontaktpunkt & Datakatalog med tilgang til behandlingsprotokoller & Rapporteres til personvernombud kvartalsvis \\
        \hline
        Hendelseshåndtering & Tid fra avvik til lukket tiltak i NIS2-prosess & Beredskapssystem og hendelseslogger & Drøftes i beredskapsmøte, referanse til \citet{eu2022nis2} \\
        \hline
        Brukerinvolvering & Antall forbedringsforslag fra operatører per sprint & Retrospektivlogger, læringsplattform & Følges opp av produktleder og fagforening \\
        \hline
    \end{tabular}
\end{table}

Implementeringen av panelet kan organiseres i tre steg:
\begin{enumerate}
    \item \textbf{Definer indikatorene:} Knytt hvert måltall til ansvarlig rolle, datakilde og beslutningsforum. Bruk standardkartet fra Tabell~\ref{tab:standardkart} for å kontrollere at indikatorene dekker både tekniske og organisatoriske krav.
    \item \textbf{Automatiser oppdateringen:} Integrer indikatorene i eksisterende data pipelines og modelljournaler slik at verdiene oppdateres samtidig med nye releaser. Vedlikehold scripts og dashboards i samme versjonskontroll som modellkoden.
    \item \textbf{Kommuniser resultatene:} Presenter indikatorene i kontrolltårn, ledermøter og pilotundervisning. Kombiner tallmateriale med kvalitativ status slik \citet{statnett2024kontrolltarn} anbefaler for operativt samarbeid.
\end{enumerate}

Når tillitsindikatorene er på plass, kan de kobles til fagfellelogg og tiltaksplan i Kapittel~7. Dette gir transparens i hvordan innspill følges opp og sikrer at tiltak prioriteres etter risiko og modenhet. Panelet støtter også kravene i AI-forordningen om risikostyring og dokumentasjon av menneskelig kontroll, og gjør det enklere å demonstrere etterlevelse overfor tilsynsmyndigheter.

\section{Driftsgodkjenning og modellvedlikehold}
Driftsgodkjenning er bindeleddet mellom utviklingsteamet og driftsorganisasjonen. For å unngå at nye modellversjoner introduserer ukjente risikoer må endringer loggføres, risikovurderes og dokumenteres i samme styringssystem som brukes for øvrige kritiske applikasjoner. \citet{iso10007-2017} anbefaler at alle modellartefakter inngår i en konfigurasjonsstyringsplan med unike identifikatorer, avhengigheter og historikk. Når tvillingen kjøres som en forvaltet tjeneste, bør service management-prosessene i \citet{iso20000-1-2018} brukes til å koordinere releaser, endringsråd og support slik at operatører vet hvilke funksjoner som er testet og godkjent.

I praksis kombineres konfigurasjonsstyring med sikkerhetskravene fra \citet{nsm2023grunnprinsipper} og kvalitetskriteriene i \citet{dnv2023digitalassurance}. Det betyr at hver endring skal ha en definert eier, knyttes til målbare effekter og følge en forhåndsdefinert sjekkliste for testing, dokumentasjon og sign-off. Tabell~\ref{tab:driftsgodkjenning} viser et eksempel på godkjenningspunkter som dekker både tekniske og organisatoriske kontroller. Tabellen kan brukes direkte i kvalitetsjournalen ved å registrere status og referanse til relevante rapporter eller loggføringer.

\begin{table}[ht]
    \centering
    \caption{Godkjenningspunkter for endringer i digital tvilling}
    \label{tab:driftsgodkjenning}
    \begin{tabular}{|p{3.0cm}|p{4.8cm}|p{4.8cm}|p{3.0cm}|}
        \hline
        \textbf{Steg} & \textbf{Godkjenningskriterier} & \textbf{Dokumentasjon} & \textbf{Beslutningsforum} \\
        \hline
        Endringsregistrering & Endringen er klassifisert (funksjonell, sikkerhet, data) og koblet til mål i veikartet & Konfigurasjonslogg, kobling til sprint-mål og risikologg & Produktleder + endringsråd \\
        \hline
        Test og modellkvalitet & Dekning for regresjonstester, modellavvik \(<\) definerte grenser, datasett godkjent for bruk & Testprotokoll, modelljournal, datakvalitetsrapport & Teknisk ansvarlig + fagansvarlig \\
        \hline
        Pilotering i drift & Kontrollert utrulling mot representativt miljø, overvåkning av KPI-er og hendelser & Pilotnotat, overvåkningsdashboard, beslutningslogg & Driftsteam + sikkerhetsleder \\
        \hline
        Endelig sign-off & Risikoaksept dokumentert, avvik lukket, kommunikasjon til brukere forberedt & Godkjenningsprotokoll, oppdatert kvalitetsjournal, kommunikasjonspakke & Kvalitetsråd + linjeleder \\
        \hline
        Etterkontroll & Effektmåling gjennom indikatorer, læringspunkter registrert og tiltak satt i arbeid & Retrospektivrapport, KPI-panelet, tiltaksregister & Produktleder + styringsgruppe \\
        \hline
    \end{tabular}
\end{table}

For å sikre at tabellen ikke bare blir et compliance-dokument, bør teamet etablere en fast arbeidsflyt:
\begin{enumerate}
    \item \textbf{Planlegging av endringen:} Endringsbeskrivelsen kobles til relevante standardkrav og sikkerhetsmål, og vurderes av endringsrådet sammen med kvalitetsjournalen.
    \item \textbf{Gjennomføring og overvåkning:} Under pilotering logges modellavvik, brukererfaring og sikkerhetshendelser i sanntid. Avvik som bryter terskelverdier fra kontrolltårn-caset må behandles før sign-off.
    \item \textbf{Læring og forbedring:} Etterkontrollen skal identifisere hvordan data fra produksjon kan brukes til å forbedre modell og prosess. Tiltak registreres som nye backlog-poster eller oppdateringer av styringssystemet.
\end{enumerate}

\section{Regulatoriske rammer og sikkerhetsstandarder}
Fagfellepanelet etterspurte en tydelig kobling mellom kvalitetsprosessen og de regulatoriske rammene som gjelder for norske virksomheter. NIS2-direktivet og IEC~62443-serien definerer krav til cybersikkerhet, hendelseshåndtering og styring av industrielle kontrollsystemer, mens DNV-RP-A204 beskriver hvordan digitale tvillinger skal kvalifiseres i sikkerhetskritiske domener.\citep{eu2022nis2,iec62443-2-1,dnv2021a204} Disse rammeverkene krever at valideringsplanen dokumenterer både tekniske og organisatoriske tiltak, med klar oversikt over ansvar og godkjenninger.

For prosjekter som skal inngå i pilotundervisningen anbefales følgende kontrollpunkter:
\begin{itemize}
    \item Kartlegg om tvillingen omfattes av NIS2-krav til rapportering av hendelser og sikkerhetsavvik, og etabler rutiner for å varsle relevante myndigheter innen tidsfristene.
    \item Bruk IEC~62443 som sjekkliste for å fordele ansvar mellom leverandør, drift og OT-team, inkludert segmentering, tilgangskontroll og oppdateringsprosedyrer.
    \item Dokumenter hvordan kvalifikasjonsløpet følger DNV-RP-A204, fra definisjon av bruksområde til testing og verifikasjon av oppdateringer.
\end{itemize}
\section{Laboratorieøving: Standard- og risikoanalyse}
Laboratorieøvelsen for kapittel 6 lar studentene omsette standardkrav til konkrete tiltak. Utgangspunktet er casebeskrivelsen i \textit{kap06-sikkerhetscase.md}, der et regionalt nettselskap skal dokumentere etterlevelse før utrulling av en operatørportal.

\begin{enumerate}
    \item \textbf{Forstudie:} Studentene analyserer avviksloggen og identifiserer hvilke hendelser som faller inn under NIS2 artikkel 23.
    \item \textbf{Kartlegging:} Gruppen fyller ut et compliance-canvas med standardreferanse, teknisk og organisatorisk tiltak, ansvarlig rolle og dokumentasjon.
    \item \textbf{Rapportering:} Resultatene oppsummeres i en tre-siders rapport med soner/konduiter-diagram og indikatorer for oppfølging.
\end{enumerate}

Evalueringsrubrikken vektlegger standardforståelse, risikohåndtering, dokumentasjon og tverrfaglig koordinering. En totalscore på 10 eller mer gir «godkjent» og brukes sammen med ALTAI-sjekklisten fra Kapittel~5.

\section{Beredskapssimulering i helsetjenesten}
Helsesektoren møter særskilte krav til beredskap, pasientsikkerhet og informasjonsforvaltning. Helsedirektoratet \citet{helsedir2023beredskap} anbefaler at helseforetak dokumenterer scenarioøvelser som kobler kliniske prosesser til digital infrastruktur. Digital tvilling-teknologi kan støtte dette ved å gi et felles situasjonsbilde, teste endringer i kapasitet og dokumentere sporbarhet mot hendelsesjournaler. Direktoratet for samfunnssikkerhet og beredskap \citet{dsb2023ovelser} fremhever at øvelser bør planlegges som helhetlige læringsløp med tydelige mål og indikatorer. Når tvillingen inngår i øvelsen, må både modellpresisjon og dataflyt måles i sanntid slik at kvalitetsjournalen oppdateres automatisk.

Et helsecase kan struktureres i tre hovedfaser:
\begin{enumerate}
    \item \textbf{Planlegging:} Kliniske fagansvarlige, IT-beredskap og sikkerhetsledelse kartlegger kritiske pasientforløp, datakilder og modellantakelser. Risikoene vurderes mot scenarioene i \citet{dsb2023nrb} og knyttes til tiltak i standardkartet fra Tabell~\ref{tab:standardkart}.
    \item \textbf{Gjennomføring:} Simuleringer av kapasitetsendringer, forsyningssvikt eller cyberhendelser kjøres i tvillingen parallelt med tabletop-øvelser. Operatører registrerer beslutninger og modellavvik direkte i kvalitetsjournalen slik at \citet{dnv2023digitalassurance} sine krav til sporbarhet ivaretas.
    \item \textbf{Læring:} Etter øvelsen oppdateres indikatorene i Tabell~\ref{tab:tillitsindikatorer}, og erfaringene integreres i styringssystemet for informasjonssikkerhet. Tiltak og anbefalinger synkroniseres med fagfelleloggen og undervisningsopplegget.
\end{enumerate}

Tabell~\ref{tab:helseberedskap} viser hvordan en beredskapssimulering kan struktureres for å koble kliniske behov til tvillingens datagrunnlag og indikatorer.

\begin{table}[ht]
    \centering
    \caption{Scenarioelementer for helsesektorens beredskapssimulering}
    \label{tab:helseberedskap}
    \begin{tabular}{|p{3.0cm}|p{4.6cm}|p{4.6cm}|p{3.0cm}|}
        \hline
        \textbf{Øvingsfase} & \textbf{Hovedmål} & \textbf{Datagrunnlag og tvillingstøtte} & \textbf{Indikatorer} \\
        \hline
        Før-øvelse & Oppdatere beredskapsplaner, kapasitetsgrenser og varslingslinjer & Pasientlogistikk fra EPJ, sensordata fra medisinsk teknisk utstyr, tilgangslogger & Fullført risikologg, andel datasett med behandlingsgrunnlag \\
        \hline
        Under øvelse & Validere beslutningspunkter ved samtidige hendelser (kapasitetsøkning og cyberangrep) & Sanntidskopi av driftstavle, simulering av ressursbelastning, automatisk hendelseslogging & Responstid på alarm, prediksjonsfeil i kapasitetsmodell, antall eskalerte tiltak \\
        \hline
        Etter øvelse & Oppdatere tiltak og ansvar, dele læringspunkter med hele virksomheten & Kvalitetsjournal, fagfellelogg, ALTAI-sjekkliste & Lukketiltak innen frist, implementerte forbedringer i styringssystemet, deltagertilfredshet \\
        \hline
    \end{tabular}
\end{table}

For undervisning anbefales det å kombinere beredskapssimuleringen med laboratorieøvelsen over. Studentgrupper kan få tildelt ulike roller (beredskapsleder, klinikksjef, IKT-sikkerhetsansvarlig) og bruke tabellen som sjekkliste for hvordan tvillingens data skal brukes i hver fase. Resultatene synliggjør sammenhengen mellom modellkvalitet, regulatorisk etterlevelse og pasientsikkerhet, og bygger bro til helsesektorcaset i Kapittel~8.

\section{Helsesektor-case: pasientlogistikk og beredskap}
Sykehus og kommunale akuttmottak bruker i økende grad digitale tvillinger for å balansere sengekapasitet, personell og logistikk når innleggelser varierer gjennom døgnet. \citet{helsedir2020dho} anbefaler at tjenesteinnovasjon kombineres med kontinuerlig datadeling mellom kommune, fastlege og spesialisthelsetjeneste for å sikre trygg utskriving og oppfølging. Et helhetlig valideringsopplegg må derfor koble operativ planlegging med beredskapskravene i \citet{hod2020beredskap} og informasjonssikkerhetsrammeverket i \citet{norm2023}. Dette krever at pasientflytmodellen dokumenterer hvordan prediksjoner påvirker beredskap, personvern og kliniske beslutninger.

Et anbefalt arbeidsforløp for helsecaset består av tre sløyfer som støtter både drift og kvalitetssikring:
\begin{enumerate}
    \item \textbf{Data- og integrasjonskontroll:} Oppdatér tvillingen med sanntidsdata fra triage, laboratorieprøver og kommunale tjenester, og loggfør behandlingsgrunnlag og tilgangsrettigheter i tråd med Normens minimumskrav.
    \item \textbf{Beslutningsstøtte og varsling:} Valider prediksjoner for beleggsgrad og ventetider mot historikk, og definer terskelverdier som utløser beredskapsplanens trinnvise tiltak når bemanningen blir kritisk.
    \item \textbf{Læring og etikk:} Dokumenter avvik, pasientsikkerhetstiltak og brukermedvirkning i kvalitetsjournalen slik at faglige råd fra tilsyn og pasientutvalg blir sporbare.
\end{enumerate}

\section{Personvern, dataminimering og tilsynslogg}
Personvernregimet i helse- og energisektoren krever at valideringsaktivitetene dokumenterer både behandlingsgrunnlag og nødvendighet. Normen understreker at datatilgang skal begrenses til det som er strengt nødvendig for formålet, og at alle oppslag skal kunne spores gjennom revisjonslogger.\citep{norm2023} Når digitale tvillinger kombinerer pasientdata, produksjonsmålinger og syntetiske scenarier må modellteamet derfor vise hvordan hvert datasett støtter et konkret beslutningspunkt i kvalitetsjournalen.

\subsection{Personvernkonsekvensvurdering som del av V\&V}
Digitale tvillinger som behandler helse- eller personopplysninger med høy risiko skal gjennomføre en personvernkonsekvensvurdering (DPIA) før nye funksjoner tas i bruk. \citet{datatilsynet2023dpia} beskriver fire hovedelementer som bør inngå i kapittel 6 sitt valideringsløp:
\begin{enumerate}
    \item \textbf{Beskriv behandling og formål:} Kartlegg hvilke datasett som inngår i tvillingens opplæring og operativ drift, og gjør rede for hvorfor hvert felt er nødvendig for å oppnå ønsket beslutningsstøtte.
    \item \textbf{Vurder nødvendighet og forholdsmessighet:} Dokumenter hvordan dataminimering er implementert i datainnsamling, modelltrening og dashboards. For helsesektoren betyr det å vise at simuleringsvariabler ikke inkluderer identifiserende felter når aggregerte indikatorer er tilstrekkelige.
    \item \textbf{Analyse av risiko og tiltak:} Knyt identifiserte risikoer til sikkerhetstiltakene i standardkartet (Tabell~\ref{tab:standardkart}), inkludert tilgangsstyring, pseudonymisering og logging.
    \item \textbf{Plan for oppfølging:} Definer milepæler for nye vurderinger når tvillingen får tilgang til nye datakilder eller når algoritmene oppdateres. Resultatene bør forankres i fagfelleloggen og i styringssystemet for informasjonssikkerhet.
\end{enumerate}

\subsection{Innebygd personvern og etterlevelsesbevis}
Innebygd personvern innebærer å bake kontrolltiltak inn i arkitekturen fremfor å legge dem på toppen etterpå. \citet{datatilsynet2022innebygd} anbefaler at løsninger bruker standardinnstillinger som beskytter sluttbrukere, detaljerte tilgangsmatriser og tekniske sikringstiltak som begrenser eksport av rådata. For kapittel 6 bør dette oversettes til konkrete arkitekturkrav:
\begin{itemize}
    \item \textbf{Standardiserte rolleprofiler:} Tilgang til treningsdata, modelljournal og operasjonelle dashboards gis kun via roller som er forhåndsgodkjent av personvernombud og sikkerhetsleder.
    \item \textbf{Automatisert sporbarhet:} Revisjonslogg for modellendringer, datauttrekk og hendelseshåndtering lagres i et separat tilsynsregister som kan eksporteres til Datatilsynet eller Helsetilsynet ved behov. Loggen bør kobles til tiltaksplanen i Kapittel~7 for å sikre helhetlig styring.
    \item \textbf{Tidsavgrenset datalagring:} Sett eksplisitte tidsgrenser for hvor lenge trenings- og valideringsdata lagres, og verifiser sletting i retrospektive møter. Kontrollpunktene kan inngå som indikator i Tabell~\ref{tab:tillitsindikatorer}.
\end{itemize}

For energisektoren anbefales tilsvarende praksis i kontrolltårn-caset: adgang til hendelseslogger begrenses til operatører på vakt, og eksport av loggdata skjer gjennom godkjente sikkerhetskanaler. Ved å samle DPIA-resultater, tilgangskontroller og loggutdrag i én tilsynslogg kan virksomheten dokumentere etterlevelse av både personvernforordningen og NIS2-direktivets rapporteringskrav. Dette styrker tillitspanelene i kapittelet og gir studentene en konkret mal for hvordan juridiske krav integreres i modellstyring.

Tabell~\ref{tab:helsevalidering} viser hvordan elementene over kan struktureres i en valideringspakke som dekker både tekniske tester og pasientsikkerhet.

\begin{table}[ht]
    \centering
    \caption{Valideringspakke for pasientlogistikk i helsesektorens digitale tvilling}
    \label{tab:helsevalidering}
    \begin{tabular}{|p{3.6cm}|p{4.2cm}|p{4.2cm}|p{3.0cm}|}
        \hline
        \textbf{Kontrollpunkt} & \textbf{Målemetode} & \textbf{Dokumentasjon} & \textbf{Ansvarlig} \\
        \hline
        Datakvalitet og sporbarhet & Kryssjekk mot elektronisk pasientjournal og kommunale meldinger & Tilgangslogg, behandlingsgrunnlag, samtykkeskjema & Klinisk informasjonsforvalter \\
        \hline
        Prediksjonsnøyaktighet & Sammenlikn prognoser for liggetid og ventetid mot historiske målepunkter & Modelljournal, ukentlig rapport til beredskapsråd & Analytiker for pasientlogistikk \\
        \hline
        Beredskapsrespons & Øvelse av beredskapstrinn (grønn–rød) og eskalering til kommunal koordineringsenhet & Hendelseslogger, evalueringsnotat etter øvelse & Akuttleder \\
        \hline
        Personvern og etikk & Revisjon av tilgangsstyring, vurdering av automatiserte anbefalinger mot kliniske retningslinjer & ROS-analyse, beslutningsprotokoll fra pasientsikkerhetsutvalg & Personvernombud \\
        \hline
    \end{tabular}
\end{table}

Når tvillingen tas i bruk i undervisningslaboratoriet, bør studentgrupper teste scenarioer for akutt pågang, influensaepidemier og samtidige kommunale beredskapssituasjoner. Resultatene kobles til kontrolltårn-panelet fra energisektoren ved å oversette indikatorene til helsesektorens måleparametere, for eksempel andel planlagte utskrivinger som må flyttes og tid fra triage til første kliniske vurdering. Slik får studentene erfaring med å overføre strukturerte valideringsmetoder til et nytt domene og ser hvordan ansvarlige roller samarbeider for å sikre trygg tjenesteleveranse.

\section{Refleksjonsspørsmål og øvinger}
\begin{enumerate}
    \item Definer en valideringsstrategi for en digital tvilling i helsevesenet, og identifiser hvilke standarder og datasett som må håndteres for å dokumentere pasientsikkerhet.
    \item Beskriv hvordan du ville gjennomføre en usikkerhetsanalyse for en energimodell, inkludert valg av statistiske metoder og hvordan resultatene skal presenteres for driftsorganisasjonen.
    \item Diskuter tiltak for å ivareta etikk og personvern i prosjektet, og foreslå hvordan funnene kan formidles til både tekniske team og eksterne tilsynsmyndigheter.
\end{enumerate}
