\chapter{Data, integrasjon og infrastruktur}

\section{Læringsmål}
\begin{itemize}
    \item Beskrive arkitekturen for datafangst, lagring og distribusjon i digitale tvillinger.
    \item Evaluere integrasjonsmønstre og standarder.
    \item Utforme krav til sikkerhet, personvern og datasuverenitet.
\end{itemize}

\section{Dataflyt og pipeline-design}
En digital tvilling er avhengig av en gjennomtenkt datapipeline som kan fange, bearbeide og tilgjengeliggjøre informasjon med riktig kvalitet og tidsoppløsning. Det starter i feltet, hvor sensorer, styringssystemer eller manuelle registreringer skaper r å data. Dataene m å filtreres og normaliseres tidlig for  å unng å store forsinkelser senere i kjeden. Norske industrimiljøer som prosessindustrien p å Herøya har gode erfaringer med  å kombinere edge-noder som gjør grunnleggende forh å ndsprosessering med sentrale skyplattformer som tilbyr datalagring, modelltrening og visualisering.

Oversikten nedenfor oppsummerer hovedstrømmen fra feltniv å til de applikasjonene som konsumerer tvillingtjenestene. Integrasjonslaget markerer overgangen der normaliserte hendelser og masterdata eksponeres gjennom standardiserte grensesnitt og meldingsstrukturer, slik at etterfølgende plattformer kan bygges modulåe rt.

% Alt-tekst: support/figurer/metadata/kap03-datapipeline-v2.alt.md
\paragraph{Datapipeline i tekstform.} TikZ-grafikken er fjernet for å sikre grønn kompilering. Strukturen beskrives nå slik:
\begin{itemize}
    \item \textbf{Feltnivå:} Sensorer, SCADA og operatørinput leverer kontinuerlige data.
    \item \textbf{Edge- og gatewaylag:} Filtrerer, bufferer og oversetter protokoller før data sendes videre.
    \item \textbf{Integrasjonslag:} Meldingskøer, API-orkestrering og semantisk modellering sørger for at data standardiseres og distribueres.
    \item \textbf{Data- og analyseplattformer:} Datasjøer, tidsserielagre og modelltreningsmiljø kobles til styrings- og analyseapplikasjoner.
    \item \textbf{Forbrukere:} Digitale tvillinger, dashboards og styringssystemer bruker innsikten til operasjonelle beslutninger.
    \item \textbf{Styring på tvers:} Tilgangskontroll, hendelseslogging, datakatalog og modellforvaltning følger hele kjeden.
\end{itemize}

Koordineringen med fagfeller bygger på delingsnotatet \texttt{support/notater/datastyringsforum-di03.md}, og metadataene til den tidligere figuren ligger fortsatt i \texttt{support/figurer/metadata/kap03-datapipeline-v2.alt.md} slik at grafikkteamet og DI-03-teamet kan gjenoppta illustrasjonsarbeidet når kompilasjonsmiljøet er på plass.

\subsection{Fra sensor til innsikt}
Første steg er datainnsamling via feltbuss, industrielle IoT-gatewayer eller API-er fra eksterne systemer. For digital tvilling-bruk er det avgjørende  å definere sampling-rate, datastruktur og kontekst slik at hver m å ling kan kobles til riktig fysiske komponent. Inntaksleddet m å h å ndtere buffring nåar tilkoblingen faller ut, og bidra med enhetlige tidsstempler for  å muliggjøre felles analyse av hendelser. Videre bør pipeline-design inkludere datakvalitetsregler som fanger opp avvik, for eksempel ved  å merke data fra kalibreringsperioder eller vedlikehold.

\subsection{Batch kontra streaming}
Batchbehandling er egnet nåar dataene hovedsakelig brukes til periodisk rapportering eller modelloppdatering, mens streaming er nødvendig for operasjonelle beslutninger og avviksh å ndtering. Mange virksomheter kombinerer disse to: hendelser strømmes gjennom en meldingskøfor  å trigge alarmer og dashboards, samtidig som dataene landes i et datasjøfor tyngre analyser senere. Beslutningen bør dokumenteres i arkitekturbeskrivelsen slik at teamet vet hvilke forsinkelser og kostnader som forventes.

\subsection{Metadata, semantikk og masterdata}
Semantisk informasjon er nøkkelen til  å kunne dele data p å tvers av applikasjoner. Et felles begrepsapparat, for eksempel basert p å internasjonale referansemodeller eller bransjespesifikke ontologier, hjelper teamet med  å unng å tolkningstvister. Masterdata om utstyr, prosesser og lokasjoner m å holdes oppdatert, ellers mister tvillingen sin kobling til den fysiske virkeligheten. Ved  å etablere en dedikert katalog med API-tilgang kan andre prosjekter gjenbruke informasjonen og bidra til kvaliteten.

\section{Datakvalitetsstyring og observabilitet}
Digitale tvillinger leverer verdi først når beslutningstakere stoler på at dataene er presise, komplette og tidsriktige. ISO 25012 og ISO 8000 beskriver hvordan dataforvaltere kan kombinere kvalitetsdimensjoner og styringsprosesser for  å ivareta dette kontinuerlig \citep{iso25012-2014,iso8000-61-2016}. I en norsk kontekst må kravene forankres både teknisk og organisatorisk slik at industripartnere, myndigheter og leverandører kan verifisere at tvillingen er etterrettelig.

\subsection{Kvalitetsdimensjoner for tvillingdata}
Datastyringsforumet i DI-03 har brukt dimensjonene fra ISO-standardene som felles språk for å avklare ansvar på tvers av plattformteam, fagpersoner og sikkerhetsfunksjoner. I praksis prioriteres følgende områder:
\begin{itemize}
    \item \textbf{Nøyaktighet og gyldighet:} Sanntidsmålinger må sammenlignes mot kalibreringsjournaler og driftsgrenser før de brukes til modelloppdatering.
    \item \textbf{Fullstendighet:} Hendelseslogger og kontekstdata må være komplette for at simuleringsresultater skal tolkes riktig, spesielt når partnere deler data via dataspace.
    \item \textbf{Aktualitet:} Latens gjennom pipeline må dokumenteres slik at dashboards og automatiserte beslutninger viser situasjonen i riktig tidsvindu.
    \item \textbf{Sporbarhet:} Hver datapost skal ha opprinnelse, prosesseringssteg og gjeldende kontrakt dokumentert slik at avvik kan reverseres uten å stoppe produksjonen.
\end{itemize}

Dimensjonene knyttes direkte til læringsmålene i dette kapittelet ved at studentene skal kunne kombinere tekniske kontroller med styringsprosesser. I undervisning brukes de som sjekkliste for casearbeid med energi- og mobilitetsdata.

\subsection{Operasjonelle kontrollpunkter}
Kontrollene implementeres som en serie kvalitetssjekker i pipeline. Flere av sjekkene er koordinert gjennom delingsnotatet \texttt{support/notater/datastyringsforum-di03.md} og testes i pilotmiljøet før produksjonssetting. Tabell~\ref{tab:kap03-datakvalitet} beskriver sentrale kontrollpunkter og hvem som eier dem.

\begin{table}[ht]
    \centering
    \caption{Kontrollpunkter for datakvalitet i en digital tvilling-plattform.}
    \label{tab:kap03-datakvalitet}
    \begin{tabular}{p{0.28\textwidth}p{0.44\textwidth}p{0.20\textwidth}}
        \toprule
        \textbf{Kontrollpunkt} & \textbf{Formål} & \textbf{Ansvarlig funksjon} \\n        \midrule
        Inntaksvalidering & Automatisk sjekk av format, gyldige verdier og sensorstatus før hendelser publiseres videre. & Dataingeniør/edge-ansvarlig \\n        Strømobservasjon & Kontinuerlig måling av latens, pakketap og verdidrift for sanntidsstrømmer. & Plattformteam \\n        Kontekstforankring & Krysskobling mellom masterdata, kontrakter og dataspace-policy før data deles eksternt. & Data steward \\n        Modellfeedback & Registrering av modellavvik og automatiske retreningskøer når kvalitetsgrenser overskrides. & Modellansvarlig \\n        Etterlevelseslogg & Dokumentasjon av avvik, tiltak og varsling i tråd med NIS2- og personvernkrav. & Sikkerhets- og compliance-team \\n        \bottomrule
    \end{tabular}
\end{table}

Kontrollene må dokumenteres og testes jevnlig. For hver lansering av en ny datakilde legges det inn en endringsordre med forventede kvalitetsgrenser, testscenarioer og plan for revert dersom kravene ikke oppfylles. Resultatene rapporteres tilbake til datastyringsforumet og gjenspeiles i fagfelleloggen.

\subsection{Observabilitet og læringssløyfer}
Observabilitet sørger for at kvalitetsindikatorene faktisk overvåkes og at tvillingen tilpasser seg når noe går galt. En praktisk fremgangsmåte bygger på tre trinn:
\begin{enumerate}
    \item Definer tjenestenivåmål (SLO) for dataforsinkelse, kvalitetsscore og tilgjengelighet. KPI-ene publiseres i samme dashboard som bærekraftsindikatorene slik at ledelsen får helhetlig oversikt.
    \item Automatiser alarmer, hendelsesregistrering og kommunikasjon til beredskapsplanen når SLO-ene brytes. Hendelser får tydelig ansvarlig og frist for korrigering.
    \item Knytt læringssløyfer til modellene ved å bruke avvik som input til feilanalyse, nye sensorbehov eller oppdatering av dataspace-kontrakter.
\end{enumerate}

\paragraph{Praktisk sjekkliste.} Før en pilot settes i drift, bør teamet bekrefte at indikatorene over kan testes ende-til-ende i testmiljøet, at varslingsrutiner er synkronisert med beredskapsplanen i dette kapittelet, og at partnere i dataspace har signert på hvilke kvalitetsmålinger de forventer. Dette sikrer at datakvalitet ikke blir en engangsaktivitet, men en kontinuerlig del av plattformstyringen.

\subsection{Operativ driftsmøtemodell for dataspace-tilknyttede tvillinger}
Norske pilotmiljøer som Statnett, Energi Norge og kommunale datasamarbeid organiserer driften rundt faste møtearenaer slik at dataavvik, modelltiltak og kontraktskrav fanges opp før de får følgefeil i styringen.\citep{statnett2023digital,energinorge2023beredskap} Driftsmøtene bygger videre på modelljournal- og dataspace-prinsippene fra \autoref{tab:kap03-datakvalitet} og \autoref{tab:dataspace-governance}, og sørger for at loggene fra kapittel~6 og fagfelleloggen holdes oppdatert.\citep{digdir2023modelljournal}

En anbefalt møtesyklus består av tre nivåer:
\begin{enumerate}
    \item \textbf{Ukentlig hendelsesreview:} Plattformteamet og dataspace-operatøren gjennomgår avvik i strømobservasjon og etterlevelseslogg. Tiltak prioriteres og testes i sandkasse før de tas inn i kontrolltårnet i kapittel~6.
    \item \textbf{Månedlig dataspace-forum:} Domeneeiere, sikkerhetsfunksjoner og programleder samkjører indikatorpanelet i \autoref{tab:kommunaleindikatorer} og \autoref{tab:kap06-tilsynsplan} med pågående endringer i dataspace-kontraktene. Resultatet dokumenteres i fagfelleloggen og i gevinstplanen i kapittel~7.
    \item \textbf{Kvartalsvis styringsråd:} Strategiske beslutningstakere bekrefter at gevinstmål, policy og investeringer henger sammen. Rådet vedtar eventuelle eskaleringer eller prioriteringer for neste kvartal og bestiller revisjoner ved behov.
\end{enumerate}

Tabell~\ref{tab:dataspace-drift} kan brukes som mal når studentgrupper skal planlegge sine egne driftsmøter. Tabellen kobler agenda, inndata og leveranser slik at møtene leverer konkrete forbedringer.

\begin{table}[ht]
    \centering
    \caption{Driftsmøtemal for dataspace-tilknyttede tvillinger}
    \label{tab:dataspace-drift}
    \begin{tabular}{p{3.2cm}p{4.6cm}p{4.0cm}p{3.4cm}}
        \toprule
        Møte & Hovedfokus & Kritiske inndata & Utfall og oppfølging \\
        \midrule
        Ukentlig hendelsesreview & Analyse av avvik fra \autoref{tab:kap03-datakvalitet} og prioritering av midlertidige tiltak. & Hendelseslogg, kvalitetsmålinger og modelljournalutdrag. & Tiltak loggføres med ansvarlig og testplan i sandkasse. \\
        Månedlig dataspace-forum & Samordne indikatorpanel, tilgangspolicy og planlagte endringer i konnektorer. & KPI-rapport fra kontrolltårn, dataspace-kontrakter og fagfellelogg. & Oppdatert tiltakslogg og versjonsnotat til deltakerne. \\
        Kvartalsvis styringsråd & Vurdere gevinstrealisering, risiko og investeringsbehov. & Gevinstplan, rapporteringspakke fra kapittel~6 og revisjonsfunn. & Beslutningsprotokoll med prioriterte forbedringer og eventuelle eskaleringer. \\
        \bottomrule
    \end{tabular}
\end{table}

Når møtene dokumenteres i samme versjonskontroll som tvillingmodeller og dataspace-artefakter, blir det enklere å vise samsvar med kravene i \autoref{tab:kap06-tilsynsplan} og å koordinere tiltak med styringsmodellene i kapittel~7. Studentteam bør bruke malen aktivt når de planlegger pilotundervisning slik at gevinstplan, datakvalitet og regulatorisk etterlevelse henger sammen i praksis.


\subsection{Case: Sanntidsobservabilitet i kraftnettet}
Statnett har brukt digital tvilling-teknologi for å overvåke sanntidsbelastningen i transmisjonsnettet og gi driftssenteret et felles situasjonsbilde \citep{statnett2023digital,statnett2024kontrolltarn}. Caset bygger på PMU-strømmer, SCADA-data og vedlikeholdslogger som kombineres i et digitalt kontrolltårn. Erfaringene viser at observabilitet må sikres både teknisk og organisatorisk for å unngå at avvik forplanter seg i kraftsystemet.

Tre designgrep er testet i pilotene:
\begin{itemize}
    \item \textbf{Felles indikatorbibliotek:} Driftsteamet, cyberberedskap og modellutviklere bruker samme definisjon av latens, datadrift og måletetthet, slik at hendelser tolkes likt på tvers av funksjoner.
    \item \textbf{Sandkasse for hendelser:} Nye alarmer og SLO-er verifiseres mot historiske feilscenarier før de aktiveres i produksjon. Denne praksisen hindrer alarmutmattelse i beredskapsrommet.
    \item \textbf{Samsvar med dataspace-policy:} Når data deles med regionale nettselskaper eller forskningspartnere, publiseres indikatorene i samme katalog som tilgangsavtaler og samtykker.
\end{itemize}

Tabell~\ref{tab:kap03-observabilitet-kraft} viser et utdrag av indikatorene som brukes i driftssenteret. Hvert målepunkt er koblet til en forventet terskel og beskriver hvilket team som leder responsen dersom verdien havner utenfor normalområdet.

\begin{table}[ht]
    \centering
    \caption{Observabilitetsindikatorer for Statnett-inspirert digital tvilling.}
    \label{tab:kap03-observabilitet-kraft}
    \begin{tabular}{p{0.26\textwidth}p{0.42\textwidth}p{0.26\textwidth}}
        \toprule
        \textbf{Metrikk} & \textbf{Overvåkingspraksis} & \textbf{Tiltak ved avvik} \\
        \midrule
        Latens i PMU-strømmer & Måles per node og publiseres i dashbord med 5-sekunders oppløsning. & Plattformteam eskalerer til telekomleverandør og initierer failover mot redundante linjer. \\
        Datadrift i fasevinkel & Automatisk analyse sammenligner mot referansemodeller og vedlikeholdslogger. & Modellansvarlig rekalkulerer parametre og flagger potensielle feil i sanntidsmodellen. \\
        Tilgjengelighet for kontrolltårn & Kontinuerlig syntetisk testbruker validerer API, graf og varslingskjede. & Beredskapsteam aktiverer manuelle rutiner og informerer situasjonssenteret. \\
        Hendelsesjournal & Alle alarmer signeres kryptografisk og synkroniseres med dataspace-avtaler. & Compliance-funksjonen vurderer rapporteringsplikt mot NVE og Statnetts situasjonssenter. \\
        \bottomrule
    \end{tabular}
\end{table}

Indikatorene forvaltes gjennom en ukentlig tavle der driftssenteret, dataspace-operatøren og sikkerhetsfunksjonen koordinerer tiltak. Resultatet er at modelloppdateringer og nettoperasjon blir tett koblet til observabilitetsarbeidet, samtidig som partnere kan etterprøve datakvaliteten når de gjenbruker strømmer fra kontrolltårnet.

\subsection{Case: Energideling i lokale energisamfunn}
Kommuner og eiendomsselskap etablerer nå lokale energisamfunn der solcelleanlegg, varmepumper, batterier og ladeinfrastruktur deler energi gjennom felles kontrakter. For å lykkes trenger de en dataspace som binder sammen tekniske målepunkt, avtaleregulering og bærekraftsrapportering. Et digitalt tvillingoppsett kan kombinere AMS- og sensordata med simuleringer av energilast for å beregne fordelingsnøkler i sanntid, styre lagring og dokumentere klimanytte.

En energidelingsdataspace må håndtere ulike roller: nettselskap leverer måleverdier og nettrelaterte begrensninger, bygningseier forvalter lokal produksjon og laster, og deltakerne trenger innsyn i fordelingsreglene. Kontraktene må derfor synkroniseres mot dataspace-policyene i \autoref{tab:dataspace-governance}, samtidig som tiltak og avvik følges opp i gevinstplanen i kapittel~7. Når tvillingen fordeler energi eller kostnader automatisk, skal beslutningene gå gjennom kontrollpunktene i kapittel~6 for å sikre etterlevelse av NVE-regler og støtteordninger.

Tabell~\ref{tab:energideling-datakrav} viser et minimumssett av dataprodukter som bør inngå i energidelingsdataspace. Kolonnene gir studentgrupper en sjekkliste for å dokumentere både tekniske og organisatoriske krav før pilotering.

\begin{table}[ht]
    \centering
    \caption{Dataprodukter og styringspunkter for energidelingsdataspace.}
    \label{tab:energideling-datakrav}
    \begin{tabular}{p{3.4cm}p{4.6cm}p{4.0cm}}
        \toprule
        \textbf{Dataprodukt} & \textbf{Formål og nøkkelinnhold} & \textbf{Styringskobling} \\
        \midrule
        Måleverdier fra AMS og underfordeling & Fem-minutters energiuttak og produksjon per leilighet, felleslast og felles solcellefelt med tidsstempel og kvalitetsflagg. & Kontrolltårnindikatorer i kapittel~6 og tiltakslogg ved avvik. \\
        \addlinespace
        Fleksibilitets- og lastprofiler & Simulerte scenarier for effekt-topper, lagring og styring av varmepumper/EV-lading. & Simulerings- og gevinstramme i kapittel~4 og kapittel~7. \\
        \addlinespace
        Avtale- og fordelingsregister & Kontraktsparametere, kostnadsnøkler og bærekraftsfordeling for deltakere og fellesanlegg. & Dataspace-governance i \autoref{tab:dataspace-governance} og læringsmål i kapittel~7. \\
        \addlinespace
        Klimaregnskap og indikatorpanel & CO$_2$-reduksjon, egenforbruk, grad av lokal energiforsyning og økonomiske gevinster per deltaker. & Rapportering til bærekraftstabellene i kapittel~7 og scenariopanelet i kapittel~8. \\
        \bottomrule
    \end{tabular}
\end{table}

Før energisamfunnet settes i drift bør prosjektteamet teste dataspace-løsningen i en sandkasse sammen med nettselskap og leverandører av energistyringssystemer. Det innebærer å verifisere at måleverdiene kan aggregeres og anonymiseres før deling, at algoritmene for fordeling kan revideres gjennom modelljournalen i kapittel~2, og at gevinstmålingen harmoniseres med klimamålene som beskrives i kapittel~8. Når kontrollene er dokumentert, kan energisamfunnet bruke tvillingen til å planlegge investeringer, rapportere klimaeffekt og identifisere nye fleksibilitetstjenester.

\subsection{Case: Kraftberedskapsdataspace ved ekstremvær}
Statnett, nettselskap og beredskapsmyndigheter har de siste årene etablert nye mekanismer for å håndtere risikoen for samtidige feil i strømforsyningen under ekstremvær.\citep{statnett2023vinter,dsb2022kraft} Når ising, storm og høyt forbruk forekommer samtidig, må energiselskapene dele sanntidsdata om produksjon, belastning og avvik slik at rasjoneringstiltak og reserveplaner kan aktiveres i tide. Norges vassdrags- og energidirektorat (NVE) krever at kraftforsyningsvirksomheter dokumenterer både tekniske tiltak og informasjonsdeling i kraftberedskapsforskriften, og peker spesielt på behovet for felles situasjonsbilder som kan etterprøves i øvelser.\citep{nve2023kraftberedskap}

Et kraftberedskapsdataspace bygger videre på prinsippene fra kontrolltårnet i kapittel~6 og gevinstoppfølgingen i kapittel~7. Følgende kapabiliteter bør være på plass før pilotering:
\begin{itemize}
    \item \textbf{Felles produksjons- og lastprognoser:} Statnett, regionalt nettselskap og større produksjonsaktører publiserer harmoniserte prognoser for produksjon, import, effektuttak og flaskehalser i nettet.
    \item \textbf{Reservestatus og tiltakspakker:} Automatisk synkronisering av kraftverk, mobile aggregater og energilagre med tilgjengelig kapasitet, kostnad og aktiveringstid.
    \item \textbf{Beslutningsjournal for rasjonering:} Alle beslutninger om redusert forbruk, tilgjengelig kraft og prioritering av samfunnskritiske kunder loggføres med begrunnelse og tidsstempel, slik at tiltakene kan evalueres i etterkant.
    \item \textbf{Rapportering til tilsyn:} Standardiserte rapporter over hendelser, indikatorer og tiltak deles med NVE og Direktoratet for samfunnssikkerhet og beredskap (DSB) etter malene i \autoref{tab:dataspace-governance} og kontrolltårnstrukturen i kapittel~6.
\end{itemize}

Tabell~\ref{tab:kraftberedskap-dataspace} strukturerer de viktigste dataproduktene for studentgrupper som skal beskrive hvordan dataspace-løsningen brukes i praksis.

\begin{table}[ht]
    \centering
    \caption{Dataprodukter og indikatorer i kraftberedskapsdataspace.}
    \label{tab:kraftberedskap-dataspace}
    \begin{tabular}{p{3.6cm}p{4.8cm}p{3.2cm}}
        \toprule
        \textbf{Dataprodukt} & \textbf{Innhold og delingsformål} & \textbf{Styringskobling} \\
        \midrule
        Sanntidslast og produksjonsmålinger & SCADA-verdier for linjer, transformatorer og produksjonsenheter, aggregert etter region og kritikalitet. & Kontrolltårnindikatorer og hendelsesjournal i kapittel~6; modelloppdatering i kapittel~4. \\
        \addlinespace
        Reservestatus og fleksibilitetsregister & Tilgjengelig effekt fra reservekraftverk, batterier og forbrukerfleksibilitet med aktiveringstid og pris. & Gevinst- og tiltakslogg i kapittel~7; samsvar med dataspace-policyer i \autoref{tab:dataspace-governance}. \\
        \addlinespace
        Ekstremvær- og risikoanalyse & Prognoser fra Meteorologisk institutt kombinert med nettmodell og historiske feil for å identifisere høyrisiko-soner. & Beredskapsplanen i kapittel~6 og scenarioarbeid i kapittel~8. \\
        \addlinespace
        Rasjoneringstiltak og prioriteringsjournal & Dokumentasjon av aktiverte tiltak, berørte kunder og forventet varighet, samt referanse til juridisk grunnlag. & Etterlevelseslogg i kapittel~6 og gevinstoppfølging i kapittel~7. \\
        \bottomrule
    \end{tabular}
\end{table}

Før dataspace-løsningen tas i bruk i drift anbefales det å gjennomføre en tverrsektoriell øvelse der energisektoren kobler seg på samvirkeøvelsen i kapittel~6. Det innebærer å teste varsling og beslutningsjournal sammen med kommunal beredskap, helse og samferdselsaktører, og validere at indikatorene i tabellen over kan rapporteres både til interne kontrolltårn og tilsynsmyndigheter. Resultatene bør registreres i fagfelleloggen og brukes til å justere både risikoprioritering, kontrakter og gevinstplaner før neste vintersesong.

\subsection{Case: Samvirke og beredskapsdataspace for nødetatene}
Felles innsats mellom brann, politi, helse og frivillige redningsressurser krever at alle aktører deler sanntidsdata, historikk og beslutningslogger uten at sensitiv informasjon lekkes. Direktoratet for samfunnssikkerhet og beredskap (DSB) legger i sine samvirkeprinsipper til grunn at Nødnett, fagsystemer og situasjonssentre må knyttes sammen i et nasjonalt operativt konsept med tydelig styringssløyfe og journalføring.\citep{dsb2023samvirke} Politidirektoratet peker samtidig på behovet for å kombinere tale- og meldingslogg fra Nødnett med geodata, operasjonsplaner og dronefeeds i et felles beslutningsdashboard.\citep{politidirektoratet2022nodnett} Når helseforetakene kobler inn ambulanser, AMK-sentraler og sykehuslogistikk, stiller Helsedirektoratet krav om at personvern og tilgangsstyring følger beredskapsplanene som beskrives i kapittel~6.\citep{helsedir2022akutt}

Et samvirkedataspace for nødetatene kan strukturere leveransene gjennom følgende komponenter:
\begin{itemize}
    \item \textbf{Operativ datadeling:} Hendelser, ressursstatus, posisjon og risikovurderinger fra politiets PO-logg, 110- og 113-sentralene synkroniseres via standardiserte hendelsesmodeller og publiseres til felles situasjonsbilde.
    \item \textbf{Beslutningsjournal:} Alle beslutninger, varsler og prioriteringer loggføres med tidsstempel, ansvar og referanse til gjeldende planverk. Journalen speiler kontrolltårn-loggene fra kapittel~6 og styringspanelet i kapittel~7.
    \item \textbf{Læring og øving:} Data fra øvelser og etterhåndsanalyser registreres i samme dataspace slik at indikatorer og tiltak kan måles på tvers av sektorer og bidra til kontinuerlig forbedring.
\end{itemize}

Tabell~\ref{tab:samvirke-dataspace} viser et minimumssett med datakilder og styringspunkter som må være på plass før pilotering. Tabellen kobler tydelig hvilke krav som må testes i dataspace-sandkassen i kapittel~6 og hvilke tiltak som må følges opp i gevinstplanen i kapittel~7.

\begin{table}[ht]
    \centering
    \caption{Datakilder og styringspunkter for nødetatenes samvirkedataspace.}
    \label{tab:samvirke-dataspace}
    \begin{tabular}{p{3.6cm}p{5.0cm}p{3.4cm}}
        \toprule
        \textbf{Datakilde} & \textbf{Delingsformål} & \textbf{Styrings- og beredskapskobling} \\
        \midrule
        Nødnett logg og taleregistrering & Sanntidsinformasjon om talegrupper, sambandsutnyttelse og prioriteringer, inkludert nødmeldinger og statuskoder. & Eskalering og ressursstyring i kontrolltårnet fra kapittel~6; audittrail for etterhåndsanalyser og revisjoner. \\
        \addlinespace
        Geoposisjon og ressursstatus & GPS-sporing for kjøretøy, mannskap og droner, inkludert tilgjengelige spesialressurser og estimert ankomsttid. & Sanntidsdisponering og gevinstmåling knyttet til indikatorpanelet i kapittel~7; input til beslutningsstøtte i kapittel~4. \\
        \addlinespace
        Risikoprofil og felles situasjonsbilde & Aggregert oversikt over fareområder, vær, kritisk infrastruktur, evakueringssoner og pågående tiltak. & Samvirkeplaner, avvikshåndtering og kommunikasjonstiltak i kapittel~6; kobling til planens scenarioarbeid i kapittel~8. \\
        \addlinespace
        Helse- og pasientlogg for akuttjenester & Anonymiserte oversikter over pasientstrøm, triage og kapasitetsstatus i akuttmottak. & Personvern- og tilgangskontroller fra kapittel~6; gevinst- og forbedringslogg for helsesektorcaset i kapittel~8. \\
        \bottomrule
    \end{tabular}
\end{table}

Ved å implementere tabellen i pilotprosjekter kan nødetatene dele et felles datasett for øvelser og reelle hendelser. Studentgrupper bør bruke caset til å øve på å modellere tilgangsstyring, logging og indikatorer slik at tekniske beslutninger henger sammen med beredskapsplaner, gevinstoppfølging og læring på tvers av kapitlene.

\subsection{Case: Overvannsberedskap i urbane områder}
Oslo, Bergen og flere andre kommuner bygger digitale tvillinger for overvannssystemer for å oppfylle kravene i egne strategier og NVE sine veiledere for klimatilpasning.\citep{oslo2022overvann,nve2021overvann} Tvillingen kombinerer sensordata fra kummer, vannføringsmålinger, radarbaserte nedbørprognoser og terrengmodeller for å identifisere flomsoner i sanntid. Ved å bruke dataspace-prinsipper kan kommunen dele aggregerte situasjonsrapporter med entreprenører, samvirke i interkommunale vannverk og koble hendelser direkte til beredskapsplanen.

En robust løsning krever at dataene beskrives med felles semantikk og tydelige ansvarslinjer:
\begin{itemize}
    \item \textbf{Hydrologiske tidsserier:} Sensorer i spillvannskummer og bekkeinntak publiserer data hvert minutt. Aggregatene merkes med kvalitetsscore og knyttes til masterdata for ledningsnett.
    \item \textbf{Digitale terrengmodeller:} LIDAR- og BIM-data danner grunnlaget for simulering av vannveier, og oppdateres når nye byggeprosjekter registreres i kommunens planportal.
    \item \textbf{Eksterne prognoser:} Meteorologiske data og flomvarsler hentes via API-er fra MET Norge og NVE og brukes til å kalibrere scenarioer for de neste timene.
    \item \textbf{Tiltakslogg og kommunikasjon:} Hendelser kobles til tiltakslogg, varslingsplaner og publikumsinformasjon slik at beredskapsleder raskt kan bestille tiltak og varsle innbyggere.
\end{itemize}

Tabell~\ref{tab:kap03-overvann} viser hvordan de viktigste datastrømmene kan organiseres for å støtte både operativ håndtering og oppfølging i styringsfora.

\begin{table}[ht]
    \centering
    \caption{Datastrømmer og styringspunkter for kommunal overvannsberedskap.}
    \label{tab:kap03-overvann}
    \begin{tabular}{p{3.6cm}p{4.6cm}p{3.6cm}}
        \toprule
        \textbf{Datakilde} & \textbf{Innhold og oppdateringsfrekvens} & \textbf{Styring og beslutning} \\
        \midrule
        Sensorer i kritiske kummer & Nivå, temperatur og overvannsvolum hvert minutt med kvalitetsflagg. & Varsles i kontrolltårnet og utløser tiltak etter \autoref{tab:kap06-tilsynsplan}. \\
        Radar- og varslingsdata & Nedbørintensitet, snøsmelting og flomprognoser fra NVE og MET Norge med 5--15 minutters intervall. & Oppdaterer scenariobiblioteket i kapittel~4 og prioriterer beredskapsressurser. \\
        Terreng- og ledningsmodeller & Oppdatert geometri for vannveier, kapasitet og kritiske flaskehalser. & Beslutningsunderlag til investeringsplaner og tiltakslogg i \autoref{tab:tiltakslogg}. \\
        Operatør- og entreprenørlogg & Feltobservasjoner, tiltak og bilder fra utrykning. & Dokumenteres i fagfelleloggen og gir læringspunkter til kapittel~7. \\
        \bottomrule
    \end{tabular}
\end{table}

\paragraph{Arbeidsflyt for masterstudentene.} Caset kan brukes til å la studentgrupper designe en ende-til-ende datastrøm der registrerte hendelser kobles til simuleringene fra kapittel~4 og rapporteringen i kapittel~6. Et anbefalt opplegg er å la teamene (1) beskrive datakvalitetskontroller for hver kilde, (2) modellere beslutningspunktene i tiltaksloggen og (3) definere hvilke indikatorer som skal publiseres til gevinst- og bærekraftspanelene i kapittel~7. Resultatene bør dokumenteres i både modelljournal og fagfellelogg slik at kommunen kan gjenbruke arbeidsflyten i faktiske øvelser.

\subsection{Case: Naturfare-dataspace for skred og flom}
Fjellkommuner og samvirkeorganer langs kysten må kombinere skred- og flomvarsling med sanntidsobservasjoner for å prioritere evakuering, infrastrukturtiltak og ressursdisponering. Norges vassdrags- og energidirektorat (NVE) beskriver i sin nasjonale skredfarestrategi hvordan sensornettverk, manuelle observasjoner og modellprognoser skal inngå i et felles situasjonsbilde.\citep{nve2024skredstrategi} Varsom-tjenesten samler flom- og skredvarsler fra MET Norge og NVE og stiller API-er til rådighet som kan integreres direkte i digitale tvillinger.\citep{varsom2024api} Kartverket og kommunene publiserer høyoppløselige terrengmodeller og planregister som gjør det mulig å modellere påvirkningssoner og kritiske objekter.\citep{kartverket2023ndh}

Et naturfare-dataspace binder disse datakildene sammen med beredskapsjournalen fra kapittel~6 og gevinstarbeidet i kapittel~7 gjennom fire hovedkomponenter:
\begin{itemize}
    \item \textbf{Varslingsgrunnlag:} Varsom-API, meteorologiske prognoser og hydrologiske modeller strømmer inn i dataspace med kvalitetsscore og tidsstempler slik at avvik kan spores.
    \item \textbf{Feltobservasjoner:} Skredsensorer, kameraer, droner og manuelle rapporter fra kommunale beredskapsgrupper lastes opp med geotaggede metadata og knyttes til tiltak i tiltaksloggen.
    \item \textbf{Terreng og kritisk infrastruktur:} Høydedata, eiendomsregister, vegnettinformasjon og kritiske objekter (kraftlinjer, vannverk, fiber) kobles til scenarioene fra kapittel~4 slik at konsekvensanalyser kan kjøres i sanntid.
    \item \textbf{Samvirke- og kommunikasjonslogg:} Beslutninger, varsler og ressursdisponering dokumenteres i samme struktur som samvirkedataspace-caset over, slik at rapportering til statsforvalter og DSB kan gjennomføres uten ekstra datavask.
\end{itemize}

Tabell~\ref{tab:kap03-naturfare} gir en oversikt som masterstudentene kan bruke når de beskriver roller, datasett og indikatorer for naturfarepilotene.

\begin{table}[ht]
    \centering
    \caption{Datastrømmer og styringspunkter i naturfare-dataspace for skred og flom.}
    \label{tab:kap03-naturfare}
    \begin{tabular}{p{3.6cm}p{4.8cm}p{3.4cm}}
        \toprule
        \textbf{Datakilde} & \textbf{Innhold og frekvens} & \textbf{Styring og beslutning} \\
        \midrule
        Varsom- og MET-varsler & Flom- og skredfaregrad, usikkerhet og anbefalte tiltak oppdatert 6--24 ganger per døgn. & Oppdaterer scenario- og beredskapsplaner i kapittel~4 og kontrolltårnindikatorene i kapittel~6. \\
        \addlinespace
        In-situ sensornett & Radar, geofoner, poretrykksmålere, snødybdestaker og kamera med kontinuerlig strømming. & Varsler feltmannskap, initierer tiltakslogg i \autoref{tab:kap06-tilsynsplan} og dokumenterer beslutninger i fagfelleloggen. \\
        \addlinespace
        Terreng- og planregister & FKB-data, detaljreguleringer, kritisk infrastruktur og evakueringspunkter. & Knytter modellresultater til konsekvensklassifisering i kapittel~7 og rapportering til statsforvalter. \\
        \addlinespace
        Ressurs- og samvirkelogg & Beredskapsvaktlister, utstyr, frivillige ressurser og kommunikasjon via samvirkeplattformen. & Synkroniseres med dataspace-samvirket i \autoref{tab:samvirke-dataspace} og deler status med kapittel~8 sine caser. \\
        \bottomrule
    \end{tabular}
\end{table}

\paragraph{Arbeidsopplegg.} Studentteam kan la caset fungere som en iterativ øvelse der (1) data fra Varsom og lokale sensorer hentes inn i sandkassen fra kapittel~6, (2) scenarioer for skredbaner og flomløp modelleres i samspill med kapittel~4 sine simuleringsverktøy, og (3) tiltak og læringspunkter dokumenteres i gevinst- og tiltaksloggene i kapittel~7. Resultatet bør være et delt dashboardskjema som viser hvilke indikatorer som må være grønne før normal drift kan gjenopptas etter en hendelse.

\section{Integrasjonsmønstre og standarder}
Et integrasjonslandskap for digitale tvillinger spenner fra enkle API-kall til kompleks hendelsesdrevet samhandling. God praksis er  å kartlegge datastrømmer, volum og krav til robusthet før man velger teknologier.

\subsection{Arkitekturvalg}
Tradisjonelle REST-API-er gir tydelig kontraktstyring og passer for forespørsel-/svar-scenarioer, men bør suppleres med publish/subscribe-mekanismer nåar flere systemer trenger de samme sanntidsdataene. Hendelsesdrevne arkitekturer med meldingskøer eller loggstrømmer (for eksempel Apache Kafka) gir bedre skalerbarhet og kan forenkle revisjon, fordi alle hendelser lagres i riktig rekkefølge. For kritiske styringssystemer bør man ogs å vurdere redundans og fallback-løsninger, slik at tapte meldinger ikke medfører sikkerhetsrisiko.

\subsection{Standarder i norsk praksis}
OPC UA er utbredt i norsk industri fordi det forener datapublisering med semantiske modeller. MQTT er lettere og passer godt nåar batteri- eller nettverkshensyn krever minimal overhead, som i maritime anvendelser. Asset Administration Shell (AAS) f å r støtte gjennom europeiske initiativer og gir et strukturert format for  å beskrive digitale representasjoner av produkter og systemer. Ved  å kombinere disse standardene kan man bøde integrere eldre automasjonssystemer og dele data med eksterne partnere.

\subsection{Datakvalitet og interoperabilitet}
Integrasjonen m å inkludere kontrollpunkter for datakvalitet, spesielt nåar dataene brukes til modelloppdatering eller automatiserte beslutninger. Versjonering av datakontrakter og testmiljøer der integrasjoner valideres før produksjonssetting reduserer risiko for feil. Dokumentasjon av avhengigheter og kontaktpunkter gjør det enklere  å etablere ansvar for h å ndtering av databrudd eller uforutsette endringer.

\section{Dataspace-arkitektur og samhandling}
Norge deltar i flere europeiske dataspace-initiativ, og mange virksomheter har begynt å etablere felles dataplattformer som
knytter industrielle tvillinger på tvers av selskaper. Gaia-X og International Data Spaces Association (IDSA) gir rammen for
hvordan teknisk arkitektur, tillitsmekanismer og policy-regler bør utformes \citep{gaiax2023architecture,idsa2023ram}. For å
sikre at norske aktører kan koble seg på disse økosystemene uten å gi fra seg kontroll på data, er det nyttig å beskrive
dataspace-arkitekturen lag for lag, slik Tabell~\ref{tab:kap03-dataspace-lag} viser.

\begin{table}[ht]
    \centering
    \caption{Lagdeling i en norsk dataspace-arkitektur for digitale tvillinger.}
    \label{tab:kap03-dataspace-lag}
    \begin{tabular}{p{0.28\textwidth}p{0.62\textwidth}}
        \toprule
        \textbf{Lag} & \textbf{Formål og tiltak} \\
        \midrule
        Deltakerforvaltning & Registrering av virksomheter, utstedelse av identiteter og sertifikater, samt avtaler om datadeling. \\
        Tilgangskontroll & Policy-håndheving via konnektorer som sikrer at datadelingen følger kontrakter og sanksjoner ved brudd. \\
        Semantikk og katalog & Felles begreper, datasettbeskrivelser og API-spesifikasjoner publiseres i søkbare kataloger. \\
        Datastrømmer & Konfigurasjon av sanntids- og batchkanaler med logging, kryptering og datasuverenitetsregler. \\
        Tjenester & Analyse- og simuleringsapplikasjoner som kan kjøres nær dataene og kobles til tvillingene i kapittel 4 og 5. \\
        \bottomrule
    \end{tabular}
\end{table}

Erfaringer fra mobilitetsdataspace-programmet i EU viser at det er avgjørende med tidlig avklaring av roller, tekniske krav og
juridiske mekanismer \citep{ec2023mobilitydataspace}. I Norge bør bransjeorganisasjoner og klynger definere et minimumssett med
policy-regler for datadeling slik at energiselskaper, offentlige etater og leverandører kan koble seg på uten omfattende
bilaterale avtaler. Ved å bruke referansearkitekturene kan pilotprosjekter gjenbruke sertifiseringsprosesser, logging og
komponenter for dataminimering.

\subsection{Norske roller og styringsmodeller}
Et vellykket dataspace krever kombinasjon av tekniske og organisatoriske roller. Tabellen under viser en forenklet matrise som
har blitt testet i norske pilotprosjekter i kraft- og mobilitetssektoren.

\begin{table}[ht]
    \centering
    \caption{Ansvarsfordeling i dataspace-piloter.}
    \label{tab:kap03-dataspace-ansvar}
    \begin{tabular}{p{0.32\textwidth}p{0.58\textwidth}}
        \toprule
        \textbf{Rolle} & \textbf{Hovedansvar} \\
        \midrule
        Dataspace-operatør & Drifter konnektorer, policy-tjenester og sertifiseringsmekanismer for deltakerne. \\
        Domeneeier & Definerer semantikk, datasettprioritet og kvalitetssikrer modellene som bruker informasjonen. \\
        Tilbyder & Leverer datastrømmer eller tjenester, følger policy-regler og rapporterer hendelser til operatøren. \\
        Forbruker & Integrerer data i egne tvillinger, dokumenterer formål og bidrar med forbedringsforslag til semantikken. \\
        Tilsyn/koordinator & Overvåker etterlevelse av regelverk (for eksempel NIS2) og beslutter tiltak ved alvorlige brudd. \\
        \bottomrule
    \end{tabular}
\end{table}

Rollen som dataspace-operatør kan med fordel ligge hos en nøytral aktør (for eksempel en bransjeorganisasjon eller et
forskningsinstitutt), mens domeneeier ofte er et konsortium av virksomheter som ønsker felles styring. Tilgang til tvillingens
simuleringsresultater kan skje via tjenestelag i dataspace-arkitekturen slik at partnerne får innsikt uten å kopiere hele
datasett.

\subsection{Regelverk og datakontrakter for dataspace}
Data Act og Data Governance Act stiller nye krav til hvordan norske dataspace-piloter dokumenterer delingsplikt, avtalefrihet og
tillitsmekanismer, samtidig som NIS2 skjerper sikkerhets- og hendelseshåndtering for kritiske tjenester.\citep{eu2023dataact,eu2022dga,eu2022nis2}
Digitaliseringsdirektoratet anbefaler at kontrakter, policy og operativ drift beskrives i gjenbrukbare moduler slik at partnere
raskt forstår hvilke rettigheter og plikter som følger hvert dataprodukt.\citep{digdir2024datasamarbeid} For masterstudentene betyr det at
juridiske vurderinger må kobles til indikatorene, tiltaksloggene og kvalitetsjournalene i de øvrige kapitlene.

\paragraph{Nøkkelprinsipper.}
\begin{itemize}
    \item \textbf{Rettferdig tilgang:} Data Act krever at dataspace-operatører dokumenterer hvilke vilkår som gjelder for viderebruk,
    og at små og mellomstore virksomheter får rimelige kontraktsvilkår. Dette må reflekteres i datakatalogen og i kontrolltårnsløyfene
    fra kapittel~6.\citep{eu2023dataact}
    \item \textbf{Tillitsmekanismer:} Data Governance Act forutsetter transparente logg- og sertifiseringsprosesser for dataformidlere.
    Alle delingshendelser og samtykker må derfor registreres i de samme journalene som brukes i \autoref{tab:kap03-dataspace-drift}
    og kapittel~7.\citep{eu2022dga}
    \item \textbf{Sikkerhet og hendelseshåndtering:} NIS2 utvider krav til risikovurdering, beredskapsplaner og rapporteringsfrister.
    Tiltakene fra \autoref{tab:kap03-tilgangsstyring} og \autoref{tab:kap03-dataspace-drift} skal vise hvordan avvik følges opp i kontrolltårnet
    og tiltaksjournalen.\citep{eu2022nis2}
    \item \textbf{Kontraktsmoduler:} Digdirs veiledning anbefaler standardiserte kontraktsmaler med referanser til API-spesifikasjoner,
    datakvalitetsregler og eskaleringsspor. Dette gjør det mulig å holde \autoref{tab:kap03-industripark} og de øvrige casene oppdaterte
    når nye partnere kobles på.\citep{digdir2024datasamarbeid}
\end{itemize}

Tabell~\ref{tab:kap03-regelverk} gir en oversikt over hvordan de viktigste regelverkene omsettes til konkrete leveranser i dette
kapittelet og hvordan de knyttes til styringen i kapitlene 6 og 7.

\begin{table}[ht]
    \centering
    \caption{Regelverkskrav og oppfølging i dataspace-leveranser.}
    \label{tab:kap03-regelverk}
    \begin{tabular}{p{0.24\textwidth}p{0.44\textwidth}p{0.26\textwidth}}
        \toprule
        \textbf{Regelverk} & \textbf{Implikasjon for dataspace} & \textbf{Oppfølging i kapitlet} \\
        \midrule
        Data Act & Dokumenter delingsplikt, prisingsregler og rettferdige kontrakter for dataprodukter. & Referer til API-katalogen og \autoref{tab:kap03-dataspace-drift} når vilkår og varsling loggføres. \\
        Data Governance Act & Etabler sertifiserte dataformidlere, logging og samtykkeprosesser. & Oppdater dataspace-katalogen og tiltaksloggen i kapittel~7 med henvisning til \autoref{tab:kap03-tilgangsstyring}. \\
        NIS2 & Krev risikovurdering, beredskapsplan og rapportering innen 24 timer ved alvorlige avvik. & Bruk indikatorer fra \autoref{tab:kap03-dataspace-drift} og kontrolltårnstrukturen i kapittel~6. \\
        Digdir datasamarbeid & Standardiser kontraktsmoduler, eskaleringspunkter og ansvarskart. & Synkroniser kontraktsmaler med \autoref{tab:kap03-industripark} og arbeidsflytene i kapittel~7. \\
        \bottomrule
    \end{tabular}
\end{table}

\paragraph{Arbeidsflyt for kontraktsforvaltning.} Når studentgrupper planlegger et dataspace skal de:
\begin{enumerate}
    \item Kartlegge hvilke datasett som omfattes av Data Act, hvilke som krever Data Governance Act-mekanismer og hvilke som klassifiseres
    som essensielle tjenester under NIS2. Resultatet loggføres i katalogen sammen med kontaktpunkt og risikovurdering.
    \item Utarbeide kontraktsmoduler som peker til datakvalitetsregler, API-spesifikasjoner og hendelsesrutiner, og plassere dem i samme
    Git-repo som modelljournalen slik \autoref{tab:kap03-dataspace-drift} anbefaler.
    \item Evaluere kontraktene kvartalsvis sammen med dataspace-operatøren, kontrolltårnet og gevinstpanelet fra kapittel~7, og registrere
    avvik og forbedringer i `support/notater/datastyringsforum-di03.md`.
\end{enumerate}

Denne arbeidsflyten gjør at regelverkskrav og tekniske leveranser holdes samlet, samtidig som læringsaktivitetene i masterkurset viser
hvordan dataspace kan skaleres uten å miste kontroll på etterlevelse og gevinst.

\subsection{Tilgangsstyring og dataminimering i kommunale dataspace}
Kommunal- og distriktsdepartementet anbefaler at kommunesektoren etablerer felles prinsipper for tilgangsstyring, logging og
dataminimering før dataspace deles med eksterne aktører.\citep{kdd2023datadeling} KS har fulgt opp med veiledning for hvordan
eiendomsdata og driftstjenester bør klassifiseres slik at skolene, helseetaten og teknisk sektor får innsikt uten å miste
kontrollen over personsensitive eller sikkerhetskritiske opplysninger.\citep{ks2024eiendomsdrift} For masterstudentene betyr
det at tilgangsavtaler må speile både dataeiers forventninger, personvernforordningen og beredskapskravene fra helse- og
teknisk sektor.

Tre styringsprinsipper går igjen i norske kommuner:
\begin{itemize}
    \item \textbf{Minimer delte attributter.} Datasett skal deles på lavest mulige granularitet og med aggregerte nøkler når
    individnivå ikke er nødvendig. Norm for informasjonssikkerhet i helsesektoren brukes som referanse for å sikre at
    pseudonymisering og loggkrav håndheves også når data flyttes mellom sektorer.\citep{norm2023}
    \item \textbf{Skarp rollefordeling.} Tjenesteprodusenter (for eksempel energientreprenører) får lese- og skriveadgang kun til
    de hendelsene de er kontraktsfestet til å håndtere, mens etaten som eier bygget beholder eierskap til konfigurasjon og
    modelljournal.
    \item \textbf{Sporbar beslutning.} Alle tilgangsendringer skal logges sammen med begrunnelse, kobling til risikoanalyse og
    hvem som godkjente unntak. Loggen synkroniseres med tiltaksplanene i kapittel~6 og gevinstoppfølgingen i kapittel~7 slik at
    avvik behandles i samme møtestruktur.
\end{itemize}

Tabell~\ref{tab:kap03-tilgangsstyring} viser hvordan tilgangsstyring og dataminimering dokumenteres gjennom hele livssyklusen til
et kommunalt dataspace. Kolonnene kobler tiltakene til kravene i andre kapitler slik at studentgrupper kan lage en helhetlig
leveranse uten å miste sporbarheten på tvers av kursmateriellet.

\begin{table}[ht]
    \centering
    \caption{Tilgangsstyring og dataminimering i kommunale dataspace-prosjekter.}
    \label{tab:kap03-tilgangsstyring}
    \begin{tabular}{p{0.26\textwidth}p{0.38\textwidth}p{0.20\textwidth}p{0.16\textwidth}}
        \toprule
        \textbf{Fase} & \textbf{Tiltak} & \textbf{Dokumentasjon} & \textbf{Kobling til kapitler} \\
        \midrule
        Behovsavklaring & Kartlegg hvilke datapunkter som må deles mellom etater og hvilke som kan aggregeres eller pseudonymiseres.
        & Delingsprotokoll, dataflytkart og risikovurdering. & Kapittel~4 (sandkasse-scenarioer) og Kapittel~8 (kommunale case) \\
        \addlinespace
        Tilgangsetablering & Implementer rollebasert tilgang, tidsbegrensede nøkler og policy for nødtilgang. & Autorisasjonsmatrise,
        endringslogg og hendelsesjournal. & Kapittel~6 (tilsyn og kontrolltårn) og Kapittel~7 (porteføljestyring) \\
        \addlinespace
        Løpende bruk & Overvåk bruksstatistikk, avvik og delingsforespørsler. Eskaler brudd via tiltaksloggen. & Månedlig
        rapportpakke, gevinstrapport og DPIA-oppdatering. & Kapittel~5 (modelltilpasning) og Kapittel~9 (innovasjon og regelverk) \\
        \bottomrule
    \end{tabular}
\end{table}

\subsection{Case: Dataspace for grønne industriparker}
Norske industriparker kombinerer energiintensive prosesser, karbonfangst og hydrogensatsinger som krever tett kobling mellom data,
simulering og styring.\citep{moindustripark2024klimaplan,heroya2024hydrogenhub} Et felles dataspace gjør det mulig å dele prosessorienterte data mellom aktører som Mo Industripark, Herøya Industripark og energiselskapene som leverer damp, strøm og CO$_2$-transport. Datastrømmen må samtidig forankres i klimamål og regulatoriske krav fra Enova og NVE for å dokumentere gevinstene av grønne industriprosjekter.\citep{enova2024energiledelse}

En industripark trenger dataprodukter som dekker hele verdikjeden fra energiinntak til utslippsrapportering:
\begin{itemize}
    \item \textbf{Prosess- og energistrømmer:} Sanntidsmålinger fra produksjonslinjer, dampnett og energigjenvinning som brukes i simuleringene i kapittel~4 og valideringspakken for industriparker i kapittel~6.
    \item \textbf{Karbonfangst og hydrogen:} Målepunkter for CO$_2$-strømmer, lagringstanker og hydrogenproduksjon dokumenteres i indikatorpanelet som beskrives i kapittel~6 og kobles til gevinstplanen i kapittel~7.
    \item \textbf{Samarbeids- og beredskapsdata:} Hendelseslogger, beredskapsplaner og leverandørstatus deles slik at parken kan koordinere tiltak med kommunal beredskap og nasjonale myndigheter.
\end{itemize}

Tabell~\ref{tab:kap03-industripark} viser hvordan dataprodukter kan beskrives i katalogen slik at industripartnere gjenbruker de samme indikatorene i styrings- og rapporteringsarbeidet.

\begin{table}[ht]
    \centering
    \caption{Dataprodukter i dataspace for grønne industriparker.}
    \label{tab:kap03-industripark}
    \begin{tabular}{p{0.28\textwidth}p{0.40\textwidth}p{0.26\textwidth}}
        \toprule
        \textbf{Dataprodukt} & \textbf{Formål og nøkkelinnhold} & \textbf{Kobling til kapitler} \\
        \midrule
        Energi- og dampbalanse & Minuttdata for varmevekslere, elektriske laster og gjenvinning per produksjonslinje. & Kap.~4 simulering av energiflyt, Kap.~6 indikatorpanel for industriparker. \\
        Karbonfangstjournal & CO$_2$-mengder, renhetsgrad, lagringstanker og skipningsplan med kvalitetsflagg. & Kap.~5 CCS-case, Kap.~6 kvantitative kontroller og kvartalsrapport. \\
        Hydrogen- og oksygenlogistikk & Produksjonsvolum, lagernivå, trykk og leveranseplaner for industripartnerne. & Kap.~4 hydrogenknutepunkt, Kap.~7 gevinstlogg og leverandørstyring. \\
        Felles beredskapslogg & Hendelser, tiltak og kommunikasjon med nødetater og kommuner. & Kap.~6 beredskapsøvelser, Kap.~8 sektorcaser for energi og helse. \\
        \bottomrule
    \end{tabular}
\end{table}

\paragraph{Arbeidspakke for studentgrupper.} Når caset brukes i masterkurset skal studentteam:
\begin{itemize}
    \item dokumentere hvordan dataproduktene over kobles til kvalitetsjournalen og indikatorpanelet i kapittel~6,
    \item beskrive hvordan energisamarbeid og beredskapsrutiner koordineres i tiltaksloggen i kapittel~7,
    \item foreslå hvordan resultatene rapporteres til klimaregnskap og myndigheter slik at kapittel~9s veikart kan oppdateres.
\end{itemize}

\subsection{Case: Felles tilgangsstyring for byggdrift og helsetjenester}
Oslo kommune tester et felles dataspace der eiendomsetaten og helseetaten deler sensordata for innemiljø, smittevern og
kapasitetsplanlegging.\citep{osloeiendom2023strategi,helsedir2023beredskap} I pilotfasen opplevde teamene at helsedata måtte
behandles annerledes enn energidata, samtidig som byggdriftsloggene var nødvendige for å forstå avvik i pasientflyten. Løsningen
ble å etablere en tilgangsstyringsgruppe som gjennomgår hver delingsforespørsel og sikrer at dataminimering og logging følger
kravene fra Normen og kommunens sikkerhetshåndbok. Gruppens beslutninger registreres i dataspace-katalogen og publiseres sammen
med indikatorene for kvalitetsjournal og gevinststyring.

Studentgrupper som jobber med caset skal:
\begin{itemize}
    \item dokumentere hvilke datasett som krever ekstra skjerming og hvordan de kan anonymiseres eller aggregeres før deling,
    \item beskrive beslutningsprosessen når helsesektoren trenger høyoppløselige driftsdata for smittevern,
    \item vise hvordan logging og månedlige revisjoner speiles i tiltaksloggen fra kapittel~6 og gevinstrapportene i kapittel~7.
\end{itemize}

\subsection{Implementeringssteg for norske virksomheter}
For å komme raskt i gang med dataspace-initiativ anbefales en trinnvis tilnærming:
\begin{enumerate}
    \item Kartlegg hvilke datakilder, modeller og simuleringsresultater som er mest etterspurt på tvers av organisasjoner.
    \item Identifiser regulatoriske krav (for eksempel energilovgivning, personvern, eksportkontroll) og oversett disse til
    policy-regler som kan konfigureres i konnektorene.
    \item Velg tekniske komponenter som støtter signering, kryptering og logging i tråd med IDSA-rammeverket.
    \item Etabler en felles katalog og begrepsmodell, gjerne med utgangspunkt i eksisterende bransjestandarder.
    \item Pilotér deling av et begrenset datasett og vurder ytelse, sikkerhet og gevinstrealisering før skalering.
\end{enumerate}
\subsection{API-katalog og dataprodukter}
En moden dataspace krever mer enn tekniske konnektorer. Norske virksomheter har erfart at en felles API-katalog og tydelige
dataprodukter gjør det enklere å koble kontrolltårn, bærekraftsarbeid og governance-prosesser på tvers av partnere.\citep{digdir2023apiveileder,dssc2024dataproducts}
Katalogen fungerer som kontrakt mellom tilbyder og forbruker: den beskriver formål, datakvalitet, sikkerhetstiltak og kobling
til lærings- og rapporteringsløp i de øvrige kapitlene.

Et praktisk oppsett følger tre hovedsteg:
\begin{enumerate}
    \item \textbf{Registrer dataprodukter i katalogen.} Hvert produkt får en kort beskrivelse, ansvarlig kontakt, datakvalitetskrav
    og referanse til policy-regler. Feltet \texttt{forbruk} peker på hvilke kapitler eller case som gjenbruker dataene, slik at
    forbedringer kan prioriteres.
    \item \textbf{Publiser maskinlesbare spesifikasjoner.} OpenAPI- eller AsyncAPI-beskrivelser lagres sammen med versjonslogg og
    sikkerhetsprofil. Gjennom versjonering kan fagteamene verifisere at dashboards og modelljournaler oppdateres før endringer
    tas i bruk.
    \item \textbf{Koble til styringsindikatorer.} Hvert dataprodukt må dokumentere hvilke KPI-er og hendelseslogger som oppdateres
    når nye data konsumteres. Det gjør at indikatorpakkene i kapittel~6 og gevinstplanene i kapittel~7 alltid peker på riktig
    kilde.
\end{enumerate}

Tabell~\ref{tab:kap03-api-produkter} viser hvordan tre prioriterte dataprodukter dokumenteres i katalogen for masterkurset. Kolonnen
"Kobling" hjelper studentgrupper med å finne igjen samme indikator i andre kapitler.

\begin{table}[ht]
    \centering
    \caption{Eksempel på dataprodukter i API-katalogen for dataspace-piloten.}
    \label{tab:kap03-api-produkter}
    \begin{tabular}{p{0.26\textwidth}p{0.32\textwidth}p{0.22\textwidth}p{0.16\textwidth}}
        \toprule
        \textbf{Dataprodukt/API} & \textbf{Formål} & \textbf{Sikkerhets- og kvalitetskrav} & \textbf{Kobling} \\
        \midrule
        Energiobservasjon v2 (REST) & Leverer sanntidslast og prognoser til kontrolltårnet. & OAuth2 med maskin-til-maskin-token, SLO for latens $<60$ sekunder, versjonerte OpenAPI-kontrakter. & Kap.~6 indikatorpanel, beredskapsøvelse hydrogen \\
        Klinisk logistikk (AsyncAPI) & Publiserer pasientflyt og kapasitet til helsedataspace. & Kryptert MQTT med klientsertifikat, pseudonymisering før distribusjon, kvartalsvis DPIA-gjennomgang. & Kap.~6 DPIA, Kap.~8 helsecase \\
        Bærekraftsdashboard (GraphQL) & Samler CO$_2$-regnskap og ombrukstall for rapportering. & Policy for aggregerte svar, kvalitetsscore $>0,95$ på datakatalog, signert endringslogg i dataspace. & Kap.~7 gevinststyring, Kap.~4 sirkulær simulering \\
        \bottomrule
    \end{tabular}
\end{table}

\paragraph{Samsvar med API-veilederne.} Digitaliseringsdirektoratet anbefaler at API-er merkes med lisens, kontaktpunkt og
forventet levetid i selve spesifikasjonen.\citep{digdir2023apiveileder} I dataspace-kontekst betyr det at katalogen lagrer
meta-data som \texttt{x-sikkerhetstiltak} (referanse til kapittel~6) og \texttt{x-gevinsteier} (kobling til kapittel~7). Data
Spaces Support Centre fremhever i tillegg behovet for å publisere dataprodukter som kombinerer API, hendelsesstrømmer og
styringsprosess i ett dokument, slik at deltakere kan automatisere oppfølgingen av kontrakter og indikatorer.\citep{dssc2024dataproducts}

\paragraph{Arbeidspakke for studentgrupper.} Når oppgaven brukes i masterkurset skal hver gruppe:
\begin{itemize}
    \item registrere minst to dataprodukter i katalogen, inkludert OpenAPI- eller AsyncAPI-fil og lenke til kvalitetsjournalen,
    \item beskrive hvordan indikatorene fra kapittel~6 oppdateres når produktet publiserer nye data,
    \item dokumentere gevinsthypoteser og rapporteringsløp i samsvar med kapittel~7, slik at fagfeller ser sammenhengen mellom
    data, beslutning og gevinstrealisering.
\end{itemize}

\subsection{Testregime for API-katalogen}
En katalog gir først verdi når spesifikasjoner, sikkerhet og datakvalitet kontinuerlig bekreftes. Digitaliseringsdirektoratet
anbefaler at hver API-versjon kvalitetssikres mot veilederen før den åpnes for bruk, mens Data Spaces Support Centre og IDS
Operations Handbook beskriver hvordan dataspace-operatører kan kombinere automatiske tester og revisjonsspor for å holde
kontrakter oppdatert.\citep{digdir2023apiveileder,dssc2024dataproducts,idsa2023operational} Testregimet nedenfor kan kjøres ved
hvert sprintskifte og før produksjonssetting av nye dataprodukter.

\begin{enumerate}
    \item \textbf{Valider spesifikasjonen.} Kjør linting på OpenAPI/AsyncAPI-filene, bekreft at obligatoriske felt som lisens,
    kontaktpunkt og livsløpsstatus er fylt ut, og generér kontrakttester som bekrefter at responsene følger katalogbeskrivelsen.
    \item \textbf{Sikkerhet og tilgang.} Test autentisering, autorisasjon og ratebegrensning med syntetiske brukere. Loggene
    kontrolleres for avvik i tråd med grunnprinsippene for IKT-sikkerhet og nulltillit.\citep{nsm2023grunnprinsipper}
    \item \textbf{Datakvalitet og semantikk.} Sammenlign felttyper, enheter og indikatorberegninger mot kvalitetsjournalen og
    indikatorpanelet i kapittel~6 for å sikre at endringer ikke bryter lærings- og rapporteringssløyfene.\citep{dssc2024dataproducts}
    \item \textbf{Drift og resiliens.} Simuler nettbrudd, versjonsrulling og throttling for å verifisere at overvåkingsrutiner og
    beredskapsplaner reagerer som forventet i sandkasse og produksjon.\citep{idsa2023operational}
\end{enumerate}

Tabell~\ref{tab:kap03-api-test} gir en sjekkliste som binder testene til konkrete verktøy og dokumentasjonskrav.

\begin{table}[ht]
    \centering
    \caption{Testregime for API-katalog og dataprodukter.}
    \label{tab:kap03-api-test}
    \begin{tabular}{p{0.24\textwidth}p{0.30\textwidth}p{0.28\textwidth}p{0.14\textwidth}}
        \toprule
        \textbf{Teststeg} & \textbf{Formål} & \textbf{Verktøy og sjekkpunkter} & \textbf{Dokumentasjon} \\
        \midrule
        Spesifikasjonsvalidering & Sikre at kontrakten er komplett og maskinlesbar. & OpenAPI/AsyncAPI-linting, kontrakttest i CI, versjonssammenligning. & Endringslogg, katalognotat med versjons-ID. \\
        Sikkerhetstest & Bekrefte autentisering, autorisasjon og logging. & Testtokens, automatisert ratebegrensningstest, loggrevisjon. & Sikkerhetsjournal, referanse til \autoref{tab:standardkart}. \\
        Datakvalitet & Hindre semantiske avvik og feil indikatorer. & Schema-diff, prøvepublisering med syntetiske data, kobling til kvalitetsjournal. & Kvalitetsjournal, indikatorrapport i kapittel~6-format. \\
        Driftsprøve & Verifisere resiliens og gjenoppretting. & Kaos-test i sandkasse, failover mot reserveinstanser, overvåkingsalarm. & Beredskapslogg, tiltaksoppdatering i gevinstplanen. \\
        \bottomrule
    \end{tabular}
\end{table}

\paragraph{Koordinering med laboratoriet.} Resultatene fra testene registreres i kvalitetsjournalen før de sendes til
valideringspanelet i kapittel~6. Når tiltak krever endringer i gevinstplanen eller leverandørkontraktene i kapittel~7, skal
oppfølgingen loggføres i tiltaksloggen og deles i fagfelleloggen. Studentgrupper kan bruke tabellen som grunnlag for å fordele
ansvar mellom tekniske og styringsroller i laboratorieøktene.

\subsection{Case: Dataspace-sandkasse for kommunal byggdrift}
Kommunale eiendomsforvaltere i Oslo, Bergen og Trondheim har de siste årene etablert felles sanntidsmodeller for energioppfølging, inneklima og vedlikeholdslogg slik at skole- og helsebygg kan styres mer presist \citep{osloeiendom2023strategi,bergen2024smartbygg}. Erfaringen viser at datadeling på tvers av etater krever en trygg arena for å teste nye integrasjoner uten å risikere driftsavbrudd. En dataspace-sandkasse gir nettopp dette: en kontrollert plattform der tvillingmodeller, API-er og indikatorer kan verifiseres før de kobles til produksjonsmiljøet.

Sandkassen bygges stegvis for å sikre både teknisk kvalitet og organisasjonslæring:
\begin{enumerate}
    \item \textbf{Kartlegg byggporteføljen.} Eiendomsteamet identifiserer hvilke bygg som skal inngå i piloten, og hvilke datasett som må prioriteres (energiforbruk, CO$_2$-nivå, brukerklager). Resultatet dokumenteres i dataspace-katalogen sammen med kontaktpunkt.
    \item \textbf{Etabler konnektorer og testdata.} Edge-noder og historiske datasett kopieres til sandkassen. API-ene merkes med egne sandkasse-nøkler og logges separat, slik at feil ikke forurenser produksjonsjournalen.
    \item \textbf{Koble indikatorpanel og gevinstmål.} Måleparametere fra kapittel~6 (kvalitetsjournal) og gevinstplanen i kapittel~7 lastes inn i et felles dashboard. Sandkassen verifiserer at indikatorene oppdateres riktig når nye datasett aktiveres.
    \item \textbf{Gjennomfør tverrfaglig sprint.} Driftspersonell, energiøkonomer og læringsansvarlige tester scenarioer (for eksempel vinterdrift eller smittevern). Tiltak og observasjoner loggføres i en felles beslutningsjournal.
    \item \textbf{Beslutning om produksjonssetting.} Når indikatorene holder terskler definert i Tabell~\ref{tab:kap03-sandkasse-faser}, flyttes integrasjonene over til ordinær dataspace med full hendelseslogging og beredskapsrutiner.
\end{enumerate}

Tabell~\ref{tab:kap03-sandkasse-faser} fungerer som arbeidsplan for masterstudentene. Den viser hvordan hver fase kobles til konkrete leveranser og hvilken dokumentasjon som må oppdateres for å holde samsvar med kravene i kapittel~6 og kapittel~7.

\begin{table}[ht]
    \centering
    \caption{Faser i dataspace-sandkasse for kommunal byggdrift.}
    \label{tab:kap03-sandkasse-faser}
    \begin{tabular}{p{0.22\textwidth}p{0.44\textwidth}p{0.26\textwidth}}
        \toprule
        \textbf{Fase} & \textbf{Hovedaktiviteter} & \textbf{Leveranser og koblinger} \\
        \midrule
        Porteføljekartlegging & Velge bygg, avklare datasett, forankre datadeling med eiere og vernetjeneste. & Oppdatert dataspace-katalog, referanse til personvernvurdering og tiltakslogg i kapittel~6. \\
        Teknisk sandkasse & Konfigurere konnektorer, rense testdata, etablere syntetiske datasett for stress-testing. & Konfigurasjonsfil i Git, kvalitetsjournal for sandkasse, risikovurdering mot NIS2/IEC 62443. \\
        Indikatorpanel & Koble energimål, inneklima og brukerklager til dashboard og rapportmal. & Oppdatert indikatorbibliotek, lenke til gevinstplan og budsjettoppfølging i kapittel~7. \\
        Sprint og evaluering & Teste driftsscenarier, loggføre beslutninger, gjennomføre retro med fagfeller. & Beslutningsjournal, fagfelleinnspill i `support/fagfellelogg.csv`, oppdatert læringsopplegg i kapittel~4 og 5. \\
        Produksjonsløft & Plan for utrulling, avtaleverk for dataspace og beredskapsøvelse før go-live. & Versjonsnotat i katalogen, avtaleoppdateringer, kobling til kontrolltårnprosess i kapittel~6. \\
        \bottomrule
    \end{tabular}
\end{table}

I praksis samkjøres sandkassen med kommunens energi- og miljømål. Oslo kommune har for eksempel vedtatt at alle nye bygg skal ha digitale loggbøker og klimaregnskap som oppdateres kontinuerlig \citep{oslo2024klimaeiendom}. KS anbefaler at kommuner etablerer felles styringsmodeller for eiendomsdrift som kobler energioppfølging til økonomiplan og bærekraftsrapportering \citep{ks2024eiendomsdrift}. Når sandkassen brukes i masterkurset, får studentene i oppgave å:
\begin{itemize}
    \item beskrive hvordan indikatorer for energibruk, inneklima og brukeropplevd kvalitet overføres fra sandkasse til produksjon,
    \item dokumentere hvordan personvern og driftsikkerhet ivaretas gjennom policy-regler og hendelseslogging,
    \item foreslå hvordan gevinstene følges opp i kommunens økonomi- og klimaregnskap, slik at dataspace-investeringen kan forsvares for politisk ledelse.
\end{itemize}

Sandkassen gjør det mulig å koble tekniske pilotprosjekter til ordinær tjenesteproduksjon. Når indikatorene og governance-elementene er testet i sandkassemiljøet, kan de samme artefaktene gjenbrukes i kapittel~6 (kvalitetsjournal og tilsyn) og kapittel~7 (porteføljestyring). Dette sikrer at kommunene bygger opp en dokumentert læringssløyfe fra eksperimentering til drift og kontinuerlig forbedring.

\subsection{Case: Dataspace for kulturarv og historiske bymiljøer}
Norske bykommuner kombinerer nå digitale tvillinger med kulturminnedata for å balansere bevaring og bærekraftig utvikling. Riksantikvaren har lansert en digital strategi som anbefaler at 3D-modeller, tilstandsanalyser og juridiske vernebestemmelser tilgjengeliggjøres gjennom et felles dataspace slik at planmyndigheter, utviklere og innbyggere kan vurdere konsekvenser tidlig i prosessen.\citep{riksantikvaren2023digitalarv} Bergen og Oslo har etablert pilotprosjekter der historiske bymiljøer visualiseres i sanntid, samtidig som energibruk og klimatilpasning testes i samarbeid med universiteter og lokale museer.\citep{bergen2024kulturminne,oslo2024bymiljolab} For masterstudentene gir caset en mulighet til å koble geodata, sensorer og samfunnsdialog til de styringsmodellene som beskrives i kapittel~6 og kapittel~7.

Dataspace-designet må håndtere både faglige og juridiske hensyn:
\begin{itemize}
    \item \textbf{Kulturminnedata:} Vernesoner, reguleringsplaner og objekthistorikk fra Riksantikvarens Askeladden-register må versjonsstyres og deles med plan- og byggesak.
    \item \textbf{Tilstands- og risikomålinger:} Sensorer for fukt, vibrasjon og luftkvalitet kompletterer manuelle inspeksjoner og gjør det mulig å overvåke effekten av tiltak i bygg og byrom.
    \item \textbf{Medvirknings- og scenariodata:} Høringsinnspill, AR/VR-modeller og klimascenarioer brukes for å teste alternative utbyggingsplaner og dokumentere konsekvenser for bevaringsmål.
\end{itemize}

Tabell~\ref{tab:kap03-kulturarv-datasett} viser hvordan datasett kan organiseres for å støtte bevaringsarbeid, planprosesser og undervisning.

\begin{table}[ht]
    \centering
    \caption{Datasett og styringstiltak i kulturarv-dataspace.}
    \label{tab:kap03-kulturarv-datasett}
    \begin{tabular}{p{0.28\textwidth}p{0.34\textwidth}p{0.26\textwidth}p{0.10\textwidth}}
        \toprule
        \textbf{Datasett} & \textbf{Kilde og oppdatering} & \textbf{Bruksområde} & \textbf{Kobling} \\
        \midrule
        Vernedata og reguleringsbestemmelser & Askeladden-register, kommunale planregister (kvartalsvis oppdatering) & Avklare juridiske rammer og varslingsplikt ved tiltak. & Kap.~7 governance \\
        3D-modeller og BIM for historiske bygg & Kommunale laserskanninger, museumsarkiv, NTNU/Statsbygg-modeller (årlig revisjon) & Visualisere konsekvenser av endringer og planlegge rehabilitering. & Kap.~4 simulering \\
        Tilstandssensorer og inspeksjonslogger & IoT-sensorer, dronefoto, feltinspeksjoner (sanntid/kvartal) & Overvåke fukt, vibrasjon og klimarisiko, trigge tiltak i kontrolltårn. & Kap.~6 kvalitetsjournal \\
        Medvirkningsdata og scenarioresultater & Dialogportaler, AR/VR-lab, klimaberegninger (per prosjektfase) & Dokumentere alternativer og beslutninger i planprosesser. & Kap.~8 casebank \\
        \bottomrule
    \end{tabular}
\end{table}

For å bruke caset i undervisning kan studentgrupper følge tre steg:
\begin{enumerate}
    \item \textbf{Kartlegg bevaringsmål og konflikter:} Kombiner verneregler med eiendomsdata for å identifisere hvilke tiltak som utløser krav til søknad, dispensasjon eller nye analyser. Resultatene dokumenteres i dataspace-katalogen og fagfelleloggen.
    \item \textbf{Bygg et indikatorpanel:} Beregn nøkkeltall for energibruk, CO$_2$-utslipp, turisme og bylivsindikatorer, og koble dem til tiltaksloggen i kapittel~7. Panelet viser hvordan bevaring og bærekraft kan balanseres.
    \item \textbf{Planlegg medvirkningsløp:} Lag en oppgave der AR/VR-modeller presenteres for innbyggere, og loggfør hvordan tilbakemeldinger påvirker valgte scenarier. Dokumentasjonen gjenbrukes i kapittel~5 for å evaluere AI-assistenter i planprosesser.
\end{enumerate}

Seksjonen knytter kulturminneforvaltning til dataspace-governance. Når datasett og indikatorer registreres i kvalitetsjournalen fra kapittel~6 kan kommunene vise etterlevelse av både plan- og bygningslov, kulturminnelov og NIS2-krav for datadeling. Erfaringene fra caset kan også kobles til brannberedskapscaset i kapittel~4 og porteføljestyringen i kapittel~9, slik at kulturarv inngår i helhetlige investerings- og beredskapsplaner.


\subsection{Tverrsektorielt dataspace-samvirke for kritiske tjenester}
Energi- og helsetjenester er gjensidig avhengige i krisesituasjoner: strømforsyningen må holdes stabil for å drive sykehus og
helsedata må deles sikkert for å prioritere reparasjons- og evakueringsressurser. Nasjonale beredskapsplaner og sikkerhetskrav
legger derfor opp til samvirkeøvelser som kobler energisektoren og helsetjenestene, med vekt på felles situasjonsforståelse,
datakvalitet og eskaleringslinjer.\citep{dsb2023totalberedskap,nsm2023grunnprinsipper,nhn2024dataspace,statnett2024kontrolltarn,helsedir2023beredskap}
Et tverrsektorielt dataspace gir en felles plattform for å koordinere sanntidsdata, beslutningslogger og tiltak i masterkurset.

Tabell~\ref{tab:kap03-tversektor-samvirke} oppsummerer hvilke informasjonsstrømmer som må forvaltes og hvem som leder
koordineringsarbeidet. Kolonnene speiler koblingen til kontrolltårn-arbeidet i kapittel~6 og governance-strukturen i
kapittel~7.

\begin{table}[ht]
    \centering
    \caption{Datastrømmer og ansvar i tverrsektorielt dataspace-samvirke mellom energi og helse.}
    \label{tab:kap03-tversektor-samvirke}
    \begin{tabular}{p{0.30\textwidth}p{0.42\textwidth}p{0.22\textwidth}}
        \toprule
        \textbf{Datastrøm} & \textbf{Formål i samvirkeøvelsen} & \textbf{Primær eier og kobling} \\
        \midrule
        Kritiske energipunkter & Viser tilgjengelig kapasitet på sykehus, reservekraft og nettnoder slik at tiltak kan prioriteres. & Statnett kontrolltårn \& regionalt nettselskap (Kap.~6) \\
        Kliniske logistikkdata & Gir oversikt over pasientflyt, ambulanser og logistikktjenester for evakuering og fordeling. & Helseforetakets operasjonssenter (Kap.~8) \\
        Hendelses- og risikologg & Dokumenterer avvik, beslutninger og varsling som grunnlag for myndighetsrapportering. & Dataspace-operatør i fellesskap med beredskapsledelse (Kap.~6/7) \\
        Ressurs- og kompetansebank & Kartlegger tilgjengelige mannskaper, leverandører og støttefunksjoner. & Kommunal beredskap og frivillige organisasjoner (Appendiks) \\
        \bottomrule
    \end{tabular}
\end{table}

\paragraph{Øvingsscenario for masterkurset.} Studentgrupper kan bygge videre på tabellen ved å gjennomføre en tredelt øvelse:
\begin{enumerate}
    \item \textbf{Forberedelse:} Kartlegg hvilke datasett som må deles og oppdater policy-regler for tilgang, inkludert logging av
    samtykker fra helsesektoren. Resultatet registreres i dataspace-katalogen og kvalitetsjournalen i kapittel~6.
    \item \textbf{Simulert hendelse:} Start en hendelse der en transformatorfeil påvirker et sykehusområde. Energisiden publiserer
    sanntidsdata om last, mens helsetjenesten deler pasientprioritet og behov for kritisk utstyr. Deltakerne bruker kontrolltårn-
    panelet til å koordinere tiltak og oppdaterer hendelsesloggen.
    \item \textbf{Etteranalyse:} Evaluer datakvalitet, beslutningslogg og responstid. Læringspunkter føres inn i governance-planen
    fra kapittel~7 og deles med fagfellegruppen via `support/fagfellelogg.csv`.
\end{enumerate}

\paragraph{Indikatorpakke.} For å måle effekten av samvirket anbefales indikatorer for energitilgjengelighet, prioriteringstid
for kritiske pasienter, kvalitet på hendelseslogg og etterlevelse av sikkerhetskrav. Tersklene bør hentes fra kontrolltårn-
seksjonen i kapittel~6 og kombineres med gevinstplanene i kapittel~7 slik at resultatene kan rapporteres til både styre og
myndigheter.

\subsection{Case: Mobilitetsdataspace for bylogistikk}
Oslo-regionen har etablert et mobilitetsdataspace som kobler sammen bylogistikkaktører, kollektivselskap og teknologipartnere
for å redusere utslipp og øke punktligheten i vareleveranser. Piloten bygger på veikartet for European Mobility Data Space og
bruker Gaia-X-kompatible konnektorer for å sikre at data deles med tydelige policy-regler \citep{ec2023mobilitydataspaceblueprint}.
Ruter og Bymiljøetaten har samlet sanntidsdata fra mobilitetspunkter, ladeinfrastruktur og mikromobilitetsoperatører i en
felles katalog. Hver deltaker beholder kontroll over egne datasett, men deler metadata, kvalitetsindikatorer og hendelseslogger
slik at tvillingene i kapittel~4 og kapittel~6 kan bruke oppdaterte trafikkprofiler.

\paragraph{Formål og arbeidsdeling.} Partnerskapet prioriterer tre hovedområder:
\begin{itemize}
    \item \textbf{Styring av bylogistikk:} Samordne leveransevinduer og lavutslippssoner gjennom et felles datalag der
    mobilitetsdata, trafikkstyring og last-mile-løsninger kobles til de simuleringsmetodene som beskrives i kapittel~4.
    \item \textbf{Beredskap:} Varsle avvik som påvirker sikkerhet og fremkommelighet (for eksempel stengte ruter eller
    energimangel i ladepunkter) og sende signaler til beredskapsprosessene i kapittel~6.
    \item \textbf{Læring og innovasjon:} Tilby anonymiserte datasett til forskningsprosjekter og studentteam som skal teste nye
    logistikkløsninger, med styringskrav for personvern og datasuverenitet \citep{ruter2023dataplattform}.
\end{itemize}

\begin{table}[ht]
    \centering
    \caption{Tiltakspakke for mobilitetsdataspace i Oslo-regionen.}
    \label{tab:kap03-mobilitet-tiltak}
    \begin{tabular}{p{0.30\textwidth}p{0.46\textwidth}p{0.18\textwidth}}
        \toprule
        \textbf{Tiltak} & \textbf{Beskrivelse} & \textbf{Ansvarlig} \\
        \midrule
        Felles datakatalog & Harmoniserer metadata for kollektivtrafikk, varelevering og mikromobilitet slik at datasett kan
        oppdages på tvers av deltakere. & Dataspace-operatør (ITS Norway) \\
        Tilgangsstyring med policy & Automatiserer kontraktvilkår for datasett, inkludert vilkår for kommersiell bruk og deling
        med forskningspartnere. & Juridisk koordinator hos Ruter \\
        Observabilitetsdashboard & Viser sanntidsscore for punktlighet, utnyttelse av lastepunkt og hendelser som påvirker
        sikkerhet, koblet til indikatorene i kapittel~6. & Bymiljøetaten og transportetatene \\
        Innovasjonsarena & Gir sandkasse for start-up-prosjekter med syntetiske datasett og støtte til
        eksperimenter dokumentert i kapittel~8. & Mobilitetslab og universitetspartner \\
        \bottomrule
    \end{tabular}
\end{table}

\paragraph{Indikatorer.} For å måle effekten av mobilitetsdataspace defineres indikatorer som gjenbrukes i styringspanelet for
kapittel~6 og gevinstplanene i kapittel~7:
\begin{itemize}
    \item Utslippsreduksjon (CO$_2$-ekvivalenter per leveranse) sammenlignet med baseline-scenarioer.
    \item Punktlighet og fyllingsgrad per transportkategori, basert på sanntidsposisjoner og bookingsdata.
    \item Andel datasett med komplett metadata og kvalitetsscore, koblet til datakvalitetsprosessen i dette kapittelet.
    \item Respons- og gjenopprettingstid ved hendelser som krever koordinering med beredskapsplanen i kapittel~6.
\end{itemize}

Caset gir studentene innsikt i hvordan datasamarbeid gir konkrete tiltak for bylogistikk, samtidig som det viser hvordan
dataspace-løsninger må koordineres med juridiske krav, indikatorarbeid og de tverrsektorielle casene i kapittel~8.

\subsection{Driftsrammeverk og hendelseshåndtering i dataspace}
Når flere virksomheter deler tvillingdata gjennom et dataspace, må drift og hendelseshåndtering organiseres som en felles
tjeneste. IDS-operasjonshåndboken anbefaler å definere operative roller, varslingsregler og sporbarhet i samme
styringssystem som benyttes for sikkerhet og kvalitet \citep{idsa2023operational}. Digitaliseringsdirektoratet legger
til grunn at offentlige aktører skal kunne dokumentere hvordan sanntidsdata stanses eller filtreres ved avvik, og at
hendelsesloggen deles med partnere som påvirkes \citep{digdir2024sanntidsdata}. For norske dataspace-piloter betyr det at
kontrolltårnene i kapittel~6 må ha en tydelig kobling til dataspace-operatøren, slik at tiltak og varsler oppdateres i samme
beslutningslogg.

Tabell~\ref{tab:kap03-dataspace-drift} gir et utgangspunkt for driftstavlen i masterkurset. Den fungerer som sjekkliste når
studentgrupper setter opp samarbeidsavtaler mellom dataspace-operatør, datasettleverandører og kontrolltårnteam.

\begin{table}[ht]
    \centering
    \caption{Operativ tavle for dataspace-tvillinger i norske pilotprosjekter.}
    \label{tab:kap03-dataspace-drift}
    \begin{tabular}{p{0.24\textwidth}p{0.40\textwidth}p{0.24\textwidth}}
        \toprule
        \textbf{Scenario} & \textbf{Varslings- og tiltaksspor} & \textbf{Kobling til kapittel 6} \\
        \midrule
        Kvalitetssvikt i sanntidsstrøm & Automatisk stopp i konnektor, varsle datasettansvarlig og publisere avvik i dataspace-logger. & Oppdaterer tillitspanel og hendelseslogg i kontrolltårnseksjonen. \\
        Uautorisert datapågang & Policy-motor sperrer tilkoblingen, sikkerhetsteamet aktiverer beredskapsprosess og registrerer hendelsen. & Viser sporbarhet til NIS2-kravene og responsflyten i valideringskapittelet. \\
        Endring i modellversjon & Dataspace-operatør sender endringsvarsel til alle forbrukere og krever sign-off før distribusjon fortsetter. & Knyttes til kvalitetsjournal og driftsgodkjenning for modellvedlikehold. \\
        Planlagt vedlikehold & Publiser vedlikeholdsvindu i katalogen, rute trafikk til syntetiske datasett og oppdatere statusdashboard. & Sikrer at kontrolltårn-planer og beredskapsøvelser bruker samme tidsplan. \\
        \bottomrule
    \end{tabular}
\end{table}

For å bruke tabellen i praksis anbefales følgende arbeidsflyt:
\begin{enumerate}
    \item \textbf{Felles driftsmøte hver måned:} Dataspace-operatøren, kontrolltårnteamet og datasettleverandørene går gjennom
    kommende endringer og avvik, og oppdaterer tiltakene i tabellen. Referat lenkes til fagfelleloggen.
    \item \textbf{Hendelsesøvelser:} Studentgrupper kjører miniscenarioer der de simulerer datatap, policy-brudd eller modellendringer.
    Øvelsene loggføres i samme kvalitetsjournal som beskrives i kapittel~6.
    \item \textbf{Tilbakemelding til dataspace-design:} Erfaringene brukes til å forbedre kontrakter, indikatorer og
    observabilitetsdashbordene som allerede er beskrevet i dette kapittelet.
\end{enumerate}

\paragraph{Leveransekrav i masterkurset.} Når oppgaven brukes i undervisning, skal gruppene levere en kort driftshåndbok som
viser hvordan varslingsreglene implementeres i dataspace-konfiguratoren, og hvordan kontrolltårn-panelet i kapittel~6
oppdateres. Dokumentasjonen må inneholde lenker til hendelsesloggen, kontaktpunkter og beslutningsreferat.

\subsection{Helsesektorens dataspace og kliniske tvillinger}
Helsesektoren stiller ekstra krav til datasuverenitet, pasientsikkerhet og dokumentasjon når digitale tvillinger tas i bruk.
Norsk helsenett beskriver i sin referansearkitektur hvordan en nasjonal dataspace skal sørge for at datadeling skjer gjennom
sertifiserte konnektorer, policy-motorer og revisjonsspor som ivaretar både helseregisterlov og personvernforordningen.\citep{nhn2024dataspace}
For kapittelteksten betyr det at kliniske tvillinger må koble dataplattformene til tydelige adgangsregler, samtykkelagre og
loggmekanismer slik at både behandlingsansvarlige og pasienter kan etterprøve bruken. Helsedirektoratets veiledere for digital
hjemmeoppfølging og beredskap anbefaler at virksomheter dokumenterer datakilder, risikovurderinger og kontinuitetsplaner før
nye tjenester lanseres, noe som kan gjenbrukes direkte i dataspace-oppsettet.\citep{helsedir2020dho,helsedir2023beredskap}

En praktisk arbeidsflyt for studentgrupper og fagteam kan beskrives i tre steg:
\begin{enumerate}
    \item \textbf{Forankre behandlingsgrunnlag og samtykker}: Kartlegg hvilke datasett som har hjemmel i helseregisterlov eller krever eksplisitt samtykke, og lagre beslutningene i samme katalog som dataspace-kontraktene.
    \item \textbf{Etabler sikre delingskanaler}: Konfigurer sertifiserte konnektorer og dataminimering slik at bare nødvendige felter eksponeres til forsknings- og kvalitetsregistre. Automatisk pseudonymisering dokumenteres i modelljournalen.\citep{datatilsynet2023dpia}
    \item \textbf{Knytt overvåking til kvalitetsjournalen}: Synkroniser hendelses- og revisjonslogger med kontrollpunktene i Kapittel~6 slik at avvik, revisjoner og læringspunkter kan spores tilbake til pasientsikkerhetskravene.\citep{ehelse2024tilsyn}
\end{enumerate}

Tabell~\ref{tab:kap03-helsedataspace} viser hvilke styringstiltak som bør beskrives for hver komponent i dataspace-arkitekturen.
Kolonnene speiler rollene i kapittel~7 og gjør det enklere å koble kliniske tvillinger til governance-modellene og casene i
kapittel~8.

\begin{table}[htbp]
    \centering
    \caption{Styringstiltak for helsedataspace i kliniske tvillingprosjekter}
    \label{tab:kap03-helsedataspace}
    \begin{tabular}{p{3.6cm}p{5.0cm}p{4.4cm}}
        \toprule
        \textbf{Komponent} & \textbf{Tiltak og dokumentasjon} & \textbf{Kobling til andre kapitler} \\
        \midrule
        Samtykke- og tilgangsstyring & Dokumenter behandlingsgrunnlag, samtykkestatus og tilgangsnivå i katalogen; oppdater policy-regler ved nye tjenester. & Kapittel~6 (DPIA og tilsynslogg), Kapittel~7 (ansvarsfordeling) \\
        Kliniske datakanaler & Bruk konnektorer med end-to-end-kryptering og dataminimering; loggfør pseudonymisering og filtrering i modelljournalen. & Kapittel~5 (modelljournal), Kapittel~8 (helsecase) \\
        Revisjon og beredskap & Etabler prosedyrer for hendelsesrapportering, fallback-datasett og kontinuitetsplaner som kan testes i sandkasse. & Kapittel~6 (rapporteringspakke), Appendiks (ressursbank for øvelser) \\
        Læring og faglig deling & Publiser aggregerte indikatorer og læringspunkter i dataspace-portalen og fagfelleloggen, med referanse til gevinstplanen. & Kapittel~7 (gevinststyring), `support/notater/pilotundervisning-materiell.md` \\
        \bottomrule
    \end{tabular}
\end{table}

\paragraph{Samspill med øvrige kapitler.} Seksjonen bør vise hvordan dataspace-designet gjør det mulig å følge kravene til
personvern, kvalitet og beredskap som beskrives i Kapittel~6, samtidig som caset i Kapittel~8 om pasientlogistikk får konkrete
datakilder og indikatorer. I masterkurset kan tabellen brukes som sjekkliste når grupper leverer modelljournal og kvalitetsrapport,
samtidig som `support/oppgavetavle.md` og planfilen oppdateres med ansvarslinjer for helsedata.

\section{Sirkulære materialstrømmer og klimaregnskap}
Sirkulærøkonomi krever at ressursdata knyttes tett til beslutningsprosesser for innkjøp, drift og avvikling. \citet{norskindustri2023sirkular}
peker på at digitale tvillinger gjør det mulig å følge materialstrømmer i sanntid, mens \citet{miljodir2023materialstrommer} framhever
behovet for standardiserte indikatorer for klima- og miljørapportering. Når datastrømmene kobles til tvillingens livssyklus kan
virksomheter planlegge ombruk, spore klimafotavtrykk og dokumentere etterlevelse av nye krav fra EUs taksonomi.

\subsection{Datagrunnlag for ressurs- og klimastyring}
Et sirkulært datagrunnlag bør dekke hele verdikjeden fra innkjøp til ombruk. Følgende elementer er spesielt viktige:
\begin{itemize}
    \item \textbf{Material- og komponentregister:} Strukturerte data om type, mengde, miljøegenskaper og demonterbarhet for hver komponent.
    \item \textbf{Tilstands- og inspeksjonsdata:} Sensormålinger, termografering og manuelle vurderinger som dokumenterer kvalitet før ombruk.
    \item \textbf{CO$_2$-regnskap og energi:} Klimafaktor per materialstrøm, energibruk i bearbeiding og logistikk, samt referanse mot klimabudsjett.
    \item \textbf{Logistikk- og kontraktsinformasjon:} Sporbarhet på leverandører, transport og avtalevilkår for tilbakekjøp eller materialbank.
    \item \textbf{Dokumentasjon til myndigheter:} Rapportpakker for miljødeklarasjoner (EPD) og krav fra taksonomi og anskaffelsesregelverk.
\end{itemize}

Tabell~\ref{tab:kap03-sirkular-data} viser hvordan datasettene kan organiseres for å støtte tvillingens analyser og rapportering.

\begin{table}[ht]
    \centering
    \caption{Datasett for sirkulære materialstrømmer og klimaregnskap.}
    \label{tab:kap03-sirkular-data}
    \begin{tabular}{p{0.26\textwidth}p{0.30\textwidth}p{0.28\textwidth}p{0.10\textwidth}}
        \toprule
        \textbf{Datasett} & \textbf{Primærkilde} & \textbf{Bruksområde} & \textbf{Indikator} \\n        \midrule
        Materialregister & BIM-modell, produktpass, lageroversikt & Kartlegger gjenbrukspotensial og tilgjengelige komponenter. & Ombruksgrad \% \\n        Klimafotavtrykk & Miljødeklarasjoner (EPD), energimålere & Beregner spart CO$_2$ ved ombruk versus nyanskaffelse. & Tonn CO$_2$ spart \\n        Tilstandslogger & IoT-sensorer, inspeksjonsapp & Prioriterer komponenter til videre bruk og planlegger vedlikehold. & Kvalitetsscore (1--5) \\n        Logistikkspor & Transport-API, avfallssystemer & Optimaliserer transport og sikrer dokumentasjon til myndigheter. & Km med lavutslipp \\n        Kontraktsarkiv & Anskaffelsessystem, dataspace-avtaler & Verifiserer eierskap, garantier og tilbakekjøpsrett. & Avtaledekning \% \\n        \bottomrule
    \end{tabular}
\end{table}

Datasett i tabellen kan distribueres gjennom dataspace-arkitekturen ved å definere egne policy-regler for klimadata. Det gjør det enklere å
dele informasjon med leverandører, kommuner og ombrukssentre uten å kompromittere forretningssensitiv informasjon.

\subsection{Case: Ombrukslab for offentlige bygg}
Statsbygg og flere kommuner har etablert laboratorier for å teste ombruk av byggematerialer, der digitale tvillinger kobler sensordata,
materialbank og klimaregnskap. \citet{statsbygg2022ombruk} beskriver hvordan prosjektteam bruker en felles modell for å planlegge demontering,
klassifisere komponenter og visualisere utslippseffekten av ulike beslutninger. En vellykket arbeidsflyt kan organiseres slik:
\begin{enumerate}
    \item \textbf{Kartlegg porteføljen:} Tvillingen henter BIM-data og lager en materialinventarliste med ombrukskategorier og forventet levetid.
    \item \textbf{Utfør tilstandsdiagnose:} Feltinspeksjoner oppdaterer tilstandslogger og utløser sensormålinger for fukt, vibrasjon eller energiavvik.
    \item \textbf{Planlegg logistikk og kontrakter:} Systemet matcher komponenter med nye byggeprosjekter og synkroniserer kontraktsvilkår via dataspace.
    \item \textbf{Evaluer klimaeffekt:} Klimapanelet beregner spart CO$_2$ og rapporterer direkte til virksomhetens bærekraftsrapport.
\end{enumerate}

\paragraph{Oppgaveforslag.} Studentgrupper kan bruke caset til å utvikle dashboards som viser ombruksgrad, klimabesparelse og risikonivå per komponent.
Resultatene kan kobles til tiltaksplanene i Kapittel~7 slik at gevinster og oppfølging får egne ansvarslinjer.

\section{Infrastruktur og sikkerhet}
Arkitekturen bør balansere behovet for lav responstid med kravene til sikkerhet, kostnad og etterlevelse. Norske organisasjoner møter ofte krav om datasuverenitet, samtidig som de ønsker  å dra nytte av skytjenester for elastisitet og tung regnekraft.

\subsection{Edge, sky og hybrid}
Edge-plattformer n å r fysiske prosesser gir rask respons og reduserer b å ndebreddebruken ved  å filtrere data før videresending. Skybaserte løsninger gir tilgang til avanserte analysetjenester og fleksibel lagring. En hybrid tilnærming er vanlig: modellene trenes i skyen og pakkes som containere som kan distribueres tilbake til fabrikkgulvet. God orkestrering (for eksempel med Kubernetes eller Azure Arc) sikrer at oppdateringer kan rulles ut kontrollert.

\subsection{Tilgangsstyring og zero-trust-prinsipper}
Zero-trust innebåe rer at alle forespørsler autentiseres og autoriseres, uansett hvor de kommer fra. I en digital tvilling m å dette omfatte sensorer, API-klienter og mennesker. Bruk av identitetsplattformer, automatisert nøkkellagring og segmenterte nettverk reduserer angrepsflaten. Logsikkerhet, inkludert kryptografisk signering av hendelser, hjelper virksomheten med  å dokumentere etterlevelse og spore hendelser i etterkant.

\subsection{Juridiske hensyn}
GDPR krever at personopplysninger behandles med tydelig hjemmel og strenge tilgangsregler. Selv n å r dataene i utgangspunktet er tekniske, kan kombinasjonen av sensorinformasjon og arbeidsplaner gjøre dem identifiserbare. Datasuverenitet har f å tt økt fokus etter Schrems II-dommen; virksomheter m å vite hvor dataene fysisk lagres og hvilke underleverandører som involveres. Mange velger  å knytte seg til norske eller europeiske skyer, eller  å bruke konfigurerbare sovereign cloud-løsninger.

\section{Beredskap og kontinuitet}
Digitale tvillinger som understøtter kritisk infrastruktur må inngå i organisasjonens beredskapsplaner. Direktoratet for
samfunnssikkerhet og beredskap (DSB) anbefaler at både teknologiske og organisatoriske avhengigheter kartlegges for å sikre
responsevne ved hendelser \citep{dsb2023nrb}. En beredskapsplan for tvillingplattformen kan organiseres i fire hoveddeler:
\begin{enumerate}
    \item \textbf{Forebygging:} Risikovurderinger av datakilder, tilgangsstyring og avhengigheter mot tredjepartsleverandører.
    \item \textbf{Beredskap:} Varslingsplaner, alternative datakilder og manuelle prosedyrer for operatører dersom tvillingen blir utilgjengelig.
    \item \textbf{Respons:} Hendelseshåndtering med tydelig ansvarslinje, logging av beslutninger og kommunikasjon mot myndigheter.
    \item \textbf{Gjenoppretting:} Prioritert plan for å gjenopprette datapipeline, modeller og dashboard, inkludert tester som verifiserer integriteten.
\end{enumerate}

Organisasjonen bør gjennomføre regelmessige øvelser der dataspace-partnere, driftspersonell og sikkerhetsteam deltar. Øvelsene
kan bygge på scenarier fra Nasjonalt risikobilde og inkludere bortfall av leverandører, cyberangrep eller feil i sensornett.
Resultatene dokumenteres i samme fagfellelogg som brukes for datastyringsforumet slik at tiltak følges opp.

\subsection{Kontinuitetsplan for dataplattformen}
Et praktisk hjelpemiddel er å etablere en kontinuitetsmatrise som viser hvilke komponenter som krever redundans. Tabell~\ref{tab:kap03-kontinuitet}
gir et eksempel fra energisektoren.

\begin{table}[ht]
    \centering
    \caption{Kontinuitetstiltak for kritiske komponenter.}
    \label{tab:kap03-kontinuitet}
    \begin{tabular}{p{0.34\textwidth}p{0.56\textwidth}}
        \toprule
        \textbf{Komponent} & \textbf{Tiltak og ansvar} \\
        \midrule
        Sensor- og edgeinfrastruktur & Redundant strøm og kommunikasjon, avtaler om nødlager for reservedeler, ansvarlig: teknisk drift. \\
        Meldingsplattform & Klyngeoppsett på tvers av datasentre, automatisert failover-test, ansvarlig: plattformteam. \\
        Datakatalog & Daglig backup til isolert sone og plan for manuell distribusjon av kritiske metadata, ansvarlig: data steward. \\
        Modelltjenester & Containerimages lagres i sikker registry med signering, gjenoppretting automatisert via IaC, ansvarlig: DevOps. \\
        Dashboards & Offline-snapshots av nøkkelindikatorer og utsending til beredskapsrom, ansvarlig: forretningskontinuitet. \\
        \bottomrule
    \end{tabular}
\end{table}

Kontinuitetsplanen bør oppdateres etter hver øvelse og ved større endringer i dataspace-arkitekturen. Ved å koble planen til
organisasjonens helhetlige beredskapsrammeverk blir tvillingen en integrert del av den operative beredskapen.

\subsection{Case: Oslo kommunes vann- og avløpsberedskap}
Oslo kommunes vann- og avløpsetat (VAV) har etablert en digital tvilling som samler sensordata, hydrauliske modeller og
beredskapsplaner for hovedforsyningen til byen. Tvillingen følger både råvannsinntak, vannbehandlingsanlegg og soner med høy
lekkasjerisiko, og kobles til varslingstjenestene i DSB sitt CIM-system. Resultatet er at planleggere kan teste konsekvensen av
ledningsbrudd eller forurensede magasiner før tiltak iverksettes \citep{oslovav2023digital}. Data fra SCADA-systemet strømmes til
tvillingen i sanntid, mens historiske vedlikeholdslogger og entreprenørdata gjøres tilgjengelig som batchoppdateringer hver
natt. Plattformen deler dessuten aggregert informasjon med nabokommuner via en begrenset dataspace slik at alternativ
vannforsyning kan koordineres når hendelser oppstår \citep{norskvann2023digitaltvilling}.

Etaten har utviklet en tiltakstabell som kobler hendelsestyper til datasett, ansvarslinjer og beslutningstid. Tabell~\ref{tab:kap03-oslo-vav}
viser et utdrag som brukes i øvelser og faktisk drift.

\begin{table}[ht]
    \centering
    \caption{Tiltakstabell for Oslo VAV sin digitale tvilling.}
    \label{tab:kap03-oslo-vav}
    \begin{tabular}{p{0.30\textwidth}p{0.28\textwidth}p{0.28\textwidth}}
        \toprule
        \textbf{Hendelse} & \textbf{Aktiverte datasett} & \textbf{Tiltak og ansvar} \\
        \midrule
        Ledningsbrudd i kritisk sone & Trykksensorer, lekkasjeindikatorer, gravepåvirkning fra entreprenør & Beredskapsvakt varsles via CIM, tvillingen simulerer avstengningsscenario, entreprenør mobiliseres innen 60 minutter. \\
        Forurenset råvann & Vannkvalitetslogger, værdata, historikk for innsjø & Tvillingen beregner alternative inntak, VAV informerer Folkehelseinstituttet, bydelene får varslingsmaler for kokevarsel. \\
        Strømutfall på pumpestasjon & SCADA-hendelser, kontinuitetsplan for nødstrøm, energimåler & Driftssentral koordinerer reservestrøm og tanker, tvillingen prioriterer soner med kritisk infrastruktur. \\
        \bottomrule
    \end{tabular}
\end{table}

Øvelsene kjøres i samarbeid med kommunens kriseledelse og nabokommuner gjennom interkommunalt vannverk. Erfaringen er at
kombinasjonen av simulering og faktisk hendelseslogg gir raskere beslutninger når hendelser oppstår, og at tiltak kan prioriteres
etter både helseeffekt og samfunnskritiske funksjoner \citep{oslovav2023digital}. Tvillingen er også koblet til et publikt
informasjonspanel som viser status for lekkasjearbeid og planlagte avbrudd, noe som reduserer antall henvendelser til
kundesenteret.

\paragraph{Overføringspunkter til undervisning og praksis.} Caset brukes i masterprosjekter ved OsloMet og NMBU der studenter
skal utvikle forbedrede lekkasjemodeller og planlegge deling av data med private entreprenører. Øvelsesopplegget gir en
praktisk ramme for å trene på samspill mellom teknisk drift, kommunikasjon og beredskapsledelse. Materialet kan gjenbrukes i
kapittel~6 når DPIA og hendelsesjournal skal dokumenteres, og i kapittel~7 når kommunal governance og datasamarbeid diskuteres.

\section{Helsesektorens dataspace og kontinuitet}
Helsesektoren forvalter sensitive pasientdata, kritiske samfunnsfunksjoner og komplekse leverandørkjeder. Derfor må dataspace-
arkitekturen kombineres med kontinuitetsplaner som oppfyller kravene i nasjonal helseberedskap og personvernforordningen.
Helsedirektoratet beskriver hvordan digitale tvillinger kan brukes til å dele situasjonsrapporter og ressursoversikter mellom
kommuner, sykehus og nasjonale koordineringssentre, forutsatt at tilgangsstyring og logging er innebygget \citep{helsedir2023beredskap}.
Norsk helsenett utvikler samtidig en nasjonal plattform for helsedataspace der konnektorer og sertifisering legger grunnlaget
for sikker datadeling mellom journal- og analysesystemer \citep{nhn2024dataspace}. Når slike plattformer kobles til digitale
tvillinger, kan sanntidsdata om kapasitet, smitte og logistikk gjøres tilgjengelig uten at pasientidentiteter eksponeres.

\subsection{Felles datagrunnlag for beredskap}
For å støtte kontinuerlig drift må dataspace-laget synliggjøre hvilke dataprodukter som dekker beredskapssituasjoner. Helseaktører
har identifisert tre kategorier som bør prioriteres i første fase \citep{helsedir2023beredskap,helseplattformen2023kontinuitet}:
\begin{itemize}
    \item \textbf{Kapasitet og logistikk:} Oppdaterte tall for intensivkapasitet, bemanning, lager av legemidler og medisinsk utstyr,
    hentet fra journalsystemer og logistikkapplikasjoner.
    \item \textbf{Pasientstrømmer og hendelser:} Aggregert informasjon om innlagte, hastegrad og smitteutbrudd, kombinert med
    hendelseslogger fra legevakt og AMK-sentraler.
    \item \textbf{Støtteprosesser:} Tilgjengelighet på laboratorietjenester, digitale konsultasjoner og eksterne leverandører,
    slik at alternative behandlingsløp kan planlegges dersom ordinær drift svikter.
\end{itemize}
Dataspace-katalogen bør tydelig merke hvilke datasett som krever behandlingsgrunnlag etter helsepersonelloven, og hvilke som
kan deles som aggregerte beslutningsstøttepakker. Dette legger til rette for at modellene i kapittel~6 kan kobles direkte til
beredskapsplanene uten å kompromittere personvern.

\subsection{Roller og kontinuitetskontroller}
Kontinuitetsstyring i helsesektoren krever en tydelig fordeling av ansvar mellom tekniske og kliniske roller. Tabell~\ref{tab:kap03-helse-dataspace}
viser hvordan roller, nøkkelkontroller og rapporteringspunkter kan organiseres når en digital tvilling skal støtte beredskap.

\begin{table}[ht]
    \centering
    \caption{Roller og kontroller i helsesektorens dataspace for digitale tvillinger.}
    \label{tab:kap03-helse-dataspace}
    \begin{tabular}{p{0.24\textwidth}p{0.37\textwidth}p{0.27\textwidth}}
        \toprule
        \textbf{Rolle} & \textbf{Kontroller og ansvar} & \textbf{Rapportering og samhandling} \\
        \midrule
        Nasjonalt koordineringssenter & Godkjenner datasett for deling, vurderer beredskapsscenarier og initierer nasjonale tiltak. & Ukentlig status mot Helsedirektoratet og regionale helseforetak. \\
        Regional helsetjeneste & Synkroniserer kapasitetstall, driftsscenarier og kritiske hendelser i dataspace-konnektoren. & Daglige rapporter til kommuner og sykehus via felles situasjonsrom. \\
        Kommune og fastlegekontor & Oppdaterer lokale tiltak, pasientstrømmer og tilgjengelige tjenester, og mottar varsler fra tvillingmodellen. & Deler læringspunkter i samhandlingsportal og kvalitetsutvalg. \\
        Teknologileverandør & Sikrer kryptering, anonymisering og audit-logg i tvillingplattformen, samt tester beredskapsskript. & Leverer revisjonsrapporter og hendelseslogger til helseforetakets sikkerhetsråd. \\
        Personvernombud & Verifiserer behandlingsgrunnlag, lagringspolicy og dataminimering for hvert datasett i katalogen. & Kvartalsvis rapport til styret og Datatilsynet ved avvik. \\
        \bottomrule
    \end{tabular}
\end{table}

Tabellen kan brukes som en sjekkliste når nye datasett eller modeller kobles til tvillingen. For hver rad bør ansvarlige fylle ut
konkrete KPI-er i tillitsindikatorpanelet fra kapittel~6 slik at endringer i kapasitet eller regelverk raskt oppdages.

\subsection{Implementering og læringssløyfer}
Erfaringer fra Helseplattformen viser at kontinuitetsplanlegging må følges av strukturerte læringssløyfer for å holde dataspace-
komponentene oppdaterte \citep{helseplattformen2023kontinuitet}. Et anbefalt løp er:
\begin{enumerate}
    \item \textbf{Etabler beredskapslogger:} Knytt dataspace-hendelser til loggformatet i beredskapssystemet slik at varsler kan
    spores gjennom hele livssyklusen.
    \item \textbf{Test alternative arbeidsprosesser:} Bruk tvillingen til å simulere omdirigering av pasientstrømmer, mobilisering
    av ekstra kapasitet og fallback-løsninger for kritiske tjenester.
    \item \textbf{Koble til evaluering:} Etter øvelser eller hendelser analyseres beslutningslogg, datakvalitet og responstid i
    fellesskap, og resultatene publiseres i dataspace-katalogens læringsseksjon.
\end{enumerate}
Når denne prosessen er forankret i både teknologi- og klinikkmiljøene, kan helsesektoren bruke digitale tvillinger til å støtte
koordinert respons ved neste krise samtidig som kontinuerlig drift ivaretas.

\subsection{Case: Velferdsteknologisk dataspace for hjemmeoppfølging}
Kommuner bygger nå responssentertjenester som kobler trygghetsalarmer, medisinsk avstandsoppfølging og digitale tvillinger av hjemmebaserte tjenester.\citep{helsedir2020dho} For å gi helse- og velferdsteam et felles situasjonsbilde må data fra sensorer, elektronisk pasientjournal og mobile omsorgsløsninger deles gjennom et dataspace som ivaretar både personvern og beredskap.\citep{ks2024responssenter} Tvillingen beregner risiko for forverring, prioriterer hjemmebesøk og dokumenterer tiltak slik at responsteam kan koordinere med fastlege, legevakt og sykehus uten å miste kontroll på tilgangen til sensitive opplysninger.\citep{nhn2024dataspace}

\paragraph{Datastrømmer og styringspunkter.} En velferdsteknologisk dataspace består av flere dataprodukter som må ha klare kontrakter og kvalitetsindikatorer før de tas i bruk i drift. Tabell~\ref{tab:velferd-dataprodukter} viser et minimumssett som kommunale team kan bruke når de etablerer et felles kontrollrom for hjemmeoppfølging.

\begin{table}[ht]
    \centering
    \caption{Dataprodukter for velferdsteknologisk dataspace i hjemmebaserte tjenester.}
    \label{tab:velferd-dataprodukter}
    \begin{tabular}{p{0.32\textwidth}p{0.38\textwidth}p{0.26\textwidth}}
        \toprule
        \textbf{Dataprodukt} & \textbf{Kjerneinnhold og datakilder} & \textbf{Styringskobling} \\
        \midrule
        Trygghetsalarm- og sensorstrømmer & Hendelser fra smarthus-sensorer, medisinske målinger og mobile trygghetsalarmer med kvalitetsflagg. & Hendelseslogg og kvalitetsjournal i kapittel~6, varsling til responssenter innen 2 minutter. \\
        Pleieplan- og oppgavekart & Oppdatert pasientoversikt med vedtak, tiltak og planlagte besøk fra fagsystem og journal. & Gevinst- og porteføljeloggen i kapittel~7, koordinering med bemanningsplan. \\
        Risiko- og prioriteringsmodell & Modellscore som rangerer pasienter etter klinisk risiko og sosialt støttebehov, bygget på kombinerte sensor- og journaldatasett. & Valideringsprotokoller for kliniske beslutningsstøtter i kapittel~6, sign-off fra ansvarlig lege. \\
        Samhandlings- og eskaleringslogg & Tidsstempler fra meldingsutveksling med fastleger, legevakt og spesialisthelsetjenesten. & Dataspace-governance fra \autoref{tab:kap03-helse-dataspace}, rapportering til fagfellelogg og kapittel~7. \\
        Bærekrafts- og gevinstpanel & Aggregert effekt på liggedøgn, hjemmeboende tid og frigjorte årsverk, med kobling til klima- og transportdata. & Gevinstmålingene i kapittel~7 og bærekraftsindikatorene i kapittel~4 og \autoref{tab:kap03-dashboard}. \\
        \bottomrule
    \end{tabular}
\end{table}

Dataspace-operatøren må sikre at hvert dataprodukt har definerte eierroller, databehandlingsgrunnlag og terskler for når avvik skal varsles. Dette gjør det mulig å koble alarmstrømmene direkte til hendelsesresponsen i kapittel~6 uten å miste kontekst om vedtak eller planlagte tjenester. Når indikatorer for gevinst og arbeidsbelastning er tilgjengelige i samme tvilling, kan kommunene dokumentere effekten av hjemmeoppfølging i gevinstplanen fra kapittel~7.

\paragraph{Implementering og kvalitetssikring.} Kommuner som tar i bruk digitale tvillinger for hjemmeoppfølging anbefales å jobbe i fire steg:
\begin{enumerate}
    \item \textbf{Forankre behandlingsgrunnlag og informasjonsmodeller:} Sikre at pasient- og tjenesteinformasjon følger Normen for informasjonssikkerhet og at datasett beskrives med felles informasjonsmodell for velferdsteknologi.\citep{helsedir2020dho}
    \item \textbf{Synkroniser responssenter og hjemmebaserte tjenester:} Responssenteret bruker dataspace-konnektorer for å dele alarmdata og tiltakslister i sanntid, mens hjemmebaserte team bekrefter utførte oppgaver via mobile løsninger.\citep{ks2024responssenter}
    \item \textbf{Valider risiko- og prioriteringsmodeller:} Modellscore testes mot historiske hendelser og kliniske vurderinger før de aktiveres i drift, og dokumenteres i kvalitetsjournalen fra kapittel~6.
    \item \textbf{Følg opp gevinst og bærekraft:} Måleparametere for selvstendig boende tid, redusert transport og arbeidsbesparelser rapporteres i gevinstloggen. Resultatene deles med politisk nivå og fagfellelogg slik at videre skalering kan planlegges.
\end{enumerate}

Når dataspace, responssenter og hjemmebaserte tjenester deler samme tvillinggrunnlag, kan kommunene tilby mer helhetlig oppfølging samtidig som de dokumenterer gevinst og etterlevelse. Caset gir også masterstudentene en arena for å koble dataintegrasjon, helsefaglig vurdering og gevinststyring i én sammenhengende øvingsoppgave.

\section{Bærekrafts-dashboard og indikatorstyring}
For å gjøre bærekraft og klimaeffekter synlige i operative beslutninger bør digitale tvillinger levere dashboards som kombinerer
miljøindikatorer, energieffektivitet og økonomiske nøkkeltall. Statnett og andre norske infrastruktureiere rapporterer at slike
dashboards er viktige for å dokumentere innsparingstiltak og rapportere til myndigheter \citep{statnett2024baerekraft}. Et
bærekraftsdashboard kan struktureres rundt tre nivåer:
\begin{itemize}
    \item \textbf{Strategisk nivå:} KPI-er for utslippsreduksjoner, fornybarandel og taksonomi-etterlevelse.
    \item \textbf{Taktisk nivå:} Energibruk, materialforbruk og logistikkindikatorer fordelt på anlegg og tidsperioder.
    \item \textbf{Operativt nivå:} Varsler på avvik fra planlagte tiltak, prediksjoner fra tvillingmodeller og anbefalte tiltak.
\end{itemize}

Ved å koble indikatorene til dataspace-laget kan partnere dele aggregerte resultater uten å avsløre forretningssensitiv informasjon.
Det gir samtidig beslutningstakere i kapitlene 4 og 7 et faktagrunnlag for å prioritere tiltak.

\subsection{Eksempel på indikatorstruktur}
Tabell~\ref{tab:kap03-dashboard} viser hvordan indikatorer kan knyttes til datakilder og ansvarlige roller i en energioperasjon.

\begin{table}[ht]
    \centering
    \caption{Indikatorer i et bærekraftsdashboard.}
    \label{tab:kap03-dashboard}
    \begin{tabular}{p{0.28\textwidth}p{0.32\textwidth}p{0.30\textwidth}}
        \toprule
        \textbf{Indikator} & \textbf{Datakilder} & \textbf{Ansvarlig rolle} \\
        \midrule
        Scope 1-utslipp & Sensorer fra forbrenningsanlegg, drivstoffrapporter & Bærekraftsleder \\
        Energitap i nett & SCADA-målinger, værdata, prognoser fra tvilling & Driftssentral \\
        Materialgjenvinningsgrad & ERP, materialpass, dataspace for leverandører & Innkjøp og sirkulærteam \\
        Leveransetid for tiltak & Prosjektverktøy, avvikslogger & Programkontor \\
        Kundepåvirkning & Hendelseslogger, kundeservice, sosiale medier & Kommunikasjon \\
        \bottomrule
    \end{tabular}
\end{table}

For å unngå «greenwashing» må indikatorene revideres av uavhengige fagmiljøer. Resultatene kan kobles til læringssløyfene i
Kapittel~5 slik at modeller kalibreres med faktiske effekter av tiltakene.

\section{Datastyring og kontinuerlig forbedring}
For  å sikre at datapipelinen leverer konsistent kvalitet m å tekniske tiltak kobles til tydelige roller og arbeidsprosesser. Norske virksomheter etablerer ofte et datastyringsforum som samler produkteier, data steward, sikkerhetsansvarlig og representanter fra drift. Forumet prioriterer tiltak basert p å indikatorer for datakvalitet, hendelseslogger og fagfelleinnspill.

\subsection{Roller og møtearenaer}
\begin{itemize}
    \item \textbf{Data steward:} Overv å ker datakvalitet og initierer korrigerende tiltak n å r regler brytes, for eksempel ved  å sende saker til endringsstyret eller oppdatere datakontrakter.
    \item \textbf{Produkteier for tvillingen:} Sikrer at hendelser med høy forretningsrisiko følges opp og at prioriteringene synkroniseres med gevinstplaner i Kapittel~7.
    \item \textbf{Teknisk plattformteam:} Forvalter overv å kingsdashbord, versjonering av dataprosesser og automatiserte tester som kjøres hver gang datastrukturer endres.
\end{itemize}
Forumet kan møtes ukentlig i innføringsfasen og deretter m å nedlig for moden drift. Referat og tiltak registreres i fagfelleloggen (DI-03) slik at status er synlig for hele redaksjonen.

\subsection{Operasjonelle kontrollpunkter}
\begin{enumerate}
    \item \textbf{Datakvalitetsdashbord:} Visualiserer fullstendighet, tidsforsinkelser og valideringsfeil for hver datastrøm. Alarmgrenser defineres i samarbeid med domeneeksperter og dokumenteres i styringspakken til Kapittel~6.
    \item \textbf{Hendelsesrespons:} Avvik loggføres i samme system som sikkerhetshendelser. Hvert avvik f å r ansvarlig person, frist og forslag til kompenserende tiltak (for eksempel fallback-modell eller manuelt kontrollsteg).
    \item \textbf{Låe ringssløyfe:} Kvartalsvise retrospektiv analyserer mønstre i datakvalitetsavvik og oppdaterer pipeline-design eller opplåe ringsprogram for operatører.
\end{enumerate}
Detaljert praksis for indikatorene, møtesstruktur og låe ringssløyfe er nedfelt i delingsnotatet `support/notater/datastyringsforum-di03.md`,
som deles med fagfeller i arbeidet med kommentar DI-03.

\subsection{Samsvar med personvern og etikk}
N å r nye datakilder tas i bruk bør datastyringsforumet sjekke behandlingsgrunnlag, vurdering av dataminimering og rutiner for sletting. Et eget sjekkpunkt sikrer at dokumentasjonen speiler kravene fra Helsedirektoratet, Normen og Kapittel~6 om validering og tillit. N å r tvillingen skaleres p å tvers av anlegg, anbefales det  å utvide datakatalogen med tydelige beskrivelser av form å l og kontaktpunkt for hvert datasett.

\section{Sirkulåe røkonomi og materialsløyfer}
Sirkulåe røkonomi krever at data følger materialene gjennom hele verdikjeden slik at gjenbruk, reparasjon og resirkulasjon kan dokumenteres. Den nasjonale strategien for sirkulåe r økonomi fremhever digitale tvillinger som en nøkkel for  å spore klimaavtrykk og ressursbruk \citep{regjeringen2021sirkulaer}. EU sitt handlingsprogram for sirkulåe røkonomi understreker behovet for  å pne standarder, produktpass og datadeling mellom produsenter, brukere og resirkuleringsaktører \citep{eu2020circulareconomy}. Oversikten under viser hvordan data flyter i en materialsløyfe n å r en digital tvilling kobler sammen design, produksjon, bruk og retur:
\begin{itemize}
    \item \textbf{Designfase:} Produktpass definerer materialinnhold, moduloppbygging og forventet levetid.
    \item \textbf{Produksjon:} Kvalitetstester, prosessdata og energibruk logges og knyttes til partinumre.
    \item \textbf{Bruk:} Sensorer og vedlikeholdslogger oppdaterer tilstand og restlevetid for komponentene.
    \item \textbf{Retur og ombruk:} Inspeksjonsdata avgjør om komponenter gjenbrukes, repareres eller materialgjenvinnes.
    \item \textbf{Resirkulasjon:} Massebalanser og sporbarhet sikrer at resirkulert innhold dokumenteres mot nye produktserier.
\end{itemize}

\subsection{Indikatorer og datakilder}
For  å støtte sirkulåe re beslutninger bør digitale tvillinger fange indikatorer for materialinnhold, karbonavtrykk, levetid og restverdi. Dataene m å våe re tilgjengelige i produktpass eller digitale loggbøker slik at b å de produsenter og resirkulatører vet hvilke komponenter som kan demonteres og gjenbrukes. Kombinasjonen av sensorinformasjon, ERP-data og kvalitetstester gjør det mulig  å beregne hvor mye av en komponent som kan gjenbrukes uten ytterligere bearbeiding. Norske virksomheter som deltar i grønn plattform-programmet rapporterer at slike indikatorer er nødvendige for  å f å støtte til pilotering \citep{miljodir2022sirkular}.

\subsection{Case: Hydro CIRCAL og materialsporing}
Hydro har etablert en digital tvilling for  å dokumentere innholdet av resirkulert aluminium i CIRCAL-produktene sine. Plattformen kobler produksjonsdata fra pressverkene med kvalitetstester og kundesertifikater, slik at hele verdikjeden kan verifisere klimap å standen \citep{hydro2023traceability}. Tvillingen gjør det mulig  å følge materialet fra innsamling av skrap via smelteprosesser til ferdig profil, og bidrar til  å oppn å tredjeparts verifisering i henhold til EN~15088.

\begin{table}[ht]
    \centering
    \caption{Datakomponenter i Hydro sitt CIRCAL-program.}
    \label{tab:kap03-hydro}
    \begin{tabular}{p{0.34\textwidth}p{0.56\textwidth}}
        \toprule
        \textbf{Komponent} & \textbf{Beskrivelse} \\
        \midrule
        Materialpass & Digitalt pass med andel resirkulert aluminium, partinummer og opprinnelse. \\
        Prosessdata & Temperatur- og energilogger fra smelteovner og presslinjer, koblet til kvalitetstester. \\
        Kunderapportering & Automatisk generering av sertifikater og CO$_2$-beregninger til arkitekter og byggherrer. \\
        Effekter & 75\% resirkulert innhold dokumentert og redusert tid p å revisjoner med 40\%. \\
        Samarbeid & Deler data med partnere i EU sitt Circular Aluminium-program for  å harmonisere indikatorer. \\
        \bottomrule
    \end{tabular}
\end{table}

Caset viser hvordan en industriell aktør bruker digitale tvillinger til  å skape tillit i markedet og møte regulatoriske krav om  å penhet.

\subsection{Case: Loopfront og Statsbyggs ombruksprosjekter}
Loopfront leverer en plattform for ombruk av byggematerialer som brukes av Statsbygg og flere kommuner til  å kartlegge materialbanker før rehabilitering \citep{statsbygg2023loopfront}. Gjennom digitale tvillinger av bygningsmassen kombineres BIM-modeller, tilstandsanalyser og logistikkdata for  å planlegge demontering og ny bruk. Systemet beregner potensielle CO$_2$-besparelser og økonomiske gevinster per komponent, og gjør det mulig  å reservere materialer direkte i prosjektporteføljen.

\begin{table}[ht]
    \centering
    \caption{Nøkkelindikatorer i Statsbyggs ombruksportefølje.}
    \label{tab:kap03-loopfront}
    \begin{tabular}{p{0.36\textwidth}p{0.54\textwidth}}
        \toprule
        \textbf{Indikator} & \textbf{Resultater fra pilotprosjekter} \\
        \midrule
        Kartlagt volum & 28\,000 m$^2$ bygningsmasse med digitalt materialkart i 2024. \\
        Datakilder & BIM-modeller, laserskanning, materialpass og logistikkdata fra entreprenør. \\
        Beslutningsstøtte & Dashboard som viser CO$_2$-besparelser, kostnader og tilgjengelighet per komponent. \\
        Effekt & 1\,200 tonn materialer ombrukt med estimert 1,9 kilotonn CO$_2$ spart i første fase. \\
        Samhandling & Deling av datagrunnlag med kommuner og private eiendomsforvaltere for  å koordinere etterspørsel. \\
        \bottomrule
    \end{tabular}
\end{table}

Eksemplet illustrerer hvordan kombinasjonen av digitale tvillinger og sirkulåe røkonomi gir nye arbeidsprosesser for offentlige byggherrer og leverandører.

\section{Refleksjonsspørsm å l og øvinger}
\begin{enumerate}
    \item Tegn et dataflytdiagram for en digital tvilling i en norsk industribedrift.
    \item Vurder nåar det er hensiktsmessig  å bruke sky kontra edge for sanntidsanalyse.
    \item Beskriv hvordan du ville etablere en governance-modell for datatilgang.
    \item Foresl å indikatorer for datakvalitet som bør overv å kes kontinuerlig i tvillingen.
\end{enumerate}
