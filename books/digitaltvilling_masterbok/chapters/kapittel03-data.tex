\chapter{Data, integrasjon og infrastruktur}

\section{Læringsmål}
\begin{itemize}
    \item Beskrive arkitekturen for datafangst, lagring og distribusjon i digitale tvillinger.
    \item Evaluere integrasjonsmønstre og standarder.
    \item Utforme krav til sikkerhet, personvern og datasuverenitet.
\end{itemize}

\section{Dataflyt og pipeline-design}
\begin{itemize}
    \item Fra sensor til innsikt: innsamling, behandling, lagring.
    \item Batch kontra streaming og krav til latens.
    \item Metadata, semantikk og masterdata.
\end{itemize}

\section{Integrasjonsmønstre og standarder}
\begin{itemize}
    \item API-er, meldingskøer og eventdrevne arkitekturer.
    \item Standarder: OPC UA, MQTT, Asset Administration Shell.
    \item Datakvalitet og interoperabilitet.
\end{itemize}

\section{Infrastruktur og sikkerhet}
\begin{itemize}
    \item Edge, sky og hybrid arkitektur.
    \item Tilgangsstyring, IAM og zero-trust-prinsipper.
    \item Juridiske hensyn: GDPR, datasuverenitet, datadeling.
\end{itemize}

\section{Refleksjonsspørsmål og øvinger}
\begin{enumerate}
    \item Tegn et dataflytdiagram for en digital tvilling i en norsk industribedrift.
    \item Vurder når det er hensiktsmessig å bruke sky kontra edge for sanntidsanalyse.
    \item Beskriv hvordan du ville etablere en governance-modell for datatilgang.
\end{enumerate}
