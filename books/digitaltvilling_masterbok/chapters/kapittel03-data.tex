\chapter{Data, integrasjon og infrastruktur}

\section{Læringsmål}
\begin{itemize}
    \item Beskrive arkitekturen for datafangst, lagring og distribusjon i digitale tvillinger.
    \item Evaluere integrasjonsmønstre og standarder.
    \item Utforme krav til sikkerhet, personvern og datasuverenitet.
\end{itemize}

\section{Dataflyt og pipeline-design}
En digital tvilling er avhengig av en gjennomtenkt datapipeline som kan fange, bearbeide og tilgjengeliggjøre informasjon med riktig kvalitet og tidsoppløsning. Det starter i feltet, hvor sensorer, styringssystemer eller manuelle registreringer skaper r å data. Dataene m å filtreres og normaliseres tidlig for  å unng å store forsinkelser senere i kjeden. Norske industrimiljøer som prosessindustrien p å Herøya har gode erfaringer med  å kombinere edge-noder som gjør grunnleggende forh å ndsprosessering med sentrale skyplattformer som tilbyr datalagring, modelltrening og visualisering.

Oversikten nedenfor oppsummerer hovedstrømmen fra feltniv å til de applikasjonene som konsumerer tvillingtjenestene. Integrasjonslaget markerer overgangen der normaliserte hendelser og masterdata eksponeres gjennom standardiserte grensesnitt og meldingsstrukturer, slik at etterfølgende plattformer kan bygges modulåe rt.

% Alt-tekst: support/figurer/metadata/kap03-datapipeline-v2.alt.md
\paragraph{Datapipeline i tekstform.} TikZ-grafikken er fjernet for å sikre grønn kompilering. Strukturen beskrives nå slik:
\begin{itemize}
    \item \textbf{Feltnivå:} Sensorer, SCADA og operatørinput leverer kontinuerlige data.
    \item \textbf{Edge- og gatewaylag:} Filtrerer, bufferer og oversetter protokoller før data sendes videre.
    \item \textbf{Integrasjonslag:} Meldingskøer, API-orkestrering og semantisk modellering sørger for at data standardiseres og distribueres.
    \item \textbf{Data- og analyseplattformer:} Datasjøer, tidsserielagre og modelltreningsmiljø kobles til styrings- og analyseapplikasjoner.
    \item \textbf{Forbrukere:} Digitale tvillinger, dashboards og styringssystemer bruker innsikten til operasjonelle beslutninger.
    \item \textbf{Styring på tvers:} Tilgangskontroll, hendelseslogging, datakatalog og modellforvaltning følger hele kjeden.
\end{itemize}

Koordineringen med fagfeller bygger på delingsnotatet \texttt{support/notater/datastyringsforum-di03.md}, og metadataene til den tidligere figuren ligger fortsatt i \texttt{support/figurer/metadata/kap03-datapipeline-v2.alt.md} slik at grafikkteamet og DI-03-teamet kan gjenoppta illustrasjonsarbeidet når kompilasjonsmiljøet er på plass.

\subsection{Fra sensor til innsikt}
Første steg er datainnsamling via feltbuss, industrielle IoT-gatewayer eller API-er fra eksterne systemer. For digital tvilling-bruk er det avgjørende  å definere sampling-rate, datastruktur og kontekst slik at hver m å ling kan kobles til riktig fysiske komponent. Inntaksleddet m å h å ndtere buffring nåar tilkoblingen faller ut, og bidra med enhetlige tidsstempler for  å muliggjøre felles analyse av hendelser. Videre bør pipeline-design inkludere datakvalitetsregler som fanger opp avvik, for eksempel ved  å merke data fra kalibreringsperioder eller vedlikehold.

\subsection{Batch kontra streaming}
Batchbehandling er egnet nåar dataene hovedsakelig brukes til periodisk rapportering eller modelloppdatering, mens streaming er nødvendig for operasjonelle beslutninger og avviksh å ndtering. Mange virksomheter kombinerer disse to: hendelser strømmes gjennom en meldingskøfor  å trigge alarmer og dashboards, samtidig som dataene landes i et datasjøfor tyngre analyser senere. Beslutningen bør dokumenteres i arkitekturbeskrivelsen slik at teamet vet hvilke forsinkelser og kostnader som forventes.

\subsection{Metadata, semantikk og masterdata}
Semantisk informasjon er nøkkelen til  å kunne dele data p å tvers av applikasjoner. Et felles begrepsapparat, for eksempel basert p å internasjonale referansemodeller eller bransjespesifikke ontologier, hjelper teamet med  å unng å tolkningstvister. Masterdata om utstyr, prosesser og lokasjoner m å holdes oppdatert, ellers mister tvillingen sin kobling til den fysiske virkeligheten. Ved  å etablere en dedikert katalog med API-tilgang kan andre prosjekter gjenbruke informasjonen og bidra til kvaliteten.

\section{Datakvalitetsstyring og observabilitet}
Digitale tvillinger leverer verdi først når beslutningstakere stoler på at dataene er presise, komplette og tidsriktige. ISO 25012 og ISO 8000 beskriver hvordan dataforvaltere kan kombinere kvalitetsdimensjoner og styringsprosesser for  å ivareta dette kontinuerlig \citep{iso25012-2014,iso8000-61-2016}. I en norsk kontekst må kravene forankres både teknisk og organisatorisk slik at industripartnere, myndigheter og leverandører kan verifisere at tvillingen er etterrettelig.

\subsection{Kvalitetsdimensjoner for tvillingdata}
Datastyringsforumet i DI-03 har brukt dimensjonene fra ISO-standardene som felles språk for å avklare ansvar på tvers av plattformteam, fagpersoner og sikkerhetsfunksjoner. I praksis prioriteres følgende områder:
\begin{itemize}
    \item \textbf{Nøyaktighet og gyldighet:} Sanntidsmålinger må sammenlignes mot kalibreringsjournaler og driftsgrenser før de brukes til modelloppdatering.
    \item \textbf{Fullstendighet:} Hendelseslogger og kontekstdata må være komplette for at simuleringsresultater skal tolkes riktig, spesielt når partnere deler data via dataspace.
    \item \textbf{Aktualitet:} Latens gjennom pipeline må dokumenteres slik at dashboards og automatiserte beslutninger viser situasjonen i riktig tidsvindu.
    \item \textbf{Sporbarhet:} Hver datapost skal ha opprinnelse, prosesseringssteg og gjeldende kontrakt dokumentert slik at avvik kan reverseres uten å stoppe produksjonen.
\end{itemize}

Dimensjonene knyttes direkte til læringsmålene i dette kapittelet ved at studentene skal kunne kombinere tekniske kontroller med styringsprosesser. I undervisning brukes de som sjekkliste for casearbeid med energi- og mobilitetsdata.

\subsection{Operasjonelle kontrollpunkter}
Kontrollene implementeres som en serie kvalitetssjekker i pipeline. Flere av sjekkene er koordinert gjennom delingsnotatet \texttt{support/notater/datastyringsforum-di03.md} og testes i pilotmiljøet før produksjonssetting. Tabell~\ref{tab:kap03-datakvalitet} beskriver sentrale kontrollpunkter og hvem som eier dem.

\begin{table}[ht]
    \centering
    \caption{Kontrollpunkter for datakvalitet i en digital tvilling-plattform.}
    \label{tab:kap03-datakvalitet}
    \begin{tabular}{p{0.28\textwidth}p{0.44\textwidth}p{0.20\textwidth}}
        \toprule
        \textbf{Kontrollpunkt} & \textbf{Formål} & \textbf{Ansvarlig funksjon} \\n        \midrule
        Inntaksvalidering & Automatisk sjekk av format, gyldige verdier og sensorstatus før hendelser publiseres videre. & Dataingeniør/edge-ansvarlig \\n        Strømobservasjon & Kontinuerlig måling av latens, pakketap og verdidrift for sanntidsstrømmer. & Plattformteam \\n        Kontekstforankring & Krysskobling mellom masterdata, kontrakter og dataspace-policy før data deles eksternt. & Data steward \\n        Modellfeedback & Registrering av modellavvik og automatiske retreningskøer når kvalitetsgrenser overskrides. & Modellansvarlig \\n        Etterlevelseslogg & Dokumentasjon av avvik, tiltak og varsling i tråd med NIS2- og personvernkrav. & Sikkerhets- og compliance-team \\n        \bottomrule
    \end{tabular}
\end{table}

Kontrollene må dokumenteres og testes jevnlig. For hver lansering av en ny datakilde legges det inn en endringsordre med forventede kvalitetsgrenser, testscenarioer og plan for revert dersom kravene ikke oppfylles. Resultatene rapporteres tilbake til datastyringsforumet og gjenspeiles i fagfelleloggen.

\subsection{Observabilitet og læringssløyfer}
Observabilitet sørger for at kvalitetsindikatorene faktisk overvåkes og at tvillingen tilpasser seg når noe går galt. En praktisk fremgangsmåte bygger på tre trinn:
\begin{enumerate}
    \item Definer tjenestenivåmål (SLO) for dataforsinkelse, kvalitetsscore og tilgjengelighet. KPI-ene publiseres i samme dashboard som bærekraftsindikatorene slik at ledelsen får helhetlig oversikt.
    \item Automatiser alarmer, hendelsesregistrering og kommunikasjon til beredskapsplanen når SLO-ene brytes. Hendelser får tydelig ansvarlig og frist for korrigering.
    \item Knytt læringssløyfer til modellene ved å bruke avvik som input til feilanalyse, nye sensorbehov eller oppdatering av dataspace-kontrakter.
\end{enumerate}

\paragraph{Praktisk sjekkliste.} Før en pilot settes i drift, bør teamet bekrefte at indikatorene over kan testes ende-til-ende i testmiljøet, at varslingsrutiner er synkronisert med beredskapsplanen i dette kapittelet, og at partnere i dataspace har signert på hvilke kvalitetsmålinger de forventer. Dette sikrer at datakvalitet ikke blir en engangsaktivitet, men en kontinuerlig del av plattformstyringen.

\subsection{Case: Sanntidsobservabilitet i kraftnettet}
Statnett har brukt digital tvilling-teknologi for å overvåke sanntidsbelastningen i transmisjonsnettet og gi driftssenteret et felles situasjonsbilde \citep{statnett2023digital,statnett2024kontrolltarn}. Caset bygger på PMU-strømmer, SCADA-data og vedlikeholdslogger som kombineres i et digitalt kontrolltårn. Erfaringene viser at observabilitet må sikres både teknisk og organisatorisk for å unngå at avvik forplanter seg i kraftsystemet.

Tre designgrep er testet i pilotene:
\begin{itemize}
    \item \textbf{Felles indikatorbibliotek:} Driftsteamet, cyberberedskap og modellutviklere bruker samme definisjon av latens, datadrift og måletetthet, slik at hendelser tolkes likt på tvers av funksjoner.
    \item \textbf{Sandkasse for hendelser:} Nye alarmer og SLO-er verifiseres mot historiske feilscenarier før de aktiveres i produksjon. Denne praksisen hindrer alarmutmattelse i beredskapsrommet.
    \item \textbf{Samsvar med dataspace-policy:} Når data deles med regionale nettselskaper eller forskningspartnere, publiseres indikatorene i samme katalog som tilgangsavtaler og samtykker.
\end{itemize}

Tabell~\ref{tab:kap03-observabilitet-kraft} viser et utdrag av indikatorene som brukes i driftssenteret. Hvert målepunkt er koblet til en forventet terskel og beskriver hvilket team som leder responsen dersom verdien havner utenfor normalområdet.

\begin{table}[ht]
    \centering
    \caption{Observabilitetsindikatorer for Statnett-inspirert digital tvilling.}
    \label{tab:kap03-observabilitet-kraft}
    \begin{tabular}{p{0.26\textwidth}p{0.42\textwidth}p{0.26\textwidth}}
        \toprule
        \textbf{Metrikk} & \textbf{Overvåkingspraksis} & \textbf{Tiltak ved avvik} \\
        \midrule
        Latens i PMU-strømmer & Måles per node og publiseres i dashbord med 5-sekunders oppløsning. & Plattformteam eskalerer til telekomleverandør og initierer failover mot redundante linjer. \\
        Datadrift i fasevinkel & Automatisk analyse sammenligner mot referansemodeller og vedlikeholdslogger. & Modellansvarlig rekalkulerer parametre og flagger potensielle feil i sanntidsmodellen. \\
        Tilgjengelighet for kontrolltårn & Kontinuerlig syntetisk testbruker validerer API, graf og varslingskjede. & Beredskapsteam aktiverer manuelle rutiner og informerer situasjonssenteret. \\
        Hendelsesjournal & Alle alarmer signeres kryptografisk og synkroniseres med dataspace-avtaler. & Compliance-funksjonen vurderer rapporteringsplikt mot NVE og Statnetts situasjonssenter. \\
        \bottomrule
    \end{tabular}
\end{table}

Indikatorene forvaltes gjennom en ukentlig tavle der driftssenteret, dataspace-operatøren og sikkerhetsfunksjonen koordinerer tiltak. Resultatet er at modelloppdateringer og nettoperasjon blir tett koblet til observabilitetsarbeidet, samtidig som partnere kan etterprøve datakvaliteten når de gjenbruker strømmer fra kontrolltårnet.

\section{Integrasjonsmønstre og standarder}
Et integrasjonslandskap for digitale tvillinger spenner fra enkle API-kall til kompleks hendelsesdrevet samhandling. God praksis er  å kartlegge datastrømmer, volum og krav til robusthet før man velger teknologier.

\subsection{Arkitekturvalg}
Tradisjonelle REST-API-er gir tydelig kontraktstyring og passer for forespørsel-/svar-scenarioer, men bør suppleres med publish/subscribe-mekanismer nåar flere systemer trenger de samme sanntidsdataene. Hendelsesdrevne arkitekturer med meldingskøer eller loggstrømmer (for eksempel Apache Kafka) gir bedre skalerbarhet og kan forenkle revisjon, fordi alle hendelser lagres i riktig rekkefølge. For kritiske styringssystemer bør man ogs å vurdere redundans og fallback-løsninger, slik at tapte meldinger ikke medfører sikkerhetsrisiko.

\subsection{Standarder i norsk praksis}
OPC UA er utbredt i norsk industri fordi det forener datapublisering med semantiske modeller. MQTT er lettere og passer godt nåar batteri- eller nettverkshensyn krever minimal overhead, som i maritime anvendelser. Asset Administration Shell (AAS) f å r støtte gjennom europeiske initiativer og gir et strukturert format for  å beskrive digitale representasjoner av produkter og systemer. Ved  å kombinere disse standardene kan man bøde integrere eldre automasjonssystemer og dele data med eksterne partnere.

\subsection{Datakvalitet og interoperabilitet}
Integrasjonen m å inkludere kontrollpunkter for datakvalitet, spesielt nåar dataene brukes til modelloppdatering eller automatiserte beslutninger. Versjonering av datakontrakter og testmiljøer der integrasjoner valideres før produksjonssetting reduserer risiko for feil. Dokumentasjon av avhengigheter og kontaktpunkter gjør det enklere  å etablere ansvar for h å ndtering av databrudd eller uforutsette endringer.

\section{Dataspace-arkitektur og samhandling}
Norge deltar i flere europeiske dataspace-initiativ, og mange virksomheter har begynt å etablere felles dataplattformer som
knytter industrielle tvillinger på tvers av selskaper. Gaia-X og International Data Spaces Association (IDSA) gir rammen for
hvordan teknisk arkitektur, tillitsmekanismer og policy-regler bør utformes \citep{gaiax2023architecture,idsa2023ram}. For å
sikre at norske aktører kan koble seg på disse økosystemene uten å gi fra seg kontroll på data, er det nyttig å beskrive
dataspace-arkitekturen lag for lag, slik Tabell~\ref{tab:kap03-dataspace-lag} viser.

\begin{table}[ht]
    \centering
    \caption{Lagdeling i en norsk dataspace-arkitektur for digitale tvillinger.}
    \label{tab:kap03-dataspace-lag}
    \begin{tabular}{p{0.28\textwidth}p{0.62\textwidth}}
        \toprule
        \textbf{Lag} & \textbf{Formål og tiltak} \\
        \midrule
        Deltakerforvaltning & Registrering av virksomheter, utstedelse av identiteter og sertifikater, samt avtaler om datadeling. \\
        Tilgangskontroll & Policy-håndheving via konnektorer som sikrer at datadelingen følger kontrakter og sanksjoner ved brudd. \\
        Semantikk og katalog & Felles begreper, datasettbeskrivelser og API-spesifikasjoner publiseres i søkbare kataloger. \\
        Datastrømmer & Konfigurasjon av sanntids- og batchkanaler med logging, kryptering og datasuverenitetsregler. \\
        Tjenester & Analyse- og simuleringsapplikasjoner som kan kjøres nær dataene og kobles til tvillingene i kapittel 4 og 5. \\
        \bottomrule
    \end{tabular}
\end{table}

Erfaringer fra mobilitetsdataspace-programmet i EU viser at det er avgjørende med tidlig avklaring av roller, tekniske krav og
juridiske mekanismer \citep{ec2023mobilitydataspace}. I Norge bør bransjeorganisasjoner og klynger definere et minimumssett med
policy-regler for datadeling slik at energiselskaper, offentlige etater og leverandører kan koble seg på uten omfattende
bilaterale avtaler. Ved å bruke referansearkitekturene kan pilotprosjekter gjenbruke sertifiseringsprosesser, logging og
komponenter for dataminimering.

\subsection{Norske roller og styringsmodeller}
Et vellykket dataspace krever kombinasjon av tekniske og organisatoriske roller. Tabellen under viser en forenklet matrise som
har blitt testet i norske pilotprosjekter i kraft- og mobilitetssektoren.

\begin{table}[ht]
    \centering
    \caption{Ansvarsfordeling i dataspace-piloter.}
    \label{tab:kap03-dataspace-ansvar}
    \begin{tabular}{p{0.32\textwidth}p{0.58\textwidth}}
        \toprule
        \textbf{Rolle} & \textbf{Hovedansvar} \\
        \midrule
        Dataspace-operatør & Drifter konnektorer, policy-tjenester og sertifiseringsmekanismer for deltakerne. \\
        Domeneeier & Definerer semantikk, datasettprioritet og kvalitetssikrer modellene som bruker informasjonen. \\
        Tilbyder & Leverer datastrømmer eller tjenester, følger policy-regler og rapporterer hendelser til operatøren. \\
        Forbruker & Integrerer data i egne tvillinger, dokumenterer formål og bidrar med forbedringsforslag til semantikken. \\
        Tilsyn/koordinator & Overvåker etterlevelse av regelverk (for eksempel NIS2) og beslutter tiltak ved alvorlige brudd. \\
        \bottomrule
    \end{tabular}
\end{table}

Rollen som dataspace-operatør kan med fordel ligge hos en nøytral aktør (for eksempel en bransjeorganisasjon eller et
forskningsinstitutt), mens domeneeier ofte er et konsortium av virksomheter som ønsker felles styring. Tilgang til tvillingens
simuleringsresultater kan skje via tjenestelag i dataspace-arkitekturen slik at partnerne får innsikt uten å kopiere hele
datasett.

\subsection{Implementeringssteg for norske virksomheter}
For å komme raskt i gang med dataspace-initiativ anbefales en trinnvis tilnærming:
\begin{enumerate}
    \item Kartlegg hvilke datakilder, modeller og simuleringsresultater som er mest etterspurt på tvers av organisasjoner.
    \item Identifiser regulatoriske krav (for eksempel energilovgivning, personvern, eksportkontroll) og oversett disse til
    policy-regler som kan konfigureres i konnektorene.
    \item Velg tekniske komponenter som støtter signering, kryptering og logging i tråd med IDSA-rammeverket.
    \item Etabler en felles katalog og begrepsmodell, gjerne med utgangspunkt i eksisterende bransjestandarder.
    \item Pilotér deling av et begrenset datasett og vurder ytelse, sikkerhet og gevinstrealisering før skalering.
\end{enumerate}

\section{Infrastruktur og sikkerhet}
Arkitekturen bør balansere behovet for lav responstid med kravene til sikkerhet, kostnad og etterlevelse. Norske organisasjoner møter ofte krav om datasuverenitet, samtidig som de ønsker  å dra nytte av skytjenester for elastisitet og tung regnekraft.

\subsection{Edge, sky og hybrid}
Edge-plattformer n å r fysiske prosesser gir rask respons og reduserer b å ndebreddebruken ved  å filtrere data før videresending. Skybaserte løsninger gir tilgang til avanserte analysetjenester og fleksibel lagring. En hybrid tilnærming er vanlig: modellene trenes i skyen og pakkes som containere som kan distribueres tilbake til fabrikkgulvet. God orkestrering (for eksempel med Kubernetes eller Azure Arc) sikrer at oppdateringer kan rulles ut kontrollert.

\subsection{Tilgangsstyring og zero-trust-prinsipper}
Zero-trust innebåe rer at alle forespørsler autentiseres og autoriseres, uansett hvor de kommer fra. I en digital tvilling m å dette omfatte sensorer, API-klienter og mennesker. Bruk av identitetsplattformer, automatisert nøkkellagring og segmenterte nettverk reduserer angrepsflaten. Logsikkerhet, inkludert kryptografisk signering av hendelser, hjelper virksomheten med  å dokumentere etterlevelse og spore hendelser i etterkant.

\subsection{Juridiske hensyn}
GDPR krever at personopplysninger behandles med tydelig hjemmel og strenge tilgangsregler. Selv n å r dataene i utgangspunktet er tekniske, kan kombinasjonen av sensorinformasjon og arbeidsplaner gjøre dem identifiserbare. Datasuverenitet har f å tt økt fokus etter Schrems II-dommen; virksomheter m å vite hvor dataene fysisk lagres og hvilke underleverandører som involveres. Mange velger  å knytte seg til norske eller europeiske skyer, eller  å bruke konfigurerbare sovereign cloud-løsninger.

\section{Beredskap og kontinuitet}
Digitale tvillinger som understøtter kritisk infrastruktur må inngå i organisasjonens beredskapsplaner. Direktoratet for
samfunnssikkerhet og beredskap (DSB) anbefaler at både teknologiske og organisatoriske avhengigheter kartlegges for å sikre
responsevne ved hendelser \citep{dsb2023nrb}. En beredskapsplan for tvillingplattformen kan organiseres i fire hoveddeler:
\begin{enumerate}
    \item \textbf{Forebygging:} Risikovurderinger av datakilder, tilgangsstyring og avhengigheter mot tredjepartsleverandører.
    \item \textbf{Beredskap:} Varslingsplaner, alternative datakilder og manuelle prosedyrer for operatører dersom tvillingen blir utilgjengelig.
    \item \textbf{Respons:} Hendelseshåndtering med tydelig ansvarslinje, logging av beslutninger og kommunikasjon mot myndigheter.
    \item \textbf{Gjenoppretting:} Prioritert plan for å gjenopprette datapipeline, modeller og dashboard, inkludert tester som verifiserer integriteten.
\end{enumerate}

Organisasjonen bør gjennomføre regelmessige øvelser der dataspace-partnere, driftspersonell og sikkerhetsteam deltar. Øvelsene
kan bygge på scenarier fra Nasjonalt risikobilde og inkludere bortfall av leverandører, cyberangrep eller feil i sensornett.
Resultatene dokumenteres i samme fagfellelogg som brukes for datastyringsforumet slik at tiltak følges opp.

\subsection{Kontinuitetsplan for dataplattformen}
Et praktisk hjelpemiddel er å etablere en kontinuitetsmatrise som viser hvilke komponenter som krever redundans. Tabell~\ref{tab:kap03-kontinuitet}
gir et eksempel fra energisektoren.

\begin{table}[ht]
    \centering
    \caption{Kontinuitetstiltak for kritiske komponenter.}
    \label{tab:kap03-kontinuitet}
    \begin{tabular}{p{0.34\textwidth}p{0.56\textwidth}}
        \toprule
        \textbf{Komponent} & \textbf{Tiltak og ansvar} \\
        \midrule
        Sensor- og edgeinfrastruktur & Redundant strøm og kommunikasjon, avtaler om nødlager for reservedeler, ansvarlig: teknisk drift. \\
        Meldingsplattform & Klyngeoppsett på tvers av datasentre, automatisert failover-test, ansvarlig: plattformteam. \\
        Datakatalog & Daglig backup til isolert sone og plan for manuell distribusjon av kritiske metadata, ansvarlig: data steward. \\
        Modelltjenester & Containerimages lagres i sikker registry med signering, gjenoppretting automatisert via IaC, ansvarlig: DevOps. \\
        Dashboards & Offline-snapshots av nøkkelindikatorer og utsending til beredskapsrom, ansvarlig: forretningskontinuitet. \\
        \bottomrule
    \end{tabular}
\end{table}

Kontinuitetsplanen bør oppdateres etter hver øvelse og ved større endringer i dataspace-arkitekturen. Ved å koble planen til
organisasjonens helhetlige beredskapsrammeverk blir tvillingen en integrert del av den operative beredskapen.

\subsection{Case: Oslo kommunes vann- og avløpsberedskap}
Oslo kommunes vann- og avløpsetat (VAV) har etablert en digital tvilling som samler sensordata, hydrauliske modeller og
beredskapsplaner for hovedforsyningen til byen. Tvillingen følger både råvannsinntak, vannbehandlingsanlegg og soner med høy
lekkasjerisiko, og kobles til varslingstjenestene i DSB sitt CIM-system. Resultatet er at planleggere kan teste konsekvensen av
ledningsbrudd eller forurensede magasiner før tiltak iverksettes \citep{oslovav2023digital}. Data fra SCADA-systemet strømmes til
tvillingen i sanntid, mens historiske vedlikeholdslogger og entreprenørdata gjøres tilgjengelig som batchoppdateringer hver
natt. Plattformen deler dessuten aggregert informasjon med nabokommuner via en begrenset dataspace slik at alternativ
vannforsyning kan koordineres når hendelser oppstår \citep{norskvann2023digitaltvilling}.

Etaten har utviklet en tiltakstabell som kobler hendelsestyper til datasett, ansvarslinjer og beslutningstid. Tabell~\ref{tab:kap03-oslo-vav}
viser et utdrag som brukes i øvelser og faktisk drift.

\begin{table}[ht]
    \centering
    \caption{Tiltakstabell for Oslo VAV sin digitale tvilling.}
    \label{tab:kap03-oslo-vav}
    \begin{tabular}{p{0.30\textwidth}p{0.28\textwidth}p{0.28\textwidth}}
        \toprule
        \textbf{Hendelse} & \textbf{Aktiverte datasett} & \textbf{Tiltak og ansvar} \\
        \midrule
        Ledningsbrudd i kritisk sone & Trykksensorer, lekkasjeindikatorer, gravepåvirkning fra entreprenør & Beredskapsvakt varsles via CIM, tvillingen simulerer avstengningsscenario, entreprenør mobiliseres innen 60 minutter. \\
        Forurenset råvann & Vannkvalitetslogger, værdata, historikk for innsjø & Tvillingen beregner alternative inntak, VAV informerer Folkehelseinstituttet, bydelene får varslingsmaler for kokevarsel. \\
        Strømutfall på pumpestasjon & SCADA-hendelser, kontinuitetsplan for nødstrøm, energimåler & Driftssentral koordinerer reservestrøm og tanker, tvillingen prioriterer soner med kritisk infrastruktur. \\
        \bottomrule
    \end{tabular}
\end{table}

Øvelsene kjøres i samarbeid med kommunens kriseledelse og nabokommuner gjennom interkommunalt vannverk. Erfaringen er at
kombinasjonen av simulering og faktisk hendelseslogg gir raskere beslutninger når hendelser oppstår, og at tiltak kan prioriteres
etter både helseeffekt og samfunnskritiske funksjoner \citep{oslovav2023digital}. Tvillingen er også koblet til et publikt
informasjonspanel som viser status for lekkasjearbeid og planlagte avbrudd, noe som reduserer antall henvendelser til
kundesenteret.

\paragraph{Overføringspunkter til undervisning og praksis.} Caset brukes i masterprosjekter ved OsloMet og NMBU der studenter
skal utvikle forbedrede lekkasjemodeller og planlegge deling av data med private entreprenører. Øvelsesopplegget gir en
praktisk ramme for å trene på samspill mellom teknisk drift, kommunikasjon og beredskapsledelse. Materialet kan gjenbrukes i
kapittel~6 når DPIA og hendelsesjournal skal dokumenteres, og i kapittel~7 når kommunal governance og datasamarbeid diskuteres.

\section{Bærekrafts-dashboard og indikatorstyring}
For å gjøre bærekraft og klimaeffekter synlige i operative beslutninger bør digitale tvillinger levere dashboards som kombinerer
miljøindikatorer, energieffektivitet og økonomiske nøkkeltall. Statnett og andre norske infrastruktureiere rapporterer at slike
dashboards er viktige for å dokumentere innsparingstiltak og rapportere til myndigheter \citep{statnett2024baerekraft}. Et
bærekraftsdashboard kan struktureres rundt tre nivåer:
\begin{itemize}
    \item \textbf{Strategisk nivå:} KPI-er for utslippsreduksjoner, fornybarandel og taksonomi-etterlevelse.
    \item \textbf{Taktisk nivå:} Energibruk, materialforbruk og logistikkindikatorer fordelt på anlegg og tidsperioder.
    \item \textbf{Operativt nivå:} Varsler på avvik fra planlagte tiltak, prediksjoner fra tvillingmodeller og anbefalte tiltak.
\end{itemize}

Ved å koble indikatorene til dataspace-laget kan partnere dele aggregerte resultater uten å avsløre forretningssensitiv informasjon.
Det gir samtidig beslutningstakere i kapitlene 4 og 7 et faktagrunnlag for å prioritere tiltak.

\subsection{Eksempel på indikatorstruktur}
Tabell~\ref{tab:kap03-dashboard} viser hvordan indikatorer kan knyttes til datakilder og ansvarlige roller i en energioperasjon.

\begin{table}[ht]
    \centering
    \caption{Indikatorer i et bærekraftsdashboard.}
    \label{tab:kap03-dashboard}
    \begin{tabular}{p{0.28\textwidth}p{0.32\textwidth}p{0.30\textwidth}}
        \toprule
        \textbf{Indikator} & \textbf{Datakilder} & \textbf{Ansvarlig rolle} \\
        \midrule
        Scope 1-utslipp & Sensorer fra forbrenningsanlegg, drivstoffrapporter & Bærekraftsleder \\
        Energitap i nett & SCADA-målinger, værdata, prognoser fra tvilling & Driftssentral \\
        Materialgjenvinningsgrad & ERP, materialpass, dataspace for leverandører & Innkjøp og sirkulærteam \\
        Leveransetid for tiltak & Prosjektverktøy, avvikslogger & Programkontor \\
        Kundepåvirkning & Hendelseslogger, kundeservice, sosiale medier & Kommunikasjon \\
        \bottomrule
    \end{tabular}
\end{table}

For å unngå «greenwashing» må indikatorene revideres av uavhengige fagmiljøer. Resultatene kan kobles til læringssløyfene i
Kapittel~5 slik at modeller kalibreres med faktiske effekter av tiltakene.

\section{Datastyring og kontinuerlig forbedring}
For  å sikre at datapipelinen leverer konsistent kvalitet m å tekniske tiltak kobles til tydelige roller og arbeidsprosesser. Norske virksomheter etablerer ofte et datastyringsforum som samler produkteier, data steward, sikkerhetsansvarlig og representanter fra drift. Forumet prioriterer tiltak basert p å indikatorer for datakvalitet, hendelseslogger og fagfelleinnspill.

\subsection{Roller og møtearenaer}
\begin{itemize}
    \item \textbf{Data steward:} Overv å ker datakvalitet og initierer korrigerende tiltak n å r regler brytes, for eksempel ved  å sende saker til endringsstyret eller oppdatere datakontrakter.
    \item \textbf{Produkteier for tvillingen:} Sikrer at hendelser med høy forretningsrisiko følges opp og at prioriteringene synkroniseres med gevinstplaner i Kapittel~7.
    \item \textbf{Teknisk plattformteam:} Forvalter overv å kingsdashbord, versjonering av dataprosesser og automatiserte tester som kjøres hver gang datastrukturer endres.
\end{itemize}
Forumet kan møtes ukentlig i innføringsfasen og deretter m å nedlig for moden drift. Referat og tiltak registreres i fagfelleloggen (DI-03) slik at status er synlig for hele redaksjonen.

\subsection{Operasjonelle kontrollpunkter}
\begin{enumerate}
    \item \textbf{Datakvalitetsdashbord:} Visualiserer fullstendighet, tidsforsinkelser og valideringsfeil for hver datastrøm. Alarmgrenser defineres i samarbeid med domeneeksperter og dokumenteres i styringspakken til Kapittel~6.
    \item \textbf{Hendelsesrespons:} Avvik loggføres i samme system som sikkerhetshendelser. Hvert avvik f å r ansvarlig person, frist og forslag til kompenserende tiltak (for eksempel fallback-modell eller manuelt kontrollsteg).
    \item \textbf{Låe ringssløyfe:} Kvartalsvise retrospektiv analyserer mønstre i datakvalitetsavvik og oppdaterer pipeline-design eller opplåe ringsprogram for operatører.
\end{enumerate}
Detaljert praksis for indikatorene, møtesstruktur og låe ringssløyfe er nedfelt i delingsnotatet `support/notater/datastyringsforum-di03.md`,
som deles med fagfeller i arbeidet med kommentar DI-03.

\subsection{Samsvar med personvern og etikk}
N å r nye datakilder tas i bruk bør datastyringsforumet sjekke behandlingsgrunnlag, vurdering av dataminimering og rutiner for sletting. Et eget sjekkpunkt sikrer at dokumentasjonen speiler kravene fra Helsedirektoratet, Normen og Kapittel~6 om validering og tillit. N å r tvillingen skaleres p å tvers av anlegg, anbefales det  å utvide datakatalogen med tydelige beskrivelser av form å l og kontaktpunkt for hvert datasett.

\section{Sirkulåe røkonomi og materialsløyfer}
Sirkulåe røkonomi krever at data følger materialene gjennom hele verdikjeden slik at gjenbruk, reparasjon og resirkulasjon kan dokumenteres. Den nasjonale strategien for sirkulåe r økonomi fremhever digitale tvillinger som en nøkkel for  å spore klimaavtrykk og ressursbruk \citep{regjeringen2021sirkulaer}. EU sitt handlingsprogram for sirkulåe røkonomi understreker behovet for  å pne standarder, produktpass og datadeling mellom produsenter, brukere og resirkuleringsaktører \citep{eu2020circulareconomy}. Oversikten under viser hvordan data flyter i en materialsløyfe n å r en digital tvilling kobler sammen design, produksjon, bruk og retur:
\begin{itemize}
    \item \textbf{Designfase:} Produktpass definerer materialinnhold, moduloppbygging og forventet levetid.
    \item \textbf{Produksjon:} Kvalitetstester, prosessdata og energibruk logges og knyttes til partinumre.
    \item \textbf{Bruk:} Sensorer og vedlikeholdslogger oppdaterer tilstand og restlevetid for komponentene.
    \item \textbf{Retur og ombruk:} Inspeksjonsdata avgjør om komponenter gjenbrukes, repareres eller materialgjenvinnes.
    \item \textbf{Resirkulasjon:} Massebalanser og sporbarhet sikrer at resirkulert innhold dokumenteres mot nye produktserier.
\end{itemize}

\subsection{Indikatorer og datakilder}
For  å støtte sirkulåe re beslutninger bør digitale tvillinger fange indikatorer for materialinnhold, karbonavtrykk, levetid og restverdi. Dataene m å våe re tilgjengelige i produktpass eller digitale loggbøker slik at b å de produsenter og resirkulatører vet hvilke komponenter som kan demonteres og gjenbrukes. Kombinasjonen av sensorinformasjon, ERP-data og kvalitetstester gjør det mulig  å beregne hvor mye av en komponent som kan gjenbrukes uten ytterligere bearbeiding. Norske virksomheter som deltar i grønn plattform-programmet rapporterer at slike indikatorer er nødvendige for  å f å støtte til pilotering \citep{miljodir2022sirkular}.

\subsection{Case: Hydro CIRCAL og materialsporing}
Hydro har etablert en digital tvilling for  å dokumentere innholdet av resirkulert aluminium i CIRCAL-produktene sine. Plattformen kobler produksjonsdata fra pressverkene med kvalitetstester og kundesertifikater, slik at hele verdikjeden kan verifisere klimap å standen \citep{hydro2023traceability}. Tvillingen gjør det mulig  å følge materialet fra innsamling av skrap via smelteprosesser til ferdig profil, og bidrar til  å oppn å tredjeparts verifisering i henhold til EN~15088.

\begin{table}[ht]
    \centering
    \caption{Datakomponenter i Hydro sitt CIRCAL-program.}
    \label{tab:kap03-hydro}
    \begin{tabular}{p{0.34\textwidth}p{0.56\textwidth}}
        \toprule
        \textbf{Komponent} & \textbf{Beskrivelse} \\
        \midrule
        Materialpass & Digitalt pass med andel resirkulert aluminium, partinummer og opprinnelse. \\
        Prosessdata & Temperatur- og energilogger fra smelteovner og presslinjer, koblet til kvalitetstester. \\
        Kunderapportering & Automatisk generering av sertifikater og CO$_2$-beregninger til arkitekter og byggherrer. \\
        Effekter & 75\% resirkulert innhold dokumentert og redusert tid p å revisjoner med 40\%. \\
        Samarbeid & Deler data med partnere i EU sitt Circular Aluminium-program for  å harmonisere indikatorer. \\
        \bottomrule
    \end{tabular}
\end{table}

Caset viser hvordan en industriell aktør bruker digitale tvillinger til  å skape tillit i markedet og møte regulatoriske krav om  å penhet.

\subsection{Case: Loopfront og Statsbyggs ombruksprosjekter}
Loopfront leverer en plattform for ombruk av byggematerialer som brukes av Statsbygg og flere kommuner til  å kartlegge materialbanker før rehabilitering \citep{statsbygg2023loopfront}. Gjennom digitale tvillinger av bygningsmassen kombineres BIM-modeller, tilstandsanalyser og logistikkdata for  å planlegge demontering og ny bruk. Systemet beregner potensielle CO$_2$-besparelser og økonomiske gevinster per komponent, og gjør det mulig  å reservere materialer direkte i prosjektporteføljen.

\begin{table}[ht]
    \centering
    \caption{Nøkkelindikatorer i Statsbyggs ombruksportefølje.}
    \label{tab:kap03-loopfront}
    \begin{tabular}{p{0.36\textwidth}p{0.54\textwidth}}
        \toprule
        \textbf{Indikator} & \textbf{Resultater fra pilotprosjekter} \\
        \midrule
        Kartlagt volum & 28\,000 m$^2$ bygningsmasse med digitalt materialkart i 2024. \\
        Datakilder & BIM-modeller, laserskanning, materialpass og logistikkdata fra entreprenør. \\
        Beslutningsstøtte & Dashboard som viser CO$_2$-besparelser, kostnader og tilgjengelighet per komponent. \\
        Effekt & 1\,200 tonn materialer ombrukt med estimert 1,9 kilotonn CO$_2$ spart i første fase. \\
        Samhandling & Deling av datagrunnlag med kommuner og private eiendomsforvaltere for  å koordinere etterspørsel. \\
        \bottomrule
    \end{tabular}
\end{table}

Eksemplet illustrerer hvordan kombinasjonen av digitale tvillinger og sirkulåe røkonomi gir nye arbeidsprosesser for offentlige byggherrer og leverandører.

\section{Refleksjonsspørsm å l og øvinger}
\begin{enumerate}
    \item Tegn et dataflytdiagram for en digital tvilling i en norsk industribedrift.
    \item Vurder nåar det er hensiktsmessig  å bruke sky kontra edge for sanntidsanalyse.
    \item Beskriv hvordan du ville etablere en governance-modell for datatilgang.
    \item Foresl å indikatorer for datakvalitet som bør overv å kes kontinuerlig i tvillingen.
\end{enumerate}
