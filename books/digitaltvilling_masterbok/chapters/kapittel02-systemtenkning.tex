\chapter{Systemtenkning og modellering}

\section{Læringsmål}
\begin{itemize}
    \item Analysere komplekse systemer ved hjelp av systemtenkning.
    \item Vurdere ulike modelleringsstrategier for digitale tvillinger.
    \item Beskrive hvordan modeller kobles til måledata og styringssystemer.
\end{itemize}

\section{Systemperspektiv}
Systemtenkning starter med å forstå hvilke deler av virkeligheten som påvirker hverandre og hvordan verdistrømmer beveger seg gjennom systemet. Første steg er å definere klare systemgrenser, identifisere interessenter og beskrive hva de forsøker å oppnå. Deretter kan vi visualisere samspillet mellom teknologi, prosesser og mennesker ved hjelp av systemkart, påvirkningsdiagrammer og kausale sløyfer. Figurene under gir to typiske fremstillinger for en digital tvilling i en norsk industribedrift.

\textbf{Nøkkelspørsmål når systemperspektivet etableres:}
\begin{itemize}
    \item Hvilke organisatoriske enheter og tekniske komponenter inngår i systemet?
    \item Hvilke datakilder, beslutninger og tilbakemeldingsløp knytter dem sammen?
    \item Hvilke målkonflikter kan oppstå mellom interessentene?
\end{itemize}

\subsection{Figur 2.1 -- Systemkart for produksjonslinje med digital tvilling}
\begin{verbatim}
flowchart LR
    subgraph Felt
        Sensorer["IoT-sensorer i utstyr"]
        Styring["PLC og lokale styringer"]
    end
    Edge["Edge-gateway"]
    Mes["MES"]
    Historian["Prosesshistorian"]
    Analyse["Analyse- og ML-plattform"]
    DT["Digital tvilling"]
    Dashboard["Operatørdashboard"]
    Vedlikehold["Vedlikeholdsteam"]
    Ledelse["Ledelsesrapporter"]

    Sensorer --> Edge --> Mes
    Styring --> Mes
    Edge --> Historian
    Mes --> Ledelse
    Historian --> Analyse --> DT
    DT --> Dashboard
    DT --> Vedlikehold
    Vedlikehold --> Mes
\end{verbatim}

Figuren illustrerer et typisk informasjonsløp der feltdata fra sensorer strømmer via en edge-gateway inn i produksjonsstyringssystemet. Historiske data kombineres med analyseplattformen for å oppdatere den digitale tvillingen, som igjen leverer beslutningsstøtte til operatører, vedlikeholdsteam og ledelsen. Når du lager et eget systemkart, noter eksplisitt hvilke datatyper som flyter mellom nodene, og marker hvor ansvar og eierskap ligger.

\subsection{Figur 2.2 -- Kausalsløyfe mellom vedlikehold og energibruk}
\begin{verbatim}
graph LR
    Vedlikeholdsfrekvens((Planlagt vedlikehold))
    SensorStatus((Tilstandsovervåkning))
    Nedetid((Planlagt nedetid))
    Energiforbruk((Energiforbruk))
    Produksjonskvalitet((Produktkvalitet))
    Overforbruk((Uplanlagt stopp))

    Vedlikeholdsfrekvens -- øker --> SensorStatus
    SensorStatus -- reduserer --> Overforbruk
    Overforbruk -- øker --> Nedetid
    Nedetid -- reduserer --> Energiforbruk
    Energiforbruk -- påvirker --> Produksjonskvalitet
    Produksjonskvalitet -- påvirker --> Vedlikeholdsfrekvens
\end{verbatim}

Kausalsløyfen viser hvordan systemkart kan brukes til å fange både ønskede og uønskede tilbakemeldinger. Økt planlagt vedlikehold forbedrer sensorenes status, som reduserer risiko for uplanlagte stopp. Lavere uplanlagt nedetid gir stabilt energiforbruk og høyere produktkvalitet, noe som igjen påvirker vedlikeholdsstrategien. Når slike forhold beskrives eksplisitt, blir det enklere å diskutere scenarier med interessenter og identifisere hvor den digitale tvillingen bør levere mest verdi.

\textbf{Arbeidsmåte for å utvikle systemkart:}
\begin{enumerate}
    \item Start med en workshop hvor interessenter beskriver mål og smertepunkter.
    \item Tegn det overordnede systemkartet (som i Figur 2.1) med fokus på dataflyt og ansvar.
    \item Utvid kartet med kausale sløyfer (som i Figur 2.2) for å avdekke dynamikk og mulige ubalanser.
    \item Forankre kartene i virksomhetens enterprise-arkitektur og eksisterende prosesskart slik at språk og symboler er gjenkjennelige for organisasjonen.
\end{enumerate}

Den konseptuelle modellen fra systemkartet danner grunnlaget for videre modellering, enten du velger fysikkbaserte, datadrevne eller hybride tilnærminger i de følgende seksjonene.

\section{Modelleringsparadigmer}
\begin{itemize}
    \item Fysikkbaserte modeller (differensialligninger, FEM, CFD).
    \item Datadrevne modeller (maskinlæring, statistikk, digitale skygger).
    \item Hybride modeller og ko-simulering.
\end{itemize}

\section{Modellintegrasjon og kalibrering}
\begin{itemize}
    \item Hvordan koble modeller til sanntidsdata.
    \item Parameteridentifikasjon og kalibreringsmetoder.
    \item Modellreduksjon og modularisering.
\end{itemize}

\section{Refleksjonsspørsmål og øvinger}
\begin{enumerate}
    \item Lag et systemkart for en valgt industriell prosess.
    \item Diskuter fordeler og ulemper ved å kombinere fysikkbaserte og datadrevne modeller.
    \item Beskriv hvordan du ville gjennomføre kalibrering av en modell som drifter over tid.
\end{enumerate}
