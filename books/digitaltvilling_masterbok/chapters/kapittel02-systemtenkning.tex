\chapter{Systemtenkning og modellering}


\section{Læringsmål}
\begin{itemize}
    \item Analysere komplekse systemer ved hjelp av systemtenkning.
    \item Vurdere ulike modelleringsstrategier for digitale tvillinger.
    \item Beskrive hvordan modeller kobles til måledata og styringssystemer.
\end{itemize}

\section{Systemperspektiv}
Systemtenkning starter med å forstå hvilke deler av virkeligheten som påvirker hverandre og hvordan verdistrømmer beveger seg gjennom systemet. Første steg er å definere klare systemgrenser, identifisere interessenter og beskrive hva de forsøker å oppnå. Deretter kan vi visualisere samspillet mellom teknologi, prosesser og mennesker ved hjelp av systemkart, påvirkningsdiagrammer og kausale sløyfer. Beskrivelsene under gir to typiske fremstillinger for en digital tvilling i en norsk industribedrift.

\textbf{Nøkkelspørsmål når systemperspektivet etableres:}
\begin{itemize}
    \item Hvilke organisatoriske enheter og tekniske komponenter inngår i systemet?
    \item Hvilke datakilder, beslutninger og tilbakemeldingsløp knytter dem sammen?
    \item Hvilke målkonflikter kan oppstå mellom interessentene?
\end{itemize}

\subsection{Systemkart for produksjonslinje med digital tvilling}
% Alt-tekst: support/figurer/metadata/kap02-systemkart-v2.alt.md
\paragraph{Systemkart som tekst.} For å eliminere kompilasjonsfeil erstattes TikZ-grafikken med en strukturert tekstforklaring. Systemkartet består av følgende lag:
\begin{itemize}
    \item \textbf{Feltnivå:} Sensorer, pumper, ventiler og andre fysiske komponenter som leverer sanntidsdata.
    \item \textbf{Edge-infrastruktur:} Lokale noder som filtrerer data, utfører hurtige analyser og sikrer robust kommunikasjon.
    \item \textbf{Produksjonsstyring:} MES- og SCADA-lag som koordinerer operasjoner og gjør data tilgjengelig for analyse.
    \item \textbf{Analyse- og tvillingtjenester:} Skyplattformer og beslutningsgrensesnitt som kombinerer historikk og sanntidsinformasjon.
    \item \textbf{Brukergrupper:} Operatører, vedlikeholdsteam og ledelse som mottar innsikt og foreslåtte tiltak.
\end{itemize}

Beskrivelsen over viser et typisk informasjonsløp der feltnivå, edge-infrastruktur og produksjonsstyringssystemer er tydelig gruppert. Når lagene omtales eksplisitt, blir ansvarsdelingen enklere å lese enn i en kompakt figur. Historiske data kombineres med analyseplattformen for å oppdatere den digitale tvillingen, som igjen leverer beslutningsstøtte til operatører, vedlikeholdsteam og ledelse. Når du lager et eget systemkart, noter eksplisitt hvilke datatyper som flyter mellom nodene, og marker hvor ansvar og eierskap ligger i hvert lag.

\subsection{Kausalsløyfe mellom vedlikehold og energibruk}
% Alt-tekst: support/figurer/metadata/kap02-kausal-v1.alt.md
\paragraph{Kausalsammenhenger forklart i tekst.} Fremfor en grafisk sløyfe beskrives koblingene slik:
\begin{itemize}
    \item Mer planlagt vedlikehold gir bedre sensordata og færre uforutsette stopp.
    \item Stabil drift reduserer energiforbruket og forbedrer produktkvaliteten.
    \item Høy produktkvalitet frigjør tid til videre læring og justering av vedlikeholdsstrategien.
    \item Operasjonelle erfaringer føres tilbake som tiltak i styringssystemene.
\end{itemize}

Denne tekstlige sløyfen viser hvordan systemkart kan brukes til å fange både ønskede og uønskede tilbakemeldinger. Økt planlagt vedlikehold forbedrer sensorenes status, som reduserer risiko for uplanlagte stopp og stabiliserer energibruken. Et stabilt energinivå gir høyere produktkvalitet, noe som igjen påvirker vedlikeholdsstrategien gjennom læring og prioritering av tiltak. Når slike forhold beskrives eksplisitt, blir det enklere å diskutere scenarier med interessenter og identifisere hvor den digitale tvillingen bør levere mest verdi.

\textbf{Arbeidsmåte for å utvikle systemkart:}
\begin{enumerate}
    \item Start med en workshop hvor interessenter beskriver mål og smertepunkter.
    \item Tegn det overordnede systemkartet (slik det er beskrevet i punktlisten over) med fokus på dataflyt og ansvar.
    \item Utvid kartet med kausale sløyfer (slik tekstbeskrivelsen over viser) for å avdekke dynamikk og mulige ubalanser.
    \item Forankre kartene i virksomhetens enterprise-arkitektur og eksisterende prosesskart slik at språk og symboler er gjenkjennelige for organisasjonen.
\end{enumerate}

Den konseptuelle modellen fra systemkartet danner grunnlaget for videre modellering, enten du velger fysikkbaserte, datadrevne eller hybride tilnærminger i de følgende seksjonene.

\section{Modelleringsparadigmer}
Systemkartet gir en felles forståelse av hvilke fenomener som må representeres i en digital tvilling. Neste steg er å velge et
modelleringsparadigme som balanserer nøyaktighet, forklarbarhet og implementasjonskostnad. Masterstudenter må kunne vurdere
hvordan ulike typer modeller beskriver sammenhengene i systemet og hvilke beslutninger de støtter. I praksis innebærer dette å
kombinere domenekunnskap fra feltet med metoder fra matematikk, statistikk og informatikk.

\subsection{Fysikkbaserte modeller}
Fysikkbaserte modeller er bygd på eksplisitte ligninger som beskriver energi-, masse- eller informasjonsstrømmer. Typiske
teknikker er differensialligninger, finite element-metoder (FEM) og computational fluid dynamics (CFD). Fordelen er at
modellene er transparente og muliggjør følsomhetsanalyser, men de krever inngående forståelse av prosessen og god kvalitet på
parametere. I norske prosessindustrier brukes slike modeller blant annet til å simulere varmebalanser i ovner, fukttransport i
trebaserte materialer og hydrodynamikk i fiskemerder.

\subsection{Datadrevne modeller}
Datadrevne modeller lærer systemets oppførsel direkte fra historiske eller sanntidsdata. Maskinlæring og statistiske metoder
kan identifisere mønstre uten å kjenne den underliggende fysikken. En digital skygge kan trenes på tidsserier fra sensorer for
å predikere avvik i produksjonen. Fordelen er rask implementering når rike datasett finnes, men modellene kan bli sårbare for
driftsendringer og må overvåkes for skjevheter.

\subsection{Hybride modeller og ko-simulering}
I mange digitale tvillinger kombineres fysikkbaserte og datadrevne komponenter. En hybrid modell kan bruke førsteprinsipper for
energi- og materialbalanser, mens maskinlæring estimerer friksjonstap eller operatørpåvirkning som er vanskelig å beskrive
analytisk. Ko-simulering gjør det mulig å koble sammen flere spesialiserte modeller i en helhetlig tidslinje, for eksempel når
mekanikk, termodynamikk og styringslogikk må løses samtidig \citep{boschert2018digital}. Denne kombinasjonen gir robusthet og
fleksibilitet, men krever tydelige API-kontrakter, delte dataskjema og versjonskontroll av hvert delsystem.

\subsection{Flerfidelitetsstrategier}
Flerfidelitetsmodellering kombinerer raske, grovkornede modeller med detaljerte høyfidelitetsmodeller for kritiske delsystemer.
Lavfidelitetsmodellen gir raske scenarioberegninger og støtte til beslutninger med stramme tidsfrister, mens høyfidelitetsmodellen
brukes til kalibrering og validering av sentrale parametere \citep{kennedy2000predicting}. Strategien er særlig nyttig i norsk
prosess- og energisektor der tilgangen til sensordata varierer mellom installasjoner \citep{sintef2021digital}. Et vellykket
flerfidelitetsoppsett krever at teamet:
\begin{itemize}
    \item definerer hvilke variabler som skal utveksles mellom modellene og hvordan usikkerhet skal rapporteres,
    \item etablerer en synkroniseringsplan for når høyfidelitetsmodellen skal oppdateres og hvordan endringene mates inn i
    lavfidelitetsmodellen,
    \item dokumenterer toleranser for avvik slik at driftsorganisasjonen vet når det er nødvendig med ny kalibrering eller
    eksperimentelle målinger.
\end{itemize}

\subsection{Metodebibliotek for modellutvikling}
Arbeidet med flerfidelitetsmodeller følger et repeterbart mønster der analyse, modellering og drift veves sammen. Tabellen under
kan brukes som sjekkliste i prosjekt- og masteroppgaver for å sikre at alle leveranser spores.

\begin{table}[ht]
    \centering
    \caption{Metodebibliotek for utvikling og forvaltning av flerfidelitetsmodeller.}
    \label{tab:kap02-metodebibliotek}
    \begin{tabular}{p{0.18\textwidth}p{0.31\textwidth}p{0.25\textwidth}p{0.22\textwidth}}
        \toprule
        \textbf{Fase} & \textbf{Formål} & \textbf{Typiske verktøy} & \textbf{Dokumentasjon} \\
        \midrule
        Behovsavklaring & Avgrense caset, identifisere beslutningstakere og målebehov. & Interessentintervjuer, målhierarki, kontekstdiagram. & Oppdragsbeskrivelse, interessentlogg og systemskisser. \\
        Modellering & Utforme og koble lav- og høyfidelitetsmodeller. & FEM/CFD, tidsserieanalyse, modellreduksjon. & Modelljournal, parameterark og antagelsesregister. \\
        Integrasjon & Koble modellene til dataflyt og driftsprosesser. & API-kontrakter, datastrømskart, containeriserte tjenester. & Integrasjonsdiagram, driftsprotokoll og sikkerhetsvurdering. \\
        Drift og forbedring & Overvåke ytelse, avdekke driftsavvik og planlegge revisjoner. & Monitoreringstavler, avviksanalyse, A/B-eksperimenter. & Kalibreringslogg, endringsnotat og forbedringsplan. \\
        \bottomrule
    \end{tabular}
\end{table}

Metodebiblioteket gjør det lettere å fordele ansvar i teamet og tydeliggjøre hvilke artefakter som må oppdateres etter hver
revisjon. Studentgrupper kan utvide tabellen med kolonner for risiko, estimerte timer og behov for fagfellegjennomgang for å få
et konkret styringsverktøy.

\subsection{Sensitivitetsanalyse og parameterprioritering}
Selv solide flerfidelitetsoppsett taper verdi dersom nøkkelparametere ikke overvåkes systematisk. En sensitivitetstilnærming
gjør det mulig å identifisere hvilke antakelser og måledata som påvirker beslutninger mest, og å planlegge målrettede
kalibreringer før modellen tas videre til valideringsarbeidet i kapittel~6. Global
sensitivitetsanalyse, der parameterinteraksjoner kartlegges i stedet for å variere én parameter om gangen, gir et mer robust
grunnlag enn manuelle scenariotester \citep{saltelli2019sensitivity}. Norske infrastruktureiere som Statnett anbefaler å kombinere
slike analyser med periodiserte vedlikeholdsplaner for å sikre at modelljournalen reflekterer faktiske måleusikkerheter
\citep{statnett2023modellforvaltning}.

\paragraph{Praktisk arbeidsflyt.} Følgende arbeidssteg kan brukes som mal når sensitivitet skal kobles til
modelljournalen og tiltaksloggene i kapittel 6 og 7:
\begin{enumerate}
    \item Definer beslutningsindikatorer (for eksempel energiutnyttelse, tilgjengelighet eller CO$_2$-intensitet) og knytt dem til
    relevante kapitler slik at endringer kan spores.
    \item Grupper parametere etter datakilde og målehyppighet, og etabler sannsynlighetsfordelinger som fanger opp målefeil eller
    ekspertvurderinger.
    \item Kjør en global sensitivitetstest (Sobol, Morris eller variansbaserte metoder) og dokumenter resultatene i
    modelljournalen med tydelig referanse til versjoner og datagrunnlag.
    \item Prioriter kalibreringstiltak og overvåkingsrutiner basert på hvilke parametere som påvirker indikatorene mest, og oppdater
    tiltaksloggen i kapittel 7 dersom endringer krever organisatorisk oppfølging.
\end{enumerate}

\paragraph{Indikatorer for oppfølging.} Resultatene bør visualiseres i et felles dashboard slik at både modellutviklere og
driftsorganisasjonen kan reagere raskt. Følgende indikatorer er anbefalt:
\begin{itemize}
    \item \textbf{Sensitivitetspoeng}: Summert bidrag fra hver parameter til total varians i beslutningsindikatorene.
    \item \textbf{Datakvalitetsgrad}: Forholdet mellom måleusikkerhet og aksepterte toleranser i modellen.
    \item \textbf{Kalibreringsstatus}: Antall dager siden sist parameter ble validert eller målt i felt.
    \item \textbf{Tiltaksavhengighet}: Markerer hvilke parametere som krever koordinering med dataspace- og valideringsteamene.
\end{itemize}

\begin{table}[ht]
    \centering
    \caption{Eksempel på prioritering av parametergrupper basert på sensitivitet og datatilgang.}
    \label{tab:kap02-sensitivitet}
    \begin{tabular}{p{0.22\textwidth}p{0.26\textwidth}p{0.27\textwidth}p{0.21\textwidth}}
        \toprule
        \textbf{Parameterkategori} & \textbf{Typiske indikatorer} & \textbf{Anbefalte tiltak} & \textbf{Kobling til øvrige kapitler} \\
        \midrule
        Feltmålinger med høy varians & Energiflyt, vibrasjon, temperatur & Øk målefrekvens, bruk redundante sensorer, planlegg feltkalibrering hvert kvartal. & Kapittel 6: Valideringsjournal og hendelsesrespons. \\
        Avledede modellparametere & Effektivitet, CO$_2$-intensitet, vedlikeholdsprognoser & Dokumenter antakelser, innfør automatiske regressjonstester og oppdater modelljournalen etter hver versjon. & Kapittel 5: Læringssløyfer og modelloppdateringer. \\
        Forretningsdrivere & Kost/nytte, leveringsgrad, sikkerhetsnivå & Synkroniser indikatorer med gevinst- og tiltakslogg, eskaler endringer i styringsfora. & Kapittel 7: Porteføljestyring og gevinstoppfølging. \\
        Exogene scenarioinput & Markedspriser, værdata, reguleringskrav & Etabler scenarioarkiv, loggfør kilder og oppdater beslutningsgrunnlag ved nye regulatoriske føringer. & Kapittel 3: Dataintegrasjon og beredskap. \\
        \bottomrule
    \end{tabular}
\end{table}

Ved å kombinere tabellen med metodestegene over får studentene en konkret mal for å prioritere feltarbeid, laboratorietester og
styringsmøter. Det gjør det enklere å dokumentere hvorfor enkelte parametere får mer oppmerksomhet enn andre, og hvordan
sensitivitetsfunnenes tiltak forplanter seg til dataspace- og governance-arbeidet.

\subsection{Case: Flerfidel modell for subsea-kompressorer}
Et norsk leverandørkonsortium har utviklet en digital tvilling for subsea-kompressorer på sokkelen. Plattformen kombinerer en
hurtig maskinlæringsmodell som estimerer produksjon og energiforbruk fra tilgjengelige trykk- og temperaturmålinger, med en
høyfidel CFD-modell som simulerer detaljerte turbulenseffekter i kompressoren. Hver natt kjøres CFD-modellen mot oppdaterte
operasjonsdata for å korrigere parametere i hurtigmodellen. Resultatene deles med driftsteamet gjennom et dashboard der
avvik merkes når usikkerheten overstiger definerte grenser. Prosjektet viser hvordan flerfidelitetsstrategier gir bedre
tilgjengelighet og vedlikeholdsplanlegging samtidig som modellene er dokumentert og revideres i samarbeid mellom operatør og
teknologipartnere \citep{sintef2021digital}.

\subsection{Case: Modellbibliotek for produksjonsoptimalisering offshore}
Aker BP og Cognite har etablert et modellbibliotek som kombinerer produksjonsmodeller, maskinlæring og sanntidsdata for å støtte
beslutninger på norsk sokkel \citep{cognite2023akerbp}. Lavfidelitetsmodeller gir kontinuerlige prognoser for produksjons- og
energiflyt, mens høyfidelitetsmodeller brukes til detaljert analyse av brønnrespons og kompressoreffektivitet. Modelljournalen
inneholder versjonerte parameterark, lenker til automatiserte tester og prosedyrer for hvordan avvik løftes i daglige
driftsmøter. Erfaringene viser at tydelig dokumentasjon gjør det enklere å koble modelleringsarbeidet til eksterne leverandører
og fagmiljø som jobber med miljøovervåking og sikkerhetsanalyser.

\subsection{Vurdering av modellvalg}
Når et paradigme skal velges, bør teamet diskutere hvor kritisk forklarbarhet er, hvilke datakilder som er tilgjengelige og hvor
raskt modellen må oppdateres. Tabell~\ref{tab:kap02-modellvalg} gir et utgangspunkt for å sammenligne alternativer og velge en
portefølje av modeller som utfyller hverandre.

\begin{table}[ht]
    \centering
    \caption{Sammenligning av modelleringsparadigmer for digitale tvillinger.}
    \label{tab:kap02-modellvalg}
    \begin{tabular}{p{0.2\textwidth}p{0.27\textwidth}p{0.24\textwidth}p{0.23\textwidth}}
        \toprule
        \textbf{Paradigme} & \textbf{Styrker} & \textbf{Utfordringer} & \textbf{Typisk norsk case} \\
        \midrule
        Fysikkbasert & Forklarbare sammenhenger og mulighet for sensitivitetstester. & Krever detaljerte parametere og høy beregningskostnad. & Energiproduksjon med krav til termisk balanse. \\
        Datadrevet & Rask modellering når historiske data er rike. & Sårbar for konseptdrift og bias i data. & Produksjonslinjer med omfattende sensordekning. \\
        Hybrid & Kombinerer robusthet og fleksibilitet gjennom ko-simulering. & Krever koordinering av grensesnitt og felles semantikk. & Integrerte olje- og gassanlegg med flere disipliner. \\
        Agent-/hendelsesbasert & Fanger interaksjon mellom aktører og logistikkflyt. & Vanskelig å kalibrere uten detaljerte prosesslogger. & Transport- og beredskapsøvelser på lufthavner. \\
    \end{tabular}
\end{table}

I praksis bør valget dokumenteres i en modelljournal som beskriver antagelser, datakrav og ansvarlige personer for videre vedlikehold
\citep{iso23247-2021}. Dokumentér alltid hvilke måledata som trengs for å holde modellen presis over tid og hvordan modellene skal
gjennomgås i fagfelleprosesser.

\section{Modellintegrasjon og kalibrering}
Valgt modell må kobles til dataflyten som ble kartlagt tidligere. Integrasjon handler både om teknisk infrastruktur og
organisatorisk forankring. Kalibrering sikrer at modellen forblir relevant når systemet endrer seg.

\subsection{Integrasjonsmønstre mot datakilder}
Modellen kan oppdateres batchvis fra historiske datalagre eller kontinuerlig via hendelsesstrømmer. Et vanlig mønster er å bruke
en meldingskø eller et datamesh som mellomlag, slik at flere modeller kan abonnere på samme datasett uten å skape tette koblinger
til kildesystemene. Edge-komponenter håndterer ofte aggregering og filtrering før data sendes til skybaserte analyseplattformer.

\subsection{Sanntidsoppdatering og orkestrering}
For digitale tvillinger som støtter operativ beslutningstaking må modellene ha mekanismer for å oppdatere tilstandsvariabler i
sanntid. Dette kan løses med orkestreringsverktøy som støtter versjonering av modeller, utrulling i containere og overvåkning av
beregningstider. Viktige indikatorer er latens fra sensor til modell, datakvalitet og hvilken grad modellen brukes i kontrollsløyfer.

\subsection{Kalibreringsstrategier}
Parameteridentifikasjon kan gjøres manuelt ved å justere parameterne basert på eksperterfaring, eller automatisk ved hjelp av
optimaliseringsalgoritmer som minimerer forskjellen mellom modell og målinger. Vanlige metoder er minste kvadrater,
Bayesiansk oppdatering eller Kalman-filtre. For systemer som endrer seg gradvis kan modellreduksjon og adaptive filtre holde
beregningstidene nede uten å tape presisjon. Tabell~\ref{tab:kap02-kalibrering} viser hvordan ulike kalibreringsteknikker kan
kombineres for å støtte flerfidelitetsmodeller og operativ drift \citep{kennedy2000predicting}.

\begin{table}[ht]
    \centering
    \caption{Oversikt over kalibreringsmetoder for digitale tvillinger.}
    \label{tab:kap02-kalibrering}
    \begin{tabular}{p{0.2\textwidth}p{0.32\textwidth}p{0.2\textwidth}p{0.22\textwidth}}
        \toprule
        \textbf{Metode} & \textbf{Formål} & \textbf{Når brukes den?} & \textbf{Typisk dokumentasjon} \\
        \midrule
        Minste kvadrater & Justere parametere basert på referansemålinger. & Ved innkjøring av nye sensorer eller komponenter. & Måleserier, residuallogg og parameteroversikt. \\
        Bayesiansk oppdatering & Kombinere historisk kunnskap med nye observasjoner. & Når usikkerhet skal uttrykkes eksplisitt og modeller skal sammenliknes. & Prior- og posteriorfordelinger samt versjonslogg. \\
        Kalman-filtre & Sanntidsestimater av tilstand og støy. & I operativ drift med streamingdata og krav til rask respons. & Konfigurasjon av filter, kovariansmatriser og alarmintervaller. \\
        Eksperimentell design & Planlegge testkampanjer for flerfidelitetsmodeller. & Før modelloppdateringer i laboratorier eller testfelt. & Testplan, datakvalitetsprotokoll og risikovurdering. \\
    \end{tabular}
\end{table}

Dokumentasjonen må vise hvilke datasett som ligger til grunn, hvordan datakvalitet er vurdert og hvilke roller som godkjenner nye
parameterverdier før de tas i bruk i produksjon.

\subsection{Kalibreringsverksted for prosessindustri}
For å gjøre kalibreringen håndgripelig gjennomfører mange norske industripartnere egne verksteder der driftspersonell, datafaglige og leverandører møtes rundt en felles modelljournal.\citep{sintef2021digital,equinor2021johansverdrup} Verkstedet organiseres gjerne som en to-dagers arbeidsøkt der første dag fokuserer på datakvalitet og hypoteser, mens dag to brukes til å teste parameterendringer i sandkassemiljøer. Prosessen gjør det mulig å identifisere hvilke antagelser som må dokumenteres i forkant av pilotering, og hvilke målepunkter som må overvåkes når modellen tas i bruk.

En vellykket samling følger tre hovedfaser:
\begin{enumerate}
    \item \textbf{Forberedelse:} Samle historiske driftsdata, definere hvilke scenarier som skal testes og oppdatere modelljournalens avsnitt om formål og datagrunnlag.
    \item \textbf{Kalibreringssløyfe:} Teste parameterendringer mot referansemålinger, vurdere konsekvenser i kontrollromsgrensesnittet og beslutte hvilke endringer som skal foreslås for produksjon.
    \item \textbf{Beslutningsmøte:} Dokumentere resultatene i beslutningsloggen, vurdere regulatoriske konsekvenser og forankre videre tiltak i kapittel~6 sin valideringsjournal og livssyklusmodellene i kapittel~7.\citep{dnv2023digitalassurance}
\end{enumerate}

Tabell~\ref{tab:kap02-kalibreringsverksted} viser en enkel rolle- og artefaktmatrise som kan brukes når verkstedet planlegges. Den gjør det tydelig hvem som har ansvar for ulike deler av kalibreringsarbeidet, hvilke dokumenter som må være tilgjengelige, og hvordan funnene kobles til videre forbedring.

\begin{table}[ht]
    \centering
    \caption{Rolle- og artefaktmatrise for kalibreringsverksted.}
    \label{tab:kap02-kalibreringsverksted}
    \begin{tabular}{p{0.22\textwidth}p{0.32\textwidth}p{0.2\textwidth}p{0.22\textwidth}}
        \toprule
        \textbf{Rolle} & \textbf{Nøkkeloppgaver} & \textbf{Hovedleveranse} & \textbf{Verktøy og kilder} \\
        \midrule
        Driftssjef & Prioritere scenarier, vurdere konsekvenser for produksjon og beredskap. & Oppdatert beslutningslogg med anbefalte tiltak. & Kontrolltårnrapporter, ROS-analyser, KPI-panel. \\
        Data scientist & Analysere datakvalitet, teste parameterendringer og dokumentere resultater. & Kalibreringsnotat med nye parameterverdier og usikkerhet. & Notebooks, datasett fra historisk arkiv, modelljournal. \\
        Leverandørrepresentant & Verifisere at endringene er kompatible med plattform og kontrakter. & Integrasjonsrapport og oppdatert vedlikeholdsplan. & API-spesifikasjoner, tjenestekontrakter, servicehåndbøker. \\
        Kvalitets- og sikkerhetsteam & Vurdere etterlevelse av krav og koordinere sign-off. & Revisjonssjekkliste og kobling til valideringsjournal i kapittel~6. & DNV-veiledere, interne prosedyrer, samsvarsregister. \\
        \bottomrule
    \end{tabular}
\end{table}

\subsection{Oppgave: planlegg en kalibreringskampanje}
Som forberedelse til verkstedet kan studentgrupper utvikle en kalibreringsplan for egen casebedrift. Planen bør inkludere:
\begin{itemize}
    \item en prioritering av hvilke målepunkter som må innhentes i laboratoriet før testen starter,
    \item forslag til datakvalitetsindikatorer og alarmgrenser som skal overvåkes under kampanjen,
    \item en oversikt over beslutninger som må eskaleres til ledelse eller myndigheter dersom avvik oppstår,
    \item referanser til eksisterende prosedyrer, for eksempel DNV~RP-A204 og SINTEF sine anbefalinger for norske tvillingprosjekter.\citep{dnv2023digitalassurance,sintef2021digital}
\end{itemize}

Gruppene kan levere planen som vedlegg til modelljournalen og bruke den i kapitteloppgavene når de vurderer hvordan digitale tvillinger går fra laboratoriet til operativ drift. Øvingen gjør det tydelig hvordan systemtenkning, modellering og styring henger sammen på tvers av kapitlene i boken.

\subsection{Mobile feltlaboratorier for datainnsamling}
Mange norske virksomheter bruker mobile laboratorier for å koble modellutvikling på campus med måledata fra felt og produksjon. Et mobilt oppsett består gjerne av containere eller trailere med sensorer, edge-noder og sikre kommunikasjonslinjer som kobles direkte til modellplattformen. Løsningen gir raske iterasjoner mellom hypoteser som testes i akademiske laboratorier og den faktiske infrastrukturen i industrien. Statnett bruker eksempelvis mobile målerigger for å hente PMU-data og stressteste nettmodeller før de rulles ut i driftskontrollene, mens Telenor og Equinor har etablert 5G-baserte feltlabber som knytter offshoreinstallasjoner til digitale tvillinger med lav forsinkelse.\citep{statnett2023digital,telenor2021equinor5g}

Når mobile laboratorier planlegges bør teamet avklare følgende punkter før utrulling:
\begin{itemize}
    \item Hvilke sensorer og datalogger som skal brukes, og hvordan strømforsyning og sikker kommunikasjon sikres i felt.
    \item Hvordan data strømmer inn i modelljournalen, inkludert metadata, tilgangskontroll og synkronisering med eksisterende dashbord fra kapittel~6.
    \item Hvordan operatører og studenter koordinerer observasjoner, for eksempel gjennom daglige standup-møter eller digitale beslutningslogger.
    \item Hvilke regulatoriske krav som påvirker testen, inkludert sikkerhet, personvern og avtalene med anleggets eiere.\citep{dnv2023digitalassurance,digdir2023styringai}
\end{itemize}

Tabell~\ref{tab:kap02-mobilelab} kan brukes som sjekkliste for å holde orden på logistikken når mobile laboratorier etableres. Den binder roller, målepunkter og læringsutbytte og gjør det enklere å dele erfaringer mellom campus og industri.

\begin{table}[ht]
    \centering
    \caption{Planleggingspakke for mobile feltlaboratorier i digitale tvillingprosjekter.}
    \label{tab:kap02-mobilelab}
    \begin{tabular}{p{0.23\textwidth}p{0.37\textwidth}p{0.32\textwidth}}
        \toprule
        \textbf{Fase} & \textbf{Nøkkelaktiviteter} & \textbf{Leveranser og læringspunkter} \\
        \midrule
        Forprosjekt & Velge case, avtale tilgang til anlegg og kartlegge målepunkt. & Feltplan, risikovurdering og oppdatert systemkart som kobles til modelljournalen. \\
        Installasjon & Montere sensorer, sikre strøm og sette opp edge- og 5G-forbindelser. & Konfigurasjonsnotat, sikkerhetssjekk og testlogg fra pilotkoblede enheter. \\
        Testkampanje & Kjøre scenarier mot modellen, gjennomføre daglige avstemminger og loggføre avvik. & Daglige beslutningslogger, datakvalitetsrapporter og forslag til modelljusteringer. \\
        Etterarbeid & Dokumentere læringsutbytte og koble funn til validerings- og gevinstplanene. & Oppdatert modelljournal, forbedringsliste til kapittel~6 og forslag til nye øvinger for kapittel~7. \\
        \bottomrule
    \end{tabular}
\end{table}

Etter en feltkampanje bør resultatene deles i en felles læringsøkt der driftsorganisasjonen, laboratorieteamet og studentgrupper sammenligner modellprediksjoner med faktiske målinger. Oppfølgingen sikrer at funnene mates inn i valideringsjournalen i kapittel~6 og i gevinstplanene i kapittel~7, og den gjør det enklere å planlegge neste runde med mobile tester.

\subsection{Geodata for kommunale digitale tvillinger}
Digitale tvillinger i kommunal sektor bygger på et mangfold av geospatiale datakilder som beskriver terreng, bygg, infrastruktur og miljøstatus. Ved å kombinere åpne datasett fra Kartverket og Copernicus-programmet med lokale målinger fra sensornettverk, kan kommuner kontinuerlig kalibrere modellene som brukes til beredskap, eiendomsforvaltning og klimatilpasning.\citep{kartverket2023ndh,copernicus2024handbook}

Før dataene integreres bør teamet avklare hvordan geodata skal harmoniseres mot eksisterende datastrømmer:
\begin{itemize}
    \item \textbf{Koordinatsystem og oppløsning}: Sikre at alle datasett reprojiseres til kommunens referansesystem (ofte UTM sone 32 eller 33) og at rutiner for resampling dokumenteres i modelljournalen.
    \item \textbf{Tidsstempling og versjonskontroll}: Registrer dato for siste oppdatering av ortofoto, høydemodeller og temalag, og lenk revisjonene til endringsloggen i dataspace-plattformen.
    \item \textbf{Tilgangs- og delingsavtaler}: Avklar hvilke datasett som kan deles åpent, og hvilke som krever avtaler med tekniske etater eller nasjonale myndigheter for å ivareta sikkerhet og personvern.\citep{kartverket2022lisens}
\end{itemize}

Tabellen under viser et utvalg geodataressurser som ofte brukes i norske kommuneprosjekter, og hvordan de kan kobles til modelleringen fra tidligere seksjoner.

\begin{table}[ht]
    \centering
    \caption{Kobling mellom geodata og modelloppgaver i kommunale digitale tvillinger.}
    \label{tab:kap02-geodata}
    \begin{tabular}{p{0.23\textwidth}p{0.18\textwidth}p{0.24\textwidth}p{0.27\textwidth}}
        \toprule
        \textbf{Datasett} & \textbf{Typisk oppløsning} & \textbf{Bruksområde} & \textbf{Integrasjon i modellarbeidet} \\
        \midrule
        Nasjonal detaljert høydemodell (NDH) & 1 m raster & Flom- og overvannsberegninger, skredfare. & Danner grunnlag for mesh i flerfidelitetsmodeller og verifiseres mot feltmålinger fra mobile laboratorier. \\
        Sentinel-2 multispektrale bilder & 10 m raster & Vegetasjonsanalyse, varmeøyer, overvåking av byggefelt. & Kombineres med datadrevne modeller for å oppdage endringer og trigge ny kalibrering i modelljournalen. \\
        Kommunale lednings- og byggdatasett & Vektordata & Drift av vann- og avløpsnett, byggforvaltning, energiplanlegging. & Integreres i dataspace-katalogen og kobles til indikatorene i kapittel~6 for samsvars- og risikorapportering. \\
        Miljøstatusrapporter fra Miljødirektoratet & Punkt- og områdedata & Luftkvalitet, støy, naturmangfold. & Gir grunnlag for scenariokjøringer i kapittel~4 og gevinstoppfølging i kapittel~7. \\
        \bottomrule
    \end{tabular}
\end{table}

Ved å beskrive hvordan geodata flyter inn i den digitale tvillingen blir det enklere å samordne innsatsen mellom kommunens geodataforvaltere, tekniske etater og masterstudenter som gjennomfører prosjektoppgaver. Seksjonen kan kombineres med arbeidsarkene i appendiks for å standardisere tilgangskontroll og oppdateringsrytme, og den bør testes i pilotverksteder sammen med kapitlene om validering og gevinstrealisering.

\subsection{Case: Digital tvilling for fjernvarme i Oslo}
Fortum Oslo Varme har utviklet en digital tvilling for å optimalisere energiproduksjon og distribusjon i fjernvarmenettet.
Modellen kombinerer hydrauliske ligninger for rørnettet med maskinlæringsmodeller som predikerer varmebehov basert på vær,
bygningstyper og historisk forbruk. Integrasjonen skjer via en datastrøm fra sensorer i kundesentraler og produksjonsanlegg til
et skybasert kontrollrom. Kalibreringen utføres daglig ved å sammenlikne modellprediksjoner med faktiske returtemperaturer, og
parametere justeres automatisk når avvik overstiger definerte terskler. Caset viser hvordan samarbeid mellom energiselskap,
teknologipartnere og kommune gir en robust modell som støtter både operativ drift og langsiktige investeringsbeslutninger.

\subsection{Case: Kommunal energihub for bydelssenter}
Oslo kommune har satt mål om at kommunale bygg skal være selvstendige energinoder som kombinerer varme, kjøling, solkraft og
fleksible laster i én styringsplattform.\citep{osloeiendom2023strategi,oslo2024klimaeiendom} For å lykkes må kommunen og
energiselskapene dele data om energibruk, inneklima og lokal produksjon i sanntid, samtidig som vedlikeholdsplaner,
anskaffelsesstrategier og klimabudsjetter oppdateres løpende. KS anbefaler at kommunene etablerer digitale tvillinger som
binder tekniske målinger til styringsdokumenter for energioppfølging, slik at krav og tiltak kan spores per bygg og bydel.
\citep{ks2024eiendomsdrift} Når slike tvillinger kobles til lokale energilagre og fjernvarmesløyfer kan de prioritere tiltak som
reduserer både kostnader og klimagassutslipp gjennom dynamisk lasthåndtering.\citep{norskfjernvarme2024fleksibilitet}

Tabell~\ref{tab:kap02-energihub} viser et eksempel på hvordan systemet kan deles inn i lag som ivaretar dataflyt, ansvar og
indikatorer. Strukturen brukes i masterkurset for å diskutere hvordan kommunen kombinerer eiendomsdata med energimarkedsinformasjon
og mobilitetsbehov i samme modell. Studentene skal spesielt merke seg hvordan kapittel~3 sitt arbeid med dataspace-kontrakter og
kapittel~6 sin kvalitetsjournal gir grunnlaget for styring av avvik og rapportering til klimabudsjettet.

\begin{table}[ht]
    \centering
    \caption{Systemlag for kommunal energihub med digital tvilling.}
    \label{tab:kap02-energihub}
    \begin{tabular}{p{0.21\textwidth}p{0.31\textwidth}p{0.26\textwidth}p{0.18\textwidth}}
        \toprule
        \textbf{Systemlag} & \textbf{Hovedoppgaver} & \textbf{Data og indikatorer} & \textbf{Ansvarlige roller} \\
        \midrule
        Feltsystem og byggautomasjon & Samle måledata fra varmepumper, ventilasjon og energilagre, styre laster og registrere hendelser. & Energiflyt per sone, temperatur og CO$_2$-nivå, status for termiske lagre. & Driftsoperatør og energivaktmester. \\
        \addlinespace
        Energihub-kontroll & Optimalisere lastbalanse mellom solkraft, fjernvarme og batterier, samt planlegge fleksibilitet mot nettselskap. & Prognoser for forbruk og produksjon, fleksibilitetsbud, avtaler med nettselskap. & Energi- og klimakoordinator. \\
        \addlinespace
        Dataspace og delingsplattform & Sikre datakontrakter, maskere persondata og publisere API-er for samarbeidspartnere. & Metadata, tilgangslogger, hendelsesjournal og kvalitetsflagg fra kapittel~6. & Data steward og informasjonssikkerhetsrådgiver. \\
        \addlinespace
        Styring og portefølje & Koble tiltak til klimabudsjett, innkjøpsplaner og vedlikeholdsprogram. & Tiltakslogg, investeringsplaner, KPI-er for energi, utslipp og innemiljø. & Porteføljestyrer og økonomiteam. \\
        \bottomrule
    \end{tabular}
\end{table}

Arbeidsprosessen følger fire steg som studentgrupper kan bruke når de modellerer egne kommunale energihuber:
\begin{enumerate}
    \item \textbf{Kartlegg datakilder:} Samle systemkart fra byggautomasjon, energisentral og mobilitetsinfrastruktur og plasser
    dem i et felles dataspace-grensesnitt med tilgangsstyring fra kapittel~3.
    \item \textbf{Prioriter tiltak:} Bruk klimabudsjettet og energiledelsesprogrammet til å rangere hvilke bygg eller laster som
    skal inngå i første iterasjon.\citep{oslo2024klimaeiendom,enova2024energiledelse}
    \item \textbf{Modeller fleksibilitet:} Kombiner førsteprinsippmodeller for varme- og kjølesløyfer med datadrevne prognoser for
    brukeradferd, og dokumenter antagelser i modelljournalen som beskrevet i kapittel~6.
    \item \textbf{Planlegg læringssløyfer:} Design kvartalsvise workshops der drift, energi og innkjøp evaluerer effektene og
    oppdaterer tiltakslogg og gevinstkart sammen med kapittel~7.
\end{enumerate}

Resultatene fra energihuben bør kobles til mobilitets- og arealplanlegging i bydelene slik at elbillading, byggdrift og
lokale energisamfunn ses i sammenheng. Når masterstudentene dokumenterer datastrømmer, indikatorer og beslutninger i tabellen
over, får de samtidig et revisjonsspor som kan brukes i kvalitetsjournalen og i porteføljestyringen.

\subsection{Case: Systemmodellering for autonome ferger}
Trondheim havn og NTNU har etablert en digital tvilling for den autonome passasjerfergen Milliampere~2, som trafikkerer kanalene
ved Brattøra.\citep{ntnu2023milliampere2} Tvillingen koordinerer navigasjon, fremdrift og fjernoperasjon i én modell slik at
testteamet kan vurdere både tekniske og organisatoriske konsekvenser før fergen får seilasgodkjenning. Sjøfartsdirektoratets
veiledning for autonome fartøy krever at slike piloter dokumenterer risikovurderinger, datadelingsavtaler og prosedyrer for
overgang til manuell kontroll.\citep{sdir2023autonomefartoy} Teamet bruker derfor systemtenkning til å binde sammen modeller for
sensorfusjon, kraftdistribusjon og situasjonsforståelse i kontrollsenteret. DNV sine anbefalinger for autonome fartøy brukes som
rammeverk for å vurdere modenheten til kontrollalgoritmer, cybersikkerhet og beredskap.\citep{dnv2024autonomous}

Arbeidet organiseres i tre modellstrømmer som gjenspeiler kravene fra både tekniske standarder og tilsynsmyndigheter:
\begin{itemize}
    \item \textbf{Fysisk fremdrift og kraftbalanse:} CFD- og maskinlæringsmodeller beskriver hvordan fremdriftssystemet reagerer
    på bølger, vind og last, og synkroniseres med energilogger for å sikre at batterier og ladestasjoner utnyttes optimalt.
    \item \textbf{Navigasjon og trafikkbildet:} Agentbaserte modeller simulerer møter med andre fartøy, kajmanøvre og
    fartsbegrensninger. Resultatene mates inn i fjernoperasjonssenteret som anbefalinger og alarmer for navigatørene.
    \item \textbf{Operasjon og beredskap:} Diskrete hendelsesskript kombinerer nødprosedyrer, kommunikasjon med VTS og
    beredskapslogg slik at systemet raskt kan overføres til manuell styring dersom tvillingen avdekker avvik.
\end{itemize}

Tabell~\ref{tab:kap02-fergemodeller} viser hvordan modellene forankres i ansvar og datagrunnlag. Oversikten brukes i masterkurset
til å diskutere hvilke fagmiljø som må involveres i hver fase, og hvordan modelljournalen bør struktureres for å møte kravene fra
Sjøfartsdirektoratet og forsikringsselskapene.

\begin{table}[ht]
    \centering
    \caption{Systemlag og modelleringsartefakter for autonom ferge.}
    \label{tab:kap02-fergemodeller}
    \begin{tabular}{p{0.22\textwidth}p{0.32\textwidth}p{0.24\textwidth}p{0.18\textwidth}}
        \toprule
        \textbf{Lag} & \textbf{Modelleringsfokus} & \textbf{Beslutningsspørsmål} & \textbf{Primære datakilder} \\
        \midrule
        Fysisk system & Hydrodynamikk, fremdrift og energidistribusjon & Dimensjonering av batteri, reservekraft og akselerasjonsgrenser & CFD-resultater, kraftsensorer, ladelogger \\
        Navigasjon og trafikk & Sensorfusjon, kollisjonsunngåelse og rutevalg & Når skal autonom modus frakobles og hvordan håndteres trafikale avvik? & Lidar, AIS, kamerafeed og digitale sjøkart \\
        Fjernoperasjon & Kontrollromsgrensesnitt, rollefordeling og situasjonsforståelse & Hvilke alarmer og tiltak skal eskaleres til navigatør og havnevakt? & Operatørlogger, kommunikasjonssystem, hendelsesjournal \\
        Regelverk og beredskap & Risikovurdering, sikkerhetsbarrierer og rapportering & Hvilke dokumenter må oppdateres før neste testkampanje og hvem signerer? & ROS-analyser, tilsynsrapporter, øvingslogg \\
        \bottomrule
    \end{tabular}
\end{table}

Caset viser at autonome tvillinger må håndtere både teknisk kompleksitet og streng regulering. Ved å kombinere modellene i en
samlet systemjournal kan prosjektet vise at sikkerhetsbarrierer, kompetanse og dataflyt er på plass før fergen går over i
regelmessig drift. Erfaringene fra Trondheim brukes nå som referanse når nye autonome ruter planlegges i andre norske havner,
og gir studentene et konkret eksempel på hvordan systemtenkning gir tryggere innovasjon.

\subsection{Anbefalt arbeidsflyt for team}
\begin{enumerate}
    \item Kartlegg hvilke datakilder som skal kobles til og etabler nødvendige API-er eller databrokere.
    \item Implementer monitorering som fanger avvik mellom modell og observasjoner i sanntid.
    \item Planlegg regelmessige kalibreringssykluser og dokumentér endringer i parametere og antagelser.
\end{enumerate}

\subsection{Samsvarstesting i digitale laboratorier}
Før en digital tvilling tas i bruk i operativ drift trenger teamet et laboratorieløp som viser at modellen tåler regulatoriske og
tekniske krav. Norske aktører som Statnett og Energi Norge har etablert kontrolltårn der simuleringer, hendelsesscenarier og
reelle målinger sammenlignes før løsningen får grønt lys i kontrollrommet.\citep{statnett2024kontrolltarn,energinorge2023beredskap}
Ved å dokumentere testsløyfene systematisk i modelljournalen unngår man at samsvarskrav blir et ad hoc-vedlegg som bare
oppdateres ved revisjoner.

Tabell~\ref{tab:kap02-samsvar} kan brukes som sjekkliste når laboratoriet planlegges. Den kombinerer forventninger fra norske
veiledere og DNV sine anbefalinger for digital tvilling-assurance, og viser hvilke artefakter som bør produseres i hver test.
\citep{sintef2021digital,dnv2023digitalassurance}

\begin{table}[htbp]
    \centering
    \caption{Testregime for samsvarstesting av digitale tvillinger i laboratoriet.}
    \label{tab:kap02-samsvar}
    \begin{tabular}{p{0.24\textwidth}p{0.32\textwidth}p{0.18\textwidth}p{0.22\textwidth}}
        \toprule
        \textbf{Testtype} & \textbf{Formål} & \textbf{Ansvar} & \textbf{Nøkkelartefakter} \\
        \midrule
        Integrasjonstest & Verifisere at datastrømmer, API-er og sikkerhetstiltak fungerer før modellen kobles mot produksjon. &
        Plattformteam & Tilkoblingsprotokoll, datakvalitetsrapport, risikovurdering i tråd med ISO~31000. \\
        \addlinespace
        Scenario- og stresstest & Bekrefte at modellen gir stabile anbefalinger under avvik og hendelser som øves i kontrolltårnet. &
        Fagansvarlig for drift & Hendelseslogg, sammenstilling av simulerte og faktiske målinger, beslutningslogg fra kontrollrom. \\
        \addlinespace
        Etterlevelsesreview & Sikre at modelljournal, personvern og sikkerhetskrav er dokumentert før utrulling til operativ drift. &
        Kvalitets- og sikkerhetsteam & Revisjonssjekkliste, signert godkjenning, kobling til kapittel~6 sin valideringsjournal. \\
        \addlinespace
        Pilotering med operatører & Teste brukergrensesnitt, forklaringsmekanismer og arbeidsprosesser med operatørteam i lab. &
        Operasjonsleder og fagfeller & Tilbakemeldingslogg, oppdatert opplæringsplan, referat til livssyklusmodellen i kapittel~7. \\
        \bottomrule
    \end{tabular}
\end{table}

Samsvarstesting bør avsluttes med en kort beslutningspakke som viser hva som er verifisert, resterende risiko og hvilke tiltak som
må følge modellen inn i produksjon. Dokumentasjonen gjør det enklere å dele erfaringer mellom laboratorier, bruke resultatene i
fagfellelogg og koble forbedringspunkter til tiltaksplanen i kapittel~6.

\section{Systemdynamiske beslutningssløyfer for beredskap og kapasitet}
Norske virksomheter må håndtere samtidige krav til beredskap, kapasitet og bærekraft når digitale tvillinger brukes i drift. Systemdynamiske modeller gjør det mulig å koble ressursdisponering, hendelseshåndtering og læringssløyfer i én helhet slik at ledelsen får et felles beslutningsgrunnlag.\citep{dsb2023nrb} Når slike modeller etableres i kapittel~2, får teamet et rammeverk som senere kan knyttes til valideringsjournalen i kapittel~6 og gevinstplanen i kapittel~7.

\subsection{Kritiske påvirkningsfaktorer}
En systemdynamisk analyse starter med å identifisere hvilke variabler som driver kapasitet og risiko over tid:
\begin{itemize}
    \item \textbf{Belastning på infrastruktur:} Kombiner produksjons- eller pasientvolum med vær- og hendelsesdata for å beregne hvilke scenarier som presser kapasiteten.\citep{helsedir2023nasjonalberedskap}
    \item \textbf{Tilgjengelig kompetanse og ressurser:} Modellér hvor raskt vaktlag, beredskapsstyrker og leverandører kan mobiliseres, og hvordan fravær påvirker responstiden.
    \item \textbf{Styringssignaler:} Beskriv hvordan beslutninger fra beredskapsstab, kontrolltårn eller politiske myndigheter endrer prioriteringer og investeringsløp.
    \item \textbf{Læringsmekanismer:} Knytt hendelseslogger, øvelser og fagfelleevalueringer til justering av prosedyrer og indikatorgrenser.
\end{itemize}

Tabell~\ref{tab:kap02-beredskapssløyfer} viser et forslag til hvordan variablene kobles i praktiske beslutningssløyfer. Oversikten hjelper studentgrupper med å tydeliggjøre hvilke datastrømmer som må være på plass før modellen tas i bruk.

\begin{table}[htbp]
    \centering
    \caption{Eksempel på systemdynamiske sløyfer for beredskap og kapasitet.}
    \label{tab:kap02-beredskapssløyfer}
    \begin{tabular}{p{0.24\textwidth}p{0.32\textwidth}p{0.18\textwidth}p{0.20\textwidth}}
        \toprule
        \textbf{Sløyfe} & \textbf{Beskrivende mekanisme} & \textbf{Sentrale datakilder} & \textbf{Styringsartefakter}\\
        \midrule
        Belastningsbalanse & Økt etterspørsel øker ressursuttak; tiltak i kontrolltårn demper belastningen og frigjør kapasitet for kritiske tjenester. & Sanntidsmålinger, hendelseslogger, værdata & Operativ tiltakslogg, scenarioark, KPI-panel fra kapittel~6.\\
        Kompetanse og utholdenhet & Læringsøkter og øvelser styrker kompetanse, som reduserer behovet for ad hoc-ressurser og forbedrer responstid. & Øvingslogg, kompetanseregister, HR-data & Opplæringsplan, beredskapsplan og gevinsttabell i kapittel~7.\\
        Etterlevelse og tillit & Revisjoner og tilsyn avdekker avvik; forbedringstiltak forbedrer compliance og øker tiltro til modellen. & Revisjonsrapporter, kvalitetsjournal, avvikssystem & Valideringsjournal, regulatoriske sjekklister og risikoregister.\\
        Læringssløyfe for investeringer & Dokumenterte effekter fra øvelser styrer hvilke investeringer som prioriteres i neste budsjett. & Kostnadsdata, målt effekt, porteføljelogger & Investeringsbeslutningsgrunnlag, gevinstplan og kapittel~7 sin tiltakslogg.\\
        \bottomrule
    \end{tabular}
\end{table}

\subsection{Fra modell til operativ styring}
Når sløyfene er definert, bør teamet koble dem til eksisterende styringsfora:
\begin{enumerate}
    \item \textbf{Kartlegg beslutningspunkter:} Identifiser hvilke møter, rapporter eller dashboards som skal få innsikt fra modellen.\citep{dsb2023nrb}
    \item \textbf{Definer terskler:} Sett målverdier og alarmgrenser som automatisk utløser handling i beredskapsplanen og gevinstoppfølgingen.
    \item \textbf{Planlegg datatilgang:} Avklar hvordan datasett fra helse, energi eller transport skal deles i tråd med dataspace-prinsipper fra kapittel~3.
    \item \textbf{Etabler evalueringssløyfe:} Beskriv hvordan effekten av tiltak måles og mates tilbake i modelljournalen og tiltaksloggen.
\end{enumerate}

\subsection{Oppgave: modeller en tverrsektoriell øvelse}
Som arbeidsoppgave kan studentgrupper modellere en øvelse som kobler et energiselskap og en helseinstitusjon. Start med å beskrive ressursflyten når et langvarig strømbrudd oppstår, og hvordan backup-løsninger aktiveres. Deretter skal gruppen:
\begin{itemize}
    \item definere indikatorer for tilgjengelig kapasitet, responstid og pasientsikkerhet,
    \item beskrive hvilke data som må deles gjennom dataspace-avtaler før hendelsen,
    \item modellere hvordan beslutninger eskaleres fra lokale team til regional beredskapsstab,
    \item foreslå hvordan læringspunkter dokumenteres i modelljournal, valideringsjournal og gevinstplan.
\end{itemize}
Resultatet brukes som grunnlag i kapitteloppgaven der teamet skal planlegge en faktisk øvelse sammen med partnerne i kapittel~6 og kapittel~8.

\section{Modelldokumentasjon og styring}
God dokumentasjon sikrer at digitale tvillinger kan forvaltes over tid og at beslutninger er etterprøvbare. ISO~23247 beskriver hvordan modellbeskrivelser, datakataloger og styringsrutiner bør struktureres for produksjonsnære tvillinger \citep{iso23247-2021}. Ved å etablere en felles dokumentasjonsprosess blir det enklere å dele modeller på tvers av fagmiljø og følge opp krav fra myndigheter og industripartnere.

\subsection{Kjerneartefakter}
Hvert modelloppsett bør følges av en artefaktsamling som minimum inneholder:
\begin{itemize}
    \item \textbf{Modelljournal} med formål, antagelser, matematiske representasjoner og referanse til opplæringsdata.
    \item \textbf{Datakatalog} som beskriver kilder, oppdateringsfrekvens, kvalitetsindikatorer og tilgangsnivå.
    \item \textbf{Beslutningslogg} som kobler modellresultater til tiltak, ansvarlige personer og godkjenningstidspunkt.
    \item \textbf{Risikovurdering} som fanger opp driftsavvik, cybersikkerhetskrav og konsekvenser av modellfeil.
\end{itemize}

\begin{table}[ht]
    \centering
    \caption{Forslag til struktur for modelljournal i tråd med norske veiledere.}
    \label{tab:kap02-modelljournal}
    \begin{tabular}{p{0.23\textwidth}p{0.34\textwidth}p{0.33\textwidth}}
        \toprule
        \textbf{Kapittel} & \textbf{Innhold} & \textbf{Relevante kilder} \\
        \midrule
        Formål og kontekst & Definer beslutningen som skal støttes, eierskap og regulatoriske rammer. & Prosjektmandat, styringsdokumenter, politiske føringer. \\
        Modellbeskrivelse & Oppsummer antagelser, ligninger, datagrensesnitt og begrensninger. & Modellark, kildekode, integrasjonsdiagram. \\
        Data- og kvalitetsgrunnlag & Beskriv datakilder, kvalitetsindikatorer og rutiner for datavask. & Datakatalog, dataprofilering, sensoroversikt. \\
        Drift og forbedring & Forklar monitorering, alarmer, revisjonsfrekvens og ansvarlige personer. & Kalibreringslogg, avviksrapporter, forbedringsplaner. \\
        Etikk og sikkerhet & Dokumenter risiko, personvern- og sikkerhetstiltak. & ROS-analyser, DPIA, beredskapsplan. \\
        \bottomrule
    \end{tabular}
\end{table}

Digitaliseringsdirektoratet anbefaler å bruke modelljournaler for å dokumentere kunstig intelligens og avanserte analysemodeller
i offentlig sektor \citep{digdir2023modelljournal}. Strukturert dokumentasjon gjør det mulig å følge sporbarhetskrav fra
standarder som ISO~23247 og samtidig gjenbruke erfaringer mellom prosjekter.

\subsection{Versjonskontroll og sporbarhet}
Alle modeller bør ligge i et versjonsstyrt repositorium der kode, parametere og dokumentasjon utvikles samlet. Metadata for hver versjon må beskrive hvilke datasett og kalibreringsmetoder som er brukt, samt referanser til godkjenning og testresultater. For hybride og flerfidelitetsmodeller innebærer dette at også grensesnittfiler og transformasjoner sjekkes inn, slik at ko-simulering kan reproduseres ved behov \citep{boschert2018digital}. Kombiner gjerne Git-baserte prosesser med automatisert rapportering fra modellovervåkingen for å vise status i sanntid.

\subsection{Sjekkliste før fagfellegjennomgang}
Før en modell distribueres til undervisning eller pilotering, bør teamet kontrollere følgende punkter:
\begin{enumerate}
    \item Alle kilder og referanser er oppdatert i modelljournalen, og versjonsnummer samsvarer med distribuerte pakker.
    \item Kalibreringsresultater og usikkerhetsanalyser er lagret sammen med underliggende datasett og beskrivelser av kvalitetssikring.
    \item Tiltak for sikker tilgang, personvern og beredskap er dokumentert, inkludert roller og kontaktpunkter for hendelseshåndtering.
    \item Plan for kontinuerlig forbedring er definert med milepæler, måleparametere og ansvarlig fagressurs.
\end{enumerate}

\section{Norske politiske rammer}
Digitale tvillinger i Norge utvikles innenfor et politisk landskap som legger tydelige føringer for hvordan data skal forvaltes, deles og brukes. Regjeringens datastrategi for offentlig sektor legger vekt på åpne, tilgjengelige og gjenbrukbare data som skal styrke innovasjon og effektivisering \citep{regjeringen2022datastrategi}. I tillegg beskriver stortingsmeldingen \emph{Data som ressurs} hvordan sektorer må samhandle gjennom felles prinsipper for datadeling, sikkerhet og personvern \citep{meldst22datasomressurs}. Oversikten nedenfor erstatter TikZ-illustrasjonen og viser hvordan europeiske, nasjonale og sektorspesifikke initiativ henger sammen når virksomheter planlegger digitale tvillinger:
\begin{itemize}
    \item \textbf{Europeisk nivå:} Datarom, Gaia-X og sektorspesifikke reguleringer (energi, transport, helse) setter rammer for interoperabilitet og sikkerhet.
    \item \textbf{Nasjonalt nivå:} Regjeringens datastrategi, Digitaliseringsdirektoratets retningslinjer og sektormeldinger definerer politikk, finansiering og samarbeidsmodeller.
    \item \textbf{Sektornivå:} Veiledere fra NVE, Samferdselsdepartementet, Helsedirektoratet m.fl. oversetter krav til konkrete tiltak i prosjektene.
\end{itemize}

\subsection{Statlige føringer og standardisering}
EU sitt arbeid med datarom og Gaia-X gir retningslinjer for interoperabilitet, sikker tilgang og styring av dataøkosystem \citep{eu2020circulareconomy}. Disse prinsippene videreføres i nasjonale strategier og i oppfølgingen av personvern-, sikkerhets- og energilovgivning. For digitaliseringsprosjekter i offentlig sektor betyr det at arkitekturen må støtte datadeling på tvers av etater, samtidig som den ivaretar krav til informasjonssikkerhet gjennom standarder som ISO~27001 og NIS2. NVE og Samferdselsdepartementet stiller tilsvarende krav i sektorveiledninger for energi og transport \citep{nve2023nettplan,avinor2022digital}.

\subsection{Case: Statnett sitt nettplanleggingsprogram}
Statnett bruker digitale tvillinger til å planlegge nettutbygging og driftsoptimalisering i takt med økende elektrifisering. Plattformen integrerer kraftsystemmodeller, sanntidsmålinger og scenarioer for energioverganger og må følge både energiloven og nye EØS-krav til datasamarbeid \citep{statnett2023digital}. Caset illustrerer hvordan politiske føringer styrer både modellstrukturen og arbeidsprosessene.

\begin{table}[ht]
    \centering
    \caption{Regulatoriske og tekniske komponenter i Statnett sitt nettplanleggingsprogram.}
    \label{tab:kap02-statnett}
    \begin{tabular}{p{0.33\textwidth}p{0.57\textwidth}}
        \toprule
        \textbf{Komponent} & \textbf{Beskrivelse} \\
        \midrule
        Regulatorisk ramme & Energimyndighetenes konsesjonskrav, NVE sin veileder for nettplanlegging og krav til samfunnsøkonomiske analyser. \\
        Dataintegrasjon & PMU-målinger og SCADA-data kobles med langsiktige forbruksprognoser og markedsdata. \\
        Beslutningsfora & Tverretatlige møter med NVE og OED hver kvartal for å evaluere scenarioer og prioriteringer. \\
        Effektmåling & Reduserte flaskehalser, bedre koordinering med regionale nettselskap og dokumentert klimanytte i investeringsbeslutninger. \\
        Kompetansebygging & Kurs i regulatoriske krav og modellbruk for Statnett-ansatte og samarbeidspartnere. \\
        \bottomrule
    \end{tabular}
\end{table}

Caset viser hvordan systemkart og modellvalg må reflektere både tekniske og regulatoriske perspektiver. Ved å knytte datastrømmer til konkrete konsesjonsprosesser kan Statnett dokumentere hvordan tiltak gir effekt og samtidig ivareta krav til forsyningssikkerhet.

\subsection{Case: Avinor sin lufthavntvilling}
Avinor utvikler digitale tvillinger for å koordinere kapasitetsplanlegging, energibruk og passasjerlogistikk på tvers av norske lufthavner \citep{avinor2022digital}. Modellene bygger på reiseinformasjon, værdata og sanntidsmålinger fra terminalene, og må samsvare med nasjonale mål om utslippskutt og universell utforming. I tillegg følges kravene i nasjonal transportplan og EASA-regelverk for sikkerhet.

\begin{table}[ht]
    \centering
    \caption{Nøkkelfunn fra Avinor sitt fagprogram for digitale tvillinger.}
    \label{tab:kap02-avinor}
    \begin{tabular}{p{0.34\textwidth}p{0.56\textwidth}}
        \toprule
        \textbf{Dimensjon} & \textbf{Observasjoner} \\
        \midrule
        Datagrunnlag & Kombinerer passasjerdata, bagasjehåndtering, energimålinger og værvarsel i felles dataplattform. \\
        Politisk kobling & Understøtter nasjonalt klimamål om 50\% reduksjon i utslipp fra lufthavndrift innen 2030 og krav om universell utforming. \\
        Operativ bruk & Scenarioanalyse for terminaldrift, bemanning og nye ruteplaner, samt simulering av beredskap. \\
        Resultater & 15\% redusert energibruk i utvalgte terminaler og bedre flyt for passasjerer ved toppbelastning. \\
        Deling & Faglig nettverk med regionale lufthavner og internasjonale partnere for å dele modeller og indikatorer. \\
        \bottomrule
    \end{tabular}
\end{table}

Eksemplet fremhever hvordan digitale tvillinger i transportsektoren må koordineres med både nasjonale strategier og europeiske sikkerhetskrav, og hvordan modellene blir en arena for samarbeid mellom ingeniører, planleggere og myndigheter.

\section{Refleksjonsspørsmål og øvinger}
\begin{enumerate}
    \item Lag et systemkart for en valgt industriell prosess.
    \item Diskuter fordeler og ulemper ved å kombinere fysikkbaserte og datadrevne modeller.
    \item Ta utgangspunkt i fjernvarme-caset og skisser hvordan kalibreringssløyfen kan overvåkes og forbedres over ett år.
\end{enumerate}
