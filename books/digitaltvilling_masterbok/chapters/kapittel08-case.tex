\chapter{Bruksområder og norske case}

\section{Læringsmål}
\begin{itemize}
    \item Analysere variasjon i anvendelser av digitale tvillinger på tvers av sektorer.
    \item Kritisk evaluere norske case med hensyn til teknologi, organisasjon og gevinster.
    \item Overføre læring fra case til egne prosjekter.
\end{itemize}

\section{Industrisektoren}
\begin{itemize}
    \item Prosessindustri og produksjon: optimalisering og vedlikehold.
    \item Energi og kraft: nettstabilitet, planlegging og bærekraft.
    \item Maritim og offshore: sikkerhet, logistikk og autonomi.
\end{itemize}

\section{Bygg, transport og helse}
\begin{itemize}
    \item Smarte bygg og byer: BIM, energistyring, mobilitet.
    \item Transport og logistikk: sanntidskoordinering og prediktive tjenester.
    \item Helse: pasientspesifikke modeller og klinisk beslutningsstøtte.
\end{itemize}

\section{Metodikk for caseanalyse}
\begin{itemize}
    \item Rammeverk: mål, omfang, teknologi, organisasjon, effekt.
    \item Datakilder for casearbeid (rapporter, intervjuer, åpne data).
    \item Etiske og samfunnsmessige hensyn.
\end{itemize}

\section{Casestudie-mal: Digital tvilling for batteriproduksjon i Mo i Rana}
Denne malen er inspirert av etableringen av den nasjonale batteriverdikjeden i Mo i Rana, der en digital tvilling skal binde sammen prosesslinjer, energiforbruk og logistikk. Bruk strukturen nedenfor som ramme for å dokumentere og analysere egne case, og erstatt eksemplene med data fra den faktiske virksomheten du studerer.

\subsection{Sammendrag av caset}
\begin{tabular}{p{0.32\textwidth}p{0.62\textwidth}}
\textbf{Element} & \textbf{Beskrivelse / spørsmål som skal besvares} \\
Virksomhet og lokasjon & Hvilken organisasjon studeres, og hvor er den plassert? \\
Strategisk hovedmål & Hva ønsker virksomheten å oppnå med den digitale tvillingen (for eksempel redusert energibruk eller økt kapasitet)? \\
Scope for tvillingen & Hvilke deler av verdikjeden eller prosesslinjen inngår? \\
Status ved prosjektstart & Hvilke systemer, datakilder og arbeidsprosesser fantes fra før? \\
\end{tabular}

\subsection{Forretningsmål og interessenter}
\begin{itemize}
    \item Beskriv konkrete forretningsmål og prioriteringer (for eksempel energieffektivitet, kvalitetskontroll eller fleksibel skalering).
    \item Identifiser hovedinteressenter: produksjonsledelse, operatører, IT/OT-team, energiselskap, logistikkpartnere og lokale myndigheter.
    \item Avklar hvordan målene måles, inkludert KPI-er som OEE, energiforbruk per battericelle og skraprate.
\end{itemize}

\subsection{Datagrunnlag og integrasjon}
\begin{itemize}
    \item Kartlegg hvilke datastrømmer som trengs: sensorer langs elektrodefabrikasjonen, MES/ERP-data, vær- og strømpriser, lager- og transportstatus.
    \item Vurder kvalitet, oppdateringsfrekvens og eierskap til hver datakilde.
    \item Beskriv integrasjonsarkitekturen, for eksempel OPC UA mot produksjonsutstyr og skybasert datasjø for historiske data.
\end{itemize}

\subsection{Modelleringsstrategi}
\begin{itemize}
    \item Definer hvilke modelltyper som skal inngå: prosessimulering, energimodell, logistikkmodell og prediktive vedlikeholdsmodeller.
    \item Spesifiser modelloppløsning (granularitet), valideringsdata og hvordan modeller kalibreres mot observasjoner.
    \item Marker antatte begrensninger eller usikkerheter, som begrenset historikk fra nyetablert produksjon.
\end{itemize}

\subsection{Implementering og arbeidsprosesser}
\begin{itemize}
    \item Beskriv hvordan tvillingen tas i bruk i daglig drift: dashboards i kontrollrom, digitale arbeidsordre, simulering før omstilling.
    \item Legg inn plan for opplæring av operatører og samarbeid mellom dataanalytikere og prosessingeniører.
    \item Dokumenter forvaltningsmodell: hvem eier modellen, hvordan versjoneres den, og hvordan sikres datakvalitet over tid?
\end{itemize}

\subsection{Resultater, gevinster og gevinstrealisering}
\begin{itemize}
    \item Oppgi konkrete gevinster som måles, for eksempel 8~\% lavere energiforbruk, redusert kassasjon eller raskere oppskalering av nye produktserier.
    \item Beskriv metoder for å verifisere gevinster, inkludert før- og etter-målinger og kontrollgrupper.
    \item Noter sekundære effekter: bedre HMS-oppfølging, samarbeid med kraftleverandør eller nye forretningsmodeller.
\end{itemize}

\subsection{Overføringsverdi og neste steg}
\begin{itemize}
    \item Drøft hvilke komponenter som kan gjenbrukes i andre norske industrier, for eksempel prosessmodeller, datastrømsarkitektur eller governance.
    \item Identifiser videre utvikling: kobling mot leverandørkjede, bruk av AI til kvalitetssikring, integrasjon med europeiske energimarkeder.
    \item Foreslå indikatorer og beslutningspunkter for å evaluere når tvillingen bør oppdateres eller skaleres.
\end{itemize}

\subsection*{Sjekkliste for caserapporten}
\begin{itemize}
    \item[$\square$] Har du dokumentert mål, interessenter og scope?
    \item[$\square$] Er alle relevante datakilder og integrasjoner beskrevet?
    \item[$\square$] Inneholder rapporten tydelige modelleringsvalg og begrunnelser?
    \item[$\square$] Er det lagt inn plan for drift, opplæring og governance?
    \item[$\square$] Er gevinstene målbare og knyttet til KPI-er som følges opp?
    \item[$\square$] Er læringspunkter og videre tiltak formulert?
\end{itemize}

\section{Refleksjonsspørsmål og øvinger}
\begin{enumerate}
    \item Velg en norsk case og vurder modenhet og gevinster.
    \item Utarbeid en plan for å hente inn datakilder til et nytt case.
    \item Diskuter hvordan erfaringer fra én sektor kan overføres til en annen.
\end{enumerate}
