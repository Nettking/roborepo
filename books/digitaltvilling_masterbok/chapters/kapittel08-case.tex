\chapter{Bruksområder og norske case}
\label{chap:case}

\section{Læringsmål}
\begin{itemize}
    \item Analysere variasjon i anvendelser av digitale tvillinger på tvers av sektorer.
    \item Kritisk evaluere norske case med hensyn til teknologi, organisasjon og gevinster.
    \item Overføre læring fra case til egne prosjekter.
\end{itemize}

\section{Industrisektoren}
Digitale tvillinger i norsk industri kobler ofte sammen komplekse produksjonslinjer, leverandørkjeder og energisystemer. Når du analyserer casene, bør du identifisere hvilke beslutningssituasjoner som styrer investeringen, og hvordan innsikten fra tvillingen brukes til både operativ og strategisk styring. Erfaringer fra Yara, Hydro og andre prosessaktører viser at kontinuerlig forbedring og vedlikehold er like sentralt som nye investeringer.

\subsection*{Prosessindustri og produksjon}
I ferrolegering, aluminium og batteriindustri ser vi digitale tvillinger som kombinerer sanntidsdata fra produksjonslinjene med simuleringer for å optimere varmebalanse, kjemiske prosesser og logistikk. Hydro Sunndal bruker for eksempel virtuelle modeller til å følge elektrolysecellene og planlegge vedlikehold før avvik oppstår. Ved å fylle ut casemalen kan du dokumentere hvordan datainnsamling fra sensorer, laboratorier og ERP-systemer kobles til modellene, og hvilke styringsparametre (for eksempel OEE og energibruk per enhet) som måles for å vise gevinst.

\subsection*{Energi og kraft}
Kraftbransjen kombinerer produksjonsdata, nettanalyser og markedssignaler for å sikre forsyningssikkerhet og bærekraft. Statnett utvikler digitale tvillinger av transmisjonsnettet for å planlegge vedlikehold og håndtere flaskehalser, mens Statkraft integrerer værdata og hydrologiske modeller for å optimalisere vannkraftmagasinene. Når du bruker malen på slike case, bør du beskrive hvordan datakilder som SCADA, værprognoser og regulatoriske krav påvirker modellene og beslutningsprosessene.

\subsection*{Maritim og offshore}
I maritim sektor kombinerer aktører som Equinor, Kongsberg Maritime og Ulstein simuleringer av skip og plattformer med sanntidsdata fra sensorer. Tvillingene brukes til å planlegge logistikk i Nordsjøen, forutse vedlikehold på subsea-utstyr og teste nye autonome funksjoner. Et godt utfylt case viser hvordan operasjonssentraler, kapteiner og tekniske team samarbeider rundt modellen, og hvordan tvillingen knyttes til sikkerhet og miljørapportering. Utforsk datakilder og planleggingsverktøy samlet i \autoref{sec:maritimressurser} for å beskrive logistikk, miljøforhold og regulatoriske krav i caseleveransene.

\paragraph{Autonome ferger i Trondheimsfjorden}
Pilotprosjektet rundt passasjerfergen \emph{Milliampere 2} demonstrerer hvordan autonome fartøy kan inngå i mobilitetsløsninger for norske byer og dokumenteres som en del av Trondheims satsing på nullutslippsmobilitet.\citep{ntnu2023milliampere2} FoU-programmet AutoFerry beskriver hvordan Zeabuz og forskningsmiljøene bygger tvillingen med modulære sensorer, simulatorer og styringssystemer som kan overføres til flere havner.\citep{sintef2024autoferry} Casebriefen i `support/notater/maritim-caseforberedelse.md` dokumenterer mål, datakilder og styringsstruktur og fungerer som startpunkt for å fylle casemalen. Oppgaven til studentene er å beskrive hvordan tvillingen binder sammen fartøyet, fjernoperatører og kommunens mobilitetsstyring, og hvilke regulatoriske rammer som gjelder for testtilatelser og beredskap.

\begin{itemize}
    \item Kartlegg sensorer og eksterne datakilder (MET-værdata, AIS-strømmer, havneinformasjon) og beskriv hvordan de strømmer inn i datasjøen og simulatorene.
    \item Forklar hvordan prosjektets governance fordeler ansvar mellom Zeabuz, Trondheim kommune og forskningspartnerne, og hvordan endringsrådet evaluerer nye autonome funksjoner.
    \item Definer KPI-er for punktlighet, energibruk og sikkerhetsrespons og knytt dem til læringsmålene i Kapittel~3 (dataflyt) og Kapittel~6 (tillit og etikk).
    \item Bruk vurderingsmatrisen i dette kapittelet til å sammenligne caset med andre maritime initiativer, og dokumenter funnene i fagfelleloggen slik at prioriteringen kan forankres.
\end{itemize}

Integrer gjerne en illustrasjon fra `support/illustrasjonsplan.md` som viser samspillet mellom fartøyet, fjernoperasjonssenteret og byens øvrige transportsystem. Dette gjør det lettere for lesere å forstå hvordan datastrømmene og beslutningssløyfene knyttes sammen i en urban, maritim kontekst.

\section{Prioriterte case for kommende fagfelleløp}
For å støtte fagfelleløpet prioriteres tre nye case som dekker norsk energiomlegging, grønn industri og landbruksmodernisering. Notatene i `support/notater/` utdyper datakilder, KPI-er og styringsmodeller slik at arbeidsgrupper kan forberede intervjuer og dokumentasjon.

\subsection*{Havvind – Sørlige Nordsjø II og Hywind Tampen}
Utbyggingen av flytende havvind gir et rikt læringsgrunnlag for digital tvilling. Statlige rammer for areal og nettilknytning kombineres med Equinors operasjonserfaring fra Hywind Tampen, der turbiner og kraftplattformer deler data for å balansere produksjon mot sokkelinstallasjoner.\citep{nve2023havvindfakta,equinor2023hywindtampen} Tvillingen må koordinere værprognoser, vedlikeholdslogistikk og energioppgjør i en delt dataspace mellom operatør, Statnett og tjenesteleverandører. Når studentgruppene fyller ut casemalen skal de beskrive hvordan produksjonsdata, SCADA-signaler og markedsinformasjon synkroniseres i hydrodynamiske modeller og energioptimering på tvers av plattformer og nettilknytning.

\begin{table}[h]
    \centering
    \caption{Nøkkelspor for casestudiet i havvind}
    \label{tab:havvind-spor}
    \begin{tabular}{p{3.5cm}p{8.5cm}}
        \toprule
        Fokusområde & Eksempler på spørsmål som skal besvares i casemalen \\
        \midrule
        Operasjonelle modeller & Hvordan oppdateres laster, vær og produksjonsprognoser i sanntid, og hvilke beslutningsdører (for eksempel daglig produksjonsmøte og ukentlig vedlikeholdsforum) bruker simuleringene? \\
        Energioppgjør og markeder & Hvilke datagrunnlag brukes for å beregne fleksibilitet mot sokkelinstallasjoner, og hvordan dokumenteres avregning og kostnadsdeling mellom aktører? \\
        Beredskap og HMS & Hvordan loggføres hendelser, barrierestyring og sikkerhetstiltak, og hvordan deles tilsynskrav med myndighetene? \\
        Dataspace-styring & Hvilke policy-artefakter fra Gaia-X eller IDSA brukes for å regulere tilgang til sensordata, og hvordan versjoneres dem for nye samarbeidspartnere? \\
        \bottomrule
    \end{tabular}
\end{table}

\begin{itemize}
    \item Kartlegg integrasjonspunktene mellom turbinsystemer, nettilknytning og sokkelinstallasjoner, og dokumenter hvordan datakvalitet sikres før modeller oppdateres.\citep{nve2023havvindfakta}
    \item Beskriv hvordan styringsstrukturen fordeler ansvar mellom operatør, myndigheter og leverandørkjede, og pek på beslutningsfora som må involveres før endringer godkjennes.
    \item Definer indikatorer for kapasitetsfaktor, HSE-observasjoner og tilgjengelighet per turbin, og lenk dem til gevinstplanen i Kapittel~7.
    \item Referer til ressurser i \autoref{appendix:ressurser} om energi- og offshoredata, og noter hvilke kilder som trengs for å fylle ut malen i \autoref{sec:case-maler-prioriterte-sektorer}.
\end{itemize}

\subsection*{Landbruk – Presisjonsjordbruk i Trøndelag}
Presisjonsjordbruket gir studentene innsikt i kombinasjonen av edge-prosessering og sentral analyse. Jord- og maskinsensorer prosesseres lokalt for å støtte autonome arbeidsoperasjoner, mens agronomiske beslutninger dokumenteres i felles datasett for rådgivere og forvaltning.\citep{nibio2023dataflyt,landbruksdir2023digitalisering,etsi2023mec} Caset viser hvordan data kan deles mellom gårdbruker, forskningsmiljø og kooperativ med klare policyer for gjenbruk av feltdata. Når casemalen fylles ut bør analysen omfatte agronomiske beslutningspunkter, maskinlogistikk og datasikkerhet slik at både operasjonell styring og bærekraftsmål blir transparente.

\begin{table}[h]
    \centering
    \caption{Datakilder og styringsaktiviteter i presisjonsjordbrukscaset}
    \label{tab:landbruk-datakilder}
    \begin{tabular}{p{3.5cm}p{8.5cm}}
        \toprule
        Datakategori & Eksempler og dokumentasjonskrav \\
        \midrule
        Sensor- og maskindata & Jordfuktighet, NDVI og maskintelemetri lagres lokalt (edge) før utsnitt lastes til datasjø; beskriv hvordan kalibrering og datakvalitet sikres i sesongen. \\
        Agronomiske beslutninger & Gjødselplan, vekstskifte og plantevern loggføres i felles rådgivningsportal; noter hvordan beslutninger forankres i samvirke eller rådgivningstjenester. \\
        Forvaltnings- og policykrav & Dokumenter hvordan datadeling møter krav fra Landbruksdirektoratet og personvernforordningen, inkludert samtykkeprosesser for deling av feltdata. \\
        Bærekraftsindikatorer & Forklar hvordan klimagassregnskap, jordhelse og vannforbruk beregnes, og hvilke indikatorer som rapporteres tilbake til finansierings- eller sertifiseringsordninger. \\
        \bottomrule
    \end{tabular}
\end{table}

\begin{itemize}
    \item Kartlegg beslutningsarenaer gjennom vekstsesongen (planlegging, utsåing, vekstoppfølging, innhøsting) og knytt databehov til hver hendelse.\citep{landbruksdir2023digitalisering}
    \item Beskriv hvordan edge-arkitekturen (for eksempel MEC-noder og traktorkontrollere) filtrerer data før de sendes til skyløsningen, og hvilke hendelser som krever manuell verifisering.\citep{etsi2023mec}
    \item Definer KPI-er for avlingsprognoser, ressursbruk og klimaeffekt, og koble dem til vurderingsmatrisen for modenhet i dette kapittelet.
    \item Sammenlign datastrømmer og verktøy med ressurskategoriene i \autoref{appendix:ressurser}, og bruk veiledningen i \autoref{sec:case-maler-prioriterte-sektorer} for å dokumentere læringspunkter til undervisning og fagfeller.
\end{itemize}

\subsection*{Batteriverdikjede – Giga Arctic i Mo i Rana}
Norsk batteriproduksjon kobler prosessindustri, energi og logistikk. FREYR Battery etablerer digitale tvillinger som følger råvarer, produksjonslinjer og energiforbruk for å oppfylle kravene i EUs batteriforordning og Enovas støtteordninger.\citep{freyr2024giga,enova2023batteri,gaiax2023architecture,eu2023batteryregulation} Dataspace-tilnærmingen gjør at leverandørkjeder kan dele karbondata og kvalitetsmålinger under strenge kontraktsregimer. Analyse av caset må vise hvordan tvillingen støtter rapportering mot bærekraftsindikatorer, samtidig som produksjonstakt og kvalitet optimaliseres i samspill med energisystemet i Mo Industripark.

\begin{table}[h]
    \centering
    \caption{Avklaringer som bør inngå i batterikaset}
    \label{tab:batteri-fokus}
    \begin{tabular}{p{3.6cm}p{8.4cm}}
        \toprule
        Tema & Beskrivelse av forventet dokumentasjon \\
        \midrule
        Material- og leverandørflyt & Spesifiser hvordan råvarer spores fra ankomst via produksjonslinjene til ferdige celler, og hvordan avvik håndteres i MES/ERP. \\
        Energi- og klimarapportering & Redegjør for koblingen til lokale energisystemer, fleksibilitetsavtaler og hvordan CO$_2$-intensitet rapporteres til kunder og myndigheter. \\
        Kvalitetslaboratorier & Beskriv hvordan laboratoriedata integreres i tvillingen for å justere prosesstyring og prediktivt vedlikehold. \\
        Dataspace og kontrakter & Dokumenter kontraktsmekanismer for deling av prosess- og karbondata, inkludert tilgangskontroll og revisjonspunkter. \\
        \bottomrule
    \end{tabular}
\end{table}

\begin{itemize}
    \item Kartlegg hvordan tvillingen understøtter regulatoriske rapporteringskrav i EUs batteriforordning og norske støtteprogrammer, og dokumenter hvilke indikatorer som må eksporteres fra plattformen.\citep{eu2023batteryregulation,enova2023batteri}
    \item Beskriv grensesnittet mellom produksjonsplanlegging, energistyring og logistikk, og angi hvor sanntidsdata må tilgjengeliggjøres for partnere i Mo Industripark.
    \item Vurder hvilke modeller (for eksempel energibalansesimulering, yield-analyse og kvalitetsprognoser) som bør prioriteres i casemalen for å gi undervisningsverdi.
    \item Sett gevinstmål opp mot indikatorene i vurderingsmatrisen og bruk malen i \autoref{sec:case-maler-prioriterte-sektorer} til å dokumentere læringsopplegg, risiko og neste steg.
\end{itemize}

\begin{table}[h]
    \centering
    \caption{Oversikt over prioriterte case og nøkkelindikatorer}
    \label{tab:prioriterte-case}
    \begin{tabular}{p{3.2cm}p{2.6cm}p{4.6cm}p{3.6cm}}
        \toprule
        Case & Sektor & Hovedindikatorer & Dataspace-fokus \\
        \midrule
        Sørlige Nordsjø II / Hywind Tampen & Havvind & Kapasitetsfaktor, planlagt vs. faktisk vedlikehold, HSE-observasjoner & Deling av produksjons- og værdata mellom operatør, TSO og offshore-partnere \\
        Presisjonsjordbruk i Trøndelag & Landbruk & Avlingsprognose, nitrogenutnyttelse, klimafotavtrykk per enhet & Samtykkebasert deling av feltdata og agronomiske anbefalinger mellom bønder og rådgivere \\
        Giga Arctic, Mo i Rana & Batteri & First-pass yield, energiforbruk per GWh, CO$_2$-intensitet per celle & Kontraktsstyrt tilgang til prosess-, logistikk- og bærekraftsdata i industrielt dataspace \\
        \bottomrule
    \end{tabular}
\end{table}

\subsection*{Felles leveranseplan for klimacaser}
Nasjonale satsinger som Grønt industriløft og de fremvoksende norske dataspace-pilotene viser at sektorenes klimaprosjekter må planlegges i fellesskap for å utløse investeringer og sikre datadeling på tvers av bransjer.\citep{regjeringen2023grontindustriloft,digitalnorway2024dataspace} For å gi fagfellegruppene en tydelig arbeidsflyt etableres en leveranseplan der havvind-, landbruks- og battericase har synkroniserte milepæler.

\begin{table}[h]
    \centering
    \caption{Leveranseplan for klimacase-workshop}
    \label{tab:klimacase-leveranser}
    \begin{tabular}{p{3.0cm}p{3.2cm}p{3.2cm}p{3.2cm}}
        \toprule
        Milepæl & Havvind & Landbruk & Batteri \\n        \midrule
        Forberedelse (uke $-2$) & Oppdater produksjons- og værdatasett, bekreft tilgang til Statnett og OED-notater. & Samle sesongplaner fra rådgivningstjenesten og jordprøver til datakvalitetssjekk. & Synkroniser prosesslogger og energidata med leverandørportalen i Mo Industripark. \\n        Workshopdag (uke 0) & Valider hydrodynamiske scenarier og beredskapsrutiner med fagfeller. & Test edge-arbeidsflyt og agronomisk beslutningslogg i sanntidsøkt. & Evaluer produksjonsdashbord mot krav i batteriforordningen og lokale fleksibilitetsavtaler. \\n        Oppfølging (uke $+1$) & Prioriter tiltak for dataspace-policy og publiser læringspunkter i fagfelleloggen. & Revider samtykkeprosesser og oppdater KPI-er for ressursbruk og klima. & Fastsett forbedringer i kvalitetslaboratorier og energisamarbeid før neste sprint. \\n        Skaleringsbeslutning (uke $+4$) & Beslutt om caset skal brukes i pilotundervisningen og hvilke modeller som gjenbrukes. & Bekreft hvilke gårder som inngår i videre utrulling og hvilke datasett som åpnes. & Avklar eksport av innsikt til verdikjeden og rapportering til Grønt industriløft. \\n        \bottomrule
    \end{tabular}
\end{table}

Bruk leveranseplanen i \autoref{tab:klimacase-leveranser} som agenda når fagfeller og studentgrupper møtes. Hver milepæl bør kobles til konkrete dokumenter i prosjektets samhandlingsrom (for eksempel casemaler, gevinstplaner og risikoanalyser) og til de ressursene som er listet i \autoref{appendix:ressurser}. Når planene følges, blir det enklere å gjenbruke beslutningsgrunnlag og sikre at datastrømmene er forankret i nasjonale politiske prioriteringer.

\begin{enumerate}
    \item Forankre leveranseplanen hos ansvarlige for kapittel 3 og 7 slik at dataspace-policy og gevinstoppfølging henger sammen i hele boken.
    \item Oppdater `support/notater/pilotundervisning-materiell.md` etter hver milepæl med status, kontaktpunkter og eventuelle avhengigheter mot illustrasjoner eller laboratorieoppsett.
    \item Del erfaringer i fagfelleloggen slik at neste revisjon av leveranseplanen kan bygge på konkrete innsikter fra norsk industri og offentlig sektor.
\end{enumerate}

Casebeskrivelsene bygger videre på casemalen i dette kapittelet og gevinstplanen i Kapittel~7. Sektorgruppene bør fylle ut notatmalene med intervjuer, datatilgang og planlagte figurer før fagfellemøtene.

\subsection*{Dataspace-arkitektur for sektorkase}
Casene over illustrerer hvordan dataspace-prinsipper kan gjøre deling tryggere og mer fleksibel. Start med å etablere dataprodukter (for eksempel værvinduer, gjødselplaner eller energirapporter) med tydelig lisens. Bruk standardiserte policy-rammeverk fra IDSA og Gaia-X for å modellere tilgang og forretningsregler, og koble dem til KPI-oppfølgingen.\citep{idsa2023ram,gaiax2023architecture} Resultatet dokumenteres i prosjektets gevinstplan og RACI-matriser, slik at undervisningscasene får en konsistent styringsstruktur.

\section{Bygg, transport og helse}
Offentlige og private aktører bruker digitale tvillinger til å koordinere investeringer og tjenester som påvirker hele samfunnet. Felles utfordringer er å forankre tvillingen i eksisterende forvaltningssystemer, sikre datadeling og bygge kompetanse på tvers av fagmiljøer. Casemalen hjelper deg med å tydeliggjøre roller og datakilder i prosjekter der mange interessenter må samarbeide.

\subsection*{Smarte bygg og byer}
Statsbygg, Oslo kommune og flere norske byer har tatt i bruk digitale modeller som kombinerer BIM-data med energi- og sensorinformasjon for å styre drift og vedlikehold. Ved å beskrive scope, datakilder og governance i casemalen kan du vise hvordan driftsteam, leietakere og tjenesteleverandører bruker tvillingen til å redusere energibruk, planlegge renhold og forbedre inneklima.

\subsection*{Transport og logistikk}
Bane NOR utvikler digitale tvillinger av jernbanenettet for å planlegge vedlikehold, simulere trafikkavvik og informere kundeservice. Avinor tester tilsvarende modeller på flyplasser for å koordinere bakketjenester. Når du analyserer slike case, må du kartlegge hvordan sanntidsdata fra sensorer, IoT-enheter og trafikkstyringssystemer flyter inn i tvillingen, og hvordan det støtter planlegging og krisehåndtering. Utforsk datakilder som Entur, Bane NOR og Avinor i \autoref{sec:transportressurser} for å sikre at caserapporter støttes av konkrete tidsserier, kapasitetsdata og hendelseslogger.

\subsection*{Kommunal klimaresiliens og beredskap}
Oslo VAV har etablert en digital tvilling av vann- og avløpsnettet som kombinerer hydrologiske modeller, sensordata og sanntidsvarsling for å planlegge tiltak mot styrtregn og overvann.\citep{oslovav2023digital} Tvillingen brukes sammen med kommunens overvannsstrategi for å teste konsekvenser av utbygginger, dimensjonere blågrønne tiltak og prioritere investeringer i kulverter og magasiner.\citep{oslo2023overvann} Ved å dokumentere caset kan studentene vise hvordan data fra værstasjoner, trykksensorer og manuelle inspeksjoner gir grunnlag for beslutninger i beredskapsstaber og tekniske etater.

\begin{table}[h]
    \centering
    \caption{Arbeidsflyt for beredskap mot styrtregn i kommunal digital tvilling}
    \label{tab:kommunal-beredskap}
    \begin{tabular}{p{3.4cm}p{4.5cm}p{5.0cm}}
        \toprule
        Fase & Kritiske datakilder & Typiske tiltak og beslutninger \\
        \midrule
        Forberedende analyser & Regnhendelsesstatistikk, kapasitetsmålinger i ledningsnett, planregister & Prioriterer fordrøyningsbasseng og grønne løsninger, oppdaterer investeringsplaner og byggesaker. \\
        Operativ beredskap & Sanntidsnivå i kummer, værvarsler, sensordata fra overløp & Aktiverer beredskapstropp, koordinerer innsats mellom vann- og avløpsvakt og bymiljøetat, varsler utsatte institusjoner. \\
        Etterarbeid og læring & Skadejournal, hendelsesrapport, øvelseslogg & Evaluerer tiltak i tråd med DSBs øvelsesveileder, oppdaterer tiltaksbank og deler læringspunkter i fagfelleloggen.\citep{dsb2024nser,dsb2023ovelser} \\
        \bottomrule
    \end{tabular}
\end{table}

Når casemalen fylles ut, bør du beskrive hvordan den kommunale tvillingen kobles til dataspace-prinsippene i kapittel~3 og valideringsrutiner i kapittel~6. Det innebærer å dokumentere hvilke dataprodukter som deles med nabokommuner eller statlige myndigheter, og hvordan tilgang styres under hendelser. Bruk tabellen i \autoref{tab:kommunal-beredskap} for å konkretisere hvem som tar hvilke beslutninger og hvordan indikatorer for responstid, overløpsutslipp og samfunnskritiske tjenester følges opp.

\begin{itemize}
    \item Kartlegg hvordan overvannstiltak prioriteres gjennom kommunens porteføljestyring, og lenk til gevinstplanene i kapittel~7 for å vise hvordan investeringsbeslutninger forankres politisk.
    \item Beskriv hvordan beredskapsøvelser og scenarioarbeid dokumenteres i kvalitetsjournaler, og referer til metodikken i kapittel~6 for å sikre sporbarhet i etterlevelse.
    \item Presenter forslag til undervisningsaktiviteter der studentgrupper simulerer styrtregnscenarier og foreslår kombinasjoner av tekniske og naturbaserte tiltak basert på tvillingens data.
\end{itemize}

\subsection*{Helse}
Helse Vest og Universitetssykehuset i Oslo eksperimenterer med pasientspesifikke tvillinger for kreftbehandling, kirurgisk planlegging og rehabilitering. Malen gir struktur for å beskrive hvordan kliniske data, bildediagnostikk og simuleringer integreres, hvilke etiske vurderinger som gjøres, og hvordan pasienter og fagpersonell involveres i beslutninger.

\section{Metodikk for caseanalyse}
For å sammenligne case på tvers av sektorer, start med å beskrive hvorfor tvillingen ble etablert, hvilke forutsetninger som må være til stede og hvordan resultatene måles. Det kan være nyttig å kombinere casemalen med intervjuer, offentlige rapporter og tekniske arkitekturdokumenter. Et helhetlig rammeverk bør dekke mål, omfang, teknologivalg, organisering og dokumenterte effekter.

\begin{itemize}
    \item Kartlegg datakilder systematisk, fra sanntidssensorer til historiske arkiv og åpne data, og vurder kvalitet, eierskap og tilgang.
    \item Noter hvilke styringsstrukturer, insentiver og kompetanseprogram som påvirker gjennomføring og skalering.
    \item Reflekter over etiske og samfunnsmessige hensyn, inkludert personvern, arbeidstakerperspektiv og miljøpåvirkning.
\end{itemize}

\subsection{Prioritering for pilotundervisningen}
Fagfelleinnspillene fra Statnett, Equinor og Helse Vest anbefalte at vi velger et begrenset sett case til pilotundervisningen slik at studentene kan fordype seg i datastrømmer og styringsprosesser. Prioriteringen nedenfor bygger på vurderingsmatrisen og de datadelingstiltakene som er tilgjengelige gjennom Gaia-X og norske dataspace-initiativer.\citep{gaiax2022architecture} Oversikten dokumenteres også i `support/notater/pilotundervisning-materiell.md` slik at mentorene har oppdaterte kontaktpunkter og lisenskrav.

\begin{table}[htbp]
    \centering
    \begin{tabular}{p{3.2cm}p{2.2cm}p{4.5cm}p{4.5cm}}
        \toprule
        \textbf{Case} & \textbf{Prioritet} & \textbf{Datastrømmer} & \textbf{Tiltak fra fagfelleløpet} \\
        \midrule
        Autonome ferger (Trondheim) & Høy & AIS, værdata, operativ logg, sikkerhetsrapporter & Datasett deles via kommunens dataspace; intervjuguide for fjernoperasjon oppdatert med sikkerhetsindikatorer. \\
        Kraftnett i Nord-Norge & Høy & SCADA, flaskehalsmodellering, energimarked & Avklarte tilgang til Statnetts sandkasse; tiltak for beredskap og NIS2-rapportering dokumentert. \\
        Batteriproduksjon Mo i Rana & Middels & MES, energi- og logistikkdata, kvalitetsmålinger & Datasamarbeid med industriparkens datasjø; plan for syntetiske datasett der produksjonsdata er sensitive. \\
        Virtuell pasientflyt (Helse Vest) & Middels & Kliniske registre, ressursplanlegging, ventelistestatistikk & Tidsluker for pseudonymisering av data avtalt; etikk- og samtykkestrømmer koblet til Kapittel~6. \\
        Smarte bygg (Oslo kommune) & Utforsk & Energi- og inneklimadata, arbeidsordre & Pilot holdes i reserve; krever videre avklaringer om deling via kommunal dataspace. \\
        \bottomrule
    \end{tabular}
    \caption{Prioritert caseliste for pilotundervisningen basert på fagfelleinnspill og tilgjengelige datastrømmer.}
\end{table}

\subsection*{Vurderingsmatrise for modenhet og prioritering}
Når flere potensielle case konkurrerer om oppmerksomhet eller undervisningsressurser, kan en vurderingsmatrise gi et felles
beslutningsgrunnlag. Tabellen under brukes til å skåre hvert case fra~1 (lav modenhet) til~5 (høy modenhet) på fem dimensjoner.
Noter begrunnelsen for hver skår og lenk til dokumentasjon, slik at fagfeller kan etterprøve vurderingen.

\begin{longtable}{p{0.23\textwidth}p{0.26\textwidth}p{0.45\textwidth}}
\toprule
\textbf{Kriterium} & \textbf{Hva vurderes?} & \textbf{Eksempel på skåringsskala (1--5)} \\
\midrule
\endfirsthead
\toprule
\textbf{Kriterium} & \textbf{Hva vurderes?} & \textbf{Eksempel på skåringsskala (1--5)} \\
\midrule
\endhead
Datagrunnlag og tilgang & Tilgjengelighet, kvalitet og rettigheter til data som trengs for tvillingen. & 1~= fragmenterte kilder uten tilgangsavklaringer. 3~= etablerte datastrømmer, men mangler kvalitetssikring. 5~= sanntidsdata med avtalte API-er, eierskap og datakvalitetsrutiner. \\
\addlinespace
Organisatorisk forankring & Hvor godt case er støttet av ledelse, fagmiljø og brukere. & 1~= initiativ fra enkeltperson uten mandat. 3~= prosjektgruppe med midlertidig finansiering. 5~= tverrfaglig styringsmodell med dedikerte roller og gevinstansvar. \\
\addlinespace
Teknologisk modenhet & Modenhet på modeller, integrasjonsplattform og driftsoppsett. & 1~= konseptskisse uten implementert plattform. 3~= pilot i begrenset produksjon. 5~= robust løsning med DevOps-/MLOps-sløyfer og overvåkning i drift. \\
\addlinespace
Gevinst- og læringspotensial & Forventet effekt for virksomheten og læringsverdi for studenter. & 1~= uklare mål og lite overføringsverdi. 3~= tydelige KPI-er, men begrenset dokumentasjon. 5~= målbare gevinster, åpne data/artefakter og tydelig kobling til læringsmål. \\
\addlinespace
Risiko og etterlevelse & Regulatoriske, sikkerhets- og etiske forutsetninger. & 1~= ukjente krav eller høy personvernrisiko. 3~= identifiserte tiltak, men ufullstendig dokumentasjon. 5~= komplette risikovurderinger, etterlevelsesplan og jevnlig revisjon. \\
\bottomrule
\end{longtable}

Etter at tabellen er fylt ut, summeres skårene for et samlet modenhetsnivå, men noter også hvordan svakheter skal håndteres.
Følg denne arbeidsflyten for å forankre prioriteringen:

\begin{enumerate}
    \item Sett sammen et tverrfaglig vurderingsteam (for eksempel produktansvarlig, dataarkitekt og domeneekspert) og avklar
    hva slags beslutning som skal tas.
    \item Innhent nødvendig dokumentasjon: prosjektmandat, tekniske beskrivelser, datalister og gevinstrealiseringsplaner. Bruk
    ressursoversikten i \autoref{appendix:ressurser} for å finne støttedata eller verktøy som mangler.
    \item Skår casene i fellesskap, dokumenter begrunnelser og registrer resultatet i prosjektets styrings- eller fagfellelogg.
    \item Definer tiltak for kriterier som får skår 1--2 (for eksempel avklare tilgang, etablere governance eller komplettere
    risikovurdering) og gi tydelig anbefaling om case skal prioriteres, videreutvikles eller settes på vent.
\end{enumerate}

Ved å dele vurderingsmatrisen med studenter og eksterne samarbeidspartnere blir det lettere å diskutere hvilke case som gir mest
verdi, og hvilke forberedelser som kreves før data kan deles eller modeller kobles på.

\section{Casestudie-maler for prioriterte sektorer}
\label{sec:case-maler-prioriterte-sektorer}
Denne seksjonen gir tre casemaler som speiler prioriteringene i \S~«Prioriterte case for kommende fagfelleløp». Hver mal beskriver hvilke spørsmål som skal besvares, hvilke datakilder som må dokumenteres og hvordan gevinstplanen fra Kapittel~7 kan kobles på.

\subsection{Havvind – Sørlige Nordsjø II og Hywind Tampen}
\paragraph{Sammendrag av caset}
\begin{tabular}{p{0.34\textwidth}p{0.60\textwidth}}
\textbf{Element} & \textbf{Beskrivelse / spørsmål som skal besvares} \\
Operatør og lokasjon & Hvilke selskaper deltar i tvillingen, og hvordan er havområdet delt mellom aktørene? \\
Strategiske mål & Hvordan balanseres energi til sokkelinstallasjoner mot eksport til fastlandet, og hvilke klimamål skal støttes? \\
Omfang & Hvilke anlegg og tjenester inngår (turbiner, kabler, substasjoner, logistikkfartøy)? \\
Utgangspunkt & Hvilke modeller, datakilder og driftserfaringer fantes før caset ble utvidet til en helhetlig tvilling? \\
\end{tabular}

\paragraph{Forretningsmål og interessenter}
\begin{itemize}
    \item Beskriv hvordan operatør, lisenspartnere, Statnett og leverandører bruker tvillingen til å koordinere produksjon, fleksibilitet og kraftpriser.\citep{nve2023havvindfakta}
    \item Identifiser relevante myndigheter (OED, NVE, Petroleumstilsynet) og hvilke krav til miljørapportering og HMS som tvillingen må støtte.
    \item Definer KPI-er for kapasitetsfaktor, vedlikeholdstid, CO$_2$-reduksjon og leveringssikkerhet.
\end{itemize}

\paragraph{Datagrunnlag og integrasjon}
\begin{itemize}
    \item Kartlegg SCADA-data, metocean-modeller, inspeksjonsrapporter og markedsdata, og forklar hvordan de samles i dataspace og kontrollrom.\citep{gaiax2023architecture}
    \item Dokumenter grensesnitt mot sokkelinstallasjoner, hydrodynamiske simuleringer og kraftmarkedsplattformer.
    \item Vurder hvordan datastrømmer kvalitetssikres før beslutninger fattes (for eksempel validering mot værbojer og inspeksjonsdroner).
\end{itemize}

\paragraph{Modelleringsstrategi}
\begin{itemize}
    \item Beskriv bruk av vær- og produksjonsprognoser, lastflytanalyse og strukturelle modeller av turbin og fundament.
    \item Forklar hvordan modellene kalibreres, og hvilke scenarioer som simuleres (storm, nettilkobling, vedlikeholdsvinduer).
    \item Angi hvordan beslutninger loggføres slik at modellforbedringer kan spores.
\end{itemize}

\paragraph{Implementering og arbeidsprosesser}
\begin{itemize}
    \item Forklar hvordan tvillingen støtter daglige operasjoner (for eksempel operations centre dashboards, vedlikeholdsplanlegging og energioppgjør).
    \item Beskriv opplæring for driftsteam, leverandørkjede og myndighetsdialog.
    \item Noter hvordan policy-artefakter og tilgangsstyring vedlikeholdes i dataspace-portalen.\citep{gaiax2023architecture}
\end{itemize}

\paragraph{Resultater, gevinster og gevinstrealisering}
\begin{itemize}
    \item Dokumenter gevinstmål for energiproduksjon, tidsbesparelse i vedlikehold og redusert nedetid.
    \item Beskriv hvordan gevinster verifiseres gjennom produksjons- og HMS-rapporter.
    \item Vurder sekundære effekter, som bedre sameksistens med fiskeri og skipsfart eller raskere konsesjonsoppfølging.
\end{itemize}

\paragraph{Overføringsverdi og neste steg}
\begin{itemize}
    \item Drøft hvordan modeller og datainnsikt kan gjenbrukes i nye felt (for eksempel Utsira Nord) eller andre offshoreprosjekter.
    \item Foreslå tiltak for å koble tvillingen til handelsplattformer og beredskapsøvelser.
    \item Identifiser beslutningspunkter for skalering til internasjonale partnere og testing av nye turbintyper.
\end{itemize}

\subsection{Landbruk – Presisjonsjordbruk i Trøndelag}
\paragraph{Sammendrag av caset}
\begin{tabular}{p{0.34\textwidth}p{0.60\textwidth}}
\textbf{Element} & \textbf{Beskrivelse / spørsmål som skal besvares} \\
Gård og produksjon & Hvilke produksjoner inngår (korn, gras, potet) og hvor ligger gården? \\
Strategiske mål & Skal tvillingen optimalisere avling, ressursbruk, klima eller dyrevelferd? \\
Omfang & Hvilke maskiner, sensorer og agronomiske beslutninger er inkludert? \\
Utgangspunkt & Hvilke digitale verktøy og datakilder fantes før presisjonsprogrammet startet? \\
\end{tabular}

\paragraph{Forretningsmål og interessenter}
\begin{itemize}
    \item Beskriv målene til bonden, rådgivningstjenesten, samvirket og eventuelle forskningspartnere.\citep{landbruksdir2023digitalisering}
    \item Avklar hvordan finansieringsordninger og sertifiseringer påvirker prioriteringer.
    \item Definer KPI-er for avlingsprognoser, inputfaktorbruk og klimafotavtrykk.
\end{itemize}

\paragraph{Datagrunnlag og integrasjon}
\begin{itemize}
    \item Kartlegg sensorer (jord, vær, maskinlogg), datasiloer (FMIS, ERP) og eksterne kilder (satellitt, rådgiving).\citep{nibio2023dataflyt}
    \item Beskriv edge-arkitekturen: hvilke data behandles lokalt, hvilke sendes til sky og når kreves offline-funksjonalitet?\citep{etsi2023mec}
    \item Dokumenter samtykke- og delingsprosesser for data mot rådgivere, leverandører og myndigheter.
\end{itemize}

\paragraph{Modelleringsstrategi}
\begin{itemize}
    \item Forklar hvilke modeller som kombineres (vekstmodeller, maskinlogistikk, agronomiske beslutningsstøtter).
    \item Angi hvordan modeller kalibreres mot jordprøver, avlingsdata og forsøksfelt.
    \item Beskriv hvordan usikkerhet og værvariasjon håndteres i anbefalinger.
\end{itemize}

\paragraph{Implementering og arbeidsprosesser}
\begin{itemize}
    \item Dokumenter hvordan tvillingen brukes i vekstsesongens beslutningspunkter og hvem som godkjenner tiltak.
    \item Beskriv opplæring og støtte for driftslag, inkludert digitale arbeidsflyter for maskinoperatører.
    \item Noter hvordan datakvalitet og datasikkerhet følges opp i samarbeidet med leverandører.
\end{itemize}

\paragraph{Resultater, gevinster og gevinstrealisering}
\begin{itemize}
    \item Oppgi målbare effekter for avling, innsatsfaktorer, klima og jordhelse.
    \item Beskriv hvordan resultatene dokumenteres (for eksempel i rådgivningsportal eller bærekraftsregnskap).
    \item Vurder sekundære gevinster som kompetanseheving og datadeling mot agronomi- og innovasjonsprosjekter.
\end{itemize}

\paragraph{Overføringsverdi og neste steg}
\begin{itemize}
    \item Identifiser hvordan løsningene kan skaleres til andre regioner, klima eller produksjoner.
    \item Foreslå tiltak for å koble tvillingen til forsyningskjeder, finansiering og sensorinnovasjon.
    \item Beskriv indikatorer som viser når tvillingen er klar for å inngå i landsdekkende samarbeidsprosjekter.
\end{itemize}

\subsection{Batteriverdikjede – Giga Arctic i Mo i Rana}
\paragraph{Sammendrag av caset}
\begin{tabular}{p{0.34\textwidth}p{0.60\textwidth}}
\textbf{Element} & \textbf{Beskrivelse / spørsmål som skal besvares} \\
Virksomhet og lokasjon & Hvilke anlegg og partnere inngår i Mo Industripark? \\
Strategisk hovedmål & Hvordan skal tvillingen støtte oppskalering, kvalitet og bærekraft? \\
Omfang & Hvilke produksjonslinjer, laboratorier og logistikkprosesser er inkludert? \\
Utgangspunkt & Hvilke systemer og datakilder fantes før Giga Arctic-programmet ble igangsatt? \\
\end{tabular}

\paragraph{Forretningsmål og interessenter}
\begin{itemize}
    \item Beskriv målbildet til FREYR Battery, energiselskap, logistikkpartnere og offentlige virkemidler.\citep{freyr2024giga,enova2023batteri}
    \item Angi hvordan kundenes krav til karbonavtrykk og kvalitet styrer prioriteringer.
    \item Definer KPI-er for produksjonskapasitet, first-pass yield, energibruk og CO$_2$-intensitet.
\end{itemize}

\paragraph{Datagrunnlag og integrasjon}
\begin{itemize}
    \item Kartlegg datakilder i råvarelogistikk, produksjonslinjer, laboratorier, energisystem og forsyningskjede.
    \item Beskriv integrasjoner mellom OT-systemer (OPC UA, PLC) og datasjø/analyseløsninger, samt hvordan data deles i dataspace.\citep{gaiax2023architecture}
    \item Dokumenter krav fra EUs batteriforordning og norske støtteprogrammer til rapportering og sporbarhet.\citep{eu2023batteryregulation,enova2023batteri}
\end{itemize}

\paragraph{Modelleringsstrategi}
\begin{itemize}
    \item Forklar hvordan prosessimulering, energibalansesimulering og kvalitetsprognoser kombineres.
    \item Beskriv kalibrering mot laboratoriedata og produksjonsstatistikk, samt håndtering av usikkerhet i nye produksjonsserier.
    \item Vurder behov for AI-modeller i produksjonsoptimalisering og vedlikehold.
\end{itemize}

\paragraph{Implementering og arbeidsprosesser}
\begin{itemize}
    \item Dokumenter hvordan tvillingen brukes i daglig drift, kvalitetsstyring og samarbeid med energipartnere.
    \item Beskriv opplæringsprogrammer for operatører, dataanalytikere og leverandører.
    \item Forklar hvordan versjonsstyring, datasikkerhet og revisjoner gjennomføres i dataspace og styringsmodeller.
\end{itemize}

\paragraph{Resultater, gevinster og gevinstrealisering}
\begin{itemize}
    \item Oppgi gevinster for energioptimalisering, yield, vedlikehold og bærekraftsrapportering.
    \item Beskriv hvordan gevinster dokumenteres gjennom interne dashboards og rapportering til myndigheter og kunder.
    \item Noter sekundære effekter som kompetanseheving, industrisamarbeid og nye forretningsmodeller.
\end{itemize}

\paragraph{Overføringsverdi og neste steg}
\begin{itemize}
    \item Vurder hvordan løsningen kan gjenbrukes i andre batteriprosjekter eller prosessindustrier.
    \item Definer tiltak for å koble tvillingen til europeiske dataspace-initiativer og leverandørkjeder.
    \item Foreslå indikatorer for når tvillingen bør utvides med nye produktlinjer eller geografiske lokasjoner.
\end{itemize}

\subsection*{Felles sjekkliste for caserapporten}
\begin{itemize}
    \item[$\square$] Er mål, interessenter og scope tydelig dokumentert med kobling til gevinstplanen?
    \item[$\square$] Er datakilder, integrasjoner og dataspace-policyer beskrevet med ansvar og kvalitetstiltak?
    \item[$\square$] Er modelleringsvalg, kalibrering og scenarioer forklart med tilhørende begrensninger?
    \item[$\square$] Er arbeidsprosesser, opplæring og governance definert med tydelige roller?
    \item[$\square$] Er gevinster og målemetoder forankret i KPI-er og vurderingsmatrisen?
    \item[$\square$] Er læringspunkter og neste steg formulert slik at fagfeller kan følge opp?
\end{itemize}

\subsection*{Eksempler på norske casetilpasninger}
\paragraph{Digital tvilling av kraftnett i Nord-Norge}
Statnett og regionale nettselskaper samarbeider om å modellere flaskehalser i kraftnettet mellom Ofoten og Finnmark. Når en av malene i denne seksjonen brukes, bør du beskrive hvordan værdata, planlagte industrietableringer og driftsmønstre fra vann- og vindkraft integreres, og hvordan tvillingen støtter dialogen med NVE og lokalsamfunn.

\paragraph{Jernbanedrift og vedlikehold hos Bane NOR}
En digital tvilling av jernbanestrekningen mellom Oslo og Lillehammer kombinerer BIM-modeller, sensordata fra spor og sanntidssignaler fra togene. Malstrukturen hjelper deg å dokumentere hvordan operasjonssentralen, entreprenører og leverandører samarbeider om vedlikeholdsvinduer og hvordan simuleringer brukes til å planlegge nye ruteoppsett.

\paragraph{Virtuell pasientflyt i spesialisthelsetjenesten}
Helse Sør-Øst prøver ut digitale tvillinger for å koordinere pasientflyt i kreftbehandling. Malen tydeliggjør hvilke kliniske datasett som inngår, hvordan algoritmer foreslår behandlingsplaner og hvordan pasienter får informasjon. Etisk vurdering, datasikkerhet og samtykkeprosesser må dokumenteres grundig.

\section{Refleksjonsspørsmål og øvinger}
\begin{enumerate}
    \item \textbf{Analyse av eksisterende case:} Velg ett av caseeksemplene ovenfor eller et eget norsk prosjekt. Bruk casemalene i \autoref{sec:case-maler-prioriterte-sektorer} til å skrive en kort rapport (2--3 sider) der du vurderer modenhet, måloppnåelse og hvilke ytterligere datakilder som kan styrke tvillingen.
    \item \textbf{Planlegging av nytt initiativ:} Sett sammen et tverrfaglig team (for eksempel prosessingeniør, data scientist og driftsleder) og lag et arbeidsnotat som beskriver hvordan dere vil samle data, utvikle modeller og organisere forvaltning for et nytt case i deres sektor ved å følge strukturen i \autoref{sec:case-maler-prioriterte-sektorer}.
    \item \textbf{Tverrsektoriell overføring:} Sammenlign to sektorer, for eksempel kraft og bygg. Lag en tabell som viser hvilke komponenter i casemalene som er direkte overførbare, hvilke som må tilpasses, og foreslå tiltak for å gjenbruke modeller eller datagrunnlag.
    \item \textbf{Refleksjon over etikk og verdiskaping:} Diskuter i grupper hvordan personvern, arbeidsmiljø og bærekraft bør ivaretas når tvillinger skaleres. Bruk notater fra de tre foregående oppgavene og oppdater relevante casemaler med konkrete risikoreduserende tiltak.
\end{enumerate}
