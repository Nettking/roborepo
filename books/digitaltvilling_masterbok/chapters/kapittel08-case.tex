\chapter{Bruksområder og norske case}
% Oppdatering: Utdypet sektorseksjoner, knyttet casemalen til flere norske eksempler og la til nye øvingsoppgaver.
% Oppdatert 2024-08-21: Utdypet maritim sektorseksjon med casebrief for autonome ferger i Trondheimsfjorden.
% Oppdatert 2025-10-23: Lagt til henvisning til nye transportressurser i appendiks.
% Oppdatert 2025-10-24: Refererte til maritim ressursseksjon og nytt arbeidsnotat for caseprioritering.

\section{Læringsmål}
\begin{itemize}
    \item Analysere variasjon i anvendelser av digitale tvillinger på tvers av sektorer.
    \item Kritisk evaluere norske case med hensyn til teknologi, organisasjon og gevinster.
    \item Overføre læring fra case til egne prosjekter.
\end{itemize}

\section{Industrisektoren}
Digitale tvillinger i norsk industri kobler ofte sammen komplekse produksjonslinjer, leverandørkjeder og energisystemer. Når du analyserer casene, bør du identifisere hvilke beslutningssituasjoner som styrer investeringen, og hvordan innsikten fra tvillingen brukes til både operativ og strategisk styring. Erfaringer fra Yara, Hydro og andre prosessaktører viser at kontinuerlig forbedring og vedlikehold er like sentralt som nye investeringer.

\subsection*{Prosessindustri og produksjon}
I ferrolegering, aluminium og batteriindustri ser vi digitale tvillinger som kombinerer sanntidsdata fra produksjonslinjene med simuleringer for å optimere varmebalanse, kjemiske prosesser og logistikk. Hydro Sunndal bruker for eksempel virtuelle modeller til å følge elektrolysecellene og planlegge vedlikehold før avvik oppstår. Ved å fylle ut casemalen kan du dokumentere hvordan datainnsamling fra sensorer, laboratorier og ERP-systemer kobles til modellene, og hvilke styringsparametre (for eksempel OEE og energibruk per enhet) som måles for å vise gevinst.

\subsection*{Energi og kraft}
Kraftbransjen kombinerer produksjonsdata, nettanalyser og markedssignaler for å sikre forsyningssikkerhet og bærekraft. Statnett utvikler digitale tvillinger av transmisjonsnettet for å planlegge vedlikehold og håndtere flaskehalser, mens Statkraft integrerer værdata og hydrologiske modeller for å optimalisere vannkraftmagasinene. Når du bruker malen på slike case, bør du beskrive hvordan datakilder som SCADA, værprognoser og regulatoriske krav påvirker modellene og beslutningsprosessene.

\subsection*{Maritim og offshore}
% Notat: Se `support/notater/maritim-caseforberedelse.md` for datagrunnlag og prioritering (2025-10-24).
I maritim sektor kombinerer aktører som Equinor, Kongsberg Maritime og Ulstein simuleringer av skip og plattformer med sanntidsdata fra sensorer. Tvillingene brukes til å planlegge logistikk i Nordsjøen, forutse vedlikehold på subsea-utstyr og teste nye autonome funksjoner. Et godt utfylt case viser hvordan operasjonssentraler, kapteiner og tekniske team samarbeider rundt modellen, og hvordan tvillingen knyttes til sikkerhet og miljørapportering. Utforsk datakilder og planleggingsverktøy samlet i \autoref{sec:maritimressurser} for å beskrive logistikk, miljøforhold og regulatoriske krav i caseleveransene.

\paragraph{Autonome ferger i Trondheimsfjorden}
Pilotprosjektet rundt passasjerfergen \emph{Milliampere 2} demonstrerer hvordan autonome fartøy kan inngå i mobilitetsløsninger for norske byer. Casebriefen i `support/notater/maritim-caseforberedelse.md` dokumenterer mål, datakilder og styringsstruktur og fungerer som startpunkt for å fylle casemalen. Oppgaven til studentene er å beskrive hvordan tvillingen binder sammen fartøyet, fjernoperatører og kommunens mobilitetsstyring, og hvilke regulatoriske rammer som gjelder for testtilatelser og beredskap.

\begin{itemize}
    \item Kartlegg sensorer og eksterne datakilder (MET-værdata, AIS-strømmer, havneinformasjon) og beskriv hvordan de strømmer inn i datasjøen og simulatorene.
    \item Forklar hvordan prosjektets governance fordeler ansvar mellom Zeabuz, Trondheim kommune og forskningspartnerne, og hvordan endringsrådet evaluerer nye autonome funksjoner.
    \item Definer KPI-er for punktlighet, energibruk og sikkerhetsrespons og knytt dem til læringsmålene i Kapittel~3 (dataflyt) og Kapittel~6 (tillit og etikk).
    \item Bruk vurderingsmatrisen i dette kapittelet til å sammenligne caset med andre maritime initiativer, og dokumenter funnene i fagfelleloggen slik at prioriteringen kan forankres.
\end{itemize}

Integrer gjerne en illustrasjon fra `support/illustrasjonsplan.md` som viser samspillet mellom fartøyet, fjernoperasjonssenteret og byens øvrige transportsystem. Dette gjør det lettere for lesere å forstå hvordan datastrømmene og beslutningssløyfene knyttes sammen i en urban, maritim kontekst.

\section{Bygg, transport og helse}
Offentlige og private aktører bruker digitale tvillinger til å koordinere investeringer og tjenester som påvirker hele samfunnet. Felles utfordringer er å forankre tvillingen i eksisterende forvaltningssystemer, sikre datadeling og bygge kompetanse på tvers av fagmiljøer. Casemalen hjelper deg med å tydeliggjøre roller og datakilder i prosjekter der mange interessenter må samarbeide.

\subsection*{Smarte bygg og byer}
Statsbygg, Oslo kommune og flere norske byer har tatt i bruk digitale modeller som kombinerer BIM-data med energi- og sensorinformasjon for å styre drift og vedlikehold. Ved å beskrive scope, datakilder og governance i casemalen kan du vise hvordan driftsteam, leietakere og tjenesteleverandører bruker tvillingen til å redusere energibruk, planlegge renhold og forbedre inneklima.

\subsection*{Transport og logistikk}
Bane NOR utvikler digitale tvillinger av jernbanenettet for å planlegge vedlikehold, simulere trafikkavvik og informere kundeservice. Avinor tester tilsvarende modeller på flyplasser for å koordinere bakketjenester. Når du analyserer slike case, må du kartlegge hvordan sanntidsdata fra sensorer, IoT-enheter og trafikkstyringssystemer flyter inn i tvillingen, og hvordan det støtter planlegging og krisehåndtering. Utforsk datakilder som Entur, Bane NOR og Avinor i \autoref{sec:transportressurser} for å sikre at caserapporter støttes av konkrete tidsserier, kapasitetsdata og hendelseslogger.

\subsection*{Helse}
Helse Vest og Universitetssykehuset i Oslo eksperimenterer med pasientspesifikke tvillinger for kreftbehandling, kirurgisk planlegging og rehabilitering. Malen gir struktur for å beskrive hvordan kliniske data, bildediagnostikk og simuleringer integreres, hvilke etiske vurderinger som gjøres, og hvordan pasienter og fagpersonell involveres i beslutninger.

\section{Metodikk for caseanalyse}
For å sammenligne case på tvers av sektorer, start med å beskrive hvorfor tvillingen ble etablert, hvilke forutsetninger som må være til stede og hvordan resultatene måles. Det kan være nyttig å kombinere casemalen med intervjuer, offentlige rapporter og tekniske arkitekturdokumenter. Et helhetlig rammeverk bør dekke mål, omfang, teknologivalg, organisering og dokumenterte effekter.

\begin{itemize}
    \item Kartlegg datakilder systematisk, fra sanntidssensorer til historiske arkiv og åpne data, og vurder kvalitet, eierskap og tilgang.
    \item Noter hvilke styringsstrukturer, insentiver og kompetanseprogram som påvirker gjennomføring og skalering.
    \item Reflekter over etiske og samfunnsmessige hensyn, inkludert personvern, arbeidstakerperspektiv og miljøpåvirkning.
\end{itemize}

\subsection*{Vurderingsmatrise for modenhet og prioritering}
Når flere potensielle case konkurrerer om oppmerksomhet eller undervisningsressurser, kan en vurderingsmatrise gi et felles
beslutningsgrunnlag. Tabellen under brukes til å skåre hvert case fra~1 (lav modenhet) til~5 (høy modenhet) på fem dimensjoner.
Noter begrunnelsen for hver skår og lenk til dokumentasjon, slik at fagfeller kan etterprøve vurderingen.

\begin{longtable}{p{0.23\textwidth}p{0.26\textwidth}p{0.45\textwidth}}
\toprule
\textbf{Kriterium} & \textbf{Hva vurderes?} & \textbf{Eksempel på skåringsskala (1--5)} \\
\midrule
\endfirsthead
\toprule
\textbf{Kriterium} & \textbf{Hva vurderes?} & \textbf{Eksempel på skåringsskala (1--5)} \\
\midrule
\endhead
Datagrunnlag og tilgang & Tilgjengelighet, kvalitet og rettigheter til data som trengs for tvillingen. & 1~= fragmenterte kilder uten tilgangsavklaringer. 3~= etablerte datastrømmer, men mangler kvalitetssikring. 5~= sanntidsdata med avtalte API-er, eierskap og datakvalitetsrutiner. \\
\addlinespace
Organisatorisk forankring & Hvor godt case er støttet av ledelse, fagmiljø og brukere. & 1~= initiativ fra enkeltperson uten mandat. 3~= prosjektgruppe med midlertidig finansiering. 5~= tverrfaglig styringsmodell med dedikerte roller og gevinstansvar. \\
\addlinespace
Teknologisk modenhet & Modenhet på modeller, integrasjonsplattform og driftsoppsett. & 1~= konseptskisse uten implementert plattform. 3~= pilot i begrenset produksjon. 5~= robust løsning med DevOps-/MLOps-sløyfer og overvåkning i drift. \\
\addlinespace
Gevinst- og læringspotensial & Forventet effekt for virksomheten og læringsverdi for studenter. & 1~= uklare mål og lite overføringsverdi. 3~= tydelige KPI-er, men begrenset dokumentasjon. 5~= målbare gevinster, åpne data/artefakter og tydelig kobling til læringsmål. \\
\addlinespace
Risiko og etterlevelse & Regulatoriske, sikkerhets- og etiske forutsetninger. & 1~= ukjente krav eller høy personvernrisiko. 3~= identifiserte tiltak, men ufullstendig dokumentasjon. 5~= komplette risikovurderinger, etterlevelsesplan og jevnlig revisjon. \\
\bottomrule
\end{longtable}

Etter at tabellen er fylt ut, summeres skårene for et samlet modenhetsnivå, men noter også hvordan svakheter skal håndteres.
Følg denne arbeidsflyten for å forankre prioriteringen:

\begin{enumerate}
    \item Sett sammen et tverrfaglig vurderingsteam (for eksempel produktansvarlig, dataarkitekt og domeneekspert) og avklar
    hva slags beslutning som skal tas.
    \item Innhent nødvendig dokumentasjon: prosjektmandat, tekniske beskrivelser, datalister og gevinstrealiseringsplaner. Bruk
    ressursoversikten i \autoref{appendix:ressurser} for å finne støttedata eller verktøy som mangler.
    \item Skår casene i fellesskap, dokumenter begrunnelser og registrer resultatet i prosjektets styrings- eller fagfellelogg.
    \item Definer tiltak for kriterier som får skår 1--2 (for eksempel avklare tilgang, etablere governance eller komplettere
    risikovurdering) og gi tydelig anbefaling om case skal prioriteres, videreutvikles eller settes på vent.
\end{enumerate}

Ved å dele vurderingsmatrisen med studenter og eksterne samarbeidspartnere blir det lettere å diskutere hvilke case som gir mest
verdi, og hvilke forberedelser som kreves før data kan deles eller modeller kobles på.

\section{Casestudie-mal: Digital tvilling for batteriproduksjon i Mo i Rana}
Denne malen er inspirert av etableringen av den nasjonale batteriverdikjeden i Mo i Rana, der en digital tvilling skal binde sammen prosesslinjer, energiforbruk og logistikk. Bruk strukturen nedenfor som ramme for å dokumentere og analysere egne case, og erstatt eksemplene med data fra den faktiske virksomheten du studerer.

\subsection{Sammendrag av caset}
\begin{tabular}{p{0.32\textwidth}p{0.62\textwidth}}
\textbf{Element} & \textbf{Beskrivelse / spørsmål som skal besvares} \\
Virksomhet og lokasjon & Hvilken organisasjon studeres, og hvor er den plassert? \\
Strategisk hovedmål & Hva ønsker virksomheten å oppnå med den digitale tvillingen (for eksempel redusert energibruk eller økt kapasitet)? \\
Scope for tvillingen & Hvilke deler av verdikjeden eller prosesslinjen inngår? \\
Status ved prosjektstart & Hvilke systemer, datakilder og arbeidsprosesser fantes fra før? \\
\end{tabular}

\subsection{Forretningsmål og interessenter}
\begin{itemize}
    \item Beskriv konkrete forretningsmål og prioriteringer (for eksempel energieffektivitet, kvalitetskontroll eller fleksibel skalering).
    \item Identifiser hovedinteressenter: produksjonsledelse, operatører, IT/OT-team, energiselskap, logistikkpartnere og lokale myndigheter.
    \item Avklar hvordan målene måles, inkludert KPI-er som OEE, energiforbruk per battericelle og skraprate.
\end{itemize}

\subsection{Datagrunnlag og integrasjon}
\begin{itemize}
    \item Kartlegg hvilke datastrømmer som trengs: sensorer langs elektrodefabrikasjonen, MES/ERP-data, vær- og strømpriser, lager- og transportstatus.
    \item Vurder kvalitet, oppdateringsfrekvens og eierskap til hver datakilde.
    \item Beskriv integrasjonsarkitekturen, for eksempel OPC UA mot produksjonsutstyr og skybasert datasjø for historiske data.
\end{itemize}

\subsection{Modelleringsstrategi}
\begin{itemize}
    \item Definer hvilke modelltyper som skal inngå: prosessimulering, energimodell, logistikkmodell og prediktive vedlikeholdsmodeller.
    \item Spesifiser modelloppløsning (granularitet), valideringsdata og hvordan modeller kalibreres mot observasjoner.
    \item Marker antatte begrensninger eller usikkerheter, som begrenset historikk fra nyetablert produksjon.
\end{itemize}

\subsection{Implementering og arbeidsprosesser}
\begin{itemize}
    \item Beskriv hvordan tvillingen tas i bruk i daglig drift: dashboards i kontrollrom, digitale arbeidsordre, simulering før omstilling.
    \item Legg inn plan for opplæring av operatører og samarbeid mellom dataanalytikere og prosessingeniører.
    \item Dokumenter forvaltningsmodell: hvem eier modellen, hvordan versjoneres den, og hvordan sikres datakvalitet over tid?
\end{itemize}

\subsection{Resultater, gevinster og gevinstrealisering}
\begin{itemize}
    \item Oppgi konkrete gevinster som måles, for eksempel 8~\% lavere energiforbruk, redusert kassasjon eller raskere oppskalering av nye produktserier.
    \item Beskriv metoder for å verifisere gevinster, inkludert før- og etter-målinger og kontrollgrupper.
    \item Noter sekundære effekter: bedre HMS-oppfølging, samarbeid med kraftleverandør eller nye forretningsmodeller.
\end{itemize}

\subsection{Overføringsverdi og neste steg}
\begin{itemize}
    \item Drøft hvilke komponenter som kan gjenbrukes i andre norske industrier, for eksempel prosessmodeller, datastrømsarkitektur eller governance.
    \item Identifiser videre utvikling: kobling mot leverandørkjede, bruk av AI til kvalitetssikring, integrasjon med europeiske energimarkeder.
    \item Foreslå indikatorer og beslutningspunkter for å evaluere når tvillingen bør oppdateres eller skaleres.
\end{itemize}

\subsection*{Sjekkliste for caserapporten}
\begin{itemize}
    \item[$\square$] Har du dokumentert mål, interessenter og scope?
    \item[$\square$] Er alle relevante datakilder og integrasjoner beskrevet?
    \item[$\square$] Inneholder rapporten tydelige modelleringsvalg og begrunnelser?
    \item[$\square$] Er det lagt inn plan for drift, opplæring og governance?
    \item[$\square$] Er gevinstene målbare og knyttet til KPI-er som følges opp?
    \item[$\square$] Er læringspunkter og videre tiltak formulert?
\end{itemize}

\subsection{Eksempler på norske casetilpasninger}
\paragraph{Digital tvilling av kraftnett i Nord-Norge}
Statnett og regionale nettselskaper samarbeider om å modellere flaskehalser i kraftnettet mellom Ofoten og Finnmark. Når casemalen brukes på dette prosjektet, bør du beskrive hvordan værdata, planlagte industrietableringer og driftsmønstre fra vann- og vindkraft integreres. Vurder også hvordan tvillingen støtter dialogen med NVE og lokalsamfunn.

\paragraph{Jernbanedrift og vedlikehold hos Bane NOR}
En digital tvilling av jernbanestrekningen mellom Oslo og Lillehammer kombinerer BIM-modeller, sensordata fra spor og sanntidssignaler fra togene. Ved å følge malen kan du dokumentere hvordan operasjonssentralen, entreprenører og leverandører samarbeider om vedlikeholdsvinduer, og hvordan simuleringer brukes til å planlegge nye ruteoppsett.

\paragraph{Virtuell pasientflyt i spesialisthelsetjenesten}
Helse Sør-Øst prøver ut digitale tvillinger for å koordinere pasientflyt i kreftbehandling. Malen hjelper deg å tydeliggjøre hvilke kliniske datasett som inngår, hvordan algoritmer foreslår behandlingsplaner og hvordan pasienter får informasjon. Etisk vurdering, datasikkerhet og samtykkeprosesser må dokumenteres grundig.

\section{Refleksjonsspørsmål og øvinger}
\begin{enumerate}
    \item \textbf{Analyse av eksisterende case:} Velg ett av caseeksemplene ovenfor eller et eget norsk prosjekt. Bruk casemalen til å skrive en kort rapport (2--3 sider) der du vurderer modenhet, måloppnåelse og hvilke ytterligere datakilder som kan styrke tvillingen.
    \item \textbf{Planlegging av nytt initiativ:} Sett sammen et tverrfaglig team (for eksempel prosessingeniør, data scientist og driftsleder) og lag et arbeidsnotat som beskriver hvordan dere vil samle data, utvikle modeller og organisere forvaltning for et nytt case i deres sektor.
    \item \textbf{Tverrsektoriell overføring:} Sammenlign to sektorer, for eksempel kraft og bygg. Lag en tabell som viser hvilke komponenter i casemalen som er direkte overførbare, hvilke som må tilpasses, og foreslå tiltak for å gjenbruke modeller eller datagrunnlag.
    \item \textbf{Refleksjon over etikk og verdiskaping:} Diskuter i grupper hvordan personvern, arbeidsmiljø og bærekraft bør ivaretas når tvillinger skaleres. Bruk notater fra de tre foregående oppgavene og oppdater casemalen med konkrete risikoreduserende tiltak.
\end{enumerate}
