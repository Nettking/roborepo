\chapter{Arbeidsark og maler for prosjektarbeid}
\label{appendix:arbeidsark}

Dette appendikset samler arbeidsark som støtter prosjektgjennomføringen i undervisningsopplegget.
Målgruppen er studentteam, faglærere og samarbeidspartnere som trenger strukturerte maler for
planlegging, datainnsamling og evaluering av digitale tvilling-prosjekter. Arbeidsarkene er
organisert tematisk, og hver tabell kan skrives ut eller fylles ut digitalt.

\section{Forprosjektcanvas}
Canvasset fungerer som et førstegangsrammeverk når studentgrupper skal definere caset sitt.
Fyll ut feltene i prioritert rekkefølge for å sikre felles forståelse mellom studenter, mentorer og
partnere.

\begin{longtable}{p{0.24\textwidth}p{0.35\textwidth}p{0.31\textwidth}}
\toprule
\textbf{Del} & \textbf{Hva beskrives?} & \textbf{Arbeidsnotater} \\
\midrule
\endfirsthead
\toprule
\textbf{Del} & \textbf{Hva beskrives?} & \textbf{Arbeidsnotater} \\
\midrule
\endhead
Problemforståelse & Hvilket forretnings- eller samfunnsproblem skal løses, og hvilke indikatorer viser at problemet er viktig? & \rule{0.95\linewidth}{0.4pt}\\[0.8em]
Verdiforslag & Hvilken verdi forventes skapt for organisasjon, sluttbruker og samarbeidspartnere? & \rule{0.95\linewidth}{0.4pt}\\[0.8em]
Datastrømskartlegging & Hvilke datakilder finnes (interne/eksterne), hvordan får teamet tilgang, og hvilke kvalitetstiltak må på plass? & \rule{0.95\linewidth}{0.4pt}\\[0.8em]
Modellomfang & Hvilke komponenter modelleres (fysikkbasert, datadrevet, hybride), og hvilke antakelser må dokumenteres? & \rule{0.95\linewidth}{0.4pt}\\[0.8em]
Interessenter & Hvem er involvert, hvilken gevinst forventer de, og hvilke beslutninger skal støttes? & \rule{0.95\linewidth}{0.4pt}\\[0.8em]
Rammebetingelser & Hvilke regulatoriske, etiske eller sikkerhetsmessige krav gjelder? & \rule{0.95\linewidth}{0.4pt}\\[0.8em]
Suksesskriterier & Hvilke målbare indikatorer (KPI-er, gevinsthypoteser) skal brukes for å evaluere tvillingen? & \rule{0.95\linewidth}{0.4pt}\\[0.8em]
Ressursbehov & Hvilke roller, verktøy, budsjetter og tidsvinduer kreves for å realisere forprosjektet? & \rule{0.95\linewidth}{0.4pt}\\[0.8em]
\bottomrule
\end{longtable}

\section{Arbeidsark for datainnsamling}
Bruk tabellen for å holde oversikt over datakilder og forvaltningsrutiner. Fyll ut én rad per
sensor, datasett eller API.

\begin{longtable}{p{0.19\textwidth}p{0.25\textwidth}p{0.19\textwidth}p{0.27\textwidth}}
\toprule
\textbf{Datakilde} & \textbf{Tilgangs- og lisensvilkår} & \textbf{Oppdateringsfrekvens} & \textbf{Tiltak for kvalitet og etterlevelse} \\
\midrule
\endfirsthead
\toprule
\textbf{Datakilde} & \textbf{Tilgangs- og lisensvilkår} & \textbf{Oppdateringsfrekvens} & \textbf{Tiltak for kvalitet og etterlevelse} \\
\midrule
\endhead
\rule{0.9\linewidth}{0.4pt} & \rule{0.9\linewidth}{0.4pt} & \rule{0.9\linewidth}{0.4pt} & \rule{0.9\linewidth}{0.4pt}\\[0.8em]
\rule{0.9\linewidth}{0.4pt} & \rule{0.9\linewidth}{0.4pt} & \rule{0.9\linewidth}{0.4pt} & \rule{0.9\linewidth}{0.4pt}\\[0.8em]
\rule{0.9\linewidth}{0.4pt} & \rule{0.9\linewidth}{0.4pt} & \rule{0.9\linewidth}{0.4pt} & \rule{0.9\linewidth}{0.4pt}\\[0.8em]
\rule{0.9\linewidth}{0.4pt} & \rule{0.9\linewidth}{0.4pt} & \rule{0.9\linewidth}{0.4pt} & \rule{0.9\linewidth}{0.4pt}\\[0.8em]
\bottomrule
\end{longtable}

\section{Modell- og simuleringsplan}
Arbeidsarket gir en rask sjekkliste for å planlegge modelleringen. Bruk kolonnen for sjekkpunkter
til å beskrive fremgang og eventuelle avklaringer.

\begin{longtable}{p{0.23\textwidth}p{0.28\textwidth}p{0.39\textwidth}}
\toprule
\textbf{Fase} & \textbf{Leveranse} & \textbf{Sjekkpunkter og avklaringer} \\
\midrule
\endfirsthead
\toprule
\textbf{Fase} & \textbf{Leveranse} & \textbf{Sjekkpunkter og avklaringer} \\
\midrule
\endhead
Behovs- og prosessanalyse & Funksjonell beskrivelse av systemet og tvillingens rolle. & \begin{itemize}[leftmargin=*]
    \item Har teamet identifisert grensesnitt mot eksisterende systemer?
    \item Er krav til responstid og nøyaktighet avklart med interessenter?
\end{itemize}\\
Dataforberedelse & Datakatalog, datakvalitetsmålinger og transformasjonslogg. & \begin{itemize}[leftmargin=*]
    \item Er datastruktur, enheter og semantikk dokumentert?
    \item Er data lagret slik at versjonering og reproduksjon ivaretas?
\end{itemize}\\
Modellutvikling & Dokumenterte modeller (fysikk, statistikk, hybrid) med parameteroversikt. & \begin{itemize}[leftmargin=*]
    \item Er valgte modeller begrunnet mot gevinsthypotesen?
    \item Finnes det plan for kalibrering, tuning og sensitivitetstesting?
\end{itemize}\\
Simulering og eksperimenter & Eksperimentdesign, scenarier og måleplan. & \begin{itemize}[leftmargin=*]
    \item Hvilke scenarioer skal kjøres, og hvilke beslutninger støttes?
    \item Hvordan registreres avvik mellom simulering og virkelighet?
\end{itemize}\\
Implementering og drift & Plan for deployment, overvåking og modelloppdatering. & \begin{itemize}[leftmargin=*]
    \item Er DevOps/MLOps-rutiner og rollback-plan etablert?
    \item Er ansvar for drift, risikostyring og gevinstoppfølging avklart?
\end{itemize}\\
\bottomrule
\end{longtable}

\section{Risiko- og gevinstjournal}
Journalen kombinerer risikoregister og gevinstoppfølging slik at teamet kan synliggjøre status for
kritiske forutsetninger. Fyll ut tabellen månedlig eller etter større milepæler.

\begin{longtable}{p{0.16\textwidth}p{0.20\textwidth}p{0.22\textwidth}p{0.18\textwidth}p{0.18\textwidth}}
\toprule
\textbf{Kategori} & \textbf{Beskrivelse} & \textbf{Konsekvens og sannsynlighet} & \textbf{Tiltak/ansvar} & \textbf{Status og neste steg} \\
\midrule
\endfirsthead
\toprule
\textbf{Kategori} & \textbf{Beskrivelse} & \textbf{Konsekvens og sannsynlighet} & \textbf{Tiltak/ansvar} & \textbf{Status og neste steg} \\
\midrule
\endhead
Risiko & \rule{0.9\linewidth}{0.4pt} & \rule{0.9\linewidth}{0.4pt} & \rule{0.9\linewidth}{0.4pt} & \rule{0.9\linewidth}{0.4pt}\\[0.8em]
Risiko & \rule{0.9\linewidth}{0.4pt} & \rule{0.9\linewidth}{0.4pt} & \rule{0.9\linewidth}{0.4pt} & \rule{0.9\linewidth}{0.4pt}\\[0.8em]
Gevinst & \rule{0.9\linewidth}{0.4pt} & \rule{0.9\linewidth}{0.4pt} & \rule{0.9\linewidth}{0.4pt} & \rule{0.9\linewidth}{0.4pt}\\[0.8em]
Gevinst & \rule{0.9\linewidth}{0.4pt} & \rule{0.9\linewidth}{0.4pt} & \rule{0.9\linewidth}{0.4pt} & \rule{0.9\linewidth}{0.4pt}\\[0.8em]
\bottomrule
\end{longtable}

\section{Refleksjonslogg for studentteam}
Refleksjonsloggen hjelper teamet å dokumentere læringsutbytte, beslutninger og tiltak. Hver rad
kan brukes ukentlig.

\begin{longtable}{p{0.22\textwidth}p{0.27\textwidth}p{0.24\textwidth}p{0.19\textwidth}}
\toprule
\textbf{Dato/uke} & \textbf{Hva lærte vi?} & \textbf{Åpne spørsmål eller avklaringer} & \textbf{Tiltak til neste møte} \\
\midrule
\endfirsthead
\toprule
\textbf{Dato/uke} & \textbf{Hva lærte vi?} & \textbf{Åpne spørsmål eller avklaringer} & \textbf{Tiltak til neste møte} \\
\midrule
\endhead
\rule{0.9\linewidth}{0.4pt} & \rule{0.9\linewidth}{0.4pt} & \rule{0.9\linewidth}{0.4pt} & \rule{0.9\linewidth}{0.4pt}\\[0.8em]
\rule{0.9\linewidth}{0.4pt} & \rule{0.9\linewidth}{0.4pt} & \rule{0.9\linewidth}{0.4pt} & \rule{0.9\linewidth}{0.4pt}\\[0.8em]
\rule{0.9\linewidth}{0.4pt} & \rule{0.9\linewidth}{0.4pt} & \rule{0.9\linewidth}{0.4pt} & \rule{0.9\linewidth}{0.4pt}\\[0.8em]
\rule{0.9\linewidth}{0.4pt} & \rule{0.9\linewidth}{0.4pt} & \rule{0.9\linewidth}{0.4pt} & \rule{0.9\linewidth}{0.4pt}\\[0.8em]
\bottomrule
\end{longtable}

\section{Sjekkliste for fagfelleklarering}
Bruk sjekklisten før innlevering til fagfelleteamet. Hvert punkt hukes av og begrunnes i
etterprøvbarhetsloggen.

\begin{enumerate}[label=\arabic*.]
    \item Er forprosjektcanvas og datainnsamlingsarket oppdatert med siste versjon av datastrømmer og tilgangsavklaringer?
    \item Er modell- og simuleringsplanen kvalitetssikret av fagmentor, med tydelig kobling til læringsmålene i Kapittel~4 og \nobreakspace{}5?
    \item Er risiko- og gevinstjournalen oppdatert med tiltak, ansvarlig person og forventet tidspunkt for neste revisjon?
    \item Er refleksjonsloggen delt med faglærer og eksterne partnere, og er åpne spørsmål loggført i prosjektets styringsverktøy?
    \item Er dokumentasjonen versjonert i prosjektets repo, inkludert referanse til modellkode, datakilder og beslutningslogg?
\end{enumerate}

\section{Regelverksoversikt for nøkkelsektorer}
Digitale tvilling-prosjekter må forholde seg til sektorvise regelverk, standarder og tilsynsformer.
Tabellen nedenfor fungerer som et arbeidsark for å koble prosjektets kravdokumentasjon til
rammeverkene som omtales i Kapittel~6 om validering og Kapittel~7 om styring. Kolonnene kan fylles
ut med egne kommentarer eller lenker til organisasjonens styringssystem.

\begin{table}[htbp]
    \centering
    \caption{Nøkkelregelverk og krav for digitale tvillinger i utvalgte sektorer}
    \label{tab:appendiks-regelverk}
    \begin{tabular}{p{3.0cm}p{4.8cm}p{4.8cm}p{3.0cm}}
        \toprule
        \textbf{Sektor} & \textbf{Relevante regelverk og standarder} & \textbf{Konsekvenser for digital tvilling} & \textbf{Dokumentasjon og tilsyn} \\
        \midrule
        Energi og kraft & NIS2, IEC~62443 og Energi Norge sin veileder for kontrolltårn\citep{eu2022nis2,iec62443-2-1,energinorge2023beredskap} & Krever segmentering av OT/IT, hendelsesrapportering og testbare sikkerhetskontroller i modell- og datastrømmer. & Lever dokumenterte tilsynslogger, standardkart og beredskapsøvelser til NVE og virksomhetens kvalitetsteam. \\
        \addlinespace
        Helse og velferd & AI-forordningen, Norm for informasjonssikkerhet og helseberedskapsplaner\citep{eu2024aiact,norm2023,helsedir2023beredskap} & Forutsetter risikoklassifisering av tjenester, tydelig menneskelig tilsyn og sporbarhet for data og beslutninger. & Oppdater kvalitetsjournal, modelljournal og beredskapslogger for revisjon av Datatilsynet og statsforvalter. \\
        \addlinespace
        Maritim transport & Regelverk for autonome fartøy og maritime sikkerhetsstandarder\citep{sdir2023autonomefartoy,dnv2024autonomous,massterly2023operations} & Krever kontinuerlig modellvalidering mot sjøfartsregler, sporbar testing og dokumenterte avviksprosedyrer for fjernoperasjon. & Del revisjonslogg og operasjonsprotokoller med Sjøfartsdirektoratet, klasseselskap og rederiets beredskapsråd. \\
        \addlinespace
        Kommunal beredskap & Veiledere for overvann, beredskap og kommunale tiltaksplaner\citep{nve2022kommunal,dsb2022beredskap,oslo2023overvann} & Krever kobling mellom simuleringer, tiltakslister og dokumentasjon av samfunnskritiske funksjoner i krisescenarier. & Arkiver scenariojournal, beslutningslogg og koordinering med beredskapskoordinator for kommunale revisjoner. \\
        \bottomrule
    \end{tabular}
\end{table}

Bruk oversikten sammen med arbeidsarkene over for å sikre at krav fra sikkerhet, personvern og
beredskap fanges opp i prosjektplanene. Tilpass rekkefølge, kolonneoverskrifter og detaljer etter
sektor og modenhetsnivå, men bevar kjerneelementene slik at fagteam og fagfeller kan sammenligne
leveranser på tvers av grupper.

\section{Personvernkonsekvensvurdering for helsedataspace}
Helsesektoren krever særskilt dokumentasjon fordi digitale tvillinger kan håndtere
sensitive opplysninger om pasientforløp, tilsynslogg og responstjenester. Arbeidsarket under
pakker krav fra personvernforordningen, Norm for informasjonssikkerhet og helsedirektoratets
beredskapsrutiner slik at kapittel~3 (dataspace), kapittel~6 (validering) og kapittel~7 (styring)
bruker samme sjekkliste.\citep{datatilsynet2023dpia,norm2023,helsedir2023beredskap}

Følg trinnene når teamet planlegger en ny pilot eller endring i helsedataspace:

\begin{enumerate}[label=\alph*.]
    \item \textbf{Forankring:} Bekreft behandlingsgrunnlag, formål og dataansvarlig, og koble
    prosjektet til virksomhetens internkontroll.
    \item \textbf{Risikovurdering:} Identifiser personverntrusler, tekniske svakheter og mulige
    konsekvenser for brukere. Dokumenter vurderingen i kvalitetsjournalen.
    \item \textbf{Tiltaksplan:} Definer tekniske og organisatoriske tiltak, hvem som har ansvar for
    gjennomføring og når tiltakene skal revideres.
    \item \textbf{Godkjenning og oppfølging:} Innhent signatur fra personvernombud eller ansvarlig
    leder, planlegg revisjon og del læringspunkter i gevinst- og tiltaksloggen.
\end{enumerate}

Tabell~\ref{tab:appendiks-dpia} fungerer som arbeidsark. Hver rad fylles ut når prosjektet går
inn i et nytt steg av vurderingen. Kolonnen \emph{Leveranse} peker til de dokumentene som allerede
er etablert i boken, slik at teamet gjenbruker strukturen i stedet for å opprette parallelle
maler.

\begin{table}[htbp]
    \centering
    \caption{Arbeidsark for DPIA i helsedataspace}
    \label{tab:appendiks-dpia}
    \begin{tabular}{p{3.0cm}p{5.0cm}p{5.0cm}p{2.8cm}}
        \toprule
        \textbf{Trinn} & \textbf{Nøkkelspørsmål} & \textbf{Tiltak og beslutninger} & \textbf{Leveranse} \\
        \midrule
        Forankring og omfang & Hvilke datastrømmer og aktører inngår, og hvilket rettslig grunnlag brukes? & Noter hvordan dataspace-kontraktene speiler tilgangsnivå, og hvordan ansvar er fordelt i RACI-S-matrisen. & Dataspace-governancebrief (Kapittel~3 og Appendiks) \\
        \addlinespace
        Risikovurdering & Hvilke personvernrisikoer og sikkerhetstrusler må vurderes, og hvordan påvirkes tjenestekvalitet og pasientsikkerhet? & Koble risikoene til indikatorene i kontrolltårnet og planlegg testaktiviteter i valideringspanelet. & Risiko- og gevinstjournal (Appendiks) \\
        \addlinespace
        Tiltaksplan og implementering & Hvilke tekniske og organisatoriske tiltak kreves, hvem er ansvarlig og hvordan måles effekt? & Beskriv logging, anonymisering og tilgangsstyring, samt opplæring og kommunikasjonsplan for tjenesteeiere. & Implementeringsverktøykasse for dataspace- og gevinststyring (Appendiks) \\
        \addlinespace
        Godkjenning og oppfølging & Hvem godkjenner vurderingen, hvordan dokumenteres beslutningen og hvordan følges tiltak opp over tid? & Registrer signerte beslutninger, planlegg revisjonsintervall og koble læringspunkter til gevinstlogg og fagfelleklarering. & Kontrolltårn-oppfølging og fagfellesammendrag (Kapittel~6, Kapittel~7 og Appendiks) \\
        \bottomrule
    \end{tabular}
\end{table}

Arbeidsarket kan lastes ned som eget skjema eller fylles ut direkte i prosjektets styringsverktøy.
Når studentene bruker malen i workshop, bør de lenke til relevante kapitler slik at vurderingen
kan følges opp i fagfelleløpet og under eksamen.

\section{Implementeringsverktøykasse for dataspace- og gevinststyring}
Verktøykassen under kombinerer sjekklister, møtemaler og rapporteringsstrukturer slik at
dataspace-arbeid (Kapittel~3) og gevinststyring (Kapittel~7) kan gjennomføres sammenhengende.
Hver modul kan tilpasses den enkelte organisasjon, men bør fylles ut før prosjektet går videre til
pilotering eller fagfelleklarering.

\begin{longtable}{p{0.22\textwidth}p{0.32\textwidth}p{0.30\textwidth}p{0.16\textwidth}}
\toprule
\textbf{Modul} & \textbf{Formål} & \textbf{Nøkkelsjekkpunkter} & \textbf{Kapitler} \\
\midrule
\endfirsthead
\toprule
\textbf{Modul} & \textbf{Formål} & \textbf{Nøkkelsjekkpunkter} & \textbf{Kapitler} \\
\midrule
\endhead
Dataspace-governancebrief & Oppsummerer beslutninger om datakilder, kontrakter og tilgang til sandkasser. &
\begin{itemize}[leftmargin=*]
    \item Er dataeier, behandlingsgrunnlag og delingsnivå avklart?
    \item Er indikatorene fra dataspace-tabellene koblet til kontrolltårnet?
\end{itemize} & Kapittel~3 og Kapittel~6 \\
\addlinespace
Gevinst- og tiltakslogg & Binder hypoteser, ansvar og måleplaner til porteføljestyringen. &
\begin{itemize}[leftmargin=*]
    \item Er målbare gevinstindikatorer knyttet til milepæler og budsjetter?
    \item Har hvert tiltak en definert ansvarlig og eskaleringsvei?
\end{itemize} & Kapittel~7 \\
\addlinespace
Kontrolltårn-oppfølging & Synliggjør hvordan operativ drift overvåkes og forbedres. &
\begin{itemize}[leftmargin=*]
    \item Er hendelseslogger og læringspunkter ført tilbake til valideringspanelet?
    \item Er beredskapsøvelser og stresstester planlagt med klare kriterier?
\end{itemize} & Kapittel~6 og Kapittel~8 \\
\addlinespace
Pilot- og fagfellesammendrag & Sikrer at erfaringer fra tester deles med fagteam og partnere. &
\begin{itemize}[leftmargin=*]
    \item Er styrker, svakheter og anbefalte forbedringer dokumentert per pilot?
    \item Er dokumentasjonen versjonert og delt med lærerveiledningen?
\end{itemize} & Kapittel~8 og lærerveiledningen \\
\bottomrule
\end{longtable}

Når moduler tas i bruk, anbefales følgende arbeidsflyt:

\begin{enumerate}[label=\alph*.]
    \item \textbf{Forberedelse:} Oppdater dataspace-governancebriefen sammen med ansvarlig dataeier og
    bekreft at tilgangsmodellen støtter planlagte scenarier.
    \item \textbf{Planlegging:} Bruk gevinst- og tiltaksloggen for å avtale prioriterte leveranser og
    tydeliggjøre hvordan indikatorer rapporteres i kontrolltårnet.
    \item \textbf{Gjennomføring:} Dokumenter resultater og avvik i kontrolltårn-oppfølgingen, og del
    læringspunkter i fagmøter.
    \item \textbf{Deling:} Lag et kort pilot- og fagfellesammendrag og arkiver det i prosjektets
    etterprøvbarhetslogg før materialet sendes videre til fagfellegruppen.
\end{enumerate}

Med denne strukturen får studentteam og partnere en felles sjekkliste for å holde styr på
datasamarbeid, beslutninger og gevinstoppfølging. Verktøykassen kan replikeres i prosjektstyrings-
verktøy eller utdeles som arbeidsark sammen med malene i resten av appendikset.
