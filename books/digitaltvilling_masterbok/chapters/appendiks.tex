\chapter{Arbeidsark og maler for prosjektarbeid}
\label{appendix:arbeidsark}

Dette appendikset samler arbeidsark som støtter prosjektgjennomføringen i undervisningsopplegget.
Målgruppen er studentteam, faglærere og samarbeidspartnere som trenger strukturerte maler for
planlegging, datainnsamling og evaluering av digitale tvilling-prosjekter. Arbeidsarkene er
organisert tematisk, og hver tabell kan skrives ut eller fylles ut digitalt.

\section{Forprosjektcanvas}
Canvasset fungerer som et førstegangsrammeverk når studentgrupper skal definere caset sitt.
Fyll ut feltene i prioritert rekkefølge for å sikre felles forståelse mellom studenter, mentorer og
partnere.

\begin{longtable}{p{0.24\textwidth}p{0.35\textwidth}p{0.31\textwidth}}
\toprule
\textbf{Del} & \textbf{Hva beskrives?} & \textbf{Arbeidsnotater} \\
\midrule
\endfirsthead
\toprule
\textbf{Del} & \textbf{Hva beskrives?} & \textbf{Arbeidsnotater} \\
\midrule
\endhead
Problemforståelse & Hvilket forretnings- eller samfunnsproblem skal løses, og hvilke indikatorer viser at problemet er viktig? & \rule{0.95\linewidth}{0.4pt}\\[0.8em]
Verdiforslag & Hvilken verdi forventes skapt for organisasjon, sluttbruker og samarbeidspartnere? & \rule{0.95\linewidth}{0.4pt}\\[0.8em]
Datastrømskartlegging & Hvilke datakilder finnes (interne/eksterne), hvordan får teamet tilgang, og hvilke kvalitetstiltak må på plass? & \rule{0.95\linewidth}{0.4pt}\\[0.8em]
Modellomfang & Hvilke komponenter modelleres (fysikkbasert, datadrevet, hybride), og hvilke antakelser må dokumenteres? & \rule{0.95\linewidth}{0.4pt}\\[0.8em]
Interessenter & Hvem er involvert, hvilken gevinst forventer de, og hvilke beslutninger skal støttes? & \rule{0.95\linewidth}{0.4pt}\\[0.8em]
Rammebetingelser & Hvilke regulatoriske, etiske eller sikkerhetsmessige krav gjelder? & \rule{0.95\linewidth}{0.4pt}\\[0.8em]
Suksesskriterier & Hvilke målbare indikatorer (KPI-er, gevinsthypoteser) skal brukes for å evaluere tvillingen? & \rule{0.95\linewidth}{0.4pt}\\[0.8em]
Ressursbehov & Hvilke roller, verktøy, budsjetter og tidsvinduer kreves for å realisere forprosjektet? & \rule{0.95\linewidth}{0.4pt}\\[0.8em]
\bottomrule
\end{longtable}

\section{Arbeidsark for datainnsamling}
Bruk tabellen for å holde oversikt over datakilder og forvaltningsrutiner. Fyll ut én rad per
sensor, datasett eller API.

\begin{longtable}{p{0.19\textwidth}p{0.25\textwidth}p{0.19\textwidth}p{0.27\textwidth}}
\toprule
\textbf{Datakilde} & \textbf{Tilgangs- og lisensvilkår} & \textbf{Oppdateringsfrekvens} & \textbf{Tiltak for kvalitet og etterlevelse} \\
\midrule
\endfirsthead
\toprule
\textbf{Datakilde} & \textbf{Tilgangs- og lisensvilkår} & \textbf{Oppdateringsfrekvens} & \textbf{Tiltak for kvalitet og etterlevelse} \\
\midrule
\endhead
\rule{0.9\linewidth}{0.4pt} & \rule{0.9\linewidth}{0.4pt} & \rule{0.9\linewidth}{0.4pt} & \rule{0.9\linewidth}{0.4pt}\\[0.8em]
\rule{0.9\linewidth}{0.4pt} & \rule{0.9\linewidth}{0.4pt} & \rule{0.9\linewidth}{0.4pt} & \rule{0.9\linewidth}{0.4pt}\\[0.8em]
\rule{0.9\linewidth}{0.4pt} & \rule{0.9\linewidth}{0.4pt} & \rule{0.9\linewidth}{0.4pt} & \rule{0.9\linewidth}{0.4pt}\\[0.8em]
\rule{0.9\linewidth}{0.4pt} & \rule{0.9\linewidth}{0.4pt} & \rule{0.9\linewidth}{0.4pt} & \rule{0.9\linewidth}{0.4pt}\\[0.8em]
\bottomrule
\end{longtable}

\section{Modell- og simuleringsplan}
Arbeidsarket gir en rask sjekkliste for å planlegge modelleringen. Bruk kolonnen for sjekkpunkter
til å beskrive fremgang og eventuelle avklaringer.

\begin{longtable}{p{0.23\textwidth}p{0.28\textwidth}p{0.39\textwidth}}
\toprule
\textbf{Fase} & \textbf{Leveranse} & \textbf{Sjekkpunkter og avklaringer} \\
\midrule
\endfirsthead
\toprule
\textbf{Fase} & \textbf{Leveranse} & \textbf{Sjekkpunkter og avklaringer} \\
\midrule
\endhead
Behovs- og prosessanalyse & Funksjonell beskrivelse av systemet og tvillingens rolle. & \begin{itemize}[leftmargin=*]
    \item Har teamet identifisert grensesnitt mot eksisterende systemer?
    \item Er krav til responstid og nøyaktighet avklart med interessenter?
\end{itemize}\\
Dataforberedelse & Datakatalog, datakvalitetsmålinger og transformasjonslogg. & \begin{itemize}[leftmargin=*]
    \item Er datastruktur, enheter og semantikk dokumentert?
    \item Er data lagret slik at versjonering og reproduksjon ivaretas?
\end{itemize}\\
Modellutvikling & Dokumenterte modeller (fysikk, statistikk, hybrid) med parameteroversikt. & \begin{itemize}[leftmargin=*]
    \item Er valgte modeller begrunnet mot gevinsthypotesen?
    \item Finnes det plan for kalibrering, tuning og sensitivitetstesting?
\end{itemize}\\
Simulering og eksperimenter & Eksperimentdesign, scenarier og måleplan. & \begin{itemize}[leftmargin=*]
    \item Hvilke scenarioer skal kjøres, og hvilke beslutninger støttes?
    \item Hvordan registreres avvik mellom simulering og virkelighet?
\end{itemize}\\
Implementering og drift & Plan for deployment, overvåking og modelloppdatering. & \begin{itemize}[leftmargin=*]
    \item Er DevOps/MLOps-rutiner og rollback-plan etablert?
    \item Er ansvar for drift, risikostyring og gevinstoppfølging avklart?
\end{itemize}\\
\bottomrule
\end{longtable}

\section{Risiko- og gevinstjournal}
Journalen kombinerer risikoregister og gevinstoppfølging slik at teamet kan synliggjøre status for
kritiske forutsetninger. Fyll ut tabellen månedlig eller etter større milepæler.

\begin{longtable}{p{0.16\textwidth}p{0.20\textwidth}p{0.22\textwidth}p{0.18\textwidth}p{0.18\textwidth}}
\toprule
\textbf{Kategori} & \textbf{Beskrivelse} & \textbf{Konsekvens og sannsynlighet} & \textbf{Tiltak/ansvar} & \textbf{Status og neste steg} \\
\midrule
\endfirsthead
\toprule
\textbf{Kategori} & \textbf{Beskrivelse} & \textbf{Konsekvens og sannsynlighet} & \textbf{Tiltak/ansvar} & \textbf{Status og neste steg} \\
\midrule
\endhead
Risiko & \rule{0.9\linewidth}{0.4pt} & \rule{0.9\linewidth}{0.4pt} & \rule{0.9\linewidth}{0.4pt} & \rule{0.9\linewidth}{0.4pt}\\[0.8em]
Risiko & \rule{0.9\linewidth}{0.4pt} & \rule{0.9\linewidth}{0.4pt} & \rule{0.9\linewidth}{0.4pt} & \rule{0.9\linewidth}{0.4pt}\\[0.8em]
Gevinst & \rule{0.9\linewidth}{0.4pt} & \rule{0.9\linewidth}{0.4pt} & \rule{0.9\linewidth}{0.4pt} & \rule{0.9\linewidth}{0.4pt}\\[0.8em]
Gevinst & \rule{0.9\linewidth}{0.4pt} & \rule{0.9\linewidth}{0.4pt} & \rule{0.9\linewidth}{0.4pt} & \rule{0.9\linewidth}{0.4pt}\\[0.8em]
\bottomrule
\end{longtable}

\section{Refleksjonslogg for studentteam}
Refleksjonsloggen hjelper teamet å dokumentere læringsutbytte, beslutninger og tiltak. Hver rad
kan brukes ukentlig.

\begin{longtable}{p{0.22\textwidth}p{0.27\textwidth}p{0.24\textwidth}p{0.19\textwidth}}
\toprule
\textbf{Dato/uke} & \textbf{Hva lærte vi?} & \textbf{Åpne spørsmål eller avklaringer} & \textbf{Tiltak til neste møte} \\
\midrule
\endfirsthead
\toprule
\textbf{Dato/uke} & \textbf{Hva lærte vi?} & \textbf{Åpne spørsmål eller avklaringer} & \textbf{Tiltak til neste møte} \\
\midrule
\endhead
\rule{0.9\linewidth}{0.4pt} & \rule{0.9\linewidth}{0.4pt} & \rule{0.9\linewidth}{0.4pt} & \rule{0.9\linewidth}{0.4pt}\\[0.8em]
\rule{0.9\linewidth}{0.4pt} & \rule{0.9\linewidth}{0.4pt} & \rule{0.9\linewidth}{0.4pt} & \rule{0.9\linewidth}{0.4pt}\\[0.8em]
\rule{0.9\linewidth}{0.4pt} & \rule{0.9\linewidth}{0.4pt} & \rule{0.9\linewidth}{0.4pt} & \rule{0.9\linewidth}{0.4pt}\\[0.8em]
\rule{0.9\linewidth}{0.4pt} & \rule{0.9\linewidth}{0.4pt} & \rule{0.9\linewidth}{0.4pt} & \rule{0.9\linewidth}{0.4pt}\\[0.8em]
\bottomrule
\end{longtable}

\section{Sjekkliste for fagfelleklarering}
Bruk sjekklisten før innlevering til fagfelleteamet. Hvert punkt hukes av og begrunnes i
etterprøvbarhetsloggen.

\begin{enumerate}[label=\arabic*.]
    \item Er forprosjektcanvas og datainnsamlingsarket oppdatert med siste versjon av datastrømmer og tilgangsavklaringer?
    \item Er modell- og simuleringsplanen kvalitetssikret av fagmentor, med tydelig kobling til læringsmålene i Kapittel~4 og \nobreakspace{}5?
    \item Er risiko- og gevinstjournalen oppdatert med tiltak, ansvarlig person og forventet tidspunkt for neste revisjon?
    \item Er refleksjonsloggen delt med faglærer og eksterne partnere, og er åpne spørsmål loggført i prosjektets styringsverktøy?
    \item Er dokumentasjonen versjonert i prosjektets repo, inkludert referanse til modellkode, datakilder og beslutningslogg?
\end{enumerate}

Arbeidsarkene er ment å brukes fleksibelt: tilpass rekkefølge, kolonneoverskrifter og detaljer
etter sektor og modenhetsnivå. Bevar likevel kjerneelementene slik at fagteam og fagfeller kan
sammenligne leveranser på tvers av grupper.
