\chapter{Introduksjon til digitale tvillinger}

\section{Læringsmål}
\begin{itemize}
    \item Forklare hva en digital tvilling er og hvordan begrepet har utviklet seg.
    \item Skille mellom digitale tvillinger, digitale skygger og klassiske simuleringsmodeller.
    \item Identifisere nøkkelkomponentene som trengs for å realisere en digital tvilling.
\end{itemize}

\section{Bakgrunn og definisjoner}
Begrepet «digital tvilling» dukket først opp i industrien rundt årtusenskiftet da Michael Grieves beskrev hvordan et fysisk produkt kunne speiles digitalt gjennom hele livssyklusen. NASA populariserte tankegangen tidlig på 2010-tallet som en videreføring av simulatorene de hadde brukt til Apollo-programmet og romfergen, der hver kritiske komponent hadde et matematisk «tvilling»-system som kunne testes i forkant av reelle operasjoner. Over tid har metoden blitt raffinert med økende datatilgang, IoT-infrastruktur og skytjenester som muliggjør kontinuerlig synkronisering mellom fysisk og digitalt system.

\subsection{Historiske milepæler}
Historien om digitale tvillinger springer ut av romfartens behov for å teste kritiske systemer på bakken før de ble sendt ut i verdensrommet. Simuleringsplattformene NASA utviklet for Apollo-programmet etablerte et tankesett der hver sentral komponent skulle ha et digitalt motstykke som muliggjorde analyse av feilscenarier og hurtige beslutninger underveis i oppdraget \citep{glaessgen2012digital}. I løpet av 2000-tallet ble ideen koblet til produktlivssyklusforvaltning (PLM), og begrepet «digital tvilling» ble popularisert i industrien.

\begin{figure}[ht]
    \centering
    \resizebox{0.95\textwidth}{!}{\begin{tikzpicture}[>=LaTeX, milestone/.style={circle, draw=petrol, fill=white, thick, minimum size=0.7cm}, label/.style={align=center, text width=3.5cm}]
    \draw[thick, dypblaa, -{Latex[length=3mm]}] (0,0) -- (13,0);
    \node[milestone] (apollo) at (1,0) {1969};
    \node[label, above=0.6cm of apollo] {NASA simulerer Apollo-moduler for operasjonell beredskap\citep{glaessgen2012digital}};
    \node[milestone] (plm) at (4,0) {2002};
    \node[label, below=0.6cm of plm] {Grieves introduserer digital tvilling i PLM-litteraturen\citep{grieves2017digital}};
    \node[milestone, fill=solgul!40, draw=solgul!80!black] (nasa) at (6.5,0) {2012};
    \node[label, above=0.6cm of nasa] {NASA formaliserer digital tvilling-strategien for romfart\citep{glaessgen2012digital}};
    \node[milestone] (industry) at (8.5,0) {2017};
    \node[label, below=0.6cm of industry] {Equinor lanserer tvilling for Johan Sverdrup-feltet\citep{equinor2021johansverdrup}};
    \node[milestone] (standard) at (10.5,0) {2021};
    \node[label, above=0.6cm of standard] {DNV utgir anbefalt praksis for digitale tvillinger\citep{dnv2021rp}};
    \node[milestone] (city) at (12.5,0) {2024};
    \node[label, below=0.6cm of city] {Trondheim tester bytvilling for mobilitet og klima\citep{trondheim2024bytvilling}};
\end{tikzpicture}
}
    \caption{Tidslinje for sentrale milepæler i utviklingen av digitale tvillinger fra romfart til norske industri- og byprosjekter.}
    \label{fig:kap01-tidslinje}
\end{figure}

\begin{itemize}
    \item \textbf{1960--1970-tallet:} NASA utvikler speilende simuleringssystemer for Apollo og Skylab for å planlegge oppdrag og håndtere feilscenarier \citep{glaessgen2012digital}.
    \item \textbf{2002:} Michael Grieves lanserer digital tvilling-konseptet i PLM-litteraturen, og legger grunnlaget for dagens terminologi \citep{grieves2017digital}.
    \item \textbf{2010--2016:} NASA formaliserer Digital Twin-strategien, EU inkluderer konseptet i Horizon 2020-programmer, og ISO starter standardiseringsarbeid rundt industrielle datastrømmer.
    \item \textbf{2017:} Equinor introduserer digitale tvillinger for Johan Sverdrup-feltet for å kombinere sanntidsdata med historisk produksjonsinformasjon \citep{equinor2021johansverdrup}.
    \item \textbf{2018--2021:} Kongsberg Digital etablerer sin Kognitwin-plattform for prosessindustri, mens DNV og SINTEF samarbeider om normative rammeverk for digitale tvillinger i maritim sektor \citep{dnv2021rp,sintef2021digital}.
    \item \textbf{2024:} Norske kommuner tester bytvillinger for mobilitet, energi og klimatilpasning som del av omstillingsløp til grønnere byer \citep{trondheim2024bytvilling}.
\end{itemize}

Hvert trinn i tidslinjen i figur~\ref{fig:kap01-tidslinje} markerer både teknologiske og organisatoriske nyvinninger. Fra NASA sin vektlegging av redundant beslutningsstøtte til DNVs anbefalte praksis for industrielle tvillinger \citep{dnv2021rp} har begrepet gått fra å være et teknisk støtteverktøy til å bli en driver for helhetlig virksomhetsstyring. Norske pionerprosjekter, som Equinors Johan Sverdrup-tvillinger og Trondheim kommunes bytvilling for mobilitet, viser hvordan historiske erfaringer nå brukes til å balansere klimaambisjoner, sikkerhet og verdiskaping i offentlig sektor \citep{equinor2021johansverdrup,trondheim2024bytvilling}.

\subsection{Norske initiativer}
Norge har vært tidlig ute med å bruke digitale tvillinger i kritisk infrastruktur. Equinor, Aker BP og Gassco bruker teknologien for å optimalisere produksjon og vedlikehold på sokkelen, mens Statnett tester digitale kopier av kraftnettkomponenter for å simulere belastningstopper og planlegge nettiltak. I maritim sektor har Kongsberg Gruppen og DNV utviklet sertifiseringsløp og testfasiliteter, blant annet gjennom Ocean Space Centre i Trondheim. Kommunal sektor eksperimenterer med digitale tvillinger av bygg og byrom, slik som Statsbyggs modell for det nye regjeringskvartalet og Trondheim kommunes digitale tvilling av Sluppen-området for mobilitetsanalyse. Disse eksemplene viser hvordan historikken fra romfart og internasjonal industri har blitt tatt videre og tilpasset norske behov.

\subsection{Nøkkelbegreper}
En \emph{digital tvilling} omtales ofte som en digital representasjon av et fysisk system som er koblet til den virkelige enheten gjennom en kontinuerlig datautveksling. NASA vektlegger beslutningsstøtten som oppstår når modellen og den fysiske tvillingen lærer av hverandre, mens ISO 23247 og EU-kommisjonens veikart for industrielle tvillinger fremhever livssyklusperspektivet der utvikling, drift og avvikling henger sammen. I norsk kontekst brukes begrepet gjerne for å beskrive alt fra et detaljerte prosessanlegg til et helt byområde, men kjernen er alltid denne toveis forbindelsen mellom data, modell og fysisk verden.

Det er nyttig å skille mellom \emph{digital modell}, \emph{digital skygge} og \emph{digital tvilling}. En digital modell er en statisk representasjon --- for eksempel en BIM-modell --- som ikke nødvendigvis oppdateres når virkeligheten endrer seg. En digital skygge \textit{(digital shadow)} mottar sanntidsdata, men påvirker ikke det fysiske systemet tilbake. En digital tvilling går ett steg videre og muliggjør beslutninger som sendes tilbake i form av styringssignaler, vedlikeholdsoppdrag eller endrede operasjonsgrenser. Flere norske virksomheter, som Equinor og Statnett, har vært nøye med å bruke denne tredelingen for å tydeliggjøre modenhetsnivået i sine programmer.

I boken bruker vi konsekvent norske begreper der det finnes innarbeidede ord, men beholder engelske termer når de er bransjestandard. Dette betyr at leseren vil møte både «tilstandsestimering» og \emph{state estimation}, «tilbakeskriving» og \emph{backcasting}, samt «datainnsamlingsrør» og \emph{data pipelines}. Tabeller og figurer angir alltid begge språkvarianter første gang et begrep introduseres, slik at masterstudenter kan bevege seg sømløst mellom norske prosjekter og internasjonal litteratur.

\section{Økosystemet rundt digitale tvillinger}
En digital tvilling er mer enn modellen alene; den står på skuldrene til et helt økosystem av sensorer, kommunikasjon og dataplattformer. Sensorer og IoT-enheter etablerer det kontinuerlige sanntidsbildet som gjør at tvillingen kan oppdateres. På Johan Sverdrup-feltet kombineres vibrasjonsdata, trykkmålinger og produksjonsdata fra borebrønner i samme datastrøm før de kvalitetssikres og sendes til Equinors skyplattform. Tilsvarende har Bane NOR montert sensorer på kontaktledninger og sporfundament for å kunne simulere belastningsendringer i digitale tvillinger av jernbaneinfrastruktur.

Forbindelsen mellom det fysiske og det digitale håndteres ofte gjennom en såkalt \emph{digital thread}, et sett av datatjenester som binder sammen konstruksjonsdata, operative data og vedlikeholdslogger. Når Statnett utvikler tvillinger av transformatorstasjoner, hentes grunnlagsdata fra PLM-systemer, driftsdata fra SCADA, og risikomodeller fra analyseverktøy. Integrasjonen skjer via hendelsesdrevne API-er som sikrer at endringer i én del av økosystemet raskt reflekteres i tvillingen.

Teknologien alene er ikke nok. Domeneeksperter, dataingeniører og beslutningstakere må arbeide tverrfaglig for å etablere valide modeller og tolke resultatene. Trondheim kommunes mobilitetstvilling illustrerer dette: byplanleggere definerer hvilke scenarier som skal simuleres, dataforskere bygger trafikkmodellene, og tjenestedesignere vurderer hvordan innsikten skal brukes i medvirkningsprosesser med innbyggerne. Et velfungerende økosystem krever derfor tydelige roller, forvaltningsmodeller og kontinuerlig kompetansebygging.

\section{Verdiskaping og anvendelser}
Digitale tvillinger skaper verdi når de setter organisasjoner i stand til å handle raskere, tryggere og mer bærekraftig. I prosessindustrien brukes tvillinger til å optimalisere produksjonsparametere og redusere energiforbruk. Yara Porsgrunn rapporterer at de, gjennom en tvilling av ammoniakkfabrikken, kunne identifisere driftsvinduer som reduserte CO$_2$-utslipp med flere prosent uten å gå på akkord med sikkerheten. I maritim sektor benytter Kongsberg Digital sin \emph{Kognitwin}-plattform til å overvåke dynamisk posisjonering og drivstoffbruk på offshorefartøy, noe som gjør det mulig å planlegge vedlikehold før komponenter feiler.

Helsesektoren er i ferd med å ta i bruk digitale tvillinger av pasientforløp. Helseplattformen og NTNU har etablert pilotprosjekter der anonymiserte pasientdata brukes til å simulere behandlingsforløp for kronikere, slik at kliniske beslutninger kan støtte seg på sannsynlige utfall fremfor intuisjon alene. Slike løsninger gir ikke bare bedre ressursutnyttelse, men styrker også pasientsikkerheten ved å avdekke risikoer tidlig.

Verdiskapingen handler også om samfunnsmål. Digitale tvillinger av byrom, som Bodøs planlagte Smart Bodø-tvillingen, gir kommunene mulighet til å teste klimapolitikk og mobilitetsgrep før de implementeres. Når resultatene kobles med livssyklusanalyser og klima-rapportering, kan tvillingene dokumentere effekten av tiltak for utslippsreduksjoner og klimatilpasning. Dermed blir teknologien et verktøy for å levere på både økonomiske og bærekraftige mål, og masterstudenter får et tydelig bilde av hvordan teoretiske modeller oversettes til samfunnsnytte.

\subsection{Case: Johan Sverdrup som læringsplattform}
Equinor har brukt Johan Sverdrup-feltet som et testlaboratorium for integrerte arbeidsprosesser der digitale tvillinger kobles til vedlikeholdsplanlegging, produksjonsstyring og HMS-arbeid \citep{equinor2021johansverdrup}. Tvillingen kombinerer geologiske modeller, produksjonsdata og sensorer i brønnstrømmen slik at ingeniører kan simulere tiltak før de iverksettes offshore. Samtidig brukes modellen i dialog med leverandører for å evaluere nye komponenter etter DNVs anbefalte praksis \citep{dnv2021rp}. Tabell~\ref{tab:kap01-johan-sverdrup} oppsummerer hvordan dataflyt og organisering er bygget opp.

\begin{table}[ht]
    \centering
    \caption{Nøkkelkomponenter i Johan Sverdrup-tvillingen slik de brukes i Equinors læringsprogram.}
    \label{tab:kap01-johan-sverdrup}
    \begin{tabular}{p{0.32\textwidth}p{0.58\textwidth}}
        \toprule
        \textbf{Dimensjon} & \textbf{Beskrivelse} \\
        \midrule
        Datakilder & 1\,500 sanntidssensorer fra prosessanlegget, historiske produksjonskurver og integrasjon mot vedlikeholdssystem (SAP). \\
        Modellbibliotek & Fysikkbaserte reservoarmodeller, strømning i rør, maskinlæringsmodeller for pumpestatus og energioptimalisering. \\
        Arbeidsprosesser & Tverrfaglige «control rooms» med daglige simuleringsøkter og ukentlige forbedringsmøter med leverandører. \\
        Effekter & Redusert nedetid i kritiske kompressorer med 30\% og dokumentert utslippsreduksjon per fat produsert. \\
        Læringsarena & Brukes i masterkurs med deling av syntetiske datasett og scenarioer for risikotrening. \\
        \bottomrule
    \end{tabular}
\end{table}

\subsection{Case: Trondheim bylab for mobilitet}
Trondheim kommune kombinerer trafikkdata, klimascenarioer og innbyggerdialog i en bytvilling som brukes til å planlegge mobilitets- og energioppgradering i Sluppen-området \citep{trondheim2024bytvilling}. Tvillingen gjør det mulig å teste tiltak som ladeinfrastruktur, busslinjer og overvannshåndtering før investeringer besluttes. Innsikt fra modellen kobles til nasjonale klimamål og lokale budsjetter slik at politiske beslutninger kan dokumenteres.

\begin{table}[ht]
    \centering
    \caption{Faktorer Trondheim bylab vurderer ved bruk av digital tvilling for mobilitet.}
    \label{tab:kap01-trondheim}
    \begin{tabular}{p{0.35\textwidth}p{0.55\textwidth}}
        \toprule
        \textbf{Faktor} & \textbf{Utdyping} \\
        \midrule
        Datagrunnlag & Sanntid fra IoT-sensorer i vegbanen, kollektivdata fra AtB og klima-projeksjoner fra MET. \\
        Interessenter & Kommunal planavdeling, næringsliv, kollektivselskap og innbyggerpanel. \\
        Beslutningspunkter & Prioritering av sykkelveger, plassering av hurtigladere, rekkefølge på grønn infrastruktur. \\
        Visualisering & 3D-modell i kommunens dataplattform med scenario-dashboard og AR-visning i folkemøter. \\
        Resultater & Identifiserte 12\% potensial for energireduksjon i byggforvaltningen og redusert reisetid i rushtid med 8\%. \\
        \bottomrule
    \end{tabular}
\end{table}

Kombinasjonen av slike caser viser hvordan digitale tvillinger både støtter teknisk optimalisering og demokratiske prosesser når modellene brukes som en arena for samskaping.

\section{Refleksjonsspørsmål}
\begin{enumerate}
    \item Hvilke komponenter mener du er viktigst for å lykkes med en digital tvilling i ditt fagområde?
    \item Hvordan skiller en digital tvilling seg fra tradisjonelle simuleringsmodeller?
    \item Gi et eksempel på hvordan dataflyt påvirker verdien av en digital tvilling.
\end{enumerate}

