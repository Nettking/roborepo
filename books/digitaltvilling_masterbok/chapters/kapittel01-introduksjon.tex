\chapter{Introduksjon til digitale tvillinger}

\section{Læringsmål}
\begin{itemize}
    \item Forklare hva en digital tvilling er og hvordan begrepet har utviklet seg.
    \item Skille mellom digitale tvillinger, digitale skygger og klassiske simuleringsmodeller.
    \item Identifisere nøkkelkomponentene som trengs for å realisere en digital tvilling.
\end{itemize}

\section{Bakgrunn og definisjoner}
Begrepet «digital tvilling» dukket først opp i industrien rundt årtusenskiftet da Michael Grieves beskrev hvordan et fysisk produkt kunne speiles digitalt gjennom hele livssyklusen. NASA populariserte tankegangen tidlig på 2010-tallet som en videreføring av simulatorene de hadde brukt til Apollo-programmet og romfergen, der hver kritiske komponent hadde et matematisk «tvilling»-system som kunne testes i forkant av reelle operasjoner. Over tid har metoden blitt raffinert med økende datatilgang, IoT-infrastruktur og skytjenester som muliggjør kontinuerlig synkronisering mellom fysisk og digitalt system.

\subsection{Historiske milepæler}
\begin{itemize}
    \item \textbf{1960--1970-tallet:} NASA utvikler speilende simuleringssystemer for Apollo og Skylab for å planlegge oppdrag og håndtere feilscenarier.
    \item \textbf{2002:} Michael Grieves lanserer digital tvilling-konseptet i PLM-litteraturen, og legger grunnlaget for dagens terminologi.
    \item \textbf{2010--2016:} NASA formaliserer Digital Twin-strategien, EU inkluderer konseptet i Horizon 2020-programmer, og ISO starter standardiseringsarbeid rundt industrielle datastrømmer.
    \item \textbf{2017:} Equinor introduserer digitale tvillinger for Johan Sverdrup-feltet for å kombinere sanntidsdata med historisk produksjonsinformasjon.
    \item \textbf{2018--2020:} Kongsberg Digital etablerer sin Kognitwin-plattform for prosessindustri, mens DNV og SINTEF samarbeider om normative rammeverk for digitale tvillinger i maritim sektor.
\end{itemize}

\subsection{Norske initiativer}
Norge har vært tidlig ute med å bruke digitale tvillinger i kritisk infrastruktur. Equinor, Aker BP og Gassco bruker teknologien for å optimalisere produksjon og vedlikehold på sokkelen, mens Statnett tester digitale kopier av kraftnettkomponenter for å simulere belastningstopper og planlegge nettiltak. I maritim sektor har Kongsberg Gruppen og DNV utviklet sertifiseringsløp og testfasiliteter, blant annet gjennom Ocean Space Centre i Trondheim. Kommunal sektor eksperimenterer med digitale tvillinger av bygg og byrom, slik som Statsbyggs modell for det nye regjeringskvartalet og Trondheim kommunes digitale tvilling av Sluppen-området for mobilitetsanalyse. Disse eksemplene viser hvordan historikken fra romfart og internasjonal industri har blitt tatt videre og tilpasset norske behov.

\subsection{Nøkkelbegreper}
\begin{itemize}
    \item Definisjoner brukt av sentrale aktører (ISO, NASA, EU).
    \item Terminologi på norsk og engelsk.
\end{itemize}

\section{Økosystemet rundt digitale tvillinger}
\begin{itemize}
    \item Rollen til sensorer, IoT og dataplattformer.
    \item Samspillet mellom fysisk system, digital modell og data.
    \item Viktigheten av domeneekspertise og tverrfaglighet.
\end{itemize}

\section{Verdiskaping og anvendelser}
\begin{itemize}
    \item Typiske mål: optimalisering, overvåkning, prediktivt vedlikehold.
    \item Norske eksempler fra energi, maritim sektor og helse.
    \item Hvordan digitale tvillinger støtter bærekraft og klima.
\end{itemize}

\section{Refleksjonsspørsmål}
\begin{enumerate}
    \item Hvilke komponenter mener du er viktigst for å lykkes med en digital tvilling i ditt fagområde?
    \item Hvordan skiller en digital tvilling seg fra tradisjonelle simuleringsmodeller?
    \item Gi et eksempel på hvordan dataflyt påvirker verdien av en digital tvilling.
\end{enumerate}

% Oppdatert: Utdypet historiske eksempler og norske initiativer i seksjon om bakgrunn.
