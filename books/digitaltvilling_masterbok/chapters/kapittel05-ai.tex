\chapter{Læring, optimalisering og kunstig intelligens}

\section{Læringsmål}
\begin{itemize}
    \item Forstå hvordan maskinlæring, optimalisering og kunstig intelligens støtter digitale tvillinger.
    \item Vurdere når dataassimilering og online læring er hensiktsmessig.
    \item Designe eksperimenter for å kombinere simulerings- og læringsmodeller.
\end{itemize}

\section{Maskinlæring i digitale tvillinger}
\begin{itemize}
    \item Overvåket, ikke-overvåket og forsterkende læring.
    \item Bruk av surrogate modeller og metamodeller.
    \item Transfer learning og federated learning i industrien.
\end{itemize}

\section{Optimalisering og beslutningsstøtte}
\begin{itemize}
    \item Kombinasjon av modellprediktiv kontroll og digitale tvillinger.
    \item Bayesiansk optimalisering, genetiske algoritmer og gradientbaserte metoder.
    \item Håndtering av målkonflikter og begrensninger.
\end{itemize}

\section{Dataassimilering og online oppdatering}
\begin{itemize}
    \item Filtreringsteknikker (Kalman, partikkelfiltre, ensemblestøtte).
    \item Kontinuerlig kalibrering og konseptdrift.
    \item Arkitektur for sanntidsoppdateringer.
\end{itemize}

\section{Refleksjonsspørsmål og øvinger}
\begin{enumerate}
    \item Velg et case og beskriv hvordan du ville brukt maskinlæring i den digitale tvillingen.
    \item Forklar hvordan modellprediktiv kontroll kan kombineres med digitale tvillinger.
    \item Lag et forslag til eksperimentoppsett for online læring med begrenset datatilgang.
\end{enumerate}
