\chapter{Livssyklus, organisering og styring}

\section{Læringsmål}
\begin{itemize}
    \item Planlegge livssyklusen til en digital tvilling fra idé til avhending.
    \item Beskrive governance-strukturer og roller.
    \item Utforme strategier for endringsledelse og kontinuerlig forbedring.
\end{itemize}

\section{Livssyklusfaser}
\begin{itemize}
    \item Initiativ, konsept, design, implementering, drift, avvikling.
    \item Håndtering av versjoner og konfigurasjonsstyring.
    \item Sammenheng med PLM, ALM og DevOps.
\end{itemize}

\section{Organisering og roller}
\begin{itemize}
    \item Tverrfaglige team og samarbeidsmodeller.
    \item Roller: produkteiere, domeneeksperter, data scientists, IT-drift.
    \item Samarbeid med leverandører og partnere.
\end{itemize}

\section{Endringsledelse og modenhet}
\begin{itemize}
    \item Modenhetsmodeller for digitale tvillinger.
    \item Gevinstrealisering og måleindikatorer (KPI-er, OKR-er).
    \item Kommunikasjon og opplæring.
\end{itemize}

\section{Refleksjonsspørsmål og øvinger}
\begin{enumerate}
    \item Utarbeid en RACI-matrise for et digitalt tvillingprosjekt i maritim sektor.
    \item Foreslå KPI-er for å følge opp gevinstrealisering.
    \item Beskriv tiltak for å heve organisasjonens modenhet.
\end{enumerate}
